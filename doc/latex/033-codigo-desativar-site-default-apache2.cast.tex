\documentclass{scrartcl}
\usepackage[utf8]{inputenc}
\usepackage{fancyvrb}
\usepackage[usenames,dvipsnames]{xcolor}
% \definecolor{red-sd}{HTML}{7ed2d2}

\title{}

\fvset{commandchars=\\\{\}}

\begin{document}

\begin{Verbatim}
]0;admlocal@devOps: ~/leantime_ansibleadmlocal@devOps:~/leantime_ansible$ nano tasks/tasks.yml 
[?2004h[?1049h[?7h[?1h=[?1h=[?25l\textcolor{inv_background inv_foreground}{[ A ler... ]}\textcolor{inv_background inv_foreground}{[ 173 linhas lidas ]}\textcolor{inv_background inv_foreground}{  GNU nano 4.8                                                tasks/tasks.yml                                                             }
\textcolor{inv_background inv_foreground}{^G} Ajuda\textcolor{inv_background inv_foreground}{^O} Gravar\textcolor{inv_background inv_foreground}{^W} Procurar\textcolor{inv_background inv_foreground}{^K} Cortar txt    \textcolor{inv_background inv_foreground}{^J} Justificar    \textcolor{inv_background inv_foreground}{^C} Pos cursor    \textcolor{inv_background inv_foreground}{M-U} Desfazer     \textcolor{inv_background inv_foreground}{M-A} Marcar txt
\textcolor{inv_background inv_foreground}{^X} Sair\textcolor{inv_background inv_foreground}{^R} Carregar\textcolor{inv_background inv_foreground}{^\} Substituir    \textcolor{inv_background inv_foreground}{^U} Colar txt     \textcolor{inv_background inv_foreground}{^T} Ortografia    \textcolor{inv_background inv_foreground}{^_} Ir p/ linha   \textcolor{inv_background inv_foreground}{M-E} Refazer\textcolor{inv_background inv_foreground}{M-6} Copiar txt
---- name: Atualiza a cache (equivalente a apt update)apt:update_cache: truecache_valid_time: 3600- name: "Atualiza o sistema operativo"apt:upgrade: safe- name: "Instalar as dependencias necessária ao programa leantime"apt:pkg:- mc- screen- git- apache2- mysql-server- php- php-mysql- php-ldap- php-cli- php-soap- php-json- graphviz- php-xml- php-gd- php-zip- libapache2-mod-php- php-dev- libmcrypt-dev- gcc- make- autoconf- libc-dev
[?12l[?25h[?25l - autoconf    - libc-dev  - pkg-config- pwgen- curl- unzip    - zip  - php-mbstring- expect- net-tools    - python3-mysqldb  - python3-apt- python3-pycurl
\textcolor{ansi36}{   # Fonte de inspiração: https://docs.ansible.com/ansible/latest/collections/ansible/builtin/service_module.html}- name: "Instala o serviço apache2 no arranque do sistema"service:name: apache2state: startedenabled: true
\textcolor{ansi36}{   # Fonte: https://www.mydailytutorials.com/using-ansible-find-module-search-filesfolder}- name: "Rotina Auxiliar: Procura o caminho do ficheiro php.ini do servidor apache2"find:paths: /etcpatterns: "php.ini"recurse: trueregister: caminho_php_ini
\textcolor{ansi36}{   # Fonte: https://docs.ansible.com/ansible/latest/collections/ansible/builtin/lineinfile_module.html}- name: "Ativa várias opções no ficheiro php.ini, utilizando o módulo lineinfile."ansible.builtin.lineinfile:path: "{{ caminho_php_ini.files[0].path }}"regexp: "{{ item.regexp  }}"line: "{{ item.line }}"
[?12l[?25h[?25lregexp: "{{ item.regexp  }}"line: "{{ item.line }}"with_items:regexp: "^file_uploads"  line: "file_uploads = 1"regexp: "^upload_max_filesize"  line: "upload_max_filesize = 1G"regexp: "^max_file_uploads"  line: "max_file_uploads = 20"regexp: "^post_max_size"  line: "post_max_size = 2G"regexp: "^memory_limit"  line: "memory_limit = 2G"- regexp: "^max_input_time" line: "max_input_time = 3600"
\textcolor{ansi36}{   # Fonte: https://docs.ansible.com/ansible/2.7/modules/file_module.html}- name: Cria a directoria temporária leantime e a directoria de apache leantimefile:path: "{{ item.path }}"state: directory mode: 0755    owner: www-data  group: www-datawith_items:- path: "/tmp/leantime"- path: "/var/www/leantime"
\textcolor{ansi36}{   # Fonte: Ansible for DevOps - Server and configuration management for humans de Jeff Geerling, página 98
}   - name: "Cria a base de dados de mysql para o leantime"  mysql_db: "db=leantime_database state=present"
- name: "Cria o utilizador para a base de dados leantime"mysql_user:namlogin_leantime }}"
[?12l[?25h[?25lmysql_user:namlogin_leantime }}"  password: "{{ password_leantime }}"priv: "leantime_database.*:ALL"host: localhoststate: present
- name: "Faz o download do software leantime a partir do repositório git"get_url:url: https://github.com/Leantime/leantime/releases/download/{{ versao_leantime }}/Leantime-{{ versao_leantime }}.zipdest: /tmp/leantimemode: 0755
\textcolor{ansi36}{   # O ficheiro Leantime-versao.zip necessita de ser descomprimido}- name: Descomprime o software leantimeunarchive: src: "/tmp/leantime/Leantime-{{ versao_leantime }}.zip"    dest: "/var/www/leantime"  owner: www-datagroup: www-datamode: 0755remote_src: true
\textcolor{ansi36}{   # Após a descompressão, se o ficheiro configuration.php não existir, este necessita de ser
   # criado a partir do ficheiro configuration.sample.php
   # Fonte: https://docs.ansible.com/ansible/latest/collections/ansible/builtin/copy_module.html}- name: Copia o ficheiro de configuração configuration.sample.phpansible.builtin.copy: src: /var/www/leantime/config/configuration.sample.php    dest: /var/www/leantime/config/configuration.php  owner: www-datagroup: www-data    mode: '0755'  remote_src: trueforce: false
[?12l[?25h[?25l  remote_src: trueforce: false
\textcolor{ansi36}{   # Fonte: https://docs.ansible.com/ansible/latest/collections/ansible/builtin/replace_module.html}- name: Altera as variáveis dbuser, dbpassword e dbdatabase no ficheiro configuration.php utilizando o módulo replace.ansible.builtin.replace:path: /var/www/leantime/config/configuration.php    regexp: "{{ item.regexp  }}"  replace: "{{ item.line }}"with_items:- regexp: 'dbUser=""'  line: 'dbUser="{{ login_leantime }}"'- regexp: 'dbDatabase=""' line: 'dbDatabase="leantime_database"'    - regexp: 'dbPassword=""'    line: 'dbPassword="{{ password_leantime }}"'
\textcolor{ansi36}{   # Fonte: https://docs.ansible.com/ansible/latest/collections/ansible/builtin/template_module.html}- name: Copia o template do site para a pásta /etc/apache2/sites-availableansible.builtin.template:src: leantime.template.j2dest: /etc/apache2/sites-available/leantime.confowner: root group: www-data mode: 0644 force: false
\textcolor{ansi36}{   # Fonte: https://docs.ansible.com/ansible/latest/collections/ansible/builtin/file_module.html}- name: "Cria um link simbólico para o ficheiro leantime.conf na pasta /etc/apache/sites-enabled"ansible.builtin.file:src: /etc/apache2/sites-available/leantime.confdest: /etc/apache2/sites-enabled/leantime.confstate: link
\textcolor{ansi36}{   # Fonte: https://docs.ansible.com/ansible/latest/collections/ansible/builtin/file_module.html
}[?12l[?25h[?25l\textcolor{ansi36}{   # Fonte: https://docs.ansible.com/ansible/latest/collections/ansible/builtin/file_module.html}- name: "Cria um link simbólico para o ficheiro leantime.conf na pasta /etc/apache/sites-enabled" ansible.builtin.file:    path: /etc/apache2/sites-available/leantime.conf  owner: rootgroup: www-datamode: '0777'
[?12l[?25h[?25l[?12l[?25h[?25l[?12l[?25h[?25l[?12l[?25h[?25l[?12l[?25h[?25l[?12l[?25h[?25l[?12l[?25h[?25l[?12l[?25h[?25l[?12l[?25h[?25l[?12l[?25h[?25l[?12l[?25h[?25l[?12l[?25h[?25l[?12l[?25h[?25l[?12l[?25h[?25l[?12l[?25h[?25l[?12l[?25h[?25l[?12l[?25h[?25l\textcolor{inv_background inv_foreground}{Modificado}
[?12l[?25h[?25l [?12l[?25h[?25l [?12l[?25h[?25l-[?12l[?25h[?25l
[?12l[?25h[?25ln[?12l[?25h[?25la[?12l[?25h[?25lm[?12l[?25h[?25le[?12l[?25h[?25l:[?12l[?25h[?25l [?12l[?25h[?25l"[?12l[?25h[?25lA[?12l[?25h[?25lp[?12l[?25h[?25la[?12l[?25h[?25lg[?12l[?25h[?25la[?12l[?25h[?25l [?12l[?25h[?25lo[?12l[?25h[?25l [?12l[?25h[?25ll[?12l[?25h[?25li[?12l[?25h[?25ln[?12l[?25h[?25lk[?12l[?25h[?25l [?12l[?25h[?25l0[?12l[?25h[?25l0[?12l[?25h[?25l0[?12l[?25h[?25l-[?12l[?25h[?25ld[?12l[?25h[?25le[?12l[?25h[?25lf[?12l[?25h[?25la[?12l[?25h[?25lu[?12l[?25h[?25ll[?12l[?25h[?25lt[?12l[?25h[?25l.[?12l[?25h[?25lc[?12l[?25h[?25lo[?12l[?25h[?25ln[?12l[?25h[?25lf[?12l[?25h[?25l [?12l[?25h[?25ln[?12l[?25h[?25la[?12l[?25h[?25l [?12l[?25h[?25lp[?12l[?25h[?25la[?12l[?25h[?25ls[?12l[?25h[?25lt[?12l[?25h[?25la[?12l[?25h[?25l [?12l[?25h[?25ls[?12l[?25h[?25li[?12l[?25h[?25lt[?12l[?25h[?25le[?12l[?25h[?25ls[?12l[?25h[?25l-[?12l[?25h[?25le[?12l[?25h[?25ln[?12l[?25h[?25la[?12l[?25h[?25lb[?12l[?25h[?25ll[?12l[?25h[?25le[?12l[?25h[?25ld[?12l[?25h[?25l [?12l[?25h[?25ld[?12l[?25h[?25lo[?12l[?25h[?25l [?12l[?25h[?25la[?12l[?25h[?25lp[?12l[?25h[?25la[?12l[?25h[?25lc[?12l[?25h[?25lh[?12l[?25h[?25le[?12l[?25h[?25l2[?12l[?25h[?25l"[?12l[?25h[?25l
[?12l[?25h[?25l [?12l[?25h[?25l [?12l[?25h[?25l [?12l[?25h[?25l [?12l[?25h[?25la[?12l[?25h[?25ln[?12l[?25h[?25ls[?12l[?25h[?25li[?12l[?25h[?25lb[?12l[?25h[?25ll[?12l[?25h[?25le[?12l[?25h[?25l.[?12l[?25h[?25lb[?12l[?25h[?25lu[?12l[?25h[?25li[?12l[?25h[?25ll[?12l[?25h[?25lt[?12l[?25h[?25li[?12l[?25h[?25ln[?12l[?25h[?25l.[?12l[?25h[?25lf[?12l[?25h[?25li[?12l[?25h[?25ll[?12l[?25h[?25le[?12l[?25h[?25l:[?12l[?25h[?25l
[?12l[?25h[?25l [?12l[?25h[?25l [?12l[?25h[?25l [?12l[?25h[?25l [?12l[?25h[?25l [?12l[?25h[?25l [?12l[?25h[?25lp[?12l[?25h[?25la[?12l[?25h[?25lt[?12l[?25h[?25lh[?12l[?25h[?25l:[?12l[?25h[?25l [?12l[?25h[?25l/[?12l[?25h[?25le[?12l[?25h[?25lt[?12l[?25h[?25lc[?12l[?25h[?25l/[?12l[?25h[?25la[?12l[?25h[?25lp[?12l[?25h[?25la[?12l[?25h[?25lc[?12l[?25h[?25lh[?12l[?25h[?25le[?12l[?25h[?25l2[?12l[?25h[?25l/[?12l[?25h[?25ls[?12l[?25h[?25li[?12l[?25h[?25lt[?12l[?25h[?25le[?12l[?25h[?25ls[?12l[?25h[?25l-[?12l[?25h[?25le[?12l[?25h[?25ln[?12l[?25h[?25la[?12l[?25h[?25lb[?12l[?25h[?25ll[?12l[?25h[?25le[?12l[?25h[?25ld[?12l[?25h[?25l/[?12l[?25h[?25l0[?12l[?25h[?25l0[?12l[?25h[?25l0[?12l[?25h[?25l-[?12l[?25h[?25ld[?12l[?25h[?25le[?12l[?25h[?25lf[?12l[?25h[?25la[?12l[?25h[?25lu[?12l[?25h[?25ll[?12l[?25h[?25lt[?12l[?25h[?25l.[?12l[?25h[?25lc[?12l[?25h[?25lo[?12l[?25h[?25ln[?12l[?25h[?25lf[?12l[?25h[?25l
[?12l[?25h[?25l [?12l[?25h[?25l [?12l[?25h[?25l [?12l[?25h[?25l [?12l[?25h[?25l [?12l[?25h[?25l [?12l[?25h[?25ls[?12l[?25h[?25lt[?12l[?25h[?25la[?12l[?25h[?25lt[?12l[?25h[?25le[?12l[?25h[?25l:[?12l[?25h[?25l [?12l[?25h[?25lab[?12l[?25h[?25ls[?12l[?25h[?25le[?12l[?25h[?25ln[?12l[?25h[?25lt[?12l[?25h[?25l
[?12l[?25h[?25l         \textcolor{inv_background inv_foreground}{M-D} Formato DOS\textcolor{inv_background inv_foreground}{M-A} Anexar\textcolor{inv_background inv_foreground}{B} Segurança\textcolor{inv_background inv_foreground}{C} Cancelar           \textcolor{inv_background inv_foreground}{M-M} Formato Mac\textcolor{inv_background inv_foreground}{M-P} Prepor\textcolor{inv_background inv_foreground}{^T} P/ ficheiros
\textcolor{inv_background inv_foreground}{Nome do ficheiro onde escrever: tasks/tasks.yml                                                                                           }[?12l[?25h[?25l \textcolor{inv_background inv_foreground}{[ A escrever... ]}\textcolor{inv_background inv_foreground}{          }\textcolor{inv_background inv_foreground}{[ 178 linhas escritas ]}\textcolor{inv_background inv_foreground}{^O} Gravar\textcolor{inv_background inv_foreground}{^W} Procurar      \textcolor{inv_background inv_foreground}{^K} Cortar txt    \textcolor{inv_background inv_foreground}{^J} Justificar    \textcolor{inv_background inv_foreground}{^C} Pos cursor\textcolor{inv_background inv_foreground}{U} Desfazer     \textcolor{inv_background inv_foreground}{M-A} Marcar txt\textcolor{inv_background inv_foreground}{X} Sair    \textcolor{inv_background inv_foreground}{^R} Carregar\textcolor{inv_background inv_foreground}{^\} Substituir    \textcolor{inv_background inv_foreground}{^U} Colar txt     \textcolor{inv_background inv_foreground}{^T} Ortografia    \textcolor{inv_background inv_foreground}{^_} Ir p/ linha   \textcolor{inv_background inv_foreground}{M-E} Refazer      \textcolor{inv_background inv_foreground}{M-6} Copiar txt
[?12l[?25h[?25l[?12l[?25h[?1049l
[?1l>[?2004l]0;admlocal@devOps: ~/leantime_ansibleadmlocal@devOps:~/leantime_ansible$ molecule lint
\textcolor{ansi34}{INFO    } default scenario test matrix: dependency, lint
\textcolor{ansi34}{INFO    } Performing prerun\textcolor{ansi33}{...}
\textcolor{ansi34}{INFO    } Using .cache/roles/nunomourinho.leantime_ansible symlink to current repository in order to enable Ansible to find the role using its expected full name.
\textcolor{ansi34}{INFO    } Added \textcolor{ansi33}{ANSIBLE_ROLES_PATH}=~\textcolor{ansi35}{/.ansible/}\textcolor{ansi95}{roles}:\textcolor{ansi35}{/usr/share/ansible/}\textcolor{ansi95}{roles}:\textcolor{ansi35}{/etc/ansible/}\textcolor{ansi95}{roles}:.\textcolor{ansi35}{/.cache/}\textcolor{ansi95}{roles}
\textcolor{ansi34}{INFO    } \textcolor{ansi2 ansi36}{Running }\textcolor{ansi2 ansi32}{default}\textcolor{ansi2 ansi36}{ > }\textcolor{ansi2 ansi32}{dependency}
\textcolor{ansi31}{WARNING } Skipping, missing the requirements file.
\textcolor{ansi31}{WARNING } Skipping, missing the requirements file.
\textcolor{ansi34}{INFO    } \textcolor{ansi2 ansi36}{Running }\textcolor{ansi2 ansi32}{default}\textcolor{ansi2 ansi36}{ > }\textcolor{ansi2 ansi32}{lint}
COMMAND: set -e
yamllint .
ansible-lint

./tasks/tasks.yml
  175:3     error    syntax error: expected '<document start>', but found '<block sequence start>' (syntax)

\textbf{\textcolor{ansi31}{CRITICAL}} Lint failed with error code \textbf{\textcolor{ansi36}{1}}
]0;admlocal@devOps: ~/leantime_ansibleadmlocal@devOps:~/leantime_ansible$ molecule lintnano tasks/tasks.yml 
[?2004h[?1049h[?7h[?1h=[?1h=[?25l\textcolor{inv_background inv_foreground}{[ A ler... ]}\textcolor{inv_background inv_foreground}{[ 178 linhas lidas ]}\textcolor{inv_background inv_foreground}{  GNU nano 4.8                                                tasks/tasks.yml                                                             }
\textcolor{inv_background inv_foreground}{^G} Ajuda\textcolor{inv_background inv_foreground}{^O} Gravar\textcolor{inv_background inv_foreground}{^W} Procurar\textcolor{inv_background inv_foreground}{^K} Cortar txt    \textcolor{inv_background inv_foreground}{^J} Justificar    \textcolor{inv_background inv_foreground}{^C} Pos cursor    \textcolor{inv_background inv_foreground}{M-U} Desfazer     \textcolor{inv_background inv_foreground}{M-A} Marcar txt
\textcolor{inv_background inv_foreground}{^X} Sair\textcolor{inv_background inv_foreground}{^R} Carregar\textcolor{inv_background inv_foreground}{^\} Substituir    \textcolor{inv_background inv_foreground}{^U} Colar txt     \textcolor{inv_background inv_foreground}{^T} Ortografia    \textcolor{inv_background inv_foreground}{^_} Ir p/ linha   \textcolor{inv_background inv_foreground}{M-E} Refazer\textcolor{inv_background inv_foreground}{M-6} Copiar txt
---- name: Atualiza a cache (equivalente a apt update)apt:update_cache: truecache_valid_time: 3600- name: "Atualiza o sistema operativo"apt:upgrade: safe- name: "Instalar as dependencias necessária ao programa leantime"apt:pkg:- mc- screen- git- apache2- mysql-server- php- php-mysql- php-ldap- php-cli- php-soap- php-json- graphviz- php-xml- php-gd- php-zip- libapache2-mod-php- php-dev- libmcrypt-dev- gcc- make- autoconf- libc-dev
[?12l[?25h[?25l - autoconf    - libc-dev  - pkg-config- pwgen- curl- unzip    - zip  - php-mbstring- expect- net-tools    - python3-mysqldb  - python3-apt- python3-pycurl
\textcolor{ansi36}{   # Fonte de inspiração: https://docs.ansible.com/ansible/latest/collections/ansible/builtin/service_module.html}- name: "Instala o serviço apache2 no arranque do sistema"service:name: apache2state: startedenabled: true
\textcolor{ansi36}{   # Fonte: https://www.mydailytutorials.com/using-ansible-find-module-search-filesfolder}- name: "Rotina Auxiliar: Procura o caminho do ficheiro php.ini do servidor apache2"find:paths: /etcpatterns: "php.ini"recurse: trueregister: caminho_php_ini
\textcolor{ansi36}{   # Fonte: https://docs.ansible.com/ansible/latest/collections/ansible/builtin/lineinfile_module.html}- name: "Ativa várias opções no ficheiro php.ini, utilizando o módulo lineinfile."ansible.builtin.lineinfile:path: "{{ caminho_php_ini.files[0].path }}"regexp: "{{ item.regexp  }}"line: "{{ item.line }}"
[?12l[?25h[?25lregexp: "{{ item.regexp  }}"line: "{{ item.line }}"with_items:regexp: "^file_uploads"  line: "file_uploads = 1"regexp: "^upload_max_filesize"  line: "upload_max_filesize = 1G"regexp: "^max_file_uploads"  line: "max_file_uploads = 20"regexp: "^post_max_size"  line: "post_max_size = 2G"regexp: "^memory_limit"  line: "memory_limit = 2G"- regexp: "^max_input_time" line: "max_input_time = 3600"
\textcolor{ansi36}{   # Fonte: https://docs.ansible.com/ansible/2.7/modules/file_module.html}- name: Cria a directoria temporária leantime e a directoria de apache leantimefile:path: "{{ item.path }}"state: directory mode: 0755    owner: www-data  group: www-datawith_items:- path: "/tmp/leantime"- path: "/var/www/leantime"
\textcolor{ansi36}{   # Fonte: Ansible for DevOps - Server and configuration management for humans de Jeff Geerling, página 98
}   - name: "Cria a base de dados de mysql para o leantime"  mysql_db: "db=leantime_database state=present"
- name: "Cria o utilizador para a base de dados leantime"mysql_user:namlogin_leantime }}"
[?12l[?25h[?25lmysql_user:namlogin_leantime }}"  password: "{{ password_leantime }}"priv: "leantime_database.*:ALL"host: localhoststate: present
- name: "Faz o download do software leantime a partir do repositório git"get_url:url: https://github.com/Leantime/leantime/releases/download/{{ versao_leantime }}/Leantime-{{ versao_leantime }}.zipdest: /tmp/leantimemode: 0755
\textcolor{ansi36}{   # O ficheiro Leantime-versao.zip necessita de ser descomprimido}- name: Descomprime o software leantimeunarchive: src: "/tmp/leantime/Leantime-{{ versao_leantime }}.zip"    dest: "/var/www/leantime"  owner: www-datagroup: www-datamode: 0755remote_src: true
\textcolor{ansi36}{   # Após a descompressão, se o ficheiro configuration.php não existir, este necessita de ser
   # criado a partir do ficheiro configuration.sample.php
   # Fonte: https://docs.ansible.com/ansible/latest/collections/ansible/builtin/copy_module.html}- name: Copia o ficheiro de configuração configuration.sample.phpansible.builtin.copy: src: /var/www/leantime/config/configuration.sample.php    dest: /var/www/leantime/config/configuration.php  owner: www-datagroup: www-data    mode: '0755'  remote_src: trueforce: false
[?12l[?25h[?25l  remote_src: trueforce: false
\textcolor{ansi36}{   # Fonte: https://docs.ansible.com/ansible/latest/collections/ansible/builtin/replace_module.html}- name: Altera as variáveis dbuser, dbpassword e dbdatabase no ficheiro configuration.php utilizando o módulo replace.ansible.builtin.replace:path: /var/www/leantime/config/configuration.php    regexp: "{{ item.regexp  }}"  replace: "{{ item.line }}"with_items:- regexp: 'dbUser=""'  line: 'dbUser="{{ login_leantime }}"'- regexp: 'dbDatabase=""' line: 'dbDatabase="leantime_database"'    - regexp: 'dbPassword=""'    line: 'dbPassword="{{ password_leantime }}"'
\textcolor{ansi36}{   # Fonte: https://docs.ansible.com/ansible/latest/collections/ansible/builtin/template_module.html}- name: Copia o template do site para a pásta /etc/apache2/sites-availableansible.builtin.template:src: leantime.template.j2dest: /etc/apache2/sites-available/leantime.confowner: root group: www-data mode: 0644 force: false
\textcolor{ansi36}{   # Fonte: https://docs.ansible.com/ansible/latest/collections/ansible/builtin/file_module.html}- name: "Cria um link simbólico para o ficheiro leantime.conf na pasta /etc/apache/sites-enabled"ansible.builtin.file:src: /etc/apache2/sites-available/leantime.confdest: /etc/apache2/sites-enabled/leantime.confstate: link
\textcolor{ansi36}{   # Fonte: https://docs.ansible.com/ansible/latest/collections/ansible/builtin/file_module.html
}[?12l[?25h[?25l\textcolor{ansi36}{   # Fonte: https://docs.ansible.com/ansible/latest/collections/ansible/builtin/file_module.html}- name: "Cria um link simbólico para o ficheiro leantime.conf na pasta /etc/apache/sites-enabled" ansible.builtin.file:    path: /etc/apache2/sites-available/leantime.conf  owner: rootgroup: www-datamode: '0777'
  - name: "Apaga o link 000-default.conf na pasta sites-enabled do apache2"ansible.builtin.file:path: /etc/apache2/sites-enabled/000-default.confstate: absent
[?12l[?25h[?25l\textcolor{inv_background inv_foreground}{Modificado}
   - name: "Apaga o link 000-default.conf na pasta sites-enabled do apache2"
 [?12l[?25h[?25l[?12l[?25h[?25l     ansible.builtin.file:
  [?12l[?25h[?25l[?12l[?25h[?25l
 path: /etc/apache2/sites-enabled/000-default.conf
   [?12l[?25h[?25l[?12l[?25h[?25l
 state: absent
[?12l[?25h[?25l
[?12l[?25h[?25l78
M[?12l[?25h[?25l
path: /etc/apache2/sites-enabled/000-default.conf[?12l[?25h[?25l
   [?12l[?25h[?25l
 path: /etc/apache2/sites-enabled/000-default.conf[?12l[?25h[?25l
[?12l[?25h[?25l[?12l[?25h[?25l         \textcolor{inv_background inv_foreground}{M-D} Formato DOS\textcolor{inv_background inv_foreground}{M-A} Anexar\textcolor{inv_background inv_foreground}{B} Segurança\textcolor{inv_background inv_foreground}{C} Cancelar           \textcolor{inv_background inv_foreground}{M-M} Formato Mac\textcolor{inv_background inv_foreground}{M-P} Prepor\textcolor{inv_background inv_foreground}{^T} P/ ficheiros
\textcolor{inv_background inv_foreground}{Nome do ficheiro onde escrever: tasks/tasks.yml                                                                                           }[?12l[?25h[?25l \textcolor{inv_background inv_foreground}{[ A escrever... ]}\textcolor{inv_background inv_foreground}{          }\textcolor{inv_background inv_foreground}{[ 178 linhas escritas ]}\textcolor{inv_background inv_foreground}{^O} Gravar\textcolor{inv_background inv_foreground}{^W} Procurar      \textcolor{inv_background inv_foreground}{^K} Cortar txt    \textcolor{inv_background inv_foreground}{^J} Justificar    \textcolor{inv_background inv_foreground}{^C} Pos cursor\textcolor{inv_background inv_foreground}{U} Desfazer     \textcolor{inv_background inv_foreground}{M-A} Marcar txt\textcolor{inv_background inv_foreground}{X} Sair    \textcolor{inv_background inv_foreground}{^R} Carregar\textcolor{inv_background inv_foreground}{^\} Substituir    \textcolor{inv_background inv_foreground}{^U} Colar txt     \textcolor{inv_background inv_foreground}{^T} Ortografia    \textcolor{inv_background inv_foreground}{^_} Ir p/ linha   \textcolor{inv_background inv_foreground}{M-E} Refazer      \textcolor{inv_background inv_foreground}{M-6} Copiar txt
[?12l[?25h[?25l[?12l[?25h[?1049l
[?1l>[?2004l]0;admlocal@devOps: ~/leantime_ansibleadmlocal@devOps:~/leantime_ansible$ nano tasks/tasks.yml molecule lint
\textcolor{ansi34}{INFO    } default scenario test matrix: dependency, lint
\textcolor{ansi34}{INFO    } Performing prerun\textcolor{ansi33}{...}
\textcolor{ansi34}{INFO    } Using .cache/roles/nunomourinho.leantime_ansible symlink to current repository in order to enable Ansible to find the role using its expected full name.
\textcolor{ansi34}{INFO    } Added \textcolor{ansi33}{ANSIBLE_ROLES_PATH}=~\textcolor{ansi35}{/.ansible/}\textcolor{ansi95}{roles}:\textcolor{ansi35}{/usr/share/ansible/}\textcolor{ansi95}{roles}:\textcolor{ansi35}{/etc/ansible/}\textcolor{ansi95}{roles}:.\textcolor{ansi35}{/.cache/}\textcolor{ansi95}{roles}
\textcolor{ansi34}{INFO    } \textcolor{ansi2 ansi36}{Running }\textcolor{ansi2 ansi32}{default}\textcolor{ansi2 ansi36}{ > }\textcolor{ansi2 ansi32}{dependency}
\textcolor{ansi31}{WARNING } Skipping, missing the requirements file.
\textcolor{ansi31}{WARNING } Skipping, missing the requirements file.
\textcolor{ansi34}{INFO    } \textcolor{ansi2 ansi36}{Running }\textcolor{ansi2 ansi32}{default}\textcolor{ansi2 ansi36}{ > }\textcolor{ansi2 ansi32}{lint}
COMMAND: set -e
yamllint .
ansible-lint

Loading custom .yamllint config file, this extends our internal yamllint config.
]0;admlocal@devOps: ~/leantime_ansibleadmlocal@devOps:~/leantime_ansible$ molecule deploy
Usage: molecule [OPTIONS] COMMAND [ARGS]...
Try "molecule --help" for help.

Error: No such command "deploy".
]0;admlocal@devOps: ~/leantime_ansibleadmlocal@devOps:~/leantime_ansible$ molecule deployconverge
\textcolor{ansi34}{INFO    } default scenario test matrix: dependency, create, prepare, converge
\textcolor{ansi34}{INFO    } Performing prerun\textcolor{ansi33}{...}
\textcolor{ansi34}{INFO    } Using .cache/roles/nunomourinho.leantime_ansible symlink to current repository in order to enable Ansible to find the role using its expected full name.
\textcolor{ansi34}{INFO    } Added \textcolor{ansi33}{ANSIBLE_ROLES_PATH}=~\textcolor{ansi35}{/.ansible/}\textcolor{ansi95}{roles}:\textcolor{ansi35}{/usr/share/ansible/}\textcolor{ansi95}{roles}:\textcolor{ansi35}{/etc/ansible/}\textcolor{ansi95}{roles}:.\textcolor{ansi35}{/.cache/}\textcolor{ansi95}{roles}
\textcolor{ansi34}{INFO    } \textcolor{ansi2 ansi36}{Running }\textcolor{ansi2 ansi32}{default}\textcolor{ansi2 ansi36}{ > }\textcolor{ansi2 ansi32}{dependency}
\textcolor{ansi31}{WARNING } Skipping, missing the requirements file.
\textcolor{ansi31}{WARNING } Skipping, missing the requirements file.
\textcolor{ansi34}{INFO    } \textcolor{ansi2 ansi36}{Running }\textcolor{ansi2 ansi32}{default}\textcolor{ansi2 ansi36}{ > }\textcolor{ansi2 ansi32}{create}
\textcolor{ansi31}{WARNING } Skipping, instances already created.
\textcolor{ansi34}{INFO    } \textcolor{ansi2 ansi36}{Running }\textcolor{ansi2 ansi32}{default}\textcolor{ansi2 ansi36}{ > }\textcolor{ansi2 ansi32}{prepare}
\textcolor{ansi31}{WARNING } Skipping, instances already prepared.
\textcolor{ansi34}{INFO    } \textcolor{ansi2 ansi36}{Running }\textcolor{ansi2 ansi32}{default}\textcolor{ansi2 ansi36}{ > }\textcolor{ansi2 ansi32}{converge}

PLAY [Converge] **************************************************************************************************************************

TASK [Gathering Facts] *******************************************************************************************************************
\textcolor{ansi32}{ok: [ubuntu-20.04]}

TASK [Include leantime_ansible] **********************************************************************************************************

TASK [leantime_ansible : Atualiza a cache (equivalente a apt update)] ********************************************************************
\textcolor{ansi32}{ok: [ubuntu-20.04]}

TASK [leantime_ansible : Atualiza o sistema operativo] ***********************************************************************************
\textcolor{ansi32}{ok: [ubuntu-20.04]}

TASK [leantime_ansible : Instalar as dependencias necessária ao programa leantime] *******************************************************
\textcolor{ansi32}{ok: [ubuntu-20.04]}

TASK [leantime_ansible : Instala o serviço apache2 no arranque do sistema] ***************************************************************
\textcolor{ansi32}{ok: [ubuntu-20.04]}

TASK [leantime_ansible : Rotina Auxiliar: Procura o caminho do ficheiro php.ini do servidor apache2] *************************************
\textcolor{ansi32}{ok: [ubuntu-20.04]}

TASK [leantime_ansible : Ativa várias opções no ficheiro php.ini, utilizando o módulo lineinfile.] ***************************************
\textcolor{ansi32}{ok: [ubuntu-20.04] => (item={'regexp': '^file_uploads', 'line': 'file_uploads = 1'})}
\textcolor{ansi32}{ok: [ubuntu-20.04] => (item={'regexp': '^upload_max_filesize', 'line': 'upload_max_filesize = 1G'})}
\textcolor{ansi32}{ok: [ubuntu-20.04] => (item={'regexp': '^max_file_uploads', 'line': 'max_file_uploads = 20'})}
\textcolor{ansi32}{ok: [ubuntu-20.04] => (item={'regexp': '^post_max_size', 'line': 'post_max_size = 2G'})}
\textcolor{ansi32}{ok: [ubuntu-20.04] => (item={'regexp': '^memory_limit', 'line': 'memory_limit = 2G'})}
\textcolor{ansi32}{ok: [ubuntu-20.04] => (item={'regexp': '^max_input_time', 'line': 'max_input_time = 3600'})}

TASK [leantime_ansible : Cria a directoria temporária leantime e a directoria de apache leantime] ****************************************
\textcolor{ansi32}{ok: [ubuntu-20.04] => (item={'path': '/tmp/leantime'})}
\textcolor{ansi32}{ok: [ubuntu-20.04] => (item={'path': '/var/www/leantime'})}

TASK [leantime_ansible : Cria a base de dados de mysql para o leantime] ******************************************************************
\textcolor{ansi32}{ok: [ubuntu-20.04]}

TASK [leantime_ansible : Cria o utilizador para a base de dados leantime] ****************************************************************
\textbf{\textcolor{ansi35}{[WARNING]: Module did not set no_log for update_password}}
\textcolor{ansi32}{ok: [ubuntu-20.04]}

TASK [leantime_ansible : Faz o download do software leantime a partir do repositório git] ************************************************
\textcolor{ansi32}{ok: [ubuntu-20.04]}

TASK [leantime_ansible : Descomprime o software leantime] ********************************************************************************
\textcolor{ansi32}{ok: [ubuntu-20.04]}

TASK [leantime_ansible : Copia o ficheiro de configuração configuration.sample.php] ******************************************************
\textcolor{ansi32}{ok: [ubuntu-20.04]}

TASK [leantime_ansible : Altera as variáveis dbuser, dbpassword e dbdatabase no ficheiro configuration.php utilizando o módulo replace.] ***
\textcolor{ansi32}{ok: [ubuntu-20.04] => (item={'regexp': 'dbUser=""', 'line': 'dbUser="leantimeDBadmin"'})}
\textcolor{ansi32}{ok: [ubuntu-20.04] => (item={'regexp': 'dbDatabase=""', 'line': 'dbDatabase="leantime_database"'})}
\textcolor{ansi32}{ok: [ubuntu-20.04] => (item={'regexp': 'dbPassword=""', 'line': 'dbPassword="#S3gr3d0S3cr3t0#"'})}

TASK [leantime_ansible : Copia o template do site para a pásta /etc/apache2/sites-available] *********************************************
\textcolor{ansi32}{ok: [ubuntu-20.04]}

TASK [leantime_ansible : Cria um link simbólico para o ficheiro leantime.conf na pasta /etc/apache/sites-enabled] ************************
\textcolor{ansi32}{ok: [ubuntu-20.04]}

TASK [leantime_ansible : Cria um link simbólico para o ficheiro leantime.conf na pasta /etc/apache/sites-enabled] ************************
\textcolor{ansi32}{ok: [ubuntu-20.04]}

TASK [leantime_ansible : Apaga o link 000-default.conf na pasta sites-enabled do apache2] ************************************************
\textcolor{ansi33}{changed: [ubuntu-20.04]}

PLAY RECAP *******************************************************************************************************************************
\textcolor{ansi33}{ubuntu-20.04}               : \textcolor{ansi32}{ok=18  } \textcolor{ansi33}{changed=1   } unreachable=0    failed=0    skipped=0    rescued=0    ignored=0


]0;admlocal@devOps: ~/leantime_ansibleadmlocal@devOps:~/leantime_ansible$ molecule idempotence
\textcolor{ansi34}{INFO    } default scenario test matrix: idempotence
\textcolor{ansi34}{INFO    } Performing prerun\textcolor{ansi33}{...}
\textcolor{ansi34}{INFO    } Using .cache/roles/nunomourinho.leantime_ansible symlink to current repository in order to enable Ansible to find the role using its expected full name.
\textcolor{ansi34}{INFO    } Added \textcolor{ansi33}{ANSIBLE_ROLES_PATH}=~\textcolor{ansi35}{/.ansible/}\textcolor{ansi95}{roles}:\textcolor{ansi35}{/usr/share/ansible/}\textcolor{ansi95}{roles}:\textcolor{ansi35}{/etc/ansible/}\textcolor{ansi95}{roles}:.\textcolor{ansi35}{/.cache/}\textcolor{ansi95}{roles}
\textcolor{ansi34}{INFO    } \textcolor{ansi2 ansi36}{Running }\textcolor{ansi2 ansi32}{default}\textcolor{ansi2 ansi36}{ > }\textcolor{ansi2 ansi32}{idempotence}

PLAY [Converge] **************************************************************************************************************************

TASK [Gathering Facts] *******************************************************************************************************************
\textcolor{ansi32}{ok: [ubuntu-20.04]}

TASK [Include leantime_ansible] **********************************************************************************************************

TASK [leantime_ansible : Atualiza a cache (equivalente a apt update)] ********************************************************************
\textcolor{ansi32}{ok: [ubuntu-20.04]}

TASK [leantime_ansible : Atualiza o sistema operativo] ***********************************************************************************
\textcolor{ansi32}{ok: [ubuntu-20.04]}

TASK [leantime_ansible : Instalar as dependencias necessária ao programa leantime] *******************************************************
\textcolor{ansi32}{ok: [ubuntu-20.04]}

TASK [leantime_ansible : Instala o serviço apache2 no arranque do sistema] ***************************************************************
\textcolor{ansi32}{ok: [ubuntu-20.04]}

TASK [leantime_ansible : Rotina Auxiliar: Procura o caminho do ficheiro php.ini do servidor apache2] *************************************
\textcolor{ansi32}{ok: [ubuntu-20.04]}

TASK [leantime_ansible : Ativa várias opções no ficheiro php.ini, utilizando o módulo lineinfile.] ***************************************
\textcolor{ansi32}{ok: [ubuntu-20.04] => (item={'regexp': '^file_uploads', 'line': 'file_uploads = 1'})}
\textcolor{ansi32}{ok: [ubuntu-20.04] => (item={'regexp': '^upload_max_filesize', 'line': 'upload_max_filesize = 1G'})}
\textcolor{ansi32}{ok: [ubuntu-20.04] => (item={'regexp': '^max_file_uploads', 'line': 'max_file_uploads = 20'})}
\textcolor{ansi32}{ok: [ubuntu-20.04] => (item={'regexp': '^post_max_size', 'line': 'post_max_size = 2G'})}
\textcolor{ansi32}{ok: [ubuntu-20.04] => (item={'regexp': '^memory_limit', 'line': 'memory_limit = 2G'})}
\textcolor{ansi32}{ok: [ubuntu-20.04] => (item={'regexp': '^max_input_time', 'line': 'max_input_time = 3600'})}

TASK [leantime_ansible : Cria a directoria temporária leantime e a directoria de apache leantime] ****************************************
\textcolor{ansi32}{ok: [ubuntu-20.04] => (item={'path': '/tmp/leantime'})}
\textcolor{ansi32}{ok: [ubuntu-20.04] => (item={'path': '/var/www/leantime'})}

TASK [leantime_ansible : Cria a base de dados de mysql para o leantime] ******************************************************************
\textcolor{ansi32}{ok: [ubuntu-20.04]}

TASK [leantime_ansible : Cria o utilizador para a base de dados leantime] ****************************************************************
\textbf{\textcolor{ansi35}{[WARNING]: Module did not set no_log for update_password}}
\textcolor{ansi32}{ok: [ubuntu-20.04]}

TASK [leantime_ansible : Faz o download do software leantime a partir do repositório git] ************************************************
\textcolor{ansi32}{ok: [ubuntu-20.04]}

TASK [leantime_ansible : Descomprime o software leantime] ********************************************************************************
\textcolor{ansi32}{ok: [ubuntu-20.04]}

TASK [leantime_ansible : Copia o ficheiro de configuração configuration.sample.php] ******************************************************
\textcolor{ansi32}{ok: [ubuntu-20.04]}

TASK [leantime_ansible : Altera as variáveis dbuser, dbpassword e dbdatabase no ficheiro configuration.php utilizando o módulo replace.] ***
\textcolor{ansi32}{ok: [ubuntu-20.04] => (item={'regexp': 'dbUser=""', 'line': 'dbUser="leantimeDBadmin"'})}
\textcolor{ansi32}{ok: [ubuntu-20.04] => (item={'regexp': 'dbDatabase=""', 'line': 'dbDatabase="leantime_database"'})}
\textcolor{ansi32}{ok: [ubuntu-20.04] => (item={'regexp': 'dbPassword=""', 'line': 'dbPassword="#S3gr3d0S3cr3t0#"'})}

TASK [leantime_ansible : Copia o template do site para a pásta /etc/apache2/sites-available] *********************************************
\textcolor{ansi32}{ok: [ubuntu-20.04]}

TASK [leantime_ansible : Cria um link simbólico para o ficheiro leantime.conf na pasta /etc/apache/sites-enabled] ************************
\textcolor{ansi32}{ok: [ubuntu-20.04]}

TASK [leantime_ansible : Cria um link simbólico para o ficheiro leantime.conf na pasta /etc/apache/sites-enabled] ************************
\textcolor{ansi32}{ok: [ubuntu-20.04]}

TASK [leantime_ansible : Apaga o link 000-default.conf na pasta sites-enabled do apache2] ************************************************
\textcolor{ansi32}{ok: [ubuntu-20.04]}

PLAY RECAP *******************************************************************************************************************************
\textcolor{ansi32}{ubuntu-20.04}               : \textcolor{ansi32}{ok=18  } changed=0    unreachable=0    failed=0    skipped=0    rescued=0    ignored=0


\textcolor{ansi34}{INFO    } Idempotence completed successfully.
]0;admlocal@devOps: ~/leantime_ansibleadmlocal@devOps:~/leantime_ansible$ molecule verify
\textcolor{ansi34}{INFO    } default scenario test matrix: verify
\textcolor{ansi34}{INFO    } Performing prerun\textcolor{ansi33}{...}
\textcolor{ansi34}{INFO    } Using .cache/roles/nunomourinho.leantime_ansible symlink to current repository in order to enable Ansible to find the role using its expected full name.
\textcolor{ansi34}{INFO    } Added \textcolor{ansi33}{ANSIBLE_ROLES_PATH}=~\textcolor{ansi35}{/.ansible/}\textcolor{ansi95}{roles}:\textcolor{ansi35}{/usr/share/ansible/}\textcolor{ansi95}{roles}:\textcolor{ansi35}{/etc/ansible/}\textcolor{ansi95}{roles}:.\textcolor{ansi35}{/.cache/}\textcolor{ansi95}{roles}
\textcolor{ansi34}{INFO    } \textcolor{ansi2 ansi36}{Running }\textcolor{ansi2 ansi32}{default}\textcolor{ansi2 ansi36}{ > }\textcolor{ansi2 ansi32}{verify}
\textcolor{ansi34}{INFO    } Running Ansible Verifier

PLAY [Infraestrutura conduzida por testes] ***********************************************************************************************

TASK [Variáveis] *************************************************************************************************************************
\textcolor{ansi32}{ok: [ubuntu-20.04]}

TASK [Simulação: Atualizar a cache do sistema] *******************************************************************************************
\textcolor{ansi32}{ok: [ubuntu-20.04]}

TASK [teste: a cache encontra-se actualizada?] *******************************************************************************************
\textcolor{ansi32}{ok: [ubuntu-20.04] => {}
\textcolor{ansi32}{    "changed": false,}
\textcolor{ansi32}{    "msg": "SUCESSO: A cache está atualizada"}
\textcolor{ansi32}{}}

TASK [Atualizar o sistema operativo (equivalente a apt upgrade)] *************************************************************************
\textcolor{ansi32}{ok: [ubuntu-20.04]}

TASK [teste: o sistema operativo encontra-se atualizado?] ********************************************************************************
\textcolor{ansi32}{ok: [ubuntu-20.04] => {}
\textcolor{ansi32}{    "changed": false,}
\textcolor{ansi32}{    "msg": "SUCESSO: O sistema operativo está atualizado"}
\textcolor{ansi32}{}}

TASK [Teste: a cache encontra-se atualizada?] ********************************************************************************************
\textcolor{ansi32}{ok: [ubuntu-20.04] => {}
\textcolor{ansi32}{    "changed": false,}
\textcolor{ansi32}{    "msg": "SUCESSO: A cache está atualizada"}
\textcolor{ansi32}{}}

TASK [Simulação: testa se as aplicações dependencia do software leantime se encontram instaladas] ****************************************
\textcolor{ansi32}{ok: [ubuntu-20.04]}

TASK [Teste: as dependencias encontra-se instaladas?] ************************************************************************************
\textcolor{ansi32}{ok: [ubuntu-20.04] => {}
\textcolor{ansi32}{    "changed": false,}
\textcolor{ansi32}{    "msg": "SUCESSO: As dependencias estavam instaladas"}
\textcolor{ansi32}{}}

TASK [Simulação: Testar se o serviço apache2 se encontra instalado, iniciado e ativo no arranque] ****************************************
\textcolor{ansi32}{ok: [ubuntu-20.04]}

TASK [Teste: O serviço apache2 encontra-se ativo no arranque no sistema, e está iniciado?] ***********************************************
\textcolor{ansi32}{ok: [ubuntu-20.04] => {}
\textcolor{ansi32}{    "changed": false,}
\textcolor{ansi32}{    "msg": "SUCESSO: O serviço apache2 está correctamente instalado e inicia com o arranque do sistema"}
\textcolor{ansi32}{}}

TASK [Rotina Auxiliar> Procura o caminho do ficheiro php.ini do servidor apache2] ********************************************************
\textcolor{ansi32}{ok: [ubuntu-20.04]}

TASK [Simulação: Ativa a opção file_uploads no ficheiro php.ini, utilizando o módulo lineinfile.] ****************************************
\textcolor{ansi32}{ok: [ubuntu-20.04] => (item={'regexp': '^file_uploads', 'line': 'file_uploads = 1'})}
\textcolor{ansi32}{ok: [ubuntu-20.04] => (item={'regexp': '^upload_max_filesize', 'line': 'upload_max_filesize = 1G'})}
\textcolor{ansi32}{ok: [ubuntu-20.04] => (item={'regexp': '^max_file_uploads', 'line': 'max_file_uploads = 20'})}
\textcolor{ansi32}{ok: [ubuntu-20.04] => (item={'regexp': '^post_max_size', 'line': 'post_max_size = 2G'})}
\textcolor{ansi32}{ok: [ubuntu-20.04] => (item={'regexp': '^memory_limit', 'line': 'memory_limit = 2G'})}
\textcolor{ansi32}{ok: [ubuntu-20.04] => (item={'regexp': '^max_input_time', 'line': 'max_input_time = 3600'})}

TASK [Teste: As linhas do php.ini encontram-se alteradas ?] ******************************************************************************
\textcolor{ansi32}{ok: [ubuntu-20.04] => {}
\textcolor{ansi32}{    "changed": false,}
\textcolor{ansi32}{    "msg": "SUCESSO: O ficheiro php.ini foi alterado com sucesso"}
\textcolor{ansi32}{}}

TASK [Simulação: Obtem informação sobre a pastas /var/www/leantime] **********************************************************************
\textcolor{ansi32}{ok: [ubuntu-20.04]}

TASK [Teste: A pasta /var/www/leantime existe e tem as permissões certas?] ***************************************************************
\textcolor{ansi32}{ok: [ubuntu-20.04] => {}
\textcolor{ansi32}{    "changed": false,}
\textcolor{ansi32}{    "msg": "SUCESSO: Permissões correctas no site leantime"}
\textcolor{ansi32}{}}

TASK [Simulação: Verifica se é necessário criar a base de dados leantime_database] *******************************************************
\textcolor{ansi32}{ok: [ubuntu-20.04]}

TASK [Teste: A base de dados leantime_database existe ?] *********************************************************************************
\textcolor{ansi32}{ok: [ubuntu-20.04] => {}
\textcolor{ansi32}{    "changed": false,}
\textcolor{ansi32}{    "msg": "SUCESSO: A base de dados leantime_database já se encontra criada"}
\textcolor{ansi32}{}}

TASK [Simulação e Teste: Verifica se o endereço git para a versão de leantime existe] ****************************************************
\textcolor{ansi32}{ok: [ubuntu-20.04]}

TASK [Simulação: Obtem informação sobre o ficheiro /var/www/leantime/config/configuration.php] *******************************************
\textcolor{ansi32}{ok: [ubuntu-20.04]}

TASK [Teste: O ficheiro /var/www/leantime/config/configuration.php existe e tem as permissões certas?] ***********************************
\textcolor{ansi32}{ok: [ubuntu-20.04] => {}
\textcolor{ansi32}{    "changed": false,}
\textcolor{ansi32}{    "msg": "SUCESSO: Permissões correctas e ficheiro configuration.php existente"}
\textcolor{ansi32}{}}

TASK [Simulação: Testa se o dbuser foi alterado no ficheiro configuration.php utilizando o módulo replace.] ******************************
\textcolor{ansi32}{ok: [ubuntu-20.04]}

TASK [Teste: O dbUser foi alterado?] *****************************************************************************************************
\textcolor{ansi32}{ok: [ubuntu-20.04] => {}
\textcolor{ansi32}{    "changed": false,}
\textcolor{ansi32}{    "msg": "SUCESSO: O dbUser foi alterado no ficheiro configuration.php"}
\textcolor{ansi32}{}}

TASK [Simulação: Testa se o dbDatabase foi alterado no ficheiro configuration.php utilizando o módulo replace.] **************************
\textcolor{ansi32}{ok: [ubuntu-20.04]}

TASK [Teste: a variável dbDatabase foi alterada?] ****************************************************************************************
\textcolor{ansi32}{ok: [ubuntu-20.04] => {}
\textcolor{ansi32}{    "changed": false,}
\textcolor{ansi32}{    "msg": "SUCESSO: O dbDatabase foi alterado no ficheiro configuration.php"}
\textcolor{ansi32}{}}

TASK [Simulação: Testa se o dbPassword foi alterado no ficheiro configuration.php utilizando o módulo replace.] **************************
\textcolor{ansi32}{ok: [ubuntu-20.04]}

TASK [Teste: O dbPassword foi alterado?] *************************************************************************************************
\textcolor{ansi32}{ok: [ubuntu-20.04] => {}
\textcolor{ansi32}{    "changed": false,}
\textcolor{ansi32}{    "msg": "SUCESSO: O dbPassword foi alterado no ficheiro configuration.php"}
\textcolor{ansi32}{}}

TASK [Simulação: Obtem informação sobre o ficheiro /etc/apache2/sites-available/leantime.conf] *******************************************
\textcolor{ansi32}{ok: [ubuntu-20.04]}

TASK [Teste: O ficheiro /etc/apache2/sites-available/leantime.conf existe e tem as permissões certas?] ***********************************
\textcolor{ansi32}{ok: [ubuntu-20.04] => {}
\textcolor{ansi32}{    "changed": false,}
\textcolor{ansi32}{    "msg": "SUCESSO: Permissões correctas"}
\textcolor{ansi32}{}}

TASK [Simulação: Obtem informação sobre o ficheiro /etc/apache2/sites-enabled/leantime.conf] *********************************************
\textcolor{ansi32}{ok: [ubuntu-20.04]}

TASK [Teste: O ficheiro /etc/apache2/sites-enabled/leantime.conf existe e tem as permissões certas?] *************************************
\textcolor{ansi32}{ok: [ubuntu-20.04] => {}
\textcolor{ansi32}{    "changed": false,}
\textcolor{ansi32}{    "msg": "SUCESSO: Permissões correctas"}
\textcolor{ansi32}{}}

TASK [Simulação: Obtem informação sobre o ficheiro /etc/apache2/sites-enabled/000-default.conf] ******************************************
\textcolor{ansi32}{ok: [ubuntu-20.04]}

TASK [Teste: O ficheiro /etc/apache2/sites-enabled/000-default.conf NÃO existe?] *********************************************************
\textcolor{ansi32}{ok: [ubuntu-20.04] => {}
\textcolor{ansi32}{    "changed": false,}
\textcolor{ansi32}{    "msg": "All assertions passed"}
\textcolor{ansi32}{}}

PLAY RECAP *******************************************************************************************************************************
\textcolor{ansi32}{ubuntu-20.04}               : \textcolor{ansi32}{ok=32  } changed=0    unreachable=0    failed=0    skipped=0    rescued=0    ignored=0

\textcolor{ansi34}{INFO    } Verifier completed successfully.
]0;admlocal@devOps: ~/leantime_ansibleadmlocal@devOps:~/leantime_ansible$ exit
exit

\end{Verbatim}
\end{document}
