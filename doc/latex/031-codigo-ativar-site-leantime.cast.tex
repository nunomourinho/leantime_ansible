\documentclass{scrartcl}
\usepackage[utf8]{inputenc}
\usepackage{fancyvrb}
\usepackage[usenames,dvipsnames]{xcolor}
% \definecolor{red-sd}{HTML}{7ed2d2}

\title{}

\fvset{commandchars=\\\{\}}

\begin{document}

\begin{Verbatim}
]0;admlocal@devOps: ~/leantime_ansibleadmlocal@devOps:~/leantime_ansible$ nano tasks/tasks.yml 
[?2004h[?1049h[?7h[?1h=[?1h=[?25l\textcolor{inv_background inv_foreground}{[ A ler... ]}\textcolor{inv_background inv_foreground}{[ 157 linhas lidas ]}\textcolor{inv_background inv_foreground}{  GNU nano 4.8                                                tasks/tasks.yml                                                             }
\textcolor{inv_background inv_foreground}{^G} Ajuda\textcolor{inv_background inv_foreground}{^O} Gravar\textcolor{inv_background inv_foreground}{^W} Procurar\textcolor{inv_background inv_foreground}{^K} Cortar txt    \textcolor{inv_background inv_foreground}{^J} Justificar    \textcolor{inv_background inv_foreground}{^C} Pos cursor    \textcolor{inv_background inv_foreground}{M-U} Desfazer     \textcolor{inv_background inv_foreground}{M-A} Marcar txt
\textcolor{inv_background inv_foreground}{^X} Sair\textcolor{inv_background inv_foreground}{^R} Carregar\textcolor{inv_background inv_foreground}{^\} Substituir    \textcolor{inv_background inv_foreground}{^U} Colar txt     \textcolor{inv_background inv_foreground}{^T} Ortografia    \textcolor{inv_background inv_foreground}{^_} Ir p/ linha   \textcolor{inv_background inv_foreground}{M-E} Refazer\textcolor{inv_background inv_foreground}{M-6} Copiar txt
---- name: Atualiza a cache (equivalente a apt update)apt:update_cache: truecache_valid_time: 3600- name: "Atualiza o sistema operativo"apt:upgrade: safe- name: "Instalar as dependencias necessária ao programa leantime"apt:pkg:- mc- screen- git- apache2- mysql-server- php- php-mysql- php-ldap- php-cli- php-soap- php-json- graphviz- php-xml- php-gd- php-zip- libapache2-mod-php- php-dev- libmcrypt-dev- gcc- make- autoconf- libc-dev
[?12l[?25h[?25l - autoconf    - libc-dev  - pkg-config- pwgen- curl- unzip    - zip  - php-mbstring- expect- net-tools    - python3-mysqldb  - python3-apt- python3-pycurl
\textcolor{ansi36}{   # Fonte de inspiração: https://docs.ansible.com/ansible/latest/collections/ansible/builtin/service_module.html}- name: "Instala o serviço apache2 no arranque do sistema"service:name: apache2state: startedenabled: true
\textcolor{ansi36}{   # Fonte: https://www.mydailytutorials.com/using-ansible-find-module-search-filesfolder}- name: "Rotina Auxiliar: Procura o caminho do ficheiro php.ini do servidor apache2"find:paths: /etcpatterns: "php.ini"recurse: trueregister: caminho_php_ini
\textcolor{ansi36}{   # Fonte: https://docs.ansible.com/ansible/latest/collections/ansible/builtin/lineinfile_module.html}- name: "Ativa várias opções no ficheiro php.ini, utilizando o módulo lineinfile."ansible.builtin.lineinfile:path: "{{ caminho_php_ini.files[0].path }}"regexp: "{{ item.regexp  }}"line: "{{ item.line }}"
[?12l[?25h[?25lregexp: "{{ item.regexp  }}"line: "{{ item.line }}"with_items:regexp: "^file_uploads"  line: "file_uploads = 1"regexp: "^upload_max_filesize"  line: "upload_max_filesize = 1G"regexp: "^max_file_uploads"  line: "max_file_uploads = 20"regexp: "^post_max_size"  line: "post_max_size = 2G"regexp: "^memory_limit"  line: "memory_limit = 2G"- regexp: "^max_input_time" line: "max_input_time = 3600"
\textcolor{ansi36}{   # Fonte: https://docs.ansible.com/ansible/2.7/modules/file_module.html}- name: Cria a directoria temporária leantime e a directoria de apache leantimefile:path: "{{ item.path }}"state: directory mode: 0755    owner: www-data  group: www-datawith_items:- path: "/tmp/leantime"- path: "/var/www/leantime"
\textcolor{ansi36}{   # Fonte: Ansible for DevOps - Server and configuration management for humans de Jeff Geerling, página 98
}   - name: "Cria a base de dados de mysql para o leantime"  mysql_db: "db=leantime_database state=present"
- name: "Cria o utilizador para a base de dados leantime"mysql_user:namlogin_leantime }}"
[?12l[?25h[?25lmysql_user:namlogin_leantime }}"  password: "{{ password_leantime }}"priv: "leantime_database.*:ALL"host: localhoststate: present
- name: "Faz o download do software leantime a partir do repositório git"get_url:url: https://github.com/Leantime/leantime/releases/download/{{ versao_leantime }}/Leantime-{{ versao_leantime }}.zipdest: /tmp/leantimemode: 0755
\textcolor{ansi36}{   # O ficheiro Leantime-versao.zip necessita de ser descomprimido}- name: Descomprime o software leantimeunarchive: src: "/tmp/leantime/Leantime-{{ versao_leantime }}.zip"    dest: "/var/www/leantime"  owner: www-datagroup: www-datamode: 0755remote_src: true
\textcolor{ansi36}{   # Após a descompressão, se o ficheiro configuration.php não existir, este necessita de ser
   # criado a partir do ficheiro configuration.sample.php
   # Fonte: https://docs.ansible.com/ansible/latest/collections/ansible/builtin/copy_module.html}- name: Copia o ficheiro de configuração configuration.sample.phpansible.builtin.copy: src: /var/www/leantime/config/configuration.sample.php    dest: /var/www/leantime/config/configuration.php  owner: www-datagroup: www-data    mode: '0755'  remote_src: trueforce: false
[?12l[?25h[?25l  remote_src: trueforce: false
\textcolor{ansi36}{   # Fonte: https://docs.ansible.com/ansible/latest/collections/ansible/builtin/replace_module.html}- name: Altera as variáveis dbuser, dbpassword e dbdatabase no ficheiro configuration.php utilizando o módulo replace.ansible.builtin.replace:path: /var/www/leantime/config/configuration.php    regexp: "{{ item.regexp  }}"  replace: "{{ item.line }}"with_items:- regexp: 'dbUser=""'  line: 'dbUser="{{ login_leantime }}"'- regexp: 'dbDatabase=""' line: 'dbDatabase="leantime_database"'    - regexp: 'dbPassword=""'    line: 'dbPassword="{{ password_leantime }}"'
\textcolor{ansi36}{   # Fonte: https://docs.ansible.com/ansible/latest/collections/ansible/builtin/template_module.html}- name: Copia o template do site para a pásta /etc/apache2/sites-availableansible.builtin.template:src: leantime.template.j2dest: /etc/apache2/sites-available/leantime.confowner: root group: www-data mode: 0755
[?12l[?25h[?25l[?12l[?25h[?25l[?12l[?25h[?25l[?12l[?25h[?25l[?12l[?25h[?25l[?12l[?25h[?25l[?12l[?25h[?25l[?12l[?25h[?25l[?12l[?25h[?25l[?12l[?25h[?25l[?12l[?25h[?25l[?12l[?25h[?25l[?12l[?25h[?25l[?12l[?25h[?25l\textcolor{inv_background inv_foreground}{Modificado}
[?12l[?25h[?25l[?12l[?25h[?25l [?12l[?25h[?25l [?12l[?25h[?25l
\textcolor{ansi36}{  #}[?12l[?25h[?25l\textcolor{ansi36}{ }[?12l[?25h[?25l\textcolor{ansi36}{F}[?12l[?25h[?25l\textcolor{ansi36}{o
}[?12l[?25h[?25l\textcolor{ansi36}{n}[?12l[?25h[?25l\textcolor{ansi36}{t}[?12l[?25h[?25l\textcolor{ansi36}{e}[?12l[?25h[?25l\textcolor{ansi36}{:}[?12l[?25h[?25l\textcolor{ansi36}{ }[?12l[?25h[?25l\textcolor{ansi36}{h}[?12l[?25h[?25l\textcolor{ansi36}{t}[?12l[?25h[?25l\textcolor{ansi36}{t}[?12l[?25h[?25l\textcolor{ansi36}{p}[?12l[?25h[?25l\textcolor{ansi36}{s}[?12l[?25h[?25l\textcolor{ansi36}{:}[?12l[?25h[?25l\textcolor{ansi36}{/}[?12l[?25h[?25l\textcolor{ansi36}{/}[?12l[?25h[?25l\textcolor{ansi36}{d}[?12l[?25h[?25l\textcolor{ansi36}{o}[?12l[?25h[?25l\textcolor{ansi36}{c}[?12l[?25h[?25l\textcolor{ansi36}{s}[?12l[?25h[?25l\textcolor{ansi36}{.}[?12l[?25h[?25l\textcolor{ansi36}{a}[?12l[?25h[?25l\textcolor{ansi36}{n}[?12l[?25h[?25l\textcolor{ansi36}{s}[?12l[?25h[?25l\textcolor{ansi36}{i}[?12l[?25h[?25l\textcolor{ansi36}{b}[?12l[?25h[?25l\textcolor{ansi36}{l}[?12l[?25h[?25l\textcolor{ansi36}{e}[?12l[?25h[?25l\textcolor{ansi36}{.}[?12l[?25h[?25l\textcolor{ansi36}{c}[?12l[?25h[?25l\textcolor{ansi36}{o}[?12l[?25h[?25l\textcolor{ansi36}{m}[?12l[?25h[?25l\textcolor{ansi36}{/}[?12l[?25h[?25l\textcolor{ansi36}{a}[?12l[?25h[?25l\textcolor{ansi36}{n}[?12l[?25h[?25l\textcolor{ansi36}{s}[?12l[?25h[?25l\textcolor{ansi36}{i}[?12l[?25h[?25l\textcolor{ansi36}{b}[?12l[?25h[?25l\textcolor{ansi36}{l}[?12l[?25h[?25l\textcolor{ansi36}{e}[?12l[?25h[?25l\textcolor{ansi36}{/}[?12l[?25h[?25l\textcolor{ansi36}{l}[?12l[?25h[?25l\textcolor{ansi36}{a}[?12l[?25h[?25l\textcolor{ansi36}{t}[?12l[?25h[?25l\textcolor{ansi36}{e}[?12l[?25h[?25l\textcolor{ansi36}{s}[?12l[?25h[?25l\textcolor{ansi36}{t}[?12l[?25h[?25l\textcolor{ansi36}{/}[?12l[?25h[?25l\textcolor{ansi36}{c}[?12l[?25h[?25l\textcolor{ansi36}{o}[?12l[?25h[?25l\textcolor{ansi36}{l}[?12l[?25h[?25l\textcolor{ansi36}{l}[?12l[?25h[?25l\textcolor{ansi36}{e}[?12l[?25h[?25l\textcolor{ansi36}{c}[?12l[?25h[?25l\textcolor{ansi36}{t}[?12l[?25h[?25l\textcolor{ansi36}{i}[?12l[?25h[?25l\textcolor{ansi36}{o}[?12l[?25h[?25l\textcolor{ansi36}{n}[?12l[?25h[?25l\textcolor{ansi36}{s}[?12l[?25h[?25l\textcolor{ansi36}{/}[?12l[?25h[?25l\textcolor{ansi36}{a}[?12l[?25h[?25l\textcolor{ansi36}{n}[?12l[?25h[?25l\textcolor{ansi36}{s}[?12l[?25h[?25l\textcolor{ansi36}{i}[?12l[?25h[?25l\textcolor{ansi36}{b}[?12l[?25h[?25l\textcolor{ansi36}{l}[?12l[?25h[?25l\textcolor{ansi36}{e}[?12l[?25h[?25l\textcolor{ansi36}{/}[?12l[?25h[?25l\textcolor{ansi36}{b}[?12l[?25h[?25l\textcolor{ansi36}{u}[?12l[?25h[?25l\textcolor{ansi36}{i}[?12l[?25h[?25l\textcolor{ansi36}{l}[?12l[?25h[?25l\textcolor{ansi36}{t}[?12l[?25h[?25l\textcolor{ansi36}{i}[?12l[?25h[?25l\textcolor{ansi36}{n}[?12l[?25h[?25l\textcolor{ansi36}{/}[?12l[?25h[?25l\textcolor{ansi36}{f}[?12l[?25h[?25l\textcolor{ansi36}{i}[?12l[?25h[?25l\textcolor{ansi36}{l}[?12l[?25h[?25l\textcolor{ansi36}{e}[?12l[?25h[?25l\textcolor{ansi36}{_}[?12l[?25h[?25l\textcolor{ansi36}{m}[?12l[?25h[?25l\textcolor{ansi36}{o}[?12l[?25h[?25l\textcolor{ansi36}{d}[?12l[?25h[?25l\textcolor{ansi36}{u}[?12l[?25h[?25l\textcolor{ansi36}{l}[?12l[?25h[?25l\textcolor{ansi36}{e}[?12l[?25h[?25l\textcolor{ansi36}{.}[?12l[?25h[?25l\textcolor{ansi36}{h}[?12l[?25h[?25l\textcolor{ansi36}{t}[?12l[?25h[?25l\textcolor{ansi36}{m}[?12l[?25h[?25l\textcolor{ansi36}{l}[?12l[?25h[?25l
[?12l[?25h[?25l [?12l[?25h[?25l [?12l[?25h[?25l-[?12l[?25h[?25l [?12l[?25h[?25ln[?12l[?25h[?25la[?12l[?25h[?25lm[?12l[?25h[?25le[?12l[?25h[?25l:[?12l[?25h[?25l [?12l[?25h[?25l"[?12l[?25h[?25lC[?12l[?25h[?25lr[?12l[?25h[?25li[?12l[?25h[?25la[?12l[?25h[?25l [?12l[?25h[?25lu[?12l[?25h[?25lm[?12l[?25h[?25l [?12l[?25h[?25ll[?12l[?25h[?25li[?12l[?25h[?25ln[?12l[?25h[?25lk[?12l[?25h[?25l [?12l[?25h[?25ls[?12l[?25h[?25li[?12l[?25h[?25lm[?12l[?25h[?25lb[?12l[?25h[?25ló[?12l[?25h[?25ll[?12l[?25h[?25li[?12l[?25h[?25lc[?12l[?25h[?25lo[?12l[?25h[?25l [?12l[?25h[?25lp[?12l[?25h[?25la[?12l[?25h[?25lr[?12l[?25h[?25la[?12l[?25h[?25l [?12l[?25h[?25lo[?12l[?25h[?25l [?12l[?25h[?25lf[?12l[?25h[?25li[?12l[?25h[?25lc[?12l[?25h[?25lh[?12l[?25h[?25le[?12l[?25h[?25li[?12l[?25h[?25lr[?12l[?25h[?25lo[?12l[?25h[?25l [?12l[?25h[?25ll[?12l[?25h[?25le[?12l[?25h[?25la[?12l[?25h[?25ln[?12l[?25h[?25lt[?12l[?25h[?25li[?12l[?25h[?25lm[?12l[?25h[?25le[?12l[?25h[?25l.[?12l[?25h[?25lc[?12l[?25h[?25lo[?12l[?25h[?25ln[?12l[?25h[?25lf[?12l[?25h[?25l [?12l[?25h[?25ln[?12l[?25h[?25la[?12l[?25h[?25l [?12l[?25h[?25lp[?12l[?25h[?25ls[?12l[?25h[?25lt[?12l[?25h[?25la[?12l[?25h[?25l [?12l[?25h[?25l/[?12l[?25h[?25le[?12l[?25h[?25lt[?12l[?25h[?25lc[?12l[?25h[?25l/[?12l[?25h[?25la[?12l[?25h[?25lp[?12l[?25h[?25la[?12l[?25h[?25lc[?12l[?25h[?25lh[?12l[?25h[?25le[?12l[?25h[?25l/[?12l[?25h[?25ls[?12l[?25h[?25li[?12l[?25h[?25lt[?12l[?25h[?25le[?12l[?25h[?25ls[?12l[?25h[?25l-[?12l[?25h[?25le[?12l[?25h[?25ln[?12l[?25h[?25la[?12l[?25h[?25lb[?12l[?25h[?25ll[?12l[?25h[?25le[?12l[?25h[?25ld[?12l[?25h[?25l"[?12l[?25h[?25l
[?12l[?25h[?25l [?12l[?25h[?25l [?12l[?25h[?25l [?12l[?25h[?25l [?12l[?25h[?25la[?12l[?25h[?25ln[?12l[?25h[?25ls[?12l[?25h[?25li[?12l[?25h[?25lb[?12l[?25h[?25ll[?12l[?25h[?25le[?12l[?25h[?25l.[?12l[?25h[?25lb[?12l[?25h[?25lu[?12l[?25h[?25li[?12l[?25h[?25ll[?12l[?25h[?25lt[?12l[?25h[?25li[?12l[?25h[?25ln[?12l[?25h[?25l.[?12l[?25h[?25lf[?12l[?25h[?25li[?12l[?25h[?25ll[?12l[?25h[?25le[?12l[?25h[?25l:[?12l[?25h[?25l
[?12l[?25h[?25l [?12l[?25h[?25l [?12l[?25h[?25l [?12l[?25h[?25l [?12l[?25h[?25l [?12l[?25h[?25l [?12l[?25h[?25ls[?12l[?25h[?25lr[?12l[?25h[?25lc[?12l[?25h[?25l:[?12l[?25h[?25l [?12l[?25h[?25l/[?12l[?25h[?25le[?12l[?25h[?25lt[?12l[?25h[?25lc[?12l[?25h[?25l/[?12l[?25h[?25la[?12l[?25h[?25lp[?12l[?25h[?25la[?12l[?25h[?25lc[?12l[?25h[?25lh[?12l[?25h[?25le[?12l[?25h[?25l2[?12l[?25h[?25l/[?12l[?25h[?25ls[?12l[?25h[?25li[?12l[?25h[?25lt[?12l[?25h[?25le[?12l[?25h[?25ls[?12l[?25h[?25l-[?12l[?25h[?25la[?12l[?25h[?25lv[?12l[?25h[?25la[?12l[?25h[?25li[?12l[?25h[?25ll[?12l[?25h[?25la[?12l[?25h[?25lb[?12l[?25h[?25ll[?12l[?25h[?25le[?12l[?25h[?25l/[?12l[?25h[?25ll[?12l[?25h[?25le[?12l[?25h[?25la[?12l[?25h[?25ln[?12l[?25h[?25lt[?12l[?25h[?25li[?12l[?25h[?25lm[?12l[?25h[?25le[?12l[?25h[?25l.[?12l[?25h[?25lc[?12l[?25h[?25lo[?12l[?25h[?25ln[?12l[?25h[?25lf[?12l[?25h[?25l
[?12l[?25h[?25l [?12l[?25h[?25l [?12l[?25h[?25l [?12l[?25h[?25l [?12l[?25h[?25l [?12l[?25h[?25l [?12l[?25h[?25ld[?12l[?25h[?25le[?12l[?25h[?25ls[?12l[?25h[?25lt[?12l[?25h[?25l:[?12l[?25h[?25l [?12l[?25h[?25l/[?12l[?25h[?25le[?12l[?25h[?25lt[?12l[?25h[?25lc[?12l[?25h[?25l/[?12l[?25h[?25la[?12l[?25h[?25lp[?12l[?25h[?25la[?12l[?25h[?25lc[?12l[?25h[?25lh[?12l[?25h[?25le[?12l[?25h[?25l2[?12l[?25h[?25l/[?12l[?25h[?25ls[?12l[?25h[?25li[?12l[?25h[?25lt[?12l[?25h[?25le[?12l[?25h[?25ls[?12l[?25h[?25l-[?12l[?25h[?25le[?12l[?25h[?25ln[?12l[?25h[?25la[?12l[?25h[?25lb[?12l[?25h[?25ll[?12l[?25h[?25le[?12l[?25h[?25ld[?12l[?25h[?25l
[?12l[?25h[?25l  [?12l[?25h[?25l [?12l[?25h[?25l [?12l[?25h[?25l [?12l[?25h[?25l [?12l[?25h[?25lo[?12l[?25h[?25lw[?12l[?25h[?25ln[?12l[?25h[?25le[?12l[?25h[?25lr[?12l[?25h[?25l:[?12l[?25h[?25l [?12l[?25h[?25lr[?12l[?25h[?25lo[?12l[?25h[?25lo[?12l[?25h[?25lt[?12l[?25h[?25l
[?12l[?25h[?25l [?12l[?25h[?25l [?12l[?25h[?25l [?12l[?25h[?25l [?12l[?25h[?25l [?12l[?25h[?25l [?12l[?25h[?25lg[?12l[?25h[?25lr[?12l[?25h[?25lo[?12l[?25h[?25lu[?12l[?25h[?25lp[?12l[?25h[?25l:[?12l[?25h[?25l [?12l[?25h[?25lr[?12l[?25h[?25lo[?12l[?25h[?25lo[?12l[?25h[?25lt[?12l[?25h[?25l
[?12l[?25h[?25l [?12l[?25h[?25l [?12l[?25h[?25l [?12l[?25h[?25l [?12l[?25h[?25l [?12l[?25h[?25l [?12l[?25h[?25lm[?12l[?25h[?25lo[?12l[?25h[?25ld[?12l[?25h[?25le[?12l[?25h[?25l:[?12l[?25h[?25l [?12l[?25h[?25l'[?12l[?25h[?25l0[?12l[?25h[?25l7[?12l[?25h[?25l7[?12l[?25h[?25l7[?12l[?25h[?25l'[?12l[?25h[?25l
[?12l[?25h[?25l[?12l[?25h[?25lm[?12l[?25h[?25l [?12l[?25h[?25ll[?12l[?25h[?25li[?12l[?25h[?25ln[?12l[?25h[?25lk[?12l[?25h[?25l [?12l[?25h[?25ls[?12l[?25h[?25li[?12l[?25h[?25lm[?12l[?25h[?25lb[?12l[?25h[?25l[?12l[?25h[?25ll[?12l[?25h[?25li[?12l[?25h[?25lc[?12l[?25h[?25lo[?12l[?25h[?25l [?12l[?25h[?25lp[?12l[?25h[?25la[?12l[?25h[?25lr[?12l[?25h[?25la[?12l[?25h[?25l [?12l[?25h[?25lo[?12l[?25h[?25l [?12l[?25h[?25lf[?12l[?25h[?25li[?12l[?25h[?25lc[?12l[?25h[?25lh[?12l[?25h[?25le[?12l[?25h[?25li[?12l[?25h[?25lr[?12l[?25h[?25lo[?12l[?25h[?25l [?12l[?25h[?25ll[?12l[?25h[?25le[?12l[?25h[?25la[?12l[?25h[?25ln[?12l[?25h[?25lt[?12l[?25h[?25li[?12l[?25h[?25lm[?12l[?25h[?25le[?12l[?25h[?25l.[?12l[?25h[?25lc[?12l[?25h[?25lo[?12l[?25h[?25ln[?12l[?25h[?25lf[?12l[?25h[?25l [?12l[?25h[?25ln[?12l[?25h[?25la[?12l[?25h[?25l [?12l[?25h[?25lp[?12l[?25h[?25lasta /etc/apache/sites-enabled"[?12l[?25h[?25l[?12l[?25h[?25l[?12l[?25h[?25l[?12l[?25h[?25l[?12l[?25h[?25l[?12l[?25h[?25l'[?12l[?25h[?25l[?12l[?25h[?25l78

[?12l[?25h[?25l[?12l[?25h[?25l         \textcolor{inv_background inv_foreground}{M-D} Formato DOS\textcolor{inv_background inv_foreground}{M-A} Anexar\textcolor{inv_background inv_foreground}{B} Segurança\textcolor{inv_background inv_foreground}{C} Cancelar           \textcolor{inv_background inv_foreground}{M-M} Formato Mac\textcolor{inv_background inv_foreground}{M-P} Prepor\textcolor{inv_background inv_foreground}{^T} P/ ficheiros
[?12l[?25h[?25l[?12l[?25h[?1049l
[?1l>[?2004l]0;admlocal@devOps: ~/leantime_ansibleadmlocal@devOps:~/leantime_ansible$ molecule lint
\textcolor{ansi34}{INFO    } default scenario test matrix: dependency, lint
\textcolor{ansi34}{INFO    } Performing prerun\textcolor{ansi33}{...}
\textcolor{ansi34}{INFO    } Using .cache/roles/nunomourinho.leantime_ansible symlink to current repository in order to enable Ansible to find the role using its expected full name.
\textcolor{ansi34}{INFO    } Added \textcolor{ansi33}{ANSIBLE_ROLES_PATH}=~\textcolor{ansi35}{/.ansible/}\textcolor{ansi95}{roles}:\textcolor{ansi35}{/usr/share/ansible/}\textcolor{ansi95}{roles}:\textcolor{ansi35}{/etc/ansible/}\textcolor{ansi95}{roles}:.\textcolor{ansi35}{/.cache/}\textcolor{ansi95}{roles}
\textcolor{ansi34}{INFO    } \textcolor{ansi2 ansi36}{Running }\textcolor{ansi2 ansi32}{default}\textcolor{ansi2 ansi36}{ > }\textcolor{ansi2 ansi32}{dependency}
\textcolor{ansi31}{WARNING } Skipping, missing the requirements file.
\textcolor{ansi31}{WARNING } Skipping, missing the requirements file.
\textcolor{ansi34}{INFO    } \textcolor{ansi2 ansi36}{Running }\textcolor{ansi2 ansi32}{default}\textcolor{ansi2 ansi36}{ > }\textcolor{ansi2 ansi32}{lint}
COMMAND: set -e
yamllint .
ansible-lint

./tasks/tasks.yml
  160:3     error    syntax error: expected '<document start>', but found '<block sequence start>' (syntax)

\textbf{\textcolor{ansi31}{CRITICAL}} Lint failed with error code \textbf{\textcolor{ansi36}{1}}
]0;admlocal@devOps: ~/leantime_ansibleadmlocal@devOps:~/leantime_ansible$ molecule lintnano tasks/tasks.yml 
[?2004h[?1049h[?7h[?1h=[?1h=[?25l\textcolor{inv_background inv_foreground}{[ A ler... ]}\textcolor{inv_background inv_foreground}{[ 167 linhas lidas ]}\textcolor{inv_background inv_foreground}{  GNU nano 4.8                                                tasks/tasks.yml                                                             }
\textcolor{inv_background inv_foreground}{^G} Ajuda\textcolor{inv_background inv_foreground}{^O} Gravar\textcolor{inv_background inv_foreground}{^W} Procurar\textcolor{inv_background inv_foreground}{^K} Cortar txt    \textcolor{inv_background inv_foreground}{^J} Justificar    \textcolor{inv_background inv_foreground}{^C} Pos cursor    \textcolor{inv_background inv_foreground}{M-U} Desfazer     \textcolor{inv_background inv_foreground}{M-A} Marcar txt
\textcolor{inv_background inv_foreground}{^X} Sair\textcolor{inv_background inv_foreground}{^R} Carregar\textcolor{inv_background inv_foreground}{^\} Substituir    \textcolor{inv_background inv_foreground}{^U} Colar txt     \textcolor{inv_background inv_foreground}{^T} Ortografia    \textcolor{inv_background inv_foreground}{^_} Ir p/ linha   \textcolor{inv_background inv_foreground}{M-E} Refazer\textcolor{inv_background inv_foreground}{M-6} Copiar txt
---- name: Atualiza a cache (equivalente a apt update)apt:update_cache: truecache_valid_time: 3600- name: "Atualiza o sistema operativo"apt:upgrade: safe- name: "Instalar as dependencias necessária ao programa leantime"apt:pkg:- mc- screen- git- apache2- mysql-server- php- php-mysql- php-ldap- php-cli- php-soap- php-json- graphviz- php-xml- php-gd- php-zip- libapache2-mod-php- php-dev- libmcrypt-dev- gcc- make- autoconf- libc-dev
[?12l[?25h[?25l         In. parág\textcolor{inv_background inv_foreground}{^Y} Prim.linha\textcolor{inv_background inv_foreground}{^T} Ir para txt\textcolor{inv_background inv_foreground}{C} Cancelar           \textcolor{inv_background inv_foreground}{O} Fim parág\textcolor{inv_background inv_foreground}{V} Últ. linha
\textcolor{inv_background inv_foreground}{Insira nº da linha, nº da coluna:                                                                                                         }[?12l[?25h[?25l\textcolor{inv_background inv_foreground}{1
}[?12l[?25h[?25l\textcolor{inv_background inv_foreground}{6
}[?12l[?25h[?25l\textcolor{inv_background inv_foreground}{0
}[?12l[?25h[?25l\textcolor{inv_background inv_foreground}{,
}[?12l[?25h[?25l\textcolor{inv_background inv_foreground}{3
}[?12l[?25h[?25l
\textcolor{inv_background inv_foreground}{^O} GravarProcurar      \textcolor{inv_background inv_foreground}{^K} Cortar txt    \textcolor{inv_background inv_foreground}{^J} Justificar    \textcolor{inv_background inv_foreground}{^C} Pos cursor    \textcolor{inv_background inv_foreground}{M-U} Desfazer     \textcolor{inv_background inv_foreground}{M-A} Marcar txt\textcolor{inv_background inv_foreground}{X} Sair    \textcolor{inv_background inv_foreground}{^R} Carregar\textcolor{inv_background inv_foreground}{\} Substituir    \textcolor{inv_background inv_foreground}{^U} Colar txt\textcolor{inv_background inv_foreground}{T} Ortografia    \textcolor{inv_background inv_foreground}{^_} Ir p/ linha   \textcolor{inv_background inv_foreground}{M-E} Refazer\textcolor{inv_background inv_foreground}{M-6} Copiar txt
 - regexp: 'dbUser=""' line: 'dbUser="{{ login_leantime }}"'  - regexp: 'dbDatabase=""'  line: 'dbDatabase="leantime_database"'- regexp: 'dbPassword=""'line: 'dbPassword="{{ password_leantime }}"'
\textcolor{ansi36}{   # Fonte: https://docs.ansible.com/ansible/latest/collections/ansible/builtin/template_module.html}- name: Copia o template do site para a pásta /etc/apache2/sites-availableansible.builtin.template:    src: leantime.template.j2  dest: /etc/apache2/sites-available/leantime.confowner: rootgroup: www-datamode: 0755
\textcolor{ansi36}{  # Fonte: https://docs.ansible.com/ansible/latest/collections/ansible/builtin/file_module.html}- name: "Cria um link simbólico para o ficheiro leantime.conf na pasta /etc/apache/sites-enabled"ansible.builtin.file:src: /etc/apache2/sites-available/leantime.confdest: /etc/apache2/sites-enabledowner: rootgroup: rootmode: '0777'state: link
  [?12l[?25h[?25l\textcolor{inv_background inv_foreground}{Modificado}
[?12l[?25h[?25l[?12l[?25h[?25l[?12l[?25h[?25l[?12l[?25h[?25l[?12l[?25h[?25l[?12l[?25h[?25l[?12l[?25h[?25l
[?12l[?25h[?25l[?12l[?25h[?25l[?12l[?25h[?25l
\textcolor{ansi36}{   # Fonte: https://docs.ansible.com/ansible/latest/collections/ansible/builtin/file_module.html
   }[?12l[?25h[?25l[?12l[?25h[?25l[?12l[?25h[?25l
     ansible.builtin.file:[?12l[?25h[?25l[?12l[?25h[?25l
 src: /etc/apache2/sites-available/leantime.conf[?12l[?25h[?25l[?12l[?25h[?25l
 dest: /etc/apache2/sites-enabled[?12l[?25h[?25l[?12l[?25h[?25l[?12l[?25h[?25l
 owner: root[?12l[?25h[?25l[?12l[?25h[?25l[?12l[?25h[?25l
 group: root[?12l[?25h[?25l[?12l[?25h[?25l[?12l[?25h[?25l
 mode: '0777'[?12l[?25h[?25l[?12l[?25h[?25l[?12l[?25h[?25l
 state: link[?12l[?25h[?25l
[?12l[?25h[?25l[?12l[?25h[?25l         \textcolor{inv_background inv_foreground}{M-D} Formato DOS\textcolor{inv_background inv_foreground}{M-A} Anexar\textcolor{inv_background inv_foreground}{B} Segurança\textcolor{inv_background inv_foreground}{C} Cancelar           \textcolor{inv_background inv_foreground}{M-M} Formato Mac\textcolor{inv_background inv_foreground}{M-P} Prepor\textcolor{inv_background inv_foreground}{^T} P/ ficheiros
\textcolor{inv_background inv_foreground}{Nome do ficheiro onde escrever: tasks/tasks.yml                                                                                           }[?12l[?25h[?25l \textcolor{inv_background inv_foreground}{[ A escrever... ]}\textcolor{inv_background inv_foreground}{          }\textcolor{inv_background inv_foreground}{[ 167 linhas escritas ]}\textcolor{inv_background inv_foreground}{^O} Gravar\textcolor{inv_background inv_foreground}{^W} Procurar      \textcolor{inv_background inv_foreground}{^K} Cortar txt    \textcolor{inv_background inv_foreground}{^J} Justificar    \textcolor{inv_background inv_foreground}{^C} Pos cursor\textcolor{inv_background inv_foreground}{U} Desfazer     \textcolor{inv_background inv_foreground}{M-A} Marcar txt\textcolor{inv_background inv_foreground}{X} Sair    \textcolor{inv_background inv_foreground}{^R} Carregar\textcolor{inv_background inv_foreground}{^\} Substituir    \textcolor{inv_background inv_foreground}{^U} Colar txt     \textcolor{inv_background inv_foreground}{^T} Ortografia    \textcolor{inv_background inv_foreground}{^_} Ir p/ linha   \textcolor{inv_background inv_foreground}{M-E} Refazer      \textcolor{inv_background inv_foreground}{M-6} Copiar txt
[?12l[?25h[?25l[?12l[?25h[?1049l
[?1l>[?2004l]0;admlocal@devOps: ~/leantime_ansibleadmlocal@devOps:~/leantime_ansible$ nano tasks/tasks.yml molecule lint
\textcolor{ansi34}{INFO    } default scenario test matrix: dependency, lint
\textcolor{ansi34}{INFO    } Performing prerun\textcolor{ansi33}{...}
\textcolor{ansi34}{INFO    } Using .cache/roles/nunomourinho.leantime_ansible symlink to current repository in order to enable Ansible to find the role using its expected full name.
\textcolor{ansi34}{INFO    } Added \textcolor{ansi33}{ANSIBLE_ROLES_PATH}=~\textcolor{ansi35}{/.ansible/}\textcolor{ansi95}{roles}:\textcolor{ansi35}{/usr/share/ansible/}\textcolor{ansi95}{roles}:\textcolor{ansi35}{/etc/ansible/}\textcolor{ansi95}{roles}:.\textcolor{ansi35}{/.cache/}\textcolor{ansi95}{roles}
\textcolor{ansi34}{INFO    } \textcolor{ansi2 ansi36}{Running }\textcolor{ansi2 ansi32}{default}\textcolor{ansi2 ansi36}{ > }\textcolor{ansi2 ansi32}{dependency}
\textcolor{ansi31}{WARNING } Skipping, missing the requirements file.
\textcolor{ansi31}{WARNING } Skipping, missing the requirements file.
\textcolor{ansi34}{INFO    } \textcolor{ansi2 ansi36}{Running }\textcolor{ansi2 ansi32}{default}\textcolor{ansi2 ansi36}{ > }\textcolor{ansi2 ansi32}{lint}
COMMAND: set -e
yamllint .
ansible-lint

Loading custom .yamllint config file, this extends our internal yamllint config.
]0;admlocal@devOps: ~/leantime_ansibleadmlocal@devOps:~/leantime_ansible$ molecule converge
\textcolor{ansi34}{INFO    } default scenario test matrix: dependency, create, prepare, converge
\textcolor{ansi34}{INFO    } Performing prerun\textcolor{ansi33}{...}
\textcolor{ansi34}{INFO    } Using .cache/roles/nunomourinho.leantime_ansible symlink to current repository in order to enable Ansible to find the role using its expected full name.
\textcolor{ansi34}{INFO    } Added \textcolor{ansi33}{ANSIBLE_ROLES_PATH}=~\textcolor{ansi35}{/.ansible/}\textcolor{ansi95}{roles}:\textcolor{ansi35}{/usr/share/ansible/}\textcolor{ansi95}{roles}:\textcolor{ansi35}{/etc/ansible/}\textcolor{ansi95}{roles}:.\textcolor{ansi35}{/.cache/}\textcolor{ansi95}{roles}
\textcolor{ansi34}{INFO    } \textcolor{ansi2 ansi36}{Running }\textcolor{ansi2 ansi32}{default}\textcolor{ansi2 ansi36}{ > }\textcolor{ansi2 ansi32}{dependency}
\textcolor{ansi31}{WARNING } Skipping, missing the requirements file.
\textcolor{ansi31}{WARNING } Skipping, missing the requirements file.
\textcolor{ansi34}{INFO    } \textcolor{ansi2 ansi36}{Running }\textcolor{ansi2 ansi32}{default}\textcolor{ansi2 ansi36}{ > }\textcolor{ansi2 ansi32}{create}
\textcolor{ansi31}{WARNING } Skipping, instances already created.
\textcolor{ansi34}{INFO    } \textcolor{ansi2 ansi36}{Running }\textcolor{ansi2 ansi32}{default}\textcolor{ansi2 ansi36}{ > }\textcolor{ansi2 ansi32}{prepare}
\textcolor{ansi31}{WARNING } Skipping, instances already prepared.
\textcolor{ansi34}{INFO    } \textcolor{ansi2 ansi36}{Running }\textcolor{ansi2 ansi32}{default}\textcolor{ansi2 ansi36}{ > }\textcolor{ansi2 ansi32}{converge}

PLAY [Converge] **************************************************************************************************************************

TASK [Gathering Facts] *******************************************************************************************************************
\textcolor{ansi32}{ok: [ubuntu-20.04]}

TASK [Include leantime_ansible] **********************************************************************************************************

TASK [leantime_ansible : Atualiza a cache (equivalente a apt update)] ********************************************************************
\textcolor{ansi33}{changed: [ubuntu-20.04]}

TASK [leantime_ansible : Atualiza o sistema operativo] ***********************************************************************************
\textcolor{ansi32}{ok: [ubuntu-20.04]}

TASK [leantime_ansible : Instalar as dependencias necessária ao programa leantime] *******************************************************
\textcolor{ansi32}{ok: [ubuntu-20.04]}

TASK [leantime_ansible : Instala o serviço apache2 no arranque do sistema] ***************************************************************
\textcolor{ansi32}{ok: [ubuntu-20.04]}

TASK [leantime_ansible : Rotina Auxiliar: Procura o caminho do ficheiro php.ini do servidor apache2] *************************************
\textcolor{ansi32}{ok: [ubuntu-20.04]}

TASK [leantime_ansible : Ativa várias opções no ficheiro php.ini, utilizando o módulo lineinfile.] ***************************************
\textcolor{ansi32}{ok: [ubuntu-20.04] => (item={'regexp': '^file_uploads', 'line': 'file_uploads = 1'})}
\textcolor{ansi32}{ok: [ubuntu-20.04] => (item={'regexp': '^upload_max_filesize', 'line': 'upload_max_filesize = 1G'})}
\textcolor{ansi32}{ok: [ubuntu-20.04] => (item={'regexp': '^max_file_uploads', 'line': 'max_file_uploads = 20'})}
\textcolor{ansi32}{ok: [ubuntu-20.04] => (item={'regexp': '^post_max_size', 'line': 'post_max_size = 2G'})}
\textcolor{ansi32}{ok: [ubuntu-20.04] => (item={'regexp': '^memory_limit', 'line': 'memory_limit = 2G'})}
\textcolor{ansi32}{ok: [ubuntu-20.04] => (item={'regexp': '^max_input_time', 'line': 'max_input_time = 3600'})}

TASK [leantime_ansible : Cria a directoria temporária leantime e a directoria de apache leantime] ****************************************
\textcolor{ansi32}{ok: [ubuntu-20.04] => (item={'path': '/tmp/leantime'})}
\textcolor{ansi32}{ok: [ubuntu-20.04] => (item={'path': '/var/www/leantime'})}

TASK [leantime_ansible : Cria a base de dados de mysql para o leantime] ******************************************************************
\textcolor{ansi32}{ok: [ubuntu-20.04]}

TASK [leantime_ansible : Cria o utilizador para a base de dados leantime] ****************************************************************
\textbf{\textcolor{ansi35}{[WARNING]: Module did not set no_log for update_password}}
\textcolor{ansi32}{ok: [ubuntu-20.04]}

TASK [leantime_ansible : Faz o download do software leantime a partir do repositório git] ************************************************
\textcolor{ansi32}{ok: [ubuntu-20.04]}

TASK [leantime_ansible : Descomprime o software leantime] ********************************************************************************
\textcolor{ansi32}{ok: [ubuntu-20.04]}

TASK [leantime_ansible : Copia o ficheiro de configuração configuration.sample.php] ******************************************************
\textcolor{ansi32}{ok: [ubuntu-20.04]}

TASK [leantime_ansible : Altera as variáveis dbuser, dbpassword e dbdatabase no ficheiro configuration.php utilizando o módulo replace.] ***
\textcolor{ansi32}{ok: [ubuntu-20.04] => (item={'regexp': 'dbUser=""', 'line': 'dbUser="leantimeDBadmin"'})}
\textcolor{ansi32}{ok: [ubuntu-20.04] => (item={'regexp': 'dbDatabase=""', 'line': 'dbDatabase="leantime_database"'})}
\textcolor{ansi32}{ok: [ubuntu-20.04] => (item={'regexp': 'dbPassword=""', 'line': 'dbPassword="#S3gr3d0S3cr3t0#"'})}

TASK [leantime_ansible : Copia o template do site para a pásta /etc/apache2/sites-available] *********************************************
\textcolor{ansi32}{ok: [ubuntu-20.04]}

TASK [leantime_ansible : Cria um link simbólico para o ficheiro leantime.conf na pasta /etc/apache/sites-enabled] ************************
\textcolor{ansi31}{fatal: [ubuntu-20.04]: FAILED! => {"changed": false, "gid": 0, "group": "root", "mode": "0755", "msg": "refusing to convert from directory to symlink for /etc/apache2/sites-enabled", "owner": "root", "path": "/etc/apache2/sites-enabled", "size": 4096, "state": "directory", "uid": 0}}

PLAY RECAP *******************************************************************************************************************************
\textcolor{ansi31}{ubuntu-20.04}               : \textcolor{ansi32}{ok=15  } \textcolor{ansi33}{changed=1   } unreachable=0    \textcolor{ansi31}{failed=1   } skipped=0    rescued=0    ignored=0


\textbf{\textcolor{ansi31}{CRITICAL}} Ansible return code was \textbf{\textcolor{ansi36}{2}}, command was: ansible-playbook --inventory \textcolor{ansi35}{/home/admlocal/.cache/molecule/leantime_ansible/default/}\textcolor{ansi95}{inventory} --skip-tags molecule-notest,notest \textcolor{ansi35}{/home/admlocal/leantime_ansible/molecule/default/}\textcolor{ansi95}{converge.yml}
]0;admlocal@devOps: ~/leantime_ansibleadmlocal@devOps:~/leantime_ansible$ molecule convergelintnano tasks/tasks.yml 
[?2004h[?1049h[?7h[?1h=[?1h=[?25l\textcolor{inv_background inv_foreground}{[ A ler... ]}\textcolor{inv_background inv_foreground}{[ 167 linhas lidas ]}\textcolor{inv_background inv_foreground}{  GNU nano 4.8                                                tasks/tasks.yml                                                             }
\textcolor{inv_background inv_foreground}{^G} Ajuda\textcolor{inv_background inv_foreground}{^O} Gravar\textcolor{inv_background inv_foreground}{^W} Procurar\textcolor{inv_background inv_foreground}{^K} Cortar txt    \textcolor{inv_background inv_foreground}{^J} Justificar    \textcolor{inv_background inv_foreground}{^C} Pos cursor    \textcolor{inv_background inv_foreground}{M-U} Desfazer     \textcolor{inv_background inv_foreground}{M-A} Marcar txt
\textcolor{inv_background inv_foreground}{^X} Sair\textcolor{inv_background inv_foreground}{^R} Carregar\textcolor{inv_background inv_foreground}{^\} Substituir    \textcolor{inv_background inv_foreground}{^U} Colar txt     \textcolor{inv_background inv_foreground}{^T} Ortografia    \textcolor{inv_background inv_foreground}{^_} Ir p/ linha   \textcolor{inv_background inv_foreground}{M-E} Refazer\textcolor{inv_background inv_foreground}{M-6} Copiar txt
---- name: Atualiza a cache (equivalente a apt update)apt:update_cache: truecache_valid_time: 3600- name: "Atualiza o sistema operativo"apt:upgrade: safe- name: "Instalar as dependencias necessária ao programa leantime"apt:pkg:- mc- screen- git- apache2- mysql-server- php- php-mysql- php-ldap- php-cli- php-soap- php-json- graphviz- php-xml- php-gd- php-zip- libapache2-mod-php- php-dev- libmcrypt-dev- gcc- make- autoconf- libc-dev
[?12l[?25h[?25l - autoconf    - libc-dev  - pkg-config- pwgen- curl- unzip    - zip  - php-mbstring- expect- net-tools    - python3-mysqldb  - python3-apt- python3-pycurl
\textcolor{ansi36}{   # Fonte de inspiração: https://docs.ansible.com/ansible/latest/collections/ansible/builtin/service_module.html}- name: "Instala o serviço apache2 no arranque do sistema"service:name: apache2state: startedenabled: true
\textcolor{ansi36}{   # Fonte: https://www.mydailytutorials.com/using-ansible-find-module-search-filesfolder}- name: "Rotina Auxiliar: Procura o caminho do ficheiro php.ini do servidor apache2"find:paths: /etcpatterns: "php.ini"recurse: trueregister: caminho_php_ini
\textcolor{ansi36}{   # Fonte: https://docs.ansible.com/ansible/latest/collections/ansible/builtin/lineinfile_module.html}- name: "Ativa várias opções no ficheiro php.ini, utilizando o módulo lineinfile."ansible.builtin.lineinfile:path: "{{ caminho_php_ini.files[0].path }}"regexp: "{{ item.regexp  }}"line: "{{ item.line }}"
[?12l[?25h[?25lregexp: "{{ item.regexp  }}"line: "{{ item.line }}"with_items:regexp: "^file_uploads"  line: "file_uploads = 1"regexp: "^upload_max_filesize"  line: "upload_max_filesize = 1G"regexp: "^max_file_uploads"  line: "max_file_uploads = 20"regexp: "^post_max_size"  line: "post_max_size = 2G"regexp: "^memory_limit"  line: "memory_limit = 2G"- regexp: "^max_input_time" line: "max_input_time = 3600"
\textcolor{ansi36}{   # Fonte: https://docs.ansible.com/ansible/2.7/modules/file_module.html}- name: Cria a directoria temporária leantime e a directoria de apache leantimefile:path: "{{ item.path }}"state: directory mode: 0755    owner: www-data  group: www-datawith_items:- path: "/tmp/leantime"- path: "/var/www/leantime"
\textcolor{ansi36}{   # Fonte: Ansible for DevOps - Server and configuration management for humans de Jeff Geerling, página 98
}   - name: "Cria a base de dados de mysql para o leantime"  mysql_db: "db=leantime_database state=present"
- name: "Cria o utilizador para a base de dados leantime"mysql_user:namlogin_leantime }}"
[?12l[?25h[?25lmysql_user:namlogin_leantime }}"  password: "{{ password_leantime }}"priv: "leantime_database.*:ALL"host: localhoststate: present
- name: "Faz o download do software leantime a partir do repositório git"get_url:url: https://github.com/Leantime/leantime/releases/download/{{ versao_leantime }}/Leantime-{{ versao_leantime }}.zipdest: /tmp/leantimemode: 0755
\textcolor{ansi36}{   # O ficheiro Leantime-versao.zip necessita de ser descomprimido}- name: Descomprime o software leantimeunarchive: src: "/tmp/leantime/Leantime-{{ versao_leantime }}.zip"    dest: "/var/www/leantime"  owner: www-datagroup: www-datamode: 0755remote_src: true
\textcolor{ansi36}{   # Após a descompressão, se o ficheiro configuration.php não existir, este necessita de ser
   # criado a partir do ficheiro configuration.sample.php
   # Fonte: https://docs.ansible.com/ansible/latest/collections/ansible/builtin/copy_module.html}- name: Copia o ficheiro de configuração configuration.sample.phpansible.builtin.copy: src: /var/www/leantime/config/configuration.sample.php    dest: /var/www/leantime/config/configuration.php  owner: www-datagroup: www-data    mode: '0755'  remote_src: trueforce: false
[?12l[?25h[?25l  remote_src: trueforce: false
\textcolor{ansi36}{   # Fonte: https://docs.ansible.com/ansible/latest/collections/ansible/builtin/replace_module.html}- name: Altera as variáveis dbuser, dbpassword e dbdatabase no ficheiro configuration.php utilizando o módulo replace.ansible.builtin.replace:path: /var/www/leantime/config/configuration.php    regexp: "{{ item.regexp  }}"  replace: "{{ item.line }}"with_items:- regexp: 'dbUser=""'  line: 'dbUser="{{ login_leantime }}"'- regexp: 'dbDatabase=""' line: 'dbDatabase="leantime_database"'    - regexp: 'dbPassword=""'    line: 'dbPassword="{{ password_leantime }}"'
\textcolor{ansi36}{   # Fonte: https://docs.ansible.com/ansible/latest/collections/ansible/builtin/template_module.html}- name: Copia o template do site para a pásta /etc/apache2/sites-availableansible.builtin.template:src: leantime.template.j2dest: /etc/apache2/sites-available/leantime.confowner: root group: www-data mode: 0755
\textcolor{ansi36}{   # Fonte: https://docs.ansible.com/ansible/latest/collections/ansible/builtin/file_module.html}- name: "Cria um link simbólico para o ficheiro leantime.conf na pasta /etc/apache/sites-enabled"ansible.builtin.file:src: /etc/apache2/sites-available/leantime.confdest: /etc/apache2/sites-enabledowner: rootgroup: rootmode: '0777'state: link
[?12l[?25h[?25lmode: '0777'state: link
[?12l[?25h[?25l78Mgroup: root
[?12l[?25h[?25l78Mowner: root
[?12l[?25h[?25l78Mdest: /etc/apache2/sites-enabled
[?12l[?25h[?25l78Msrc: /etc/apache2/sites-available/leantime.conf
[?12l[?25h[?25l78Mansible.builtin.file:
[?12l[?25h[?25l78M- name: "Cria um link simbólico para o ficheiro leantime.conf na pasta /etc/apache/sites-enabled"
[?12l[?25h[?25l78M\textcolor{ansi36}{   # Fonte: https://docs.ansible.com/ansible/latest/collections/ansible/builtin/file_module.html
}[?12l[?25h[?25l78M[?12l[?25h[?25l[?12l[?25h[?25l[?12l[?25h[?25l[?12l[?25h[?25l[?12l[?25h[?25l
[?12l[?25h[?25l[?12l[?25h[?25l[?12l[?25h[?25l[?12l[?25h[?25l[?12l[?25h[?25l[?12l[?25h[?25l[?12l[?25h[?25l[?12l[?25h[?25l[?12l[?25h[?25l[?12l[?25h[?25l[?12l[?25h[?25l[?12l[?25h[?25l[?12l[?25h[?25l[?12l[?25h[?25l[?12l[?25h[?25l[?12l[?25h[?25l\textcolor{inv_background inv_foreground}{Modificado}M
leantime.conf
[?12l[?25h[?25l
[?12l[?25h[?25l[?12l[?25h[?25l78
Mleantime.conf
[?12l[?25h[?25l
[?12l[?25h[?25l[?12l[?25h[?25l/[?12l[?25h[?25l78
Mleantime.conf
[?12l[?25h[?25l
[?12l[?25h[?25l         \textcolor{inv_background inv_foreground}{M-D} Formato DOS\textcolor{inv_background inv_foreground}{M-A} Anexar\textcolor{inv_background inv_foreground}{B} Segurança\textcolor{inv_background inv_foreground}{C} Cancelar           \textcolor{inv_background inv_foreground}{M-M} Formato Mac\textcolor{inv_background inv_foreground}{M-P} Prepor\textcolor{inv_background inv_foreground}{^T} P/ ficheiros
\textcolor{inv_background inv_foreground}{Nome do ficheiro onde escrever: tasks/tasks.yml                                                                                           }[?12l[?25h[?25l \textcolor{inv_background inv_foreground}{[ A escrever... ]}\textcolor{inv_background inv_foreground}{          }\textcolor{inv_background inv_foreground}{[ 167 linhas escritas ]}\textcolor{inv_background inv_foreground}{^O} Gravar\textcolor{inv_background inv_foreground}{^W} Procurar      \textcolor{inv_background inv_foreground}{^K} Cortar txt    \textcolor{inv_background inv_foreground}{^J} Justificar    \textcolor{inv_background inv_foreground}{^C} Pos cursor\textcolor{inv_background inv_foreground}{U} Desfazer     \textcolor{inv_background inv_foreground}{M-A} Marcar txt\textcolor{inv_background inv_foreground}{X} Sair    \textcolor{inv_background inv_foreground}{^R} Carregar\textcolor{inv_background inv_foreground}{^\} Substituir    \textcolor{inv_background inv_foreground}{^U} Colar txt     \textcolor{inv_background inv_foreground}{^T} Ortografia    \textcolor{inv_background inv_foreground}{^_} Ir p/ linha   \textcolor{inv_background inv_foreground}{M-E} Refazer      \textcolor{inv_background inv_foreground}{M-6} Copiar txt
[?12l[?25h[?25l
[?12l[?25h[?1049l
[?1l>[?2004l]0;admlocal@devOps: ~/leantime_ansibleadmlocal@devOps:~/leantime_ansible$ nano tasks/tasks.yml molecule converge
\textcolor{ansi34}{INFO    } default scenario test matrix: dependency, create, prepare, converge
\textcolor{ansi34}{INFO    } Performing prerun\textcolor{ansi33}{...}
\textcolor{ansi34}{INFO    } Using .cache/roles/nunomourinho.leantime_ansible symlink to current repository in order to enable Ansible to find the role using its expected full name.
\textcolor{ansi34}{INFO    } Added \textcolor{ansi33}{ANSIBLE_ROLES_PATH}=~\textcolor{ansi35}{/.ansible/}\textcolor{ansi95}{roles}:\textcolor{ansi35}{/usr/share/ansible/}\textcolor{ansi95}{roles}:\textcolor{ansi35}{/etc/ansible/}\textcolor{ansi95}{roles}:.\textcolor{ansi35}{/.cache/}\textcolor{ansi95}{roles}
\textcolor{ansi34}{INFO    } \textcolor{ansi2 ansi36}{Running }\textcolor{ansi2 ansi32}{default}\textcolor{ansi2 ansi36}{ > }\textcolor{ansi2 ansi32}{dependency}
\textcolor{ansi31}{WARNING } Skipping, missing the requirements file.
\textcolor{ansi31}{WARNING } Skipping, missing the requirements file.
\textcolor{ansi34}{INFO    } \textcolor{ansi2 ansi36}{Running }\textcolor{ansi2 ansi32}{default}\textcolor{ansi2 ansi36}{ > }\textcolor{ansi2 ansi32}{create}
\textcolor{ansi31}{WARNING } Skipping, instances already created.
\textcolor{ansi34}{INFO    } \textcolor{ansi2 ansi36}{Running }\textcolor{ansi2 ansi32}{default}\textcolor{ansi2 ansi36}{ > }\textcolor{ansi2 ansi32}{prepare}
\textcolor{ansi31}{WARNING } Skipping, instances already prepared.
\textcolor{ansi34}{INFO    } \textcolor{ansi2 ansi36}{Running }\textcolor{ansi2 ansi32}{default}\textcolor{ansi2 ansi36}{ > }\textcolor{ansi2 ansi32}{converge}

PLAY [Converge] **************************************************************************************************************************

TASK [Gathering Facts] *******************************************************************************************************************
\textcolor{ansi32}{ok: [ubuntu-20.04]}

TASK [Include leantime_ansible] **********************************************************************************************************

TASK [leantime_ansible : Atualiza a cache (equivalente a apt update)] ********************************************************************
\textcolor{ansi32}{ok: [ubuntu-20.04]}

TASK [leantime_ansible : Atualiza o sistema operativo] ***********************************************************************************
\textcolor{ansi32}{ok: [ubuntu-20.04]}

TASK [leantime_ansible : Instalar as dependencias necessária ao programa leantime] *******************************************************
\textcolor{ansi32}{ok: [ubuntu-20.04]}

TASK [leantime_ansible : Instala o serviço apache2 no arranque do sistema] ***************************************************************
\textcolor{ansi32}{ok: [ubuntu-20.04]}

TASK [leantime_ansible : Rotina Auxiliar: Procura o caminho do ficheiro php.ini do servidor apache2] *************************************
\textcolor{ansi32}{ok: [ubuntu-20.04]}

TASK [leantime_ansible : Ativa várias opções no ficheiro php.ini, utilizando o módulo lineinfile.] ***************************************
\textcolor{ansi32}{ok: [ubuntu-20.04] => (item={'regexp': '^file_uploads', 'line': 'file_uploads = 1'})}
\textcolor{ansi32}{ok: [ubuntu-20.04] => (item={'regexp': '^upload_max_filesize', 'line': 'upload_max_filesize = 1G'})}
\textcolor{ansi32}{ok: [ubuntu-20.04] => (item={'regexp': '^max_file_uploads', 'line': 'max_file_uploads = 20'})}
\textcolor{ansi32}{ok: [ubuntu-20.04] => (item={'regexp': '^post_max_size', 'line': 'post_max_size = 2G'})}
\textcolor{ansi32}{ok: [ubuntu-20.04] => (item={'regexp': '^memory_limit', 'line': 'memory_limit = 2G'})}
\textcolor{ansi32}{ok: [ubuntu-20.04] => (item={'regexp': '^max_input_time', 'line': 'max_input_time = 3600'})}

TASK [leantime_ansible : Cria a directoria temporária leantime e a directoria de apache leantime] ****************************************
\textcolor{ansi32}{ok: [ubuntu-20.04] => (item={'path': '/tmp/leantime'})}
\textcolor{ansi32}{ok: [ubuntu-20.04] => (item={'path': '/var/www/leantime'})}

TASK [leantime_ansible : Cria a base de dados de mysql para o leantime] ******************************************************************
\textcolor{ansi32}{ok: [ubuntu-20.04]}

TASK [leantime_ansible : Cria o utilizador para a base de dados leantime] ****************************************************************
\textbf{\textcolor{ansi35}{[WARNING]: Module did not set no_log for update_password}}
\textcolor{ansi32}{ok: [ubuntu-20.04]}

TASK [leantime_ansible : Faz o download do software leantime a partir do repositório git] ************************************************
\textcolor{ansi32}{ok: [ubuntu-20.04]}

TASK [leantime_ansible : Descomprime o software leantime] ********************************************************************************
\textcolor{ansi32}{ok: [ubuntu-20.04]}

TASK [leantime_ansible : Copia o ficheiro de configuração configuration.sample.php] ******************************************************
\textcolor{ansi32}{ok: [ubuntu-20.04]}

TASK [leantime_ansible : Altera as variáveis dbuser, dbpassword e dbdatabase no ficheiro configuration.php utilizando o módulo replace.] ***
\textcolor{ansi32}{ok: [ubuntu-20.04] => (item={'regexp': 'dbUser=""', 'line': 'dbUser="leantimeDBadmin"'})}
\textcolor{ansi32}{ok: [ubuntu-20.04] => (item={'regexp': 'dbDatabase=""', 'line': 'dbDatabase="leantime_database"'})}
\textcolor{ansi32}{ok: [ubuntu-20.04] => (item={'regexp': 'dbPassword=""', 'line': 'dbPassword="#S3gr3d0S3cr3t0#"'})}

TASK [leantime_ansible : Copia o template do site para a pásta /etc/apache2/sites-available] *********************************************
\textcolor{ansi32}{ok: [ubuntu-20.04]}

TASK [leantime_ansible : Cria um link simbólico para o ficheiro leantime.conf na pasta /etc/apache/sites-enabled] ************************
\textcolor{ansi33}{changed: [ubuntu-20.04]}

PLAY RECAP *******************************************************************************************************************************
\textcolor{ansi33}{ubuntu-20.04}               : \textcolor{ansi32}{ok=16  } \textcolor{ansi33}{changed=1   } unreachable=0    failed=0    skipped=0    rescued=0    ignored=0


]0;admlocal@devOps: ~/leantime_ansibleadmlocal@devOps:~/leantime_ansible$ molecule convergeidempotence
\textcolor{ansi34}{INFO    } default scenario test matrix: idempotence
\textcolor{ansi34}{INFO    } Performing prerun\textcolor{ansi33}{...}
\textcolor{ansi34}{INFO    } Using .cache/roles/nunomourinho.leantime_ansible symlink to current repository in order to enable Ansible to find the role using its expected full name.
\textcolor{ansi34}{INFO    } Added \textcolor{ansi33}{ANSIBLE_ROLES_PATH}=~\textcolor{ansi35}{/.ansible/}\textcolor{ansi95}{roles}:\textcolor{ansi35}{/usr/share/ansible/}\textcolor{ansi95}{roles}:\textcolor{ansi35}{/etc/ansible/}\textcolor{ansi95}{roles}:.\textcolor{ansi35}{/.cache/}\textcolor{ansi95}{roles}
\textcolor{ansi34}{INFO    } \textcolor{ansi2 ansi36}{Running }\textcolor{ansi2 ansi32}{default}\textcolor{ansi2 ansi36}{ > }\textcolor{ansi2 ansi32}{idempotence}

PLAY [Converge] **************************************************************************************************************************

TASK [Gathering Facts] *******************************************************************************************************************
\textcolor{ansi32}{ok: [ubuntu-20.04]}

TASK [Include leantime_ansible] **********************************************************************************************************

TASK [leantime_ansible : Atualiza a cache (equivalente a apt update)] ********************************************************************
\textcolor{ansi32}{ok: [ubuntu-20.04]}

TASK [leantime_ansible : Atualiza o sistema operativo] ***********************************************************************************
\textcolor{ansi32}{ok: [ubuntu-20.04]}

TASK [leantime_ansible : Instalar as dependencias necessária ao programa leantime] *******************************************************
\textcolor{ansi32}{ok: [ubuntu-20.04]}

TASK [leantime_ansible : Instala o serviço apache2 no arranque do sistema] ***************************************************************
\textcolor{ansi32}{ok: [ubuntu-20.04]}

TASK [leantime_ansible : Rotina Auxiliar: Procura o caminho do ficheiro php.ini do servidor apache2] *************************************
\textcolor{ansi32}{ok: [ubuntu-20.04]}

TASK [leantime_ansible : Ativa várias opções no ficheiro php.ini, utilizando o módulo lineinfile.] ***************************************
\textcolor{ansi32}{ok: [ubuntu-20.04] => (item={'regexp': '^file_uploads', 'line': 'file_uploads = 1'})}
\textcolor{ansi32}{ok: [ubuntu-20.04] => (item={'regexp': '^upload_max_filesize', 'line': 'upload_max_filesize = 1G'})}
\textcolor{ansi32}{ok: [ubuntu-20.04] => (item={'regexp': '^max_file_uploads', 'line': 'max_file_uploads = 20'})}
\textcolor{ansi32}{ok: [ubuntu-20.04] => (item={'regexp': '^post_max_size', 'line': 'post_max_size = 2G'})}
\textcolor{ansi32}{ok: [ubuntu-20.04] => (item={'regexp': '^memory_limit', 'line': 'memory_limit = 2G'})}
\textcolor{ansi32}{ok: [ubuntu-20.04] => (item={'regexp': '^max_input_time', 'line': 'max_input_time = 3600'})}

TASK [leantime_ansible : Cria a directoria temporária leantime e a directoria de apache leantime] ****************************************
\textcolor{ansi32}{ok: [ubuntu-20.04] => (item={'path': '/tmp/leantime'})}
\textcolor{ansi32}{ok: [ubuntu-20.04] => (item={'path': '/var/www/leantime'})}

TASK [leantime_ansible : Cria a base de dados de mysql para o leantime] ******************************************************************
\textcolor{ansi32}{ok: [ubuntu-20.04]}

TASK [leantime_ansible : Cria o utilizador para a base de dados leantime] ****************************************************************
\textbf{\textcolor{ansi35}{[WARNING]: Module did not set no_log for update_password}}
\textcolor{ansi32}{ok: [ubuntu-20.04]}

TASK [leantime_ansible : Faz o download do software leantime a partir do repositório git] ************************************************
\textcolor{ansi32}{ok: [ubuntu-20.04]}

TASK [leantime_ansible : Descomprime o software leantime] ********************************************************************************
\textcolor{ansi32}{ok: [ubuntu-20.04]}

TASK [leantime_ansible : Copia o ficheiro de configuração configuration.sample.php] ******************************************************
\textcolor{ansi32}{ok: [ubuntu-20.04]}

TASK [leantime_ansible : Altera as variáveis dbuser, dbpassword e dbdatabase no ficheiro configuration.php utilizando o módulo replace.] ***
\textcolor{ansi32}{ok: [ubuntu-20.04] => (item={'regexp': 'dbUser=""', 'line': 'dbUser="leantimeDBadmin"'})}
\textcolor{ansi32}{ok: [ubuntu-20.04] => (item={'regexp': 'dbDatabase=""', 'line': 'dbDatabase="leantime_database"'})}
\textcolor{ansi32}{ok: [ubuntu-20.04] => (item={'regexp': 'dbPassword=""', 'line': 'dbPassword="#S3gr3d0S3cr3t0#"'})}

TASK [leantime_ansible : Copia o template do site para a pásta /etc/apache2/sites-available] *********************************************
\textcolor{ansi33}{changed: [ubuntu-20.04]}

TASK [leantime_ansible : Cria um link simbólico para o ficheiro leantime.conf na pasta /etc/apache/sites-enabled] ************************
\textcolor{ansi33}{changed: [ubuntu-20.04]}

PLAY RECAP *******************************************************************************************************************************
\textcolor{ansi33}{ubuntu-20.04}               : \textcolor{ansi32}{ok=16  } \textcolor{ansi33}{changed=2   } unreachable=0    failed=0    skipped=0    rescued=0    ignored=0


\textbf{\textcolor{ansi31}{CRITICAL}} Idempotence test failed because of the following tasks:
*  => leantime_ansible : Copia o template do site para a pásta \textcolor{ansi35}{/etc/apache2/}\textcolor{ansi95}{sites-available}
*  => leantime_ansible : Cria um link simbólico para o ficheiro leantime.conf na pasta \textcolor{ansi35}{/etc/apache/}\textcolor{ansi95}{sites-enabled}
]0;admlocal@devOps: ~/leantime_ansibleadmlocal@devOps:~/leantime_ansible$ nano tasks/
[?2004h[?1049h[?7h[?1h=[?1h=[?25l\textcolor{ansi1}{}\textbf{\textcolor{ansi37}{}}\textcolor{ansi1 ansi37 ansi41}{[ "tasks/" é uma pasta ]}\textcolor{ansi1}{}\textcolor{inv_background inv_foreground}{  GNU nano 4.8                                                  Novo buffer                                                               }
\textcolor{inv_background inv_foreground}{^G} Ajuda\textcolor{inv_background inv_foreground}{^O} Gravar\textcolor{inv_background inv_foreground}{^W} Procurar\textcolor{inv_background inv_foreground}{^K} Cortar txt    \textcolor{inv_background inv_foreground}{^J} Justificar    \textcolor{inv_background inv_foreground}{^C} Pos cursor    \textcolor{inv_background inv_foreground}{M-U} Desfazer     \textcolor{inv_background inv_foreground}{M-A} Marcar txt
\textcolor{inv_background inv_foreground}{^X} Sair\textcolor{inv_background inv_foreground}{^R} Carregar\textcolor{inv_background inv_foreground}{^\} Substituir    \textcolor{inv_background inv_foreground}{^U} Colar txt     \textcolor{inv_background inv_foreground}{^T} Ortografia    \textcolor{inv_background inv_foreground}{^_} Ir p/ linha   \textcolor{inv_background inv_foreground}{M-E} Refazer\textcolor{inv_background inv_foreground}{M-6} Copiar txt
[?12l[?25h[?25l[?12l[?25h[?25l[?12l[?25h[?1049l
[?1l>[?2004l]0;admlocal@devOps: ~/leantime_ansibleadmlocal@devOps:~/leantime_ansible$ nano tasks/tasks.yml 
[?2004h[?1049h[?7h[?1h=[?1h=[?25l\textcolor{inv_background inv_foreground}{[ A ler... ]}\textcolor{inv_background inv_foreground}{[ 167 linhas lidas ]}\textcolor{inv_background inv_foreground}{  GNU nano 4.8                                                tasks/tasks.yml                                                             }
\textcolor{inv_background inv_foreground}{^G} Ajuda\textcolor{inv_background inv_foreground}{^O} Gravar\textcolor{inv_background inv_foreground}{^W} Procurar\textcolor{inv_background inv_foreground}{^K} Cortar txt    \textcolor{inv_background inv_foreground}{^J} Justificar    \textcolor{inv_background inv_foreground}{^C} Pos cursor    \textcolor{inv_background inv_foreground}{M-U} Desfazer     \textcolor{inv_background inv_foreground}{M-A} Marcar txt
\textcolor{inv_background inv_foreground}{^X} Sair\textcolor{inv_background inv_foreground}{^R} Carregar\textcolor{inv_background inv_foreground}{^\} Substituir    \textcolor{inv_background inv_foreground}{^U} Colar txt     \textcolor{inv_background inv_foreground}{^T} Ortografia    \textcolor{inv_background inv_foreground}{^_} Ir p/ linha   \textcolor{inv_background inv_foreground}{M-E} Refazer\textcolor{inv_background inv_foreground}{M-6} Copiar txt
---- name: Atualiza a cache (equivalente a apt update)apt:update_cache: truecache_valid_time: 3600- name: "Atualiza o sistema operativo"apt:upgrade: safe- name: "Instalar as dependencias necessária ao programa leantime"apt:pkg:- mc- screen- git- apache2- mysql-server- php- php-mysql- php-ldap- php-cli- php-soap- php-json- graphviz- php-xml- php-gd- php-zip- libapache2-mod-php- php-dev- libmcrypt-dev- gcc- make- autoconf- libc-dev
[?12l[?25h[?25l - autoconf    - libc-dev  - pkg-config- pwgen- curl- unzip    - zip  - php-mbstring- expect- net-tools    - python3-mysqldb  - python3-apt- python3-pycurl
\textcolor{ansi36}{   # Fonte de inspiração: https://docs.ansible.com/ansible/latest/collections/ansible/builtin/service_module.html}- name: "Instala o serviço apache2 no arranque do sistema"service:name: apache2state: startedenabled: true
\textcolor{ansi36}{   # Fonte: https://www.mydailytutorials.com/using-ansible-find-module-search-filesfolder}- name: "Rotina Auxiliar: Procura o caminho do ficheiro php.ini do servidor apache2"find:paths: /etcpatterns: "php.ini"recurse: trueregister: caminho_php_ini
\textcolor{ansi36}{   # Fonte: https://docs.ansible.com/ansible/latest/collections/ansible/builtin/lineinfile_module.html}- name: "Ativa várias opções no ficheiro php.ini, utilizando o módulo lineinfile."ansible.builtin.lineinfile:path: "{{ caminho_php_ini.files[0].path }}"regexp: "{{ item.regexp  }}"line: "{{ item.line }}"
[?12l[?25h[?25lregexp: "{{ item.regexp  }}"line: "{{ item.line }}"with_items:regexp: "^file_uploads"  line: "file_uploads = 1"regexp: "^upload_max_filesize"  line: "upload_max_filesize = 1G"regexp: "^max_file_uploads"  line: "max_file_uploads = 20"regexp: "^post_max_size"  line: "post_max_size = 2G"regexp: "^memory_limit"  line: "memory_limit = 2G"- regexp: "^max_input_time" line: "max_input_time = 3600"
\textcolor{ansi36}{   # Fonte: https://docs.ansible.com/ansible/2.7/modules/file_module.html}- name: Cria a directoria temporária leantime e a directoria de apache leantimefile:path: "{{ item.path }}"state: directory mode: 0755    owner: www-data  group: www-datawith_items:- path: "/tmp/leantime"- path: "/var/www/leantime"
\textcolor{ansi36}{   # Fonte: Ansible for DevOps - Server and configuration management for humans de Jeff Geerling, página 98
}   - name: "Cria a base de dados de mysql para o leantime"  mysql_db: "db=leantime_database state=present"
- name: "Cria o utilizador para a base de dados leantime"mysql_user:namlogin_leantime }}"
[?12l[?25h[?25lmysql_user:namlogin_leantime }}"  password: "{{ password_leantime }}"priv: "leantime_database.*:ALL"host: localhoststate: present
- name: "Faz o download do software leantime a partir do repositório git"get_url:url: https://github.com/Leantime/leantime/releases/download/{{ versao_leantime }}/Leantime-{{ versao_leantime }}.zipdest: /tmp/leantimemode: 0755
\textcolor{ansi36}{   # O ficheiro Leantime-versao.zip necessita de ser descomprimido}- name: Descomprime o software leantimeunarchive: src: "/tmp/leantime/Leantime-{{ versao_leantime }}.zip"    dest: "/var/www/leantime"  owner: www-datagroup: www-datamode: 0755remote_src: true
\textcolor{ansi36}{   # Após a descompressão, se o ficheiro configuration.php não existir, este necessita de ser
   # criado a partir do ficheiro configuration.sample.php
   # Fonte: https://docs.ansible.com/ansible/latest/collections/ansible/builtin/copy_module.html}- name: Copia o ficheiro de configuração configuration.sample.phpansible.builtin.copy: src: /var/www/leantime/config/configuration.sample.php    dest: /var/www/leantime/config/configuration.php  owner: www-datagroup: www-data    mode: '0755'  remote_src: trueforce: false
[?12l[?25h[?25l  remote_src: trueforce: false
\textcolor{ansi36}{   # Fonte: https://docs.ansible.com/ansible/latest/collections/ansible/builtin/replace_module.html}- name: Altera as variáveis dbuser, dbpassword e dbdatabase no ficheiro configuration.php utilizando o módulo replace.ansible.builtin.replace:path: /var/www/leantime/config/configuration.php    regexp: "{{ item.regexp  }}"  replace: "{{ item.line }}"with_items:- regexp: 'dbUser=""'  line: 'dbUser="{{ login_leantime }}"'- regexp: 'dbDatabase=""' line: 'dbDatabase="leantime_database"'    - regexp: 'dbPassword=""'    line: 'dbPassword="{{ password_leantime }}"'
\textcolor{ansi36}{   # Fonte: https://docs.ansible.com/ansible/latest/collections/ansible/builtin/template_module.html}- name: Copia o template do site para a pásta /etc/apache2/sites-availableansible.builtin.template:src: leantime.template.j2dest: /etc/apache2/sites-available/leantime.confowner: root group: www-data mode: 0755
\textcolor{ansi36}{   # Fonte: https://docs.ansible.com/ansible/latest/collections/ansible/builtin/file_module.html}- name: "Cria um link simbólico para o ficheiro leantime.conf na pasta /etc/apache/sites-enabled"ansible.builtin.file:src: /etc/apache2/sites-available/leantime.confdest: /etc/apache2/sites-enabled/leantime.confowner: rootgroup: rootmode: '0777'state: link
[?12l[?25h[?25lmode: '0777'state: link
[?12l[?25h[?25l78Mgroup: root
[?12l[?25h[?25l78Mowner: root
[?12l[?25h[?25l78Mdest: /etc/apache2/sites-enabled/leantime.conf
[?12l[?25h[?25l78Msrc: /etc/apache2/sites-available/leantime.conf
[?12l[?25h[?25l78Mansible.builtin.file:
[?12l[?25h[?25l78M- name: "Cria um link simbólico para o ficheiro leantime.conf na pasta /etc/apache/sites-enabled"
[?12l[?25h[?25l78M\textcolor{ansi36}{   # Fonte: https://docs.ansible.com/ansible/latest/collections/ansible/builtin/file_module.html
}[?12l[?25h[?25l78M[?12l[?25h[?25l78Mmode: 0755
[?12l[?25h[?25l78Mgroup: www-data
[?12l[?25h[?25l78Mowner: root
[?12l[?25h[?25l78Mdest: /etc/apache2/sites-available/leantime.conf
[?12l[?25h[?25l78Msrc: leantime.template.j2
[?12l[?25h[?25l78Mansible.builtin.template:
[?12l[?25h[?25l78M- name: Copia o template do site para a pásta /etc/apache2/sites-available
[?12l[?25h[?25l78M\textcolor{ansi36}{   # Fonte: https://docs.ansible.com/ansible/latest/collections/ansible/builtin/template_module.html
}[?12l[?25h[?25l78M[?12l[?25h[?25l78Mline: 'dbPassword="{{ password_leantime }}"'
[?12l[?25h[?25l78M- regexp: 'dbPassword=""'
[?12l[?25h[?25l78Mline: 'dbDatabase="leantime_database"'
[?12l[?25h[?25l78M- regexp: 'dbDatabase=""'
[?12l[?25h[?25l78Mline: 'dbUser="{{ login_leantime }}"'
[?12l[?25h[?25l78M- regexp: 'dbUser=""'
[?12l[?25h[?25l78Mwith_items:
[?12l[?25h[?25l78Mreplace: "{{ item.line }}"
[?12l[?25h[?25l78Mregexp: "{{ item.regexp  }}"
[?12l[?25h[?25l78Mpath: /var/www/leantime/config/configuration.php
[?12l[?25h[?25l78Mansible.builtin.replace:
[?12l[?25h[?25l78M- name: Altera as variáveis dbuser, dbpassword e dbdatabase no ficheiro configuration.php utilizando o módulo replace.
[?12l[?25h[?25l78M\textcolor{ansi36}{   # Fonte: https://docs.ansible.com/ansible/latest/collections/ansible/builtin/replace_module.html
}[?12l[?25h[?25l78M[?12l[?25h[?25l78Mforce: false
[?12l[?25h[?25l78Mremote_src: true
[?12l[?25h[?25l78Mmode: '0755'
[?12l[?25h[?25l78Mgroup: www-data
[?12l[?25h[?25l78Mowner: www-data
[?12l[?25h[?25l78Mdest: /var/www/leantime/config/configuration.php
[?12l[?25h[?25l78Msrc: /var/www/leantime/config/configuration.sample.php
[?12l[?25h[?25l78Mansible.builtin.copy:
[?12l[?25h[?25l78M- name: Copia o ficheiro de configuração configuration.sample.php
[?12l[?25h[?25l78M\textcolor{ansi36}{   # Fonte: https://docs.ansible.com/ansible/latest/collections/ansible/builtin/copy_module.html
}[?12l[?25h[?25l[?12l[?25h[?25l[?12l[?25h[?25l[?12l[?25h[?25l[?12l[?25h[?25l[?12l[?25h[?25l[?12l[?25h[?25l[?12l[?25h[?25l[?12l[?25h[?25l[?12l[?25h[?25l\textcolor{inv_background inv_foreground}{Modificado}
- name: "Cria um link simbólico para o ficheiro leantime.conf na pasta /etc/apache/sites-enabled"
[?12l[?25h[?25l78Mforce: false
[?12l[?25h[?25l[?12l[?25h[?25l[?12l[?25h[?25l[?12l[?25h[?25l[?12l[?25h[?25l[?12l[?25h[?25l[?12l[?25h[?25l[?12l[?25h[?25l[?12l[?25h[?25l[?12l[?25h[?25l[?12l[?25h[?25l[?12l[?25h[?25l[?12l[?25h[?25l[?12l[?25h[?25l[?12l[?25h[?25l[?12l[?25h[?25l[?12l[?25h[?25l[?12l[?25h[?25l[?12l[?25h[?25l[?12l[?25h[?25l[?12l[?25h[?25l[?12l[?25h[?25l[?12l[?25h[?25l[?12l[?25h[?25l[?12l[?25h[?25l78
- name: "Cria um link simbólico para o ficheiro leantime.conf na pasta /etc/apache/sites-enabled"
[?12l[?25h[?25l78
ansible.builtin.file:
[?12l[?25h[?25l78
src: /etc/apache2/sites-available/leantime.conf
[?12l[?25h[?25l78
dest: /etc/apache2/sites-enabled/leantime.conf
[?12l[?25h[?25l78
owner: root
[?12l[?25h[?25l78
group: root
[?12l[?25h[?25l78
mode: '0777'
[?12l[?25h[?25l78
state: link
[?12l[?25h[?25l78
[?12l[?25h[?25l[?12l[?25h[?25l[?12l[?25h[?25l78Mforce: false
[?12l[?25h[?25l         \textcolor{inv_background inv_foreground}{M-D} Formato DOS\textcolor{inv_background inv_foreground}{M-A} Anexar\textcolor{inv_background inv_foreground}{B} Segurança\textcolor{inv_background inv_foreground}{C} Cancelar           \textcolor{inv_background inv_foreground}{M-M} Formato Mac\textcolor{inv_background inv_foreground}{M-P} Prepor\textcolor{inv_background inv_foreground}{^T} P/ ficheiros
\textcolor{inv_background inv_foreground}{Nome do ficheiro onde escrever: tasks/tasks.yml                                                                                           }[?12l[?25h[?25l \textcolor{inv_background inv_foreground}{[ A escrever... ]}\textcolor{inv_background inv_foreground}{          }\textcolor{inv_background inv_foreground}{[ 168 linhas escritas ]}\textcolor{inv_background inv_foreground}{^O} Gravar\textcolor{inv_background inv_foreground}{^W} Procurar      \textcolor{inv_background inv_foreground}{^K} Cortar txt    \textcolor{inv_background inv_foreground}{^J} Justificar    \textcolor{inv_background inv_foreground}{^C} Pos cursor\textcolor{inv_background inv_foreground}{U} Desfazer     \textcolor{inv_background inv_foreground}{M-A} Marcar txt\textcolor{inv_background inv_foreground}{X} Sair    \textcolor{inv_background inv_foreground}{^R} Carregar\textcolor{inv_background inv_foreground}{^\} Substituir    \textcolor{inv_background inv_foreground}{^U} Colar txt     \textcolor{inv_background inv_foreground}{^T} Ortografia    \textcolor{inv_background inv_foreground}{^_} Ir p/ linha   \textcolor{inv_background inv_foreground}{M-E} Refazer      \textcolor{inv_background inv_foreground}{M-6} Copiar txt
[?12l[?25h[?25l[?12l[?25h[?1049l
[?1l>[?2004l]0;admlocal@devOps: ~/leantime_ansibleadmlocal@devOps:~/leantime_ansible$ nano tasks/tasks.yml molecule idempotenceconverge
\textcolor{ansi34}{INFO    } default scenario test matrix: dependency, create, prepare, converge
\textcolor{ansi34}{INFO    } Performing prerun\textcolor{ansi33}{...}
\textcolor{ansi34}{INFO    } Using .cache/roles/nunomourinho.leantime_ansible symlink to current repository in order to enable Ansible to find the role using its expected full name.
\textcolor{ansi34}{INFO    } Added \textcolor{ansi33}{ANSIBLE_ROLES_PATH}=~\textcolor{ansi35}{/.ansible/}\textcolor{ansi95}{roles}:\textcolor{ansi35}{/usr/share/ansible/}\textcolor{ansi95}{roles}:\textcolor{ansi35}{/etc/ansible/}\textcolor{ansi95}{roles}:.\textcolor{ansi35}{/.cache/}\textcolor{ansi95}{roles}
\textcolor{ansi34}{INFO    } \textcolor{ansi2 ansi36}{Running }\textcolor{ansi2 ansi32}{default}\textcolor{ansi2 ansi36}{ > }\textcolor{ansi2 ansi32}{dependency}
\textcolor{ansi31}{WARNING } Skipping, missing the requirements file.
\textcolor{ansi31}{WARNING } Skipping, missing the requirements file.
\textcolor{ansi34}{INFO    } \textcolor{ansi2 ansi36}{Running }\textcolor{ansi2 ansi32}{default}\textcolor{ansi2 ansi36}{ > }\textcolor{ansi2 ansi32}{create}
\textcolor{ansi31}{WARNING } Skipping, instances already created.
\textcolor{ansi34}{INFO    } \textcolor{ansi2 ansi36}{Running }\textcolor{ansi2 ansi32}{default}\textcolor{ansi2 ansi36}{ > }\textcolor{ansi2 ansi32}{prepare}
\textcolor{ansi31}{WARNING } Skipping, instances already prepared.
\textcolor{ansi34}{INFO    } \textcolor{ansi2 ansi36}{Running }\textcolor{ansi2 ansi32}{default}\textcolor{ansi2 ansi36}{ > }\textcolor{ansi2 ansi32}{converge}

PLAY [Converge] **************************************************************************************************************************

TASK [Gathering Facts] *******************************************************************************************************************
\textcolor{ansi32}{ok: [ubuntu-20.04]}

TASK [Include leantime_ansible] **********************************************************************************************************

TASK [leantime_ansible : Atualiza a cache (equivalente a apt update)] ********************************************************************
\textcolor{ansi32}{ok: [ubuntu-20.04]}

TASK [leantime_ansible : Atualiza o sistema operativo] ***********************************************************************************
\textcolor{ansi32}{ok: [ubuntu-20.04]}

TASK [leantime_ansible : Instalar as dependencias necessária ao programa leantime] *******************************************************
\textcolor{ansi32}{ok: [ubuntu-20.04]}

TASK [leantime_ansible : Instala o serviço apache2 no arranque do sistema] ***************************************************************
\textcolor{ansi32}{ok: [ubuntu-20.04]}

TASK [leantime_ansible : Rotina Auxiliar: Procura o caminho do ficheiro php.ini do servidor apache2] *************************************
\textcolor{ansi32}{ok: [ubuntu-20.04]}

TASK [leantime_ansible : Ativa várias opções no ficheiro php.ini, utilizando o módulo lineinfile.] ***************************************
\textcolor{ansi32}{ok: [ubuntu-20.04] => (item={'regexp': '^file_uploads', 'line': 'file_uploads = 1'})}
\textcolor{ansi32}{ok: [ubuntu-20.04] => (item={'regexp': '^upload_max_filesize', 'line': 'upload_max_filesize = 1G'})}
\textcolor{ansi32}{ok: [ubuntu-20.04] => (item={'regexp': '^max_file_uploads', 'line': 'max_file_uploads = 20'})}
\textcolor{ansi32}{ok: [ubuntu-20.04] => (item={'regexp': '^post_max_size', 'line': 'post_max_size = 2G'})}
\textcolor{ansi32}{ok: [ubuntu-20.04] => (item={'regexp': '^memory_limit', 'line': 'memory_limit = 2G'})}
\textcolor{ansi32}{ok: [ubuntu-20.04] => (item={'regexp': '^max_input_time', 'line': 'max_input_time = 3600'})}

TASK [leantime_ansible : Cria a directoria temporária leantime e a directoria de apache leantime] ****************************************
\textcolor{ansi32}{ok: [ubuntu-20.04] => (item={'path': '/tmp/leantime'})}
\textcolor{ansi32}{ok: [ubuntu-20.04] => (item={'path': '/var/www/leantime'})}

TASK [leantime_ansible : Cria a base de dados de mysql para o leantime] ******************************************************************
\textcolor{ansi32}{ok: [ubuntu-20.04]}

TASK [leantime_ansible : Cria o utilizador para a base de dados leantime] ****************************************************************
\textbf{\textcolor{ansi35}{[WARNING]: Module did not set no_log for update_password}}
\textcolor{ansi32}{ok: [ubuntu-20.04]}

TASK [leantime_ansible : Faz o download do software leantime a partir do repositório git] ************************************************
\textcolor{ansi32}{ok: [ubuntu-20.04]}

TASK [leantime_ansible : Descomprime o software leantime] ********************************************************************************
\textcolor{ansi32}{ok: [ubuntu-20.04]}

TASK [leantime_ansible : Copia o ficheiro de configuração configuration.sample.php] ******************************************************
\textcolor{ansi32}{ok: [ubuntu-20.04]}

TASK [leantime_ansible : Altera as variáveis dbuser, dbpassword e dbdatabase no ficheiro configuration.php utilizando o módulo replace.] ***
\textcolor{ansi32}{ok: [ubuntu-20.04] => (item={'regexp': 'dbUser=""', 'line': 'dbUser="leantimeDBadmin"'})}
\textcolor{ansi32}{ok: [ubuntu-20.04] => (item={'regexp': 'dbDatabase=""', 'line': 'dbDatabase="leantime_database"'})}
\textcolor{ansi32}{ok: [ubuntu-20.04] => (item={'regexp': 'dbPassword=""', 'line': 'dbPassword="#S3gr3d0S3cr3t0#"'})}

TASK [leantime_ansible : Copia o template do site para a pásta /etc/apache2/sites-available] *********************************************
\textcolor{ansi32}{ok: [ubuntu-20.04]}

TASK [leantime_ansible : Cria um link simbólico para o ficheiro leantime.conf na pasta /etc/apache/sites-enabled] ************************
\textcolor{ansi32}{ok: [ubuntu-20.04]}

PLAY RECAP *******************************************************************************************************************************
\textcolor{ansi32}{ubuntu-20.04}               : \textcolor{ansi32}{ok=16  } changed=0    unreachable=0    failed=0    skipped=0    rescued=0    ignored=0


]0;admlocal@devOps: ~/leantime_ansibleadmlocal@devOps:~/leantime_ansible$ molecule convergeidempotence
\textcolor{ansi34}{INFO    } default scenario test matrix: idempotence
\textcolor{ansi34}{INFO    } Performing prerun\textcolor{ansi33}{...}
\textcolor{ansi34}{INFO    } Using .cache/roles/nunomourinho.leantime_ansible symlink to current repository in order to enable Ansible to find the role using its expected full name.
\textcolor{ansi34}{INFO    } Added \textcolor{ansi33}{ANSIBLE_ROLES_PATH}=~\textcolor{ansi35}{/.ansible/}\textcolor{ansi95}{roles}:\textcolor{ansi35}{/usr/share/ansible/}\textcolor{ansi95}{roles}:\textcolor{ansi35}{/etc/ansible/}\textcolor{ansi95}{roles}:.\textcolor{ansi35}{/.cache/}\textcolor{ansi95}{roles}
\textcolor{ansi34}{INFO    } \textcolor{ansi2 ansi36}{Running }\textcolor{ansi2 ansi32}{default}\textcolor{ansi2 ansi36}{ > }\textcolor{ansi2 ansi32}{idempotence}

PLAY [Converge] **************************************************************************************************************************

TASK [Gathering Facts] *******************************************************************************************************************
\textcolor{ansi32}{ok: [ubuntu-20.04]}

TASK [Include leantime_ansible] **********************************************************************************************************

TASK [leantime_ansible : Atualiza a cache (equivalente a apt update)] ********************************************************************
\textcolor{ansi32}{ok: [ubuntu-20.04]}

TASK [leantime_ansible : Atualiza o sistema operativo] ***********************************************************************************
\textcolor{ansi32}{ok: [ubuntu-20.04]}

TASK [leantime_ansible : Instalar as dependencias necessária ao programa leantime] *******************************************************
\textcolor{ansi32}{ok: [ubuntu-20.04]}

TASK [leantime_ansible : Instala o serviço apache2 no arranque do sistema] ***************************************************************
\textcolor{ansi32}{ok: [ubuntu-20.04]}

TASK [leantime_ansible : Rotina Auxiliar: Procura o caminho do ficheiro php.ini do servidor apache2] *************************************
\textcolor{ansi32}{ok: [ubuntu-20.04]}

TASK [leantime_ansible : Ativa várias opções no ficheiro php.ini, utilizando o módulo lineinfile.] ***************************************
\textcolor{ansi32}{ok: [ubuntu-20.04] => (item={'regexp': '^file_uploads', 'line': 'file_uploads = 1'})}
\textcolor{ansi32}{ok: [ubuntu-20.04] => (item={'regexp': '^upload_max_filesize', 'line': 'upload_max_filesize = 1G'})}
\textcolor{ansi32}{ok: [ubuntu-20.04] => (item={'regexp': '^max_file_uploads', 'line': 'max_file_uploads = 20'})}
\textcolor{ansi32}{ok: [ubuntu-20.04] => (item={'regexp': '^post_max_size', 'line': 'post_max_size = 2G'})}
\textcolor{ansi32}{ok: [ubuntu-20.04] => (item={'regexp': '^memory_limit', 'line': 'memory_limit = 2G'})}
\textcolor{ansi32}{ok: [ubuntu-20.04] => (item={'regexp': '^max_input_time', 'line': 'max_input_time = 3600'})}

TASK [leantime_ansible : Cria a directoria temporária leantime e a directoria de apache leantime] ****************************************
\textcolor{ansi32}{ok: [ubuntu-20.04] => (item={'path': '/tmp/leantime'})}
\textcolor{ansi32}{ok: [ubuntu-20.04] => (item={'path': '/var/www/leantime'})}

TASK [leantime_ansible : Cria a base de dados de mysql para o leantime] ******************************************************************
\textcolor{ansi32}{ok: [ubuntu-20.04]}

TASK [leantime_ansible : Cria o utilizador para a base de dados leantime] ****************************************************************
\textbf{\textcolor{ansi35}{[WARNING]: Module did not set no_log for update_password}}
\textcolor{ansi32}{ok: [ubuntu-20.04]}

TASK [leantime_ansible : Faz o download do software leantime a partir do repositório git] ************************************************
\textcolor{ansi32}{ok: [ubuntu-20.04]}

TASK [leantime_ansible : Descomprime o software leantime] ********************************************************************************
\textcolor{ansi32}{ok: [ubuntu-20.04]}

TASK [leantime_ansible : Copia o ficheiro de configuração configuration.sample.php] ******************************************************
\textcolor{ansi32}{ok: [ubuntu-20.04]}

TASK [leantime_ansible : Altera as variáveis dbuser, dbpassword e dbdatabase no ficheiro configuration.php utilizando o módulo replace.] ***
\textcolor{ansi32}{ok: [ubuntu-20.04] => (item={'regexp': 'dbUser=""', 'line': 'dbUser="leantimeDBadmin"'})}
\textcolor{ansi32}{ok: [ubuntu-20.04] => (item={'regexp': 'dbDatabase=""', 'line': 'dbDatabase="leantime_database"'})}
\textcolor{ansi32}{ok: [ubuntu-20.04] => (item={'regexp': 'dbPassword=""', 'line': 'dbPassword="#S3gr3d0S3cr3t0#"'})}

TASK [leantime_ansible : Copia o template do site para a pásta /etc/apache2/sites-available] *********************************************
\textcolor{ansi32}{ok: [ubuntu-20.04]}

TASK [leantime_ansible : Cria um link simbólico para o ficheiro leantime.conf na pasta /etc/apache/sites-enabled] ************************
\textcolor{ansi32}{ok: [ubuntu-20.04]}

PLAY RECAP *******************************************************************************************************************************
\textcolor{ansi32}{ubuntu-20.04}               : \textcolor{ansi32}{ok=16  } changed=0    unreachable=0    failed=0    skipped=0    rescued=0    ignored=0


\textcolor{ansi34}{INFO    } Idempotence completed successfully.
]0;admlocal@devOps: ~/leantime_ansibleadmlocal@devOps:~/leantime_ansible$ molecule verify
\textcolor{ansi34}{INFO    } default scenario test matrix: verify
\textcolor{ansi34}{INFO    } Performing prerun\textcolor{ansi33}{...}
\textcolor{ansi34}{INFO    } Using .cache/roles/nunomourinho.leantime_ansible symlink to current repository in order to enable Ansible to find the role using its expected full name.
\textcolor{ansi34}{INFO    } Added \textcolor{ansi33}{ANSIBLE_ROLES_PATH}=~\textcolor{ansi35}{/.ansible/}\textcolor{ansi95}{roles}:\textcolor{ansi35}{/usr/share/ansible/}\textcolor{ansi95}{roles}:\textcolor{ansi35}{/etc/ansible/}\textcolor{ansi95}{roles}:.\textcolor{ansi35}{/.cache/}\textcolor{ansi95}{roles}
\textcolor{ansi34}{INFO    } \textcolor{ansi2 ansi36}{Running }\textcolor{ansi2 ansi32}{default}\textcolor{ansi2 ansi36}{ > }\textcolor{ansi2 ansi32}{verify}
\textcolor{ansi34}{INFO    } Running Ansible Verifier

PLAY [Infraestrutura conduzida por testes] ***********************************************************************************************

TASK [Variáveis] *************************************************************************************************************************
\textcolor{ansi32}{ok: [ubuntu-20.04]}

TASK [Simulação: Atualizar a cache do sistema] *******************************************************************************************
\textcolor{ansi32}{ok: [ubuntu-20.04]}

TASK [teste: a cache encontra-se actualizada?] *******************************************************************************************
\textcolor{ansi32}{ok: [ubuntu-20.04] => {}
\textcolor{ansi32}{    "changed": false,}
\textcolor{ansi32}{    "msg": "SUCESSO: A cache está atualizada"}
\textcolor{ansi32}{}}

TASK [Atualizar o sistema operativo (equivalente a apt upgrade)] *************************************************************************
\textcolor{ansi32}{ok: [ubuntu-20.04]}

TASK [teste: o sistema operativo encontra-se atualizado?] ********************************************************************************
\textcolor{ansi32}{ok: [ubuntu-20.04] => {}
\textcolor{ansi32}{    "changed": false,}
\textcolor{ansi32}{    "msg": "SUCESSO: O sistema operativo está atualizado"}
\textcolor{ansi32}{}}

TASK [Teste: a cache encontra-se atualizada?] ********************************************************************************************
\textcolor{ansi32}{ok: [ubuntu-20.04] => {}
\textcolor{ansi32}{    "changed": false,}
\textcolor{ansi32}{    "msg": "SUCESSO: A cache está atualizada"}
\textcolor{ansi32}{}}

TASK [Simulação: testa se as aplicações dependencia do software leantime se encontram instaladas] ****************************************
\textcolor{ansi32}{ok: [ubuntu-20.04]}

TASK [Teste: as dependencias encontra-se instaladas?] ************************************************************************************
\textcolor{ansi32}{ok: [ubuntu-20.04] => {}
\textcolor{ansi32}{    "changed": false,}
\textcolor{ansi32}{    "msg": "SUCESSO: As dependencias estavam instaladas"}
\textcolor{ansi32}{}}

TASK [Simulação: Testar se o serviço apache2 se encontra instalado, iniciado e ativo no arranque] ****************************************
\textcolor{ansi32}{ok: [ubuntu-20.04]}

TASK [Teste: O serviço apache2 encontra-se ativo no arranque no sistema, e está iniciado?] ***********************************************
\textcolor{ansi32}{ok: [ubuntu-20.04] => {}
\textcolor{ansi32}{    "changed": false,}
\textcolor{ansi32}{    "msg": "SUCESSO: O serviço apache2 está correctamente instalado e inicia com o arranque do sistema"}
\textcolor{ansi32}{}}

TASK [Rotina Auxiliar> Procura o caminho do ficheiro php.ini do servidor apache2] ********************************************************
\textcolor{ansi32}{ok: [ubuntu-20.04]}

TASK [Simulação: Ativa a opção file_uploads no ficheiro php.ini, utilizando o módulo lineinfile.] ****************************************
\textcolor{ansi32}{ok: [ubuntu-20.04] => (item={'regexp': '^file_uploads', 'line': 'file_uploads = 1'})}
\textcolor{ansi32}{ok: [ubuntu-20.04] => (item={'regexp': '^upload_max_filesize', 'line': 'upload_max_filesize = 1G'})}
\textcolor{ansi32}{ok: [ubuntu-20.04] => (item={'regexp': '^max_file_uploads', 'line': 'max_file_uploads = 20'})}
\textcolor{ansi32}{ok: [ubuntu-20.04] => (item={'regexp': '^post_max_size', 'line': 'post_max_size = 2G'})}
\textcolor{ansi32}{ok: [ubuntu-20.04] => (item={'regexp': '^memory_limit', 'line': 'memory_limit = 2G'})}
\textcolor{ansi32}{ok: [ubuntu-20.04] => (item={'regexp': '^max_input_time', 'line': 'max_input_time = 3600'})}

TASK [Teste: As linhas do php.ini encontram-se alteradas ?] ******************************************************************************
\textcolor{ansi32}{ok: [ubuntu-20.04] => {}
\textcolor{ansi32}{    "changed": false,}
\textcolor{ansi32}{    "msg": "SUCESSO: O ficheiro php.ini foi alterado com sucesso"}
\textcolor{ansi32}{}}

TASK [Simulação: Obtem informação sobre a pastas /var/www/leantime] **********************************************************************
\textcolor{ansi32}{ok: [ubuntu-20.04]}

TASK [Teste: A pasta /var/www/leantime existe e tem as permissões certas?] ***************************************************************
\textcolor{ansi32}{ok: [ubuntu-20.04] => {}
\textcolor{ansi32}{    "changed": false,}
\textcolor{ansi32}{    "msg": "SUCESSO: Permissões correctas no site leantime"}
\textcolor{ansi32}{}}

TASK [Simulação: Verifica se é necessário criar a base de dados leantime_database] *******************************************************
\textcolor{ansi32}{ok: [ubuntu-20.04]}

TASK [Teste: A base de dados leantime_database existe ?] *********************************************************************************
\textcolor{ansi32}{ok: [ubuntu-20.04] => {}
\textcolor{ansi32}{    "changed": false,}
\textcolor{ansi32}{    "msg": "SUCESSO: A base de dados leantime_database já se encontra criada"}
\textcolor{ansi32}{}}

TASK [Simulação e Teste: Verifica se o endereço git para a versão de leantime existe] ****************************************************
\textcolor{ansi32}{ok: [ubuntu-20.04]}

TASK [Simulação: Obtem informação sobre o ficheiro /var/www/leantime/config/configuration.php] *******************************************
\textcolor{ansi32}{ok: [ubuntu-20.04]}

TASK [Teste: O ficheiro /var/www/leantime/config/configuration.php existe e tem as permissões certas?] ***********************************
\textcolor{ansi32}{ok: [ubuntu-20.04] => {}
\textcolor{ansi32}{    "changed": false,}
\textcolor{ansi32}{    "msg": "SUCESSO: Permissões correctas e ficheiro configuration.php existente"}
\textcolor{ansi32}{}}

TASK [Simulação: Testa se o dbuser foi alterado no ficheiro configuration.php utilizando o módulo replace.] ******************************
\textcolor{ansi32}{ok: [ubuntu-20.04]}

TASK [Teste: O dbUser foi alterado?] *****************************************************************************************************
\textcolor{ansi32}{ok: [ubuntu-20.04] => {}
\textcolor{ansi32}{    "changed": false,}
\textcolor{ansi32}{    "msg": "SUCESSO: O dbUser foi alterado no ficheiro configuration.php"}
\textcolor{ansi32}{}}

TASK [Simulação: Testa se o dbDatabase foi alterado no ficheiro configuration.php utilizando o módulo replace.] **************************
\textcolor{ansi32}{ok: [ubuntu-20.04]}

TASK [Teste: a variável dbDatabase foi alterada?] ****************************************************************************************
\textcolor{ansi32}{ok: [ubuntu-20.04] => {}
\textcolor{ansi32}{    "changed": false,}
\textcolor{ansi32}{    "msg": "SUCESSO: O dbDatabase foi alterado no ficheiro configuration.php"}
\textcolor{ansi32}{}}

TASK [Simulação: Testa se o dbPassword foi alterado no ficheiro configuration.php utilizando o módulo replace.] **************************
\textcolor{ansi32}{ok: [ubuntu-20.04]}

TASK [Teste: O dbPassword foi alterado?] *************************************************************************************************
\textcolor{ansi32}{ok: [ubuntu-20.04] => {}
\textcolor{ansi32}{    "changed": false,}
\textcolor{ansi32}{    "msg": "SUCESSO: O dbPassword foi alterado no ficheiro configuration.php"}
\textcolor{ansi32}{}}

TASK [Simulação: Obtem informação sobre o ficheiro /etc/apache2/sites-available/leantime.conf] *******************************************
\textcolor{ansi32}{ok: [ubuntu-20.04]}

TASK [Teste: O ficheiro /etc/apache2/sites-available/leantime.conf existe e tem as permissões certas?] ***********************************
\textcolor{ansi31}{fatal: [ubuntu-20.04]: FAILED! => {}
\textcolor{ansi31}{    "assertion": "ficheiro.stat.mode == \"0755\"",}
\textcolor{ansi31}{    "changed": false,}
\textcolor{ansi31}{    "evaluated_to": false,}
\textcolor{ansi31}{    "msg": "ERRO: Permissões incorrectas"}
\textcolor{ansi31}{}}

PLAY RECAP *******************************************************************************************************************************
\textcolor{ansi31}{ubuntu-20.04}               : \textcolor{ansi32}{ok=27  } changed=0    unreachable=0    \textcolor{ansi31}{failed=1   } skipped=0    rescued=0    ignored=0


\textbf{\textcolor{ansi31}{CRITICAL}} Ansible return code was \textbf{\textcolor{ansi36}{2}}, command was: ansible-playbook --inventory \textcolor{ansi35}{/home/admlocal/.cache/molecule/leantime_ansible/default/}\textcolor{ansi95}{inventory} --skip-tags molecule-notest,notest \textcolor{ansi35}{/home/admlocal/leantime_ansible/molecule/default/}\textcolor{ansi95}{verify.yml}
]0;admlocal@devOps: ~/leantime_ansibleadmlocal@devOps:~/leantime_ansible$ molecule verifyidempotenceconvergenano tasks/tasks.yml 
[?2004h[?1049h[?7h[?1h=[?1h=[?25l\textcolor{inv_background inv_foreground}{[ A ler... ]}\textcolor{inv_background inv_foreground}{[ 168 linhas lidas ]}\textcolor{inv_background inv_foreground}{  GNU nano 4.8                                                tasks/tasks.yml                                                             }
\textcolor{inv_background inv_foreground}{^G} Ajuda\textcolor{inv_background inv_foreground}{^O} Gravar\textcolor{inv_background inv_foreground}{^W} Procurar\textcolor{inv_background inv_foreground}{^K} Cortar txt    \textcolor{inv_background inv_foreground}{^J} Justificar    \textcolor{inv_background inv_foreground}{^C} Pos cursor    \textcolor{inv_background inv_foreground}{M-U} Desfazer     \textcolor{inv_background inv_foreground}{M-A} Marcar txt
\textcolor{inv_background inv_foreground}{^X} Sair\textcolor{inv_background inv_foreground}{^R} Carregar\textcolor{inv_background inv_foreground}{^\} Substituir    \textcolor{inv_background inv_foreground}{^U} Colar txt     \textcolor{inv_background inv_foreground}{^T} Ortografia    \textcolor{inv_background inv_foreground}{^_} Ir p/ linha   \textcolor{inv_background inv_foreground}{M-E} Refazer\textcolor{inv_background inv_foreground}{M-6} Copiar txt
---- name: Atualiza a cache (equivalente a apt update)apt:update_cache: truecache_valid_time: 3600- name: "Atualiza o sistema operativo"apt:upgrade: safe- name: "Instalar as dependencias necessária ao programa leantime"apt:pkg:- mc- screen- git- apache2- mysql-server- php- php-mysql- php-ldap- php-cli- php-soap- php-json- graphviz- php-xml- php-gd- php-zip- libapache2-mod-php- php-dev- libmcrypt-dev- gcc- make- autoconf- libc-dev
[?12l[?25h[?25l - autoconf    - libc-dev  - pkg-config- pwgen- curl- unzip    - zip  - php-mbstring- expect- net-tools    - python3-mysqldb  - python3-apt- python3-pycurl
\textcolor{ansi36}{   # Fonte de inspiração: https://docs.ansible.com/ansible/latest/collections/ansible/builtin/service_module.html}- name: "Instala o serviço apache2 no arranque do sistema"service:name: apache2state: startedenabled: true
\textcolor{ansi36}{   # Fonte: https://www.mydailytutorials.com/using-ansible-find-module-search-filesfolder}- name: "Rotina Auxiliar: Procura o caminho do ficheiro php.ini do servidor apache2"find:paths: /etcpatterns: "php.ini"recurse: trueregister: caminho_php_ini
\textcolor{ansi36}{   # Fonte: https://docs.ansible.com/ansible/latest/collections/ansible/builtin/lineinfile_module.html}- name: "Ativa várias opções no ficheiro php.ini, utilizando o módulo lineinfile."ansible.builtin.lineinfile:path: "{{ caminho_php_ini.files[0].path }}"regexp: "{{ item.regexp  }}"line: "{{ item.line }}"
[?12l[?25h[?25lregexp: "{{ item.regexp  }}"line: "{{ item.line }}"with_items:regexp: "^file_uploads"  line: "file_uploads = 1"regexp: "^upload_max_filesize"  line: "upload_max_filesize = 1G"regexp: "^max_file_uploads"  line: "max_file_uploads = 20"regexp: "^post_max_size"  line: "post_max_size = 2G"regexp: "^memory_limit"  line: "memory_limit = 2G"- regexp: "^max_input_time" line: "max_input_time = 3600"
\textcolor{ansi36}{   # Fonte: https://docs.ansible.com/ansible/2.7/modules/file_module.html}- name: Cria a directoria temporária leantime e a directoria de apache leantimefile:path: "{{ item.path }}"state: directory mode: 0755    owner: www-data  group: www-datawith_items:- path: "/tmp/leantime"- path: "/var/www/leantime"
\textcolor{ansi36}{   # Fonte: Ansible for DevOps - Server and configuration management for humans de Jeff Geerling, página 98
}   - name: "Cria a base de dados de mysql para o leantime"  mysql_db: "db=leantime_database state=present"
- name: "Cria o utilizador para a base de dados leantime"mysql_user:namlogin_leantime }}"
[?12l[?25h[?25lmysql_user:namlogin_leantime }}"  password: "{{ password_leantime }}"priv: "leantime_database.*:ALL"host: localhoststate: present
- name: "Faz o download do software leantime a partir do repositório git"get_url:url: https://github.com/Leantime/leantime/releases/download/{{ versao_leantime }}/Leantime-{{ versao_leantime }}.zipdest: /tmp/leantimemode: 0755
\textcolor{ansi36}{   # O ficheiro Leantime-versao.zip necessita de ser descomprimido}- name: Descomprime o software leantimeunarchive: src: "/tmp/leantime/Leantime-{{ versao_leantime }}.zip"    dest: "/var/www/leantime"  owner: www-datagroup: www-datamode: 0755remote_src: true
\textcolor{ansi36}{   # Após a descompressão, se o ficheiro configuration.php não existir, este necessita de ser
   # criado a partir do ficheiro configuration.sample.php
   # Fonte: https://docs.ansible.com/ansible/latest/collections/ansible/builtin/copy_module.html}- name: Copia o ficheiro de configuração configuration.sample.phpansible.builtin.copy: src: /var/www/leantime/config/configuration.sample.php    dest: /var/www/leantime/config/configuration.php  owner: www-datagroup: www-data    mode: '0755'  remote_src: trueforce: false
[?12l[?25h[?25l  remote_src: trueforce: false
\textcolor{ansi36}{   # Fonte: https://docs.ansible.com/ansible/latest/collections/ansible/builtin/replace_module.html}- name: Altera as variáveis dbuser, dbpassword e dbdatabase no ficheiro configuration.php utilizando o módulo replace.ansible.builtin.replace:path: /var/www/leantime/config/configuration.php    regexp: "{{ item.regexp  }}"  replace: "{{ item.line }}"with_items:- regexp: 'dbUser=""'  line: 'dbUser="{{ login_leantime }}"'- regexp: 'dbDatabase=""' line: 'dbDatabase="leantime_database"'    - regexp: 'dbPassword=""'    line: 'dbPassword="{{ password_leantime }}"'
\textcolor{ansi36}{   # Fonte: https://docs.ansible.com/ansible/latest/collections/ansible/builtin/template_module.html}- name: Copia o template do site para a pásta /etc/apache2/sites-availableansible.builtin.template:src: leantime.template.j2dest: /etc/apache2/sites-available/leantime.confowner: root group: www-data mode: 0755 force: false
\textcolor{ansi36}{   # Fonte: https://docs.ansible.com/ansible/latest/collections/ansible/builtin/file_module.html}- name: "Cria um link simbólico para o ficheiro leantime.conf na pasta /etc/apache/sites-enabled"ansible.builtin.file:src: /etc/apache2/sites-available/leantime.confdest: /etc/apache2/sites-enabled/leantime.confowner: rootgroup: rootmode: '0777'
[?12l[?25h[?25lgroup: rootmode: '0777'state: link
[?12l[?25h[?25l[?12l[?25h[?25l[?12l[?25h[?25l\textcolor{inv_background inv_foreground}{Modificado}'[?12l[?25h[?25l'[?12l[?25h[?25l5'[?12l[?25h[?25l55'[?12l[?25h[?25l[?12l[?25h[?25l         \textcolor{inv_background inv_foreground}{M-D} Formato DOS\textcolor{inv_background inv_foreground}{M-A} Anexar\textcolor{inv_background inv_foreground}{B} Segurança\textcolor{inv_background inv_foreground}{C} Cancelar           \textcolor{inv_background inv_foreground}{M-M} Formato Mac\textcolor{inv_background inv_foreground}{M-P} Prepor\textcolor{inv_background inv_foreground}{^T} P/ ficheiros
\textcolor{inv_background inv_foreground}{Nome do ficheiro onde escrever: tasks/tasks.yml                                                                                           }[?12l[?25h[?25l \textcolor{inv_background inv_foreground}{[ A escrever... ]}\textcolor{inv_background inv_foreground}{          }\textcolor{inv_background inv_foreground}{[ 168 linhas escritas ]}\textcolor{inv_background inv_foreground}{^O} Gravar\textcolor{inv_background inv_foreground}{^W} Procurar      \textcolor{inv_background inv_foreground}{^K} Cortar txt    \textcolor{inv_background inv_foreground}{^J} Justificar    \textcolor{inv_background inv_foreground}{^C} Pos cursor\textcolor{inv_background inv_foreground}{U} Desfazer     \textcolor{inv_background inv_foreground}{M-A} Marcar txt\textcolor{inv_background inv_foreground}{X} Sair    \textcolor{inv_background inv_foreground}{^R} Carregar\textcolor{inv_background inv_foreground}{^\} Substituir    \textcolor{inv_background inv_foreground}{^U} Colar txt     \textcolor{inv_background inv_foreground}{^T} Ortografia    \textcolor{inv_background inv_foreground}{^_} Ir p/ linha   \textcolor{inv_background inv_foreground}{M-E} Refazer      \textcolor{inv_background inv_foreground}{M-6} Copiar txt
[?12l[?25h[?25l
[?12l[?25h[?1049l
[?1l>[?2004l]0;admlocal@devOps: ~/leantime_ansibleadmlocal@devOps:~/leantime_ansible$ nano tasks/tasks.yml molecule verifyidempotenceconverge
\textcolor{ansi34}{INFO    } default scenario test matrix: dependency, create, prepare, converge
\textcolor{ansi34}{INFO    } Performing prerun\textcolor{ansi33}{...}
\textcolor{ansi34}{INFO    } Using .cache/roles/nunomourinho.leantime_ansible symlink to current repository in order to enable Ansible to find the role using its expected full name.
\textcolor{ansi34}{INFO    } Added \textcolor{ansi33}{ANSIBLE_ROLES_PATH}=~\textcolor{ansi35}{/.ansible/}\textcolor{ansi95}{roles}:\textcolor{ansi35}{/usr/share/ansible/}\textcolor{ansi95}{roles}:\textcolor{ansi35}{/etc/ansible/}\textcolor{ansi95}{roles}:.\textcolor{ansi35}{/.cache/}\textcolor{ansi95}{roles}
\textcolor{ansi34}{INFO    } \textcolor{ansi2 ansi36}{Running }\textcolor{ansi2 ansi32}{default}\textcolor{ansi2 ansi36}{ > }\textcolor{ansi2 ansi32}{dependency}
\textcolor{ansi31}{WARNING } Skipping, missing the requirements file.
\textcolor{ansi31}{WARNING } Skipping, missing the requirements file.
\textcolor{ansi34}{INFO    } \textcolor{ansi2 ansi36}{Running }\textcolor{ansi2 ansi32}{default}\textcolor{ansi2 ansi36}{ > }\textcolor{ansi2 ansi32}{create}
\textcolor{ansi31}{WARNING } Skipping, instances already created.
\textcolor{ansi34}{INFO    } \textcolor{ansi2 ansi36}{Running }\textcolor{ansi2 ansi32}{default}\textcolor{ansi2 ansi36}{ > }\textcolor{ansi2 ansi32}{prepare}
\textcolor{ansi31}{WARNING } Skipping, instances already prepared.
\textcolor{ansi34}{INFO    } \textcolor{ansi2 ansi36}{Running }\textcolor{ansi2 ansi32}{default}\textcolor{ansi2 ansi36}{ > }\textcolor{ansi2 ansi32}{converge}

PLAY [Converge] **************************************************************************************************************************

TASK [Gathering Facts] *******************************************************************************************************************
\textcolor{ansi32}{ok: [ubuntu-20.04]}

TASK [Include leantime_ansible] **********************************************************************************************************

TASK [leantime_ansible : Atualiza a cache (equivalente a apt update)] ********************************************************************
\textcolor{ansi32}{ok: [ubuntu-20.04]}

TASK [leantime_ansible : Atualiza o sistema operativo] ***********************************************************************************
\textcolor{ansi32}{ok: [ubuntu-20.04]}

TASK [leantime_ansible : Instalar as dependencias necessária ao programa leantime] *******************************************************
\textcolor{ansi32}{ok: [ubuntu-20.04]}

TASK [leantime_ansible : Instala o serviço apache2 no arranque do sistema] ***************************************************************
\textcolor{ansi32}{ok: [ubuntu-20.04]}

TASK [leantime_ansible : Rotina Auxiliar: Procura o caminho do ficheiro php.ini do servidor apache2] *************************************
\textcolor{ansi32}{ok: [ubuntu-20.04]}

TASK [leantime_ansible : Ativa várias opções no ficheiro php.ini, utilizando o módulo lineinfile.] ***************************************
\textcolor{ansi32}{ok: [ubuntu-20.04] => (item={'regexp': '^file_uploads', 'line': 'file_uploads = 1'})}
\textcolor{ansi32}{ok: [ubuntu-20.04] => (item={'regexp': '^upload_max_filesize', 'line': 'upload_max_filesize = 1G'})}
\textcolor{ansi32}{ok: [ubuntu-20.04] => (item={'regexp': '^max_file_uploads', 'line': 'max_file_uploads = 20'})}
\textcolor{ansi32}{ok: [ubuntu-20.04] => (item={'regexp': '^post_max_size', 'line': 'post_max_size = 2G'})}
\textcolor{ansi32}{ok: [ubuntu-20.04] => (item={'regexp': '^memory_limit', 'line': 'memory_limit = 2G'})}
\textcolor{ansi32}{ok: [ubuntu-20.04] => (item={'regexp': '^max_input_time', 'line': 'max_input_time = 3600'})}

TASK [leantime_ansible : Cria a directoria temporária leantime e a directoria de apache leantime] ****************************************
\textcolor{ansi32}{ok: [ubuntu-20.04] => (item={'path': '/tmp/leantime'})}
\textcolor{ansi32}{ok: [ubuntu-20.04] => (item={'path': '/var/www/leantime'})}

TASK [leantime_ansible : Cria a base de dados de mysql para o leantime] ******************************************************************
\textcolor{ansi32}{ok: [ubuntu-20.04]}

TASK [leantime_ansible : Cria o utilizador para a base de dados leantime] ****************************************************************
\textbf{\textcolor{ansi35}{[WARNING]: Module did not set no_log for update_password}}
\textcolor{ansi32}{ok: [ubuntu-20.04]}

TASK [leantime_ansible : Faz o download do software leantime a partir do repositório git] ************************************************
\textcolor{ansi32}{ok: [ubuntu-20.04]}

TASK [leantime_ansible : Descomprime o software leantime] ********************************************************************************
\textcolor{ansi32}{ok: [ubuntu-20.04]}

TASK [leantime_ansible : Copia o ficheiro de configuração configuration.sample.php] ******************************************************
\textcolor{ansi32}{ok: [ubuntu-20.04]}

TASK [leantime_ansible : Altera as variáveis dbuser, dbpassword e dbdatabase no ficheiro configuration.php utilizando o módulo replace.] ***
\textcolor{ansi32}{ok: [ubuntu-20.04] => (item={'regexp': 'dbUser=""', 'line': 'dbUser="leantimeDBadmin"'})}
\textcolor{ansi32}{ok: [ubuntu-20.04] => (item={'regexp': 'dbDatabase=""', 'line': 'dbDatabase="leantime_database"'})}
\textcolor{ansi32}{ok: [ubuntu-20.04] => (item={'regexp': 'dbPassword=""', 'line': 'dbPassword="#S3gr3d0S3cr3t0#"'})}

TASK [leantime_ansible : Copia o template do site para a pásta /etc/apache2/sites-available] *********************************************
\textcolor{ansi32}{ok: [ubuntu-20.04]}

TASK [leantime_ansible : Cria um link simbólico para o ficheiro leantime.conf na pasta /etc/apache/sites-enabled] ************************
\textcolor{ansi33}{changed: [ubuntu-20.04]}

PLAY RECAP *******************************************************************************************************************************
\textcolor{ansi33}{ubuntu-20.04}               : \textcolor{ansi32}{ok=16  } \textcolor{ansi33}{changed=1   } unreachable=0    failed=0    skipped=0    rescued=0    ignored=0


]0;admlocal@devOps: ~/leantime_ansibleadmlocal@devOps:~/leantime_ansible$ molecule convergeidempotenece
Usage: molecule [OPTIONS] COMMAND [ARGS]...
Try "molecule --help" for help.

Error: No such command "idempotece".
]0;admlocal@devOps: ~/leantime_ansibleadmlocal@devOps:~/leantime_ansible$ molecule idempotecence
\textcolor{ansi34}{INFO    } default scenario test matrix: idempotence
\textcolor{ansi34}{INFO    } Performing prerun\textcolor{ansi33}{...}
\textcolor{ansi34}{INFO    } Using .cache/roles/nunomourinho.leantime_ansible symlink to current repository in order to enable Ansible to find the role using its expected full name.
\textcolor{ansi34}{INFO    } Added \textcolor{ansi33}{ANSIBLE_ROLES_PATH}=~\textcolor{ansi35}{/.ansible/}\textcolor{ansi95}{roles}:\textcolor{ansi35}{/usr/share/ansible/}\textcolor{ansi95}{roles}:\textcolor{ansi35}{/etc/ansible/}\textcolor{ansi95}{roles}:.\textcolor{ansi35}{/.cache/}\textcolor{ansi95}{roles}
\textcolor{ansi34}{INFO    } \textcolor{ansi2 ansi36}{Running }\textcolor{ansi2 ansi32}{default}\textcolor{ansi2 ansi36}{ > }\textcolor{ansi2 ansi32}{idempotence}

PLAY [Converge] **************************************************************************************************************************

TASK [Gathering Facts] *******************************************************************************************************************
\textcolor{ansi32}{ok: [ubuntu-20.04]}

TASK [Include leantime_ansible] **********************************************************************************************************

TASK [leantime_ansible : Atualiza a cache (equivalente a apt update)] ********************************************************************
\textcolor{ansi32}{ok: [ubuntu-20.04]}

TASK [leantime_ansible : Atualiza o sistema operativo] ***********************************************************************************
\textcolor{ansi32}{ok: [ubuntu-20.04]}

TASK [leantime_ansible : Instalar as dependencias necessária ao programa leantime] *******************************************************
\textcolor{ansi32}{ok: [ubuntu-20.04]}

TASK [leantime_ansible : Instala o serviço apache2 no arranque do sistema] ***************************************************************
\textcolor{ansi32}{ok: [ubuntu-20.04]}

TASK [leantime_ansible : Rotina Auxiliar: Procura o caminho do ficheiro php.ini do servidor apache2] *************************************
\textcolor{ansi32}{ok: [ubuntu-20.04]}

TASK [leantime_ansible : Ativa várias opções no ficheiro php.ini, utilizando o módulo lineinfile.] ***************************************
\textcolor{ansi32}{ok: [ubuntu-20.04] => (item={'regexp': '^file_uploads', 'line': 'file_uploads = 1'})}
\textcolor{ansi32}{ok: [ubuntu-20.04] => (item={'regexp': '^upload_max_filesize', 'line': 'upload_max_filesize = 1G'})}
\textcolor{ansi32}{ok: [ubuntu-20.04] => (item={'regexp': '^max_file_uploads', 'line': 'max_file_uploads = 20'})}
\textcolor{ansi32}{ok: [ubuntu-20.04] => (item={'regexp': '^post_max_size', 'line': 'post_max_size = 2G'})}
\textcolor{ansi32}{ok: [ubuntu-20.04] => (item={'regexp': '^memory_limit', 'line': 'memory_limit = 2G'})}
\textcolor{ansi32}{ok: [ubuntu-20.04] => (item={'regexp': '^max_input_time', 'line': 'max_input_time = 3600'})}

TASK [leantime_ansible : Cria a directoria temporária leantime e a directoria de apache leantime] ****************************************
\textcolor{ansi32}{ok: [ubuntu-20.04] => (item={'path': '/tmp/leantime'})}
\textcolor{ansi32}{ok: [ubuntu-20.04] => (item={'path': '/var/www/leantime'})}

TASK [leantime_ansible : Cria a base de dados de mysql para o leantime] ******************************************************************
\textcolor{ansi32}{ok: [ubuntu-20.04]}

TASK [leantime_ansible : Cria o utilizador para a base de dados leantime] ****************************************************************
\textbf{\textcolor{ansi35}{[WARNING]: Module did not set no_log for update_password}}
\textcolor{ansi32}{ok: [ubuntu-20.04]}

TASK [leantime_ansible : Faz o download do software leantime a partir do repositório git] ************************************************
\textcolor{ansi32}{ok: [ubuntu-20.04]}

TASK [leantime_ansible : Descomprime o software leantime] ********************************************************************************
\textcolor{ansi32}{ok: [ubuntu-20.04]}

TASK [leantime_ansible : Copia o ficheiro de configuração configuration.sample.php] ******************************************************
\textcolor{ansi32}{ok: [ubuntu-20.04]}

TASK [leantime_ansible : Altera as variáveis dbuser, dbpassword e dbdatabase no ficheiro configuration.php utilizando o módulo replace.] ***
\textcolor{ansi32}{ok: [ubuntu-20.04] => (item={'regexp': 'dbUser=""', 'line': 'dbUser="leantimeDBadmin"'})}
\textcolor{ansi32}{ok: [ubuntu-20.04] => (item={'regexp': 'dbDatabase=""', 'line': 'dbDatabase="leantime_database"'})}
\textcolor{ansi32}{ok: [ubuntu-20.04] => (item={'regexp': 'dbPassword=""', 'line': 'dbPassword="#S3gr3d0S3cr3t0#"'})}

TASK [leantime_ansible : Copia o template do site para a pásta /etc/apache2/sites-available] *********************************************
\textcolor{ansi32}{ok: [ubuntu-20.04]}

TASK [leantime_ansible : Cria um link simbólico para o ficheiro leantime.conf na pasta /etc/apache/sites-enabled] ************************
\textcolor{ansi32}{ok: [ubuntu-20.04]}

PLAY RECAP *******************************************************************************************************************************
\textcolor{ansi32}{ubuntu-20.04}               : \textcolor{ansi32}{ok=16  } changed=0    unreachable=0    failed=0    skipped=0    rescued=0    ignored=0


\textcolor{ansi34}{INFO    } Idempotence completed successfully.
]0;admlocal@devOps: ~/leantime_ansibleadmlocal@devOps:~/leantime_ansible$ molecule idempotenceverify
\textcolor{ansi34}{INFO    } default scenario test matrix: verify
\textcolor{ansi34}{INFO    } Performing prerun\textcolor{ansi33}{...}
\textcolor{ansi34}{INFO    } Using .cache/roles/nunomourinho.leantime_ansible symlink to current repository in order to enable Ansible to find the role using its expected full name.
\textcolor{ansi34}{INFO    } Added \textcolor{ansi33}{ANSIBLE_ROLES_PATH}=~\textcolor{ansi35}{/.ansible/}\textcolor{ansi95}{roles}:\textcolor{ansi35}{/usr/share/ansible/}\textcolor{ansi95}{roles}:\textcolor{ansi35}{/etc/ansible/}\textcolor{ansi95}{roles}:.\textcolor{ansi35}{/.cache/}\textcolor{ansi95}{roles}
\textcolor{ansi34}{INFO    } \textcolor{ansi2 ansi36}{Running }\textcolor{ansi2 ansi32}{default}\textcolor{ansi2 ansi36}{ > }\textcolor{ansi2 ansi32}{verify}
\textcolor{ansi34}{INFO    } Running Ansible Verifier

PLAY [Infraestrutura conduzida por testes] ***********************************************************************************************

TASK [Variáveis] *************************************************************************************************************************
\textcolor{ansi32}{ok: [ubuntu-20.04]}

TASK [Simulação: Atualizar a cache do sistema] *******************************************************************************************
\textcolor{ansi32}{ok: [ubuntu-20.04]}

TASK [teste: a cache encontra-se actualizada?] *******************************************************************************************
\textcolor{ansi32}{ok: [ubuntu-20.04] => {}
\textcolor{ansi32}{    "changed": false,}
\textcolor{ansi32}{    "msg": "SUCESSO: A cache está atualizada"}
\textcolor{ansi32}{}}

TASK [Atualizar o sistema operativo (equivalente a apt upgrade)] *************************************************************************
\textcolor{ansi32}{ok: [ubuntu-20.04]}

TASK [teste: o sistema operativo encontra-se atualizado?] ********************************************************************************
\textcolor{ansi32}{ok: [ubuntu-20.04] => {}
\textcolor{ansi32}{    "changed": false,}
\textcolor{ansi32}{    "msg": "SUCESSO: O sistema operativo está atualizado"}
\textcolor{ansi32}{}}

TASK [Teste: a cache encontra-se atualizada?] ********************************************************************************************
\textcolor{ansi32}{ok: [ubuntu-20.04] => {}
\textcolor{ansi32}{    "changed": false,}
\textcolor{ansi32}{    "msg": "SUCESSO: A cache está atualizada"}
\textcolor{ansi32}{}}

TASK [Simulação: testa se as aplicações dependencia do software leantime se encontram instaladas] ****************************************
\textcolor{ansi32}{ok: [ubuntu-20.04]}

TASK [Teste: as dependencias encontra-se instaladas?] ************************************************************************************
\textcolor{ansi32}{ok: [ubuntu-20.04] => {}
\textcolor{ansi32}{    "changed": false,}
\textcolor{ansi32}{    "msg": "SUCESSO: As dependencias estavam instaladas"}
\textcolor{ansi32}{}}

TASK [Simulação: Testar se o serviço apache2 se encontra instalado, iniciado e ativo no arranque] ****************************************
\textcolor{ansi32}{ok: [ubuntu-20.04]}

TASK [Teste: O serviço apache2 encontra-se ativo no arranque no sistema, e está iniciado?] ***********************************************
\textcolor{ansi32}{ok: [ubuntu-20.04] => {}
\textcolor{ansi32}{    "changed": false,}
\textcolor{ansi32}{    "msg": "SUCESSO: O serviço apache2 está correctamente instalado e inicia com o arranque do sistema"}
\textcolor{ansi32}{}}

TASK [Rotina Auxiliar> Procura o caminho do ficheiro php.ini do servidor apache2] ********************************************************
\textcolor{ansi32}{ok: [ubuntu-20.04]}

TASK [Simulação: Ativa a opção file_uploads no ficheiro php.ini, utilizando o módulo lineinfile.] ****************************************
\textcolor{ansi32}{ok: [ubuntu-20.04] => (item={'regexp': '^file_uploads', 'line': 'file_uploads = 1'})}
\textcolor{ansi32}{ok: [ubuntu-20.04] => (item={'regexp': '^upload_max_filesize', 'line': 'upload_max_filesize = 1G'})}
\textcolor{ansi32}{ok: [ubuntu-20.04] => (item={'regexp': '^max_file_uploads', 'line': 'max_file_uploads = 20'})}
\textcolor{ansi32}{ok: [ubuntu-20.04] => (item={'regexp': '^post_max_size', 'line': 'post_max_size = 2G'})}
\textcolor{ansi32}{ok: [ubuntu-20.04] => (item={'regexp': '^memory_limit', 'line': 'memory_limit = 2G'})}
\textcolor{ansi32}{ok: [ubuntu-20.04] => (item={'regexp': '^max_input_time', 'line': 'max_input_time = 3600'})}

TASK [Teste: As linhas do php.ini encontram-se alteradas ?] ******************************************************************************
\textcolor{ansi32}{ok: [ubuntu-20.04] => {}
\textcolor{ansi32}{    "changed": false,}
\textcolor{ansi32}{    "msg": "SUCESSO: O ficheiro php.ini foi alterado com sucesso"}
\textcolor{ansi32}{}}

TASK [Simulação: Obtem informação sobre a pastas /var/www/leantime] **********************************************************************
\textcolor{ansi32}{ok: [ubuntu-20.04]}

TASK [Teste: A pasta /var/www/leantime existe e tem as permissões certas?] ***************************************************************
\textcolor{ansi32}{ok: [ubuntu-20.04] => {}
\textcolor{ansi32}{    "changed": false,}
\textcolor{ansi32}{    "msg": "SUCESSO: Permissões correctas no site leantime"}
\textcolor{ansi32}{}}

TASK [Simulação: Verifica se é necessário criar a base de dados leantime_database] *******************************************************
\textcolor{ansi32}{ok: [ubuntu-20.04]}

TASK [Teste: A base de dados leantime_database existe ?] *********************************************************************************
\textcolor{ansi32}{ok: [ubuntu-20.04] => {}
\textcolor{ansi32}{    "changed": false,}
\textcolor{ansi32}{    "msg": "SUCESSO: A base de dados leantime_database já se encontra criada"}
\textcolor{ansi32}{}}

TASK [Simulação e Teste: Verifica se o endereço git para a versão de leantime existe] ****************************************************
\textcolor{ansi32}{ok: [ubuntu-20.04]}

TASK [Simulação: Obtem informação sobre o ficheiro /var/www/leantime/config/configuration.php] *******************************************
\textcolor{ansi32}{ok: [ubuntu-20.04]}

TASK [Teste: O ficheiro /var/www/leantime/config/configuration.php existe e tem as permissões certas?] ***********************************
\textcolor{ansi32}{ok: [ubuntu-20.04] => {}
\textcolor{ansi32}{    "changed": false,}
\textcolor{ansi32}{    "msg": "SUCESSO: Permissões correctas e ficheiro configuration.php existente"}
\textcolor{ansi32}{}}

TASK [Simulação: Testa se o dbuser foi alterado no ficheiro configuration.php utilizando o módulo replace.] ******************************
\textcolor{ansi32}{ok: [ubuntu-20.04]}

TASK [Teste: O dbUser foi alterado?] *****************************************************************************************************
\textcolor{ansi32}{ok: [ubuntu-20.04] => {}
\textcolor{ansi32}{    "changed": false,}
\textcolor{ansi32}{    "msg": "SUCESSO: O dbUser foi alterado no ficheiro configuration.php"}
\textcolor{ansi32}{}}

TASK [Simulação: Testa se o dbDatabase foi alterado no ficheiro configuration.php utilizando o módulo replace.] **************************
\textcolor{ansi32}{ok: [ubuntu-20.04]}

TASK [Teste: a variável dbDatabase foi alterada?] ****************************************************************************************
\textcolor{ansi32}{ok: [ubuntu-20.04] => {}
\textcolor{ansi32}{    "changed": false,}
\textcolor{ansi32}{    "msg": "SUCESSO: O dbDatabase foi alterado no ficheiro configuration.php"}
\textcolor{ansi32}{}}

TASK [Simulação: Testa se o dbPassword foi alterado no ficheiro configuration.php utilizando o módulo replace.] **************************
\textcolor{ansi32}{ok: [ubuntu-20.04]}

TASK [Teste: O dbPassword foi alterado?] *************************************************************************************************
\textcolor{ansi32}{ok: [ubuntu-20.04] => {}
\textcolor{ansi32}{    "changed": false,}
\textcolor{ansi32}{    "msg": "SUCESSO: O dbPassword foi alterado no ficheiro configuration.php"}
\textcolor{ansi32}{}}

TASK [Simulação: Obtem informação sobre o ficheiro /etc/apache2/sites-available/leantime.conf] *******************************************
\textcolor{ansi32}{ok: [ubuntu-20.04]}

TASK [Teste: O ficheiro /etc/apache2/sites-available/leantime.conf existe e tem as permissões certas?] ***********************************
\textcolor{ansi31}{fatal: [ubuntu-20.04]: FAILED! => {}
\textcolor{ansi31}{    "assertion": "ficheiro.stat.gr_name == \"www-data\"",}
\textcolor{ansi31}{    "changed": false,}
\textcolor{ansi31}{    "evaluated_to": false,}
\textcolor{ansi31}{    "msg": "ERRO: Permissões incorrectas"}
\textcolor{ansi31}{}}

PLAY RECAP *******************************************************************************************************************************
\textcolor{ansi31}{ubuntu-20.04}               : \textcolor{ansi32}{ok=27  } changed=0    unreachable=0    \textcolor{ansi31}{failed=1   } skipped=0    rescued=0    ignored=0


\textbf{\textcolor{ansi31}{CRITICAL}} Ansible return code was \textbf{\textcolor{ansi36}{2}}, command was: ansible-playbook --inventory \textcolor{ansi35}{/home/admlocal/.cache/molecule/leantime_ansible/default/}\textcolor{ansi95}{inventory} --skip-tags molecule-notest,notest \textcolor{ansi35}{/home/admlocal/leantime_ansible/molecule/default/}\textcolor{ansi95}{verify.yml}
]0;admlocal@devOps: ~/leantime_ansibleadmlocal@devOps:~/leantime_ansible$ nano molecule/default/verify.yml 
[?2004h[?1049h[?7h[?1h=[?1h=[?25l\textcolor{inv_background inv_foreground}{[ A ler... ]}\textcolor{inv_background inv_foreground}{[ 302 linhas lidas ]}\textcolor{inv_background inv_foreground}{  GNU nano 4.8                                          molecule/default/verify.yml                                                       }
\textcolor{inv_background inv_foreground}{^G} Ajuda\textcolor{inv_background inv_foreground}{^O} Gravar\textcolor{inv_background inv_foreground}{^W} Procurar\textcolor{inv_background inv_foreground}{^K} Cortar txt    \textcolor{inv_background inv_foreground}{^J} Justificar    \textcolor{inv_background inv_foreground}{^C} Pos cursor    \textcolor{inv_background inv_foreground}{M-U} Desfazer     \textcolor{inv_background inv_foreground}{M-A} Marcar txt
\textcolor{inv_background inv_foreground}{^X} Sair\textcolor{inv_background inv_foreground}{^R} Carregar\textcolor{inv_background inv_foreground}{^\} Substituir    \textcolor{inv_background inv_foreground}{^U} Colar txt     \textcolor{inv_background inv_foreground}{^T} Ortografia    \textcolor{inv_background inv_foreground}{^_} Ir p/ linha   \textcolor{inv_background inv_foreground}{M-E} Refazer\textcolor{inv_background inv_foreground}{M-6} Copiar txt
---
\textcolor{ansi36}{# Tutorial de inspiração para a infraestrutura conduzida por teste no site
# https://www.adictosaltrabajo.com/2020/05/08/ansible-testing-using-molecule-with-ansible-as-verifier/
}- name: "Infraestrutura conduzida por testes"hosts: allgather_facts: falsebecome: truetasks:- name: Variáveisinclude_vars:file: ../../vars/main.yml- name: "Simulação: Atualizar a cache do sistema"apt:update_cache: truecache_valid_time: 3600check_mode: trueregister: estado- name: "teste: a cache encontra-se actualizada?"assert:that:- not estado.changedsuccess_msg: "SUCESSO: A cache está atualizada"fail_msg: "ERRO: Existem actualizações pendentes"- name: "Atualizar o sistema operativo (equivalente a apt upgrade)"apt:upgrade: safecheck_mode: trueregister: estado- name: "teste: o sistema operativo encontra-se atualizado?"
[?12l[?25h[?25l  - name: "teste: o sistema operativo encontra-se atualizado?"
    assert:that: - not estado.changed    success_msg: "SUCESSO: O sistema operativo está atualizado"    fail_msg: "ERRO: Existem actualizações pendentes"
  - name: "Teste: a cache encontra-se atualizada?"  assert:    that:  - not estado.changedsuccess_msg: "SUCESSO: A cache está atualizada"fail_msg: "ERRO: Existem atualizações pendentes"
  - name: "Simulação: testa se as aplicações dependencia do software leantime se encontram instaladas"apt:pkg:  - mc  - screen- git    - apache2  - mysql-server- phpphp-mysql- php-ldap- php-cli- php-soap    - php-json  - graphviz- php-xml  - php-gd  - php-zip- libapache2-mod-php    - php-dev
[?12l[?25h[?25l- libapache2-mod-php    - php-dev  - libmcrypt-dev- gccmake- autoconf- libc-dev- pkg-config    - pwgen  - curl- unzipzip- php-mbstring- expect- net-tools    - python3-mysqldb  - python3-apt- python3-pycurlcheck_mode: trueregister: estado
  - name: "Teste: as dependencias encontra-se instaladas?"assert:that:not estado.changedsuccess_msg: "SUCESSO: As dependencias estavam instaladas"fail_msg: "ERRO: Faltam instalar algumas dependencias. O software pode estar parcialmente instalado"
\textcolor{ansi36}{  # Fonte de inspiração: https://docs.ansible.com/ansible/latest/collections/ansible/builtin/service_module.html}- name: "Simulação: Testar se o serviço apache2 se encontra instalado, iniciado e ativo no arranque"service:name: apache2state: startedenabled: truecheck_mode: true
[?12l[?25h[?25lenabled: trueO serviço apache2 encontra-se ativo no arranque no sistema, e está iniciado?"O serviço apache2 está correctamente instalado e inicia com o arranque do sistema"O serviço apache2 não se encontra correcta\textcolor{ansi36}{: https://www.mydailytutorials.com/using--find-module-search-filesfolder}Rotina Auxiliar> Procura o caminho do ficheiro php.ini do servidor apache2"find:paths: /etcpatterns: "php.ini"recurseregister: caminho_php_ini
\textcolor{ansi36}{  # Fonte: https://docs.ansible.com/ansible/latest/collections/ansible/builtin/lineinfile_module.html}- name: "Simulação: Ativa a opção file_uploads no ficheiro php.ini, utilizando o módulo lineinfile."ansible.builtin.lineinfile:path: "{{ caminho_php_ini.files[0].path }}"regexp: "{{ item.regexp  }}"line: "{{ item.line }}"with_items:- regexp: "^file_uploads"line: "file_uploads = 1"- regexp: "^upload_max_filesize"line: "upload_max_filesize = 1G"- regexp: "^max_file_uploads"line: "max_file_uploads = 20"- regexp: "^post_max_size"line: "post_max_size = 2G"- regexp: "^memory_limit"
[?12l[?25h[?25l78line: "post_max_size = 2G"- regexp: "^memory_limit"line: "memory_limit = 2G"- regexp: "^max_input_time"  line: "max_input_time = 3600"As linhas do php.ini encontram-se alteradas ?"ficheiro php.ini foi alterado com sucesso"Algumas alterações ao ficheiro php.ini não tiveram sucesso"\textcolor{ansi36}{O site deve existir na pasta /var/www/leantime. Este teste visa determinar se a pasta
  # existe e se temas permissões certas
  # Fonte de inspiração: https://docs.ansible.com/ansible/latest/collections/ansible/builtin/stat_module.html}- name: "Simulação: Obtem informação sobre a pastas /var/www/leantime"
    stat:    path: "/var/www/leantime"register: pasta
  - name: "Teste: A pasta /var/www/leantime existe e tem as permissões certas?"assert:  that:  - pasta.stat.exists- pasta.stat.isdir  - pasta.stat.mode == "0755"- pasta.stat.pw_name == "www-data"  - pasta.stat.gr_name == "www-data"success_msg: "SUCESSO: Permissões correctas no site leantime"fail_msg: "ERRO: Permissões incorrectas no site leantime ou site não existente"
\textcolor{ansi36}{  # Outro dos requesitos é a existência de uma base de dados para o site.
}[?12l[?25h[?25l\textcolor{ansi36}{  # Outro dos requesitos é a existência de uma base de dados para o site.
  # Esta base de dados deve ser criada, logo tem de ser testada para determinar se ela existe ou não.
  # Fonte de inspiração: https://serverfault.com/questions/173978/from-a-shell-script-how-can-i-check-whether-a-mysql-database-exists}- name: "Simulação: Verifica se é necessário criar a base de dados leantime_database"shell: "mysql -u root -e 'use leantime_database'"resultfailed_when: false  changed_when: false
  - name: "Teste: A base de dados leantime_database existe ?"assert:that:- result.rc == 0success_msg: "SUCESSO: A base de dados leantime_database já se encontra criada"fail_msg: "ERRO: Não existe a base de dados leantime_database"
\textcolor{ansi36}{  # Teste: O download do software leantime deve ser possível, a partir do site do leantime}- name: "Simulação e Teste: Verifica se o endereço git para a versão de leantime existe"uri:url: https://github.com/Leantime/leantime/releases/download/{{ versao_leantime }}/Leantime-{{ versao_leantime }}.zipregister: resultado\textcolor{ansi36}{Teste: Do ficheiro configuration.php no site do leantime. A existência deste ficheiro       com as permissões certas indica que o processo de unzip correu bemDocumentação oficial em: https://docs.ansible.com/ansible/latest/collections/ansible/builtin/stat_module.html}o ficheiro /var/www/leantime/config/configuration.php"/config/configuration.php"ficheiroO ficheiro /var/www/leantime/config/configuration.php existe e tem as permissões certas?"ficheiro.stat.exists
[?12l[?25h[?25lthat: - ficheiro.stat.exists - ficheiro.stat.isreg - ficheiro.stat.mode == "0755" - ficheiro.stat.pw_name == "www-data"    - ficheiro.stat.gr_name == "www-data"  success_msg: "SUCESSO: Permissões correctas e ficheiro configuration.php existente"  fail_msg: "ERRO: Permissões incorrectas ou ficheiro configuration.php não existente"
\textcolor{ansi36}{  # Teste: O ficheiro /var/www/leantime/config/configuration.php necessita de ser alterado.
  # As variáveis dbuser, dbDatabase, e dbPassword necessitam de estar configuradas.
  # Por simulação o sistema vai determinar se estas estão configuradas com o valor esperado.
  # Fonte: https://docs.ansible.com/ansible/latest/collections/ansible/builtin/replace_module.html}- name: "Simulação: Testa se o dbuser foi alterado no ficheiro configuration.php utilizando o módulo replace."ansible.builtin.replace:path: /var/www/leantime/config/configuration.phpregexp: 'dbUser=""' replace: 'dbUser="{{ login_leantime }}"'  check_mode: yesregister: conffailed_when: false
  - name: "Teste: O dbUser foi alterado?"
    assert: that: - conf is not changed    - conf is not failed  success_msg: "SUCESSO: O dbUser foi alterado no ficheiro configuration.php"fail_msg: "ERRO: O dbUser não foi alterado no ficheiro configuration.php"
\textcolor{ansi36}{  # Fonte: https://docs.ansible.com/ansible/latest/collections/ansible/builtin/replace_module.html}- name: "Simulação: Testa se o dbDatabase foi alterado no ficheiro configuration.php utilizando o módulo replace."ansible.builtin.replace:path: /var/www/leantime/config/configuration.php
[?12l[?25h[?25l    ansible.builtin.replace:path: /var/www/leantime/config/configuration.phpregexp: 'dbDatabase=""'replace: 'dbDatabase="leantime_database"'check_mode: yesregister: conffailed_when: false
  - name: "Teste: a variável dbDatabase foi alterada?"
    assert: that: - conf is not changed - conf is not failed    success_msg: "SUCESSO: O dbDatabase foi alterado no ficheiro configuration.php"  fail_msg: "ERRO: O dbDatabase não foi alterado no ficheiro configuration.php"
\textcolor{ansi36}{  # Fonte: https://docs.ansible.com/ansible/latest/collections/ansible/builtin/replace_module.html}- name: "Simulação: Testa se o dbPassword foi alterado no ficheiro configuration.php utilizando o módulo replace."ansible.builtin.replace:  path: /var/www/leantime/config/configuration.php  regexp: 'dbPassword=""'replace: 'dbPassword="{{ password_leantime }}"'  check_mode: yesregister: conffailed_when: false
  - name: "Teste: O dbPassword foi alterado?"assert:that:- conf is not changed- conf is not failed success_msg: "SUCESSO: O dbPassword foi alterado no ficheiro configuration.php"    fail_msg: "ERRO: O dbPassword não foi alterado no ficheiro configuration.php"
\textcolor{ansi36}{  # A configuração do site leantime necessita que seja criado um site através de um ficheiro de
}[?12l[?25h[?25l\textcolor{ansi36}{  # A configuração do site leantime necessita que seja criado um site através de um ficheiro de
  # configuração, dentro das pasta do apache2. Deve existir um ficheiro de configuração do
  # leantime em (/etc/apache2/sites-available}
\textcolor{ansi36}{  # Fonte: https://docs.ansible.com/ansible/latest/collections/ansible/builtin/stat_module.html}- name: "Simulação: Obtem informação sobre o ficheiro /etc/apache2/sites-available/leantime.conf"stat:path: "/etc/apache2/sites-available/leantime.conf"  register: ficheiro
  - name: "Teste: O ficheiro /etc/apache2/sites-available/leantime.conf existe e tem as permissões certas?"assert:that:  - ficheiro.stat.exists  - ficheiro.stat.isreg- ficheiro.stat.mode == "0755" - ficheiro.stat.pw_name == "root" - ficheiro.stat.gr_name == "www-data"  success_msg: "SUCESSO: Permissões correctas"fail_msg: "ERRO: Permissões incorrectas"
\textcolor{ansi36}{  # O site deve estar ativo no apache. Este teste vai verificar se o site se encontra ativo
  # https://docs.ansible.com/ansible/latest/collections/ansible/builtin/stat_module.html}- name: "Simulação: Obtem informação sobre o ficheiro /etc/apache2/sites-enabled/leantime.conf"stat:path: "/etc/apache2/sites-enabled/leantime.conf"  register: ficheiro
  - name: "Teste: O ficheiro /etc/apache2/sites-enabled/leantime.conf existe e tem as permissões certas?"assert:that:  - ficheiro.stat.exists  - ficheiro.stat.islnk- ficheiro.stat.mode == "0777" - ficheiro.stat.pw_name == "root"
[?12l[?25h[?25l77root"
[?12l[?25h[?25l78M- ficheiro.stat.islnk
[?12l[?25h[?25l78M- ficheiro.stat.exists
[?12l[?25h[?25l78Mthat:
[?12l[?25h[?25l78Massert:
[?12l[?25h[?25l78M- name: "Teste: O ficheiro /etc/apache2/sites-enabled/leantime.conf existe e tem as permissões certas?"
[?12l[?25h[?25l78M[?12l[?25h[?25l[?12l[?25h[?1049l
[?1l>[?2004l]0;admlocal@devOps: ~/leantime_ansibleadmlocal@devOps:~/leantime_ansible$ nano molecule/default/verify.yml 
[?2004h[?1049h[?7h[?1h=[?1h=[?25l\textcolor{inv_background inv_foreground}{[ A ler... ]}\textcolor{inv_background inv_foreground}{[ 302 linhas lidas ]}\textcolor{inv_background inv_foreground}{  GNU nano 4.8                                          molecule/default/verify.yml                                                       }
\textcolor{inv_background inv_foreground}{^G} Ajuda\textcolor{inv_background inv_foreground}{^O} Gravar\textcolor{inv_background inv_foreground}{^W} Procurar\textcolor{inv_background inv_foreground}{^K} Cortar txt    \textcolor{inv_background inv_foreground}{^J} Justificar    \textcolor{inv_background inv_foreground}{^C} Pos cursor    \textcolor{inv_background inv_foreground}{M-U} Desfazer     \textcolor{inv_background inv_foreground}{M-A} Marcar txt
\textcolor{inv_background inv_foreground}{^X} Sair\textcolor{inv_background inv_foreground}{^R} Carregar\textcolor{inv_background inv_foreground}{^\} Substituir    \textcolor{inv_background inv_foreground}{^U} Colar txt     \textcolor{inv_background inv_foreground}{^T} Ortografia    \textcolor{inv_background inv_foreground}{^_} Ir p/ linha   \textcolor{inv_background inv_foreground}{M-E} Refazer\textcolor{inv_background inv_foreground}{M-6} Copiar txt
---
\textcolor{ansi36}{# Tutorial de inspiração para a infraestrutura conduzida por teste no site
# https://www.adictosaltrabajo.com/2020/05/08/ansible-testing-using-molecule-with-ansible-as-verifier/
}- name: "Infraestrutura conduzida por testes"hosts: allgather_facts: falsebecome: truetasks:- name: Variáveisinclude_vars:file: ../../vars/main.yml- name: "Simulação: Atualizar a cache do sistema"apt:update_cache: truecache_valid_time: 3600check_mode: trueregister: estado- name: "teste: a cache encontra-se actualizada?"assert:that:- not estado.changedsuccess_msg: "SUCESSO: A cache está atualizada"fail_msg: "ERRO: Existem actualizações pendentes"- name: "Atualizar o sistema operativo (equivalente a apt upgrade)"apt:upgrade: safecheck_mode: trueregister: estado- name: "teste: o sistema operativo encontra-se atualizado?"
[?12l[?25h[?25l  - name: "teste: o sistema operativo encontra-se atualizado?"
    assert:that: - not estado.changed    success_msg: "SUCESSO: O sistema operativo está atualizado"    fail_msg: "ERRO: Existem actualizações pendentes"
  - name: "Teste: a cache encontra-se atualizada?"  assert:    that:  - not estado.changedsuccess_msg: "SUCESSO: A cache está atualizada"fail_msg: "ERRO: Existem atualizações pendentes"
  - name: "Simulação: testa se as aplicações dependencia do software leantime se encontram instaladas"apt:pkg:  - mc  - screen- git    - apache2  - mysql-server- phpphp-mysql- php-ldap- php-cli- php-soap    - php-json  - graphviz- php-xml  - php-gd  - php-zip- libapache2-mod-php    - php-dev
[?12l[?25h[?25l- libapache2-mod-php    - php-dev  - libmcrypt-dev- gccmake- autoconf- libc-dev- pkg-config    - pwgen  - curl- unzipzip- php-mbstring- expect- net-tools    - python3-mysqldb  - python3-apt- python3-pycurlcheck_mode: trueregister: estado
  - name: "Teste: as dependencias encontra-se instaladas?"assert:that:not estado.changedsuccess_msg: "SUCESSO: As dependencias estavam instaladas"fail_msg: "ERRO: Faltam instalar algumas dependencias. O software pode estar parcialmente instalado"
\textcolor{ansi36}{  # Fonte de inspiração: https://docs.ansible.com/ansible/latest/collections/ansible/builtin/service_module.html}- name: "Simulação: Testar se o serviço apache2 se encontra instalado, iniciado e ativo no arranque"service:name: apache2state: startedenabled: truecheck_mode: true
[?12l[?25h[?25lenabled: trueO serviço apache2 encontra-se ativo no arranque no sistema, e está iniciado?"O serviço apache2 está correctamente instalado e inicia com o arranque do sistema"O serviço apache2 não se encontra correcta\textcolor{ansi36}{: https://www.mydailytutorials.com/using--find-module-search-filesfolder}Rotina Auxiliar> Procura o caminho do ficheiro php.ini do servidor apache2"find:paths: /etcpatterns: "php.ini"recurseregister: caminho_php_ini
\textcolor{ansi36}{  # Fonte: https://docs.ansible.com/ansible/latest/collections/ansible/builtin/lineinfile_module.html}- name: "Simulação: Ativa a opção file_uploads no ficheiro php.ini, utilizando o módulo lineinfile."ansible.builtin.lineinfile:path: "{{ caminho_php_ini.files[0].path }}"regexp: "{{ item.regexp  }}"line: "{{ item.line }}"with_items:- regexp: "^file_uploads"line: "file_uploads = 1"- regexp: "^upload_max_filesize"line: "upload_max_filesize = 1G"- regexp: "^max_file_uploads"line: "max_file_uploads = 20"- regexp: "^post_max_size"line: "post_max_size = 2G"- regexp: "^memory_limit"
[?12l[?25h[?25l78line: "post_max_size = 2G"- regexp: "^memory_limit"line: "memory_limit = 2G"- regexp: "^max_input_time"  line: "max_input_time = 3600"As linhas do php.ini encontram-se alteradas ?"ficheiro php.ini foi alterado com sucesso"Algumas alterações ao ficheiro php.ini não tiveram sucesso"\textcolor{ansi36}{O site deve existir na pasta /var/www/leantime. Este teste visa determinar se a pasta
  # existe e se temas permissões certas
  # Fonte de inspiração: https://docs.ansible.com/ansible/latest/collections/ansible/builtin/stat_module.html}- name: "Simulação: Obtem informação sobre a pastas /var/www/leantime"
    stat:    path: "/var/www/leantime"register: pasta
  - name: "Teste: A pasta /var/www/leantime existe e tem as permissões certas?"assert:  that:  - pasta.stat.exists- pasta.stat.isdir  - pasta.stat.mode == "0755"- pasta.stat.pw_name == "www-data"  - pasta.stat.gr_name == "www-data"success_msg: "SUCESSO: Permissões correctas no site leantime"fail_msg: "ERRO: Permissões incorrectas no site leantime ou site não existente"
\textcolor{ansi36}{  # Outro dos requesitos é a existência de uma base de dados para o site.
}[?12l[?25h[?25l\textcolor{ansi36}{  # Outro dos requesitos é a existência de uma base de dados para o site.
  # Esta base de dados deve ser criada, logo tem de ser testada para determinar se ela existe ou não.
  # Fonte de inspiração: https://serverfault.com/questions/173978/from-a-shell-script-how-can-i-check-whether-a-mysql-database-exists}- name: "Simulação: Verifica se é necessário criar a base de dados leantime_database"shell: "mysql -u root -e 'use leantime_database'"resultfailed_when: false  changed_when: false
  - name: "Teste: A base de dados leantime_database existe ?"assert:that:- result.rc == 0success_msg: "SUCESSO: A base de dados leantime_database já se encontra criada"fail_msg: "ERRO: Não existe a base de dados leantime_database"
\textcolor{ansi36}{  # Teste: O download do software leantime deve ser possível, a partir do site do leantime}- name: "Simulação e Teste: Verifica se o endereço git para a versão de leantime existe"uri:url: https://github.com/Leantime/leantime/releases/download/{{ versao_leantime }}/Leantime-{{ versao_leantime }}.zipregister: resultado\textcolor{ansi36}{Teste: Do ficheiro configuration.php no site do leantime. A existência deste ficheiro       com as permissões certas indica que o processo de unzip correu bemDocumentação oficial em: https://docs.ansible.com/ansible/latest/collections/ansible/builtin/stat_module.html}o ficheiro /var/www/leantime/config/configuration.php"/config/configuration.php"ficheiroO ficheiro /var/www/leantime/config/configuration.php existe e tem as permissões certas?"ficheiro.stat.exists
[?12l[?25h[?25lthat: - ficheiro.stat.exists - ficheiro.stat.isreg - ficheiro.stat.mode == "0755" - ficheiro.stat.pw_name == "www-data"    - ficheiro.stat.gr_name == "www-data"  success_msg: "SUCESSO: Permissões correctas e ficheiro configuration.php existente"  fail_msg: "ERRO: Permissões incorrectas ou ficheiro configuration.php não existente"
\textcolor{ansi36}{  # Teste: O ficheiro /var/www/leantime/config/configuration.php necessita de ser alterado.
  # As variáveis dbuser, dbDatabase, e dbPassword necessitam de estar configuradas.
  # Por simulação o sistema vai determinar se estas estão configuradas com o valor esperado.
  # Fonte: https://docs.ansible.com/ansible/latest/collections/ansible/builtin/replace_module.html}- name: "Simulação: Testa se o dbuser foi alterado no ficheiro configuration.php utilizando o módulo replace."ansible.builtin.replace:path: /var/www/leantime/config/configuration.phpregexp: 'dbUser=""' replace: 'dbUser="{{ login_leantime }}"'  check_mode: yesregister: conffailed_when: false
  - name: "Teste: O dbUser foi alterado?"
    assert: that: - conf is not changed    - conf is not failed  success_msg: "SUCESSO: O dbUser foi alterado no ficheiro configuration.php"fail_msg: "ERRO: O dbUser não foi alterado no ficheiro configuration.php"
\textcolor{ansi36}{  # Fonte: https://docs.ansible.com/ansible/latest/collections/ansible/builtin/replace_module.html}- name: "Simulação: Testa se o dbDatabase foi alterado no ficheiro configuration.php utilizando o módulo replace."ansible.builtin.replace:path: /var/www/leantime/config/configuration.php
[?12l[?25h[?25l    ansible.builtin.replace:path: /var/www/leantime/config/configuration.phpregexp: 'dbDatabase=""'replace: 'dbDatabase="leantime_database"'check_mode: yesregister: conffailed_when: false
  - name: "Teste: a variável dbDatabase foi alterada?"
    assert: that: - conf is not changed - conf is not failed    success_msg: "SUCESSO: O dbDatabase foi alterado no ficheiro configuration.php"  fail_msg: "ERRO: O dbDatabase não foi alterado no ficheiro configuration.php"
\textcolor{ansi36}{  # Fonte: https://docs.ansible.com/ansible/latest/collections/ansible/builtin/replace_module.html}- name: "Simulação: Testa se o dbPassword foi alterado no ficheiro configuration.php utilizando o módulo replace."ansible.builtin.replace:  path: /var/www/leantime/config/configuration.php  regexp: 'dbPassword=""'replace: 'dbPassword="{{ password_leantime }}"'  check_mode: yesregister: conffailed_when: false
  - name: "Teste: O dbPassword foi alterado?"assert:that:- conf is not changed- conf is not failed success_msg: "SUCESSO: O dbPassword foi alterado no ficheiro configuration.php"    fail_msg: "ERRO: O dbPassword não foi alterado no ficheiro configuration.php"
\textcolor{ansi36}{  # A configuração do site leantime necessita que seja criado um site através de um ficheiro de
}[?12l[?25h[?25l\textcolor{ansi36}{  # A configuração do site leantime necessita que seja criado um site através de um ficheiro de
  # configuração, dentro das pasta do apache2. Deve existir um ficheiro de configuração do
  # leantime em (/etc/apache2/sites-available}
\textcolor{ansi36}{  # Fonte: https://docs.ansible.com/ansible/latest/collections/ansible/builtin/stat_module.html}- name: "Simulação: Obtem informação sobre o ficheiro /etc/apache2/sites-available/leantime.conf"stat:path: "/etc/apache2/sites-available/leantime.conf"  register: ficheiro
  - name: "Teste: O ficheiro /etc/apache2/sites-available/leantime.conf existe e tem as permissões certas?"assert:that:  - ficheiro.stat.exists  - ficheiro.stat.isreg- ficheiro.stat.mode == "0755" - ficheiro.stat.pw_name == "root" - ficheiro.stat.gr_name == "www-data"  success_msg: "SUCESSO: Permissões correctas"fail_msg: "ERRO: Permissões incorrectas"
\textcolor{ansi36}{  # O site deve estar ativo no apache. Este teste vai verificar se o site se encontra ativo
  # https://docs.ansible.com/ansible/latest/collections/ansible/builtin/stat_module.html}- name: "Simulação: Obtem informação sobre o ficheiro /etc/apache2/sites-enabled/leantime.conf"stat:path: "/etc/apache2/sites-enabled/leantime.conf"  register: ficheiro
  - name: "Teste: O ficheiro /etc/apache2/sites-enabled/leantime.conf existe e tem as permissões certas?"assert:that:  - ficheiro.stat.exists  - ficheiro.stat.islnk- ficheiro.stat.mode == "0777" - ficheiro.stat.pw_name == "root"
[?12l[?25h[?25l77root"
[?12l[?25h[?25l78M- ficheiro.stat.islnk
[?12l[?25h[?25l78M- ficheiro.stat.exists
[?12l[?25h[?25l78Mthat:
[?12l[?25h[?25l78Massert:
[?12l[?25h[?25l78M- name: "Teste: O ficheiro /etc/apache2/sites-enabled/leantime.conf existe e tem as permissões certas?"
[?12l[?25h[?25l78M[?12l[?25h[?25l78Mregister: ficheiro
[?12l[?25h[?25l78Mpath: "/etc/apache2/sites-enabled/leantime.conf"
[?12l[?25h[?25l78Mstat:
[?12l[?25h[?25l78M- name: "Simulação: Obtem informação sobre o ficheiro /etc/apache2/sites-enabled/leantime.conf"
[?12l[?25h[?25l78M\textcolor{ansi36}{  # https://docs.ansible.com/ansible/latest/collections/ansible/builtin/stat_module.html
}[?12l[?25h[?25l78M\textcolor{ansi36}{  # O site deve estar ativo no apache. Este teste vai verificar se o site se encontra ativo
}[?12l[?25h[?25l78M[?12l[?25h[?25l78Mfail_msg: "ERRO: Permissões incorrectas"
[?12l[?25h[?25l78Msuccess_msg: "SUCESSO: Permissões correctas"
[?12l[?25h[?25l78M- ficheiro.stat.gr_name == "www-data"
[?12l[?25h[?25l78M- ficheiro.stat.pw_name == "root"
[?12l[?25h[?25l78M- ficheiro.stat.mode == "0755"
[?12l[?25h[?25l78M- ficheiro.stat.isreg
[?12l[?25h[?25l78M- ficheiro.stat.exists
[?12l[?25h[?25l78Mthat:
[?12l[?25h[?25l78Massert:
[?12l[?25h[?25l78M- name: "Teste: O ficheiro /etc/apache2/sites-available/leantime.conf existe e tem as permissões certas?"
[?12l[?25h[?25l78M[?12l[?25h[?25l78Mregister: ficheiro
[?12l[?25h[?25l78Mpath: "/etc/apache2/sites-available/leantime.conf"
[?12l[?25h[?25l78Mstat:
[?12l[?25h[?25l78M- name: "Simulação: Obtem informação sobre o ficheiro /etc/apache2/sites-available/leantime.conf"
[?12l[?25h[?25l78M\textcolor{ansi36}{  # Fonte: https://docs.ansible.com/ansible/latest/collections/ansible/builtin/stat_module.html
}[?12l[?25h[?25l78M\textcolor{ansi36}{  # leantime em (/etc/apache2/sites-available
}[?12l[?25h[?25l[?12l[?25h[?25l[?12l[?25h[?25l[?12l[?25h[?25l[?12l[?25h[?25l[?12l[?25h[?25l[?12l[?25h[?25l[?12l[?25h[?25l[?12l[?25h[?25l[?12l[?25h[?25l[?12l[?25h[?25l[?12l[?25h[?25l[?12l[?25h[?25l[?12l[?25h[?25l[?12l[?25h[?25l[?12l[?25h[?25lata"[?12l[?25h[?25l
[?12l[?25h[?1049l
[?1l>[?2004l]0;admlocal@devOps: ~/leantime_ansibleadmlocal@devOps:~/leantime_ansible$ molecule login
\textcolor{ansi34}{INFO    } \textcolor{ansi2 ansi36}{Running }\textcolor{ansi2 ansi32}{default}\textcolor{ansi2 ansi36}{ > }\textcolor{ansi2 ansi32}{login}
Welcome to Ubuntu 20.04.2 LTS (GNU/Linux 5.4.0-72-generic x86_64)

 * Documentation:  https://help.ubuntu.com
 * Management:     https://landscape.canonical.com
 * Support:        https://ubuntu.com/advantage

  System information as of Tue May 25 20:56:10 UTC 2021

  System load:  0.26              Processes:               130
  Usage of /:   7.8% of 38.71GB   Users logged in:         0
  Memory usage: 31%               IPv4 address for enp0s3: 10.0.2.15
  Swap usage:   0%                IPv4 address for enp0s8: 172.28.128.21

 * Super-optimized for small spaces - read how we shrank the memory
   footprint of MicroK8s to make it the smallest full K8s around.

   https://ubuntu.com/blog/microk8s-memory-optimisation




*** System restart required ***
Last login: Tue May 25 20:54:44 2021 from 10.0.2.2

]0;vagrant@ubuntu-20: ~vagrant@ubuntu-20:~$ sudo /i*i-i
]0;root@ubuntu-20: ~root@ubuntu-20:~# ls /etc/apache2/sites-available/- /etc/apache2/sites-available/a /etc/apache2/sites-available/l /etc/apache2/sites-available/h /etc/apache2/sites-available/ /etc/apache2/sites-available/
ls-alh: command not found
]0;root@ubuntu-20: ~root@ubuntu-20:~# ls-alh  /etc/apache2/sites-available/[1@ 
total 24K
drwxr-xr-x 2 root root 4.0K May 25 18:57 \textbf{\textcolor{ansi34}{.}}
drwxr-xr-x 8 root root 4.0K May 20 21:19 \textbf{\textcolor{ansi34}{..}}
-rw-r--r-- 1 root root 1.4K Apr 13  2020 000-default.conf
-rw-r--r-- 1 root root 6.2K Apr 13  2020 default-ssl.conf
-rwxr-xr-x 1 root root  320 May 25 18:57 \textbf{\textcolor{ansi32}{leantime.conf}}
]0;root@ubuntu-20: ~root@ubuntu-20:~# rm leals -alh  /etc/apache2/sites-available/[1@r[1@mleantime.conf 
]0;root@ubuntu-20: ~root@ubuntu-20:~# rm  /etc/apache2/sites-available/leantime.conf 
ls -alh  /etc/apache2/sites-available/

[1@ /////////e/n/abled/
total 8.0K
drwxr-xr-x 2 root root 4.0K May 25 20:49 \textbf{\textcolor{ansi34}{.}}
drwxr-xr-x 8 root root 4.0K May 20 21:19 \textbf{\textcolor{ansi34}{..}}
lrwxrwxrwx 1 root root   35 May 20 21:19 \textbf{\textcolor{ansi36}{000-default.conf}} -> ../sites-available/000-default.conf
lrwxrwxrwx 1 root root   42 May 25 20:49 \textcolor{ansi1 ansi31 ansi40}{leantime.conf} -> \textcolor{ansi1 ansi31 ansi40}{/etc/apache2/sites-available/leantime.conf}
]0;root@ubuntu-20: ~root@ubuntu-20:~# ls -alh  /etc/apache2/sites-enabled/[1@r[1@mleantime.conf 
]0;root@ubuntu-20: ~root@ubuntu-20:~# exit
logout
]0;vagrant@ubuntu-20: ~vagrant@ubuntu-20:~$ exit
logout
Connection to 127.0.0.1 closed.

]0;admlocal@devOps: ~/leantime_ansibleadmlocal@devOps:~/leantime_ansible$ nano tasks/tasks.yml 
[?2004h[?1049h[?7h[?1h=[?1h=[?25l\textcolor{inv_background inv_foreground}{[ A ler... ]}\textcolor{inv_background inv_foreground}{[ 168 linhas lidas ]}\textcolor{inv_background inv_foreground}{  GNU nano 4.8                                                tasks/tasks.yml                                                             }
\textcolor{inv_background inv_foreground}{^G} Ajuda\textcolor{inv_background inv_foreground}{^O} Gravar\textcolor{inv_background inv_foreground}{^W} Procurar\textcolor{inv_background inv_foreground}{^K} Cortar txt    \textcolor{inv_background inv_foreground}{^J} Justificar    \textcolor{inv_background inv_foreground}{^C} Pos cursor    \textcolor{inv_background inv_foreground}{M-U} Desfazer     \textcolor{inv_background inv_foreground}{M-A} Marcar txt
\textcolor{inv_background inv_foreground}{^X} Sair\textcolor{inv_background inv_foreground}{^R} Carregar\textcolor{inv_background inv_foreground}{^\} Substituir    \textcolor{inv_background inv_foreground}{^U} Colar txt     \textcolor{inv_background inv_foreground}{^T} Ortografia    \textcolor{inv_background inv_foreground}{^_} Ir p/ linha   \textcolor{inv_background inv_foreground}{M-E} Refazer\textcolor{inv_background inv_foreground}{M-6} Copiar txt
---- name: Atualiza a cache (equivalente a apt update)apt:update_cache: truecache_valid_time: 3600- name: "Atualiza o sistema operativo"apt:upgrade: safe- name: "Instalar as dependencias necessária ao programa leantime"apt:pkg:- mc- screen- git- apache2- mysql-server- php- php-mysql- php-ldap- php-cli- php-soap- php-json- graphviz- php-xml- php-gd- php-zip- libapache2-mod-php- php-dev- libmcrypt-dev- gcc- make- autoconf- libc-dev
[?12l[?25h[?25l - autoconf    - libc-dev  - pkg-config- pwgen- curl- unzip    - zip  - php-mbstring- expect- net-tools    - python3-mysqldb  - python3-apt- python3-pycurl
\textcolor{ansi36}{   # Fonte de inspiração: https://docs.ansible.com/ansible/latest/collections/ansible/builtin/service_module.html}- name: "Instala o serviço apache2 no arranque do sistema"service:name: apache2state: startedenabled: true
\textcolor{ansi36}{   # Fonte: https://www.mydailytutorials.com/using-ansible-find-module-search-filesfolder}- name: "Rotina Auxiliar: Procura o caminho do ficheiro php.ini do servidor apache2"find:paths: /etcpatterns: "php.ini"recurse: trueregister: caminho_php_ini
\textcolor{ansi36}{   # Fonte: https://docs.ansible.com/ansible/latest/collections/ansible/builtin/lineinfile_module.html}- name: "Ativa várias opções no ficheiro php.ini, utilizando o módulo lineinfile."ansible.builtin.lineinfile:path: "{{ caminho_php_ini.files[0].path }}"regexp: "{{ item.regexp  }}"line: "{{ item.line }}"
[?12l[?25h[?25lregexp: "{{ item.regexp  }}"line: "{{ item.line }}"with_items:regexp: "^file_uploads"  line: "file_uploads = 1"regexp: "^upload_max_filesize"  line: "upload_max_filesize = 1G"regexp: "^max_file_uploads"  line: "max_file_uploads = 20"regexp: "^post_max_size"  line: "post_max_size = 2G"regexp: "^memory_limit"  line: "memory_limit = 2G"- regexp: "^max_input_time" line: "max_input_time = 3600"
\textcolor{ansi36}{   # Fonte: https://docs.ansible.com/ansible/2.7/modules/file_module.html}- name: Cria a directoria temporária leantime e a directoria de apache leantimefile:path: "{{ item.path }}"state: directory mode: 0755    owner: www-data  group: www-datawith_items:- path: "/tmp/leantime"- path: "/var/www/leantime"
\textcolor{ansi36}{   # Fonte: Ansible for DevOps - Server and configuration management for humans de Jeff Geerling, página 98
}   - name: "Cria a base de dados de mysql para o leantime"  mysql_db: "db=leantime_database state=present"
- name: "Cria o utilizador para a base de dados leantime"mysql_user:namlogin_leantime }}"
[?12l[?25h[?25lmysql_user:namlogin_leantime }}"  password: "{{ password_leantime }}"priv: "leantime_database.*:ALL"host: localhoststate: present
- name: "Faz o download do software leantime a partir do repositório git"get_url:url: https://github.com/Leantime/leantime/releases/download/{{ versao_leantime }}/Leantime-{{ versao_leantime }}.zipdest: /tmp/leantimemode: 0755
\textcolor{ansi36}{   # O ficheiro Leantime-versao.zip necessita de ser descomprimido}- name: Descomprime o software leantimeunarchive: src: "/tmp/leantime/Leantime-{{ versao_leantime }}.zip"    dest: "/var/www/leantime"  owner: www-datagroup: www-datamode: 0755remote_src: true
\textcolor{ansi36}{   # Após a descompressão, se o ficheiro configuration.php não existir, este necessita de ser
   # criado a partir do ficheiro configuration.sample.php
   # Fonte: https://docs.ansible.com/ansible/latest/collections/ansible/builtin/copy_module.html}- name: Copia o ficheiro de configuração configuration.sample.phpansible.builtin.copy: src: /var/www/leantime/config/configuration.sample.php    dest: /var/www/leantime/config/configuration.php  owner: www-datagroup: www-data    mode: '0755'  remote_src: trueforce: false
[?12l[?25h[?25l  remote_src: trueforce: false
\textcolor{ansi36}{   # Fonte: https://docs.ansible.com/ansible/latest/collections/ansible/builtin/replace_module.html}- name: Altera as variáveis dbuser, dbpassword e dbdatabase no ficheiro configuration.php utilizando o módulo replace.ansible.builtin.replace:path: /var/www/leantime/config/configuration.php    regexp: "{{ item.regexp  }}"  replace: "{{ item.line }}"with_items:- regexp: 'dbUser=""'  line: 'dbUser="{{ login_leantime }}"'- regexp: 'dbDatabase=""' line: 'dbDatabase="leantime_database"'    - regexp: 'dbPassword=""'    line: 'dbPassword="{{ password_leantime }}"'
\textcolor{ansi36}{   # Fonte: https://docs.ansible.com/ansible/latest/collections/ansible/builtin/template_module.html}- name: Copia o template do site para a pásta /etc/apache2/sites-availableansible.builtin.template:src: leantime.template.j2dest: /etc/apache2/sites-available/leantime.confowner: root group: www-data mode: 0755force: false\textcolor{ansi36}{file}"Cria um link simbólico para o ficheiro leantime.conf na pasta /etc/apache/sites-enabled"fileetc/apache2/sites-available/leantime.confetc/apache2/sites-enabled/leantime.confrootroot
[?12l[?25h[?25lgroup: rootmode: '0755'state: link
[?12l[?25h[?25l78Mowner: root
[?12l[?25h[?25l78Mdest: /etc/apache2/sites-enabled/leantime.conf
[?12l[?25h[?25l78Msrc: /etc/apache2/sites-available/leantime.conf
[?12l[?25h[?25l78Mansible.builtin.file:
[?12l[?25h[?25l78M- name: "Cria um link simbólico para o ficheiro leantime.conf na pasta /etc/apache/sites-enabled"
[?12l[?25h[?25l78M\textcolor{ansi36}{   # Fonte: https://docs.ansible.com/ansible/latest/collections/ansible/builtin/file_module.html
}[?12l[?25h[?25l78M[?12l[?25h[?25l78Mforce: false
[?12l[?25h[?25l78Mmode: 0755
[?12l[?25h[?25l78Mgroup: www-data
[?12l[?25h[?25l78Mowner: root
[?12l[?25h[?25l78Mdest: /etc/apache2/sites-available/leantime.conf
[?12l[?25h[?25l78Msrc: leantime.template.j2
[?12l[?25h[?25l78Mansible.builtin.template:
[?12l[?25h[?25l78M- name: Copia o template do site para a pásta /etc/apache2/sites-available
[?12l[?25h[?25l78M\textcolor{ansi36}{   # Fonte: https://docs.ansible.com/ansible/latest/collections/ansible/builtin/template_module.html
}[?12l[?25h[?25l78M[?12l[?25h[?25l78Mline: 'dbPassword="{{ password_leantime }}"'
[?12l[?25h[?25l[?12l[?25h[?25l[?12l[?25h[?25l[?12l[?25h[?25l[?12l[?25h[?25l[?12l[?25h[?25l[?12l[?25h[?25l[?12l[?25h[?25l[?12l[?25h[?25l[?12l[?25h[?25l[?12l[?25h[?25l[?12l[?25h[?25l[?12l[?25h[?25l[?12l[?25h[?25l[?12l[?25h[?25l[?12l[?25h[?25l[?12l[?25h[?25l[?12l[?25h[?25l[?12l[?25h[?25l[?12l[?25h[?25l[?12l[?25h[?25l\textcolor{inv_background inv_foreground}{Modificado}'[?12l[?25h[?25l'[?12l[?25h[?25l77'[?12l[?25h[?25l777'[?12l[?25h[?25l         \textcolor{inv_background inv_foreground}{M-D} Formato DOS\textcolor{inv_background inv_foreground}{M-A} Anexar\textcolor{inv_background inv_foreground}{B} Segurança\textcolor{inv_background inv_foreground}{C} Cancelar           \textcolor{inv_background inv_foreground}{M-M} Formato Mac\textcolor{inv_background inv_foreground}{M-P} Prepor\textcolor{inv_background inv_foreground}{^T} P/ ficheiros
\textcolor{inv_background inv_foreground}{Nome do ficheiro onde escrever: tasks/tasks.yml                                                                                           }[?12l[?25h[?25l \textcolor{inv_background inv_foreground}{[ A escrever... ]}\textcolor{inv_background inv_foreground}{          }\textcolor{inv_background inv_foreground}{[ 168 linhas escritas ]}\textcolor{inv_background inv_foreground}{^O} Gravar\textcolor{inv_background inv_foreground}{^W} Procurar      \textcolor{inv_background inv_foreground}{^K} Cortar txt    \textcolor{inv_background inv_foreground}{^J} Justificar    \textcolor{inv_background inv_foreground}{^C} Pos cursor\textcolor{inv_background inv_foreground}{U} Desfazer     \textcolor{inv_background inv_foreground}{M-A} Marcar txt\textcolor{inv_background inv_foreground}{X} Sair    \textcolor{inv_background inv_foreground}{^R} Carregar\textcolor{inv_background inv_foreground}{^\} Substituir    \textcolor{inv_background inv_foreground}{^U} Colar txt     \textcolor{inv_background inv_foreground}{^T} Ortografia    \textcolor{inv_background inv_foreground}{^_} Ir p/ linha   \textcolor{inv_background inv_foreground}{M-E} Refazer      \textcolor{inv_background inv_foreground}{M-6} Copiar txt
[?12l[?25h[?25l[?12l[?25h[?25l[?12l[?25h[?25ls[?12l[?25h[?25l
[?12l[?25h[?25l\textcolor{inv_background inv_foreground}{Modificado
} [?12l[?25h[?25l [?12l[?25h[?25l [?12l[?25h[?25l [?12l[?25h[?25lw[?12l[?25h[?25lw[?12l[?25h[?25lw[?12l[?25h[?25l-[?12l[?25h[?25ld[?12l[?25h[?25la[?12l[?25h[?25lt[?12l[?25h[?25la[?12l[?25h[?25l         \textcolor{inv_background inv_foreground}{M-D} Formato DOS\textcolor{inv_background inv_foreground}{M-A} Anexar\textcolor{inv_background inv_foreground}{B} Segurança\textcolor{inv_background inv_foreground}{C} Cancelar           \textcolor{inv_background inv_foreground}{M-M} Formato Mac\textcolor{inv_background inv_foreground}{M-P} Prepor\textcolor{inv_background inv_foreground}{^T} P/ ficheiros
\textcolor{inv_background inv_foreground}{Nome do ficheiro onde escrever: tasks/tasks.yml                                                                                           }[?12l[?25h[?25l \textcolor{inv_background inv_foreground}{[ A escrever... ]}\textcolor{inv_background inv_foreground}{          }\textcolor{inv_background inv_foreground}{[ 168 linhas escritas ]}\textcolor{inv_background inv_foreground}{^O} Gravar\textcolor{inv_background inv_foreground}{^W} Procurar      \textcolor{inv_background inv_foreground}{^K} Cortar txt    \textcolor{inv_background inv_foreground}{^J} Justificar    \textcolor{inv_background inv_foreground}{^C} Pos cursor\textcolor{inv_background inv_foreground}{U} Desfazer     \textcolor{inv_background inv_foreground}{M-A} Marcar txt\textcolor{inv_background inv_foreground}{X} Sair    \textcolor{inv_background inv_foreground}{^R} Carregar\textcolor{inv_background inv_foreground}{^\} Substituir    \textcolor{inv_background inv_foreground}{^U} Colar txt     \textcolor{inv_background inv_foreground}{^T} Ortografia    \textcolor{inv_background inv_foreground}{^_} Ir p/ linha   \textcolor{inv_background inv_foreground}{M-E} Refazer      \textcolor{inv_background inv_foreground}{M-6} Copiar txt
[?12l[?25h[?25l
[?12l[?25h[?1049l
[?1l>[?2004l]0;admlocal@devOps: ~/leantime_ansibleadmlocal@devOps:~/leantime_ansible$ nano tasks/tasks.yml molecule loginconvergeverify
\textcolor{ansi34}{INFO    } default scenario test matrix: verify
\textcolor{ansi34}{INFO    } Performing prerun\textcolor{ansi33}{...}
\textcolor{ansi34}{INFO    } Using .cache/roles/nunomourinho.leantime_ansible symlink to current repository in order to enable Ansible to find the role using its expected full name.
\textcolor{ansi34}{INFO    } Added \textcolor{ansi33}{ANSIBLE_ROLES_PATH}=~\textcolor{ansi35}{/.ansible/}\textcolor{ansi95}{roles}:\textcolor{ansi35}{/usr/share/ansible/}\textcolor{ansi95}{roles}:\textcolor{ansi35}{/etc/ansible/}\textcolor{ansi95}{roles}:.\textcolor{ansi35}{/.cache/}\textcolor{ansi95}{roles}
\textcolor{ansi34}{INFO    } \textcolor{ansi2 ansi36}{Running }\textcolor{ansi2 ansi32}{default}\textcolor{ansi2 ansi36}{ > }\textcolor{ansi2 ansi32}{verify}
\textcolor{ansi34}{INFO    } Running Ansible Verifier

PLAY [Infraestrutura conduzida por testes] ***********************************************************************************************

TASK [Variáveis] *************************************************************************************************************************
\textcolor{ansi32}{ok: [ubuntu-20.04]}

TASK [Simulação: Atualizar a cache do sistema] *******************************************************************************************
\textcolor{ansi32}{ok: [ubuntu-20.04]}

TASK [teste: a cache encontra-se actualizada?] *******************************************************************************************
\textcolor{ansi32}{ok: [ubuntu-20.04] => {}
\textcolor{ansi32}{    "changed": false,}
\textcolor{ansi32}{    "msg": "SUCESSO: A cache está atualizada"}
\textcolor{ansi32}{}}

TASK [Atualizar o sistema operativo (equivalente a apt upgrade)] *************************************************************************
\textcolor{ansi32}{ok: [ubuntu-20.04]}

TASK [teste: o sistema operativo encontra-se atualizado?] ********************************************************************************
\textcolor{ansi32}{ok: [ubuntu-20.04] => {}
\textcolor{ansi32}{    "changed": false,}
\textcolor{ansi32}{    "msg": "SUCESSO: O sistema operativo está atualizado"}
\textcolor{ansi32}{}}

TASK [Teste: a cache encontra-se atualizada?] ********************************************************************************************
\textcolor{ansi32}{ok: [ubuntu-20.04] => {}
\textcolor{ansi32}{    "changed": false,}
\textcolor{ansi32}{    "msg": "SUCESSO: A cache está atualizada"}
\textcolor{ansi32}{}}

TASK [Simulação: testa se as aplicações dependencia do software leantime se encontram instaladas] ****************************************
\textcolor{ansi32}{ok: [ubuntu-20.04]}

TASK [Teste: as dependencias encontra-se instaladas?] ************************************************************************************
\textcolor{ansi32}{ok: [ubuntu-20.04] => {}
\textcolor{ansi32}{    "changed": false,}
\textcolor{ansi32}{    "msg": "SUCESSO: As dependencias estavam instaladas"}
\textcolor{ansi32}{}}

TASK [Simulação: Testar se o serviço apache2 se encontra instalado, iniciado e ativo no arranque] ****************************************
\textcolor{ansi32}{ok: [ubuntu-20.04]}

TASK [Teste: O serviço apache2 encontra-se ativo no arranque no sistema, e está iniciado?] ***********************************************
\textcolor{ansi32}{ok: [ubuntu-20.04] => {}
\textcolor{ansi32}{    "changed": false,}
\textcolor{ansi32}{    "msg": "SUCESSO: O serviço apache2 está correctamente instalado e inicia com o arranque do sistema"}
\textcolor{ansi32}{}}

TASK [Rotina Auxiliar> Procura o caminho do ficheiro php.ini do servidor apache2] ********************************************************
\textcolor{ansi32}{ok: [ubuntu-20.04]}

TASK [Simulação: Ativa a opção file_uploads no ficheiro php.ini, utilizando o módulo lineinfile.] ****************************************
\textcolor{ansi32}{ok: [ubuntu-20.04] => (item={'regexp': '^file_uploads', 'line': 'file_uploads = 1'})}
\textcolor{ansi32}{ok: [ubuntu-20.04] => (item={'regexp': '^upload_max_filesize', 'line': 'upload_max_filesize = 1G'})}
\textcolor{ansi32}{ok: [ubuntu-20.04] => (item={'regexp': '^max_file_uploads', 'line': 'max_file_uploads = 20'})}
\textcolor{ansi32}{ok: [ubuntu-20.04] => (item={'regexp': '^post_max_size', 'line': 'post_max_size = 2G'})}
\textcolor{ansi32}{ok: [ubuntu-20.04] => (item={'regexp': '^memory_limit', 'line': 'memory_limit = 2G'})}
\textcolor{ansi32}{ok: [ubuntu-20.04] => (item={'regexp': '^max_input_time', 'line': 'max_input_time = 3600'})}

TASK [Teste: As linhas do php.ini encontram-se alteradas ?] ******************************************************************************
\textcolor{ansi32}{ok: [ubuntu-20.04] => {}
\textcolor{ansi32}{    "changed": false,}
\textcolor{ansi32}{    "msg": "SUCESSO: O ficheiro php.ini foi alterado com sucesso"}
\textcolor{ansi32}{}}

TASK [Simulação: Obtem informação sobre a pastas /var/www/leantime] **********************************************************************
\textcolor{ansi32}{ok: [ubuntu-20.04]}

TASK [Teste: A pasta /var/www/leantime existe e tem as permissões certas?] ***************************************************************
\textcolor{ansi32}{ok: [ubuntu-20.04] => {}
\textcolor{ansi32}{    "changed": false,}
\textcolor{ansi32}{    "msg": "SUCESSO: Permissões correctas no site leantime"}
\textcolor{ansi32}{}}

TASK [Simulação: Verifica se é necessário criar a base de dados leantime_database] *******************************************************
\textcolor{ansi32}{ok: [ubuntu-20.04]}

TASK [Teste: A base de dados leantime_database existe ?] *********************************************************************************
\textcolor{ansi32}{ok: [ubuntu-20.04] => {}
\textcolor{ansi32}{    "changed": false,}
\textcolor{ansi32}{    "msg": "SUCESSO: A base de dados leantime_database já se encontra criada"}
\textcolor{ansi32}{}}

TASK [Simulação e Teste: Verifica se o endereço git para a versão de leantime existe] ****************************************************
\textcolor{ansi32}{ok: [ubuntu-20.04]}

TASK [Simulação: Obtem informação sobre o ficheiro /var/www/leantime/config/configuration.php] *******************************************
\textcolor{ansi32}{ok: [ubuntu-20.04]}

TASK [Teste: O ficheiro /var/www/leantime/config/configuration.php existe e tem as permissões certas?] ***********************************
\textcolor{ansi32}{ok: [ubuntu-20.04] => {}
\textcolor{ansi32}{    "changed": false,}
\textcolor{ansi32}{    "msg": "SUCESSO: Permissões correctas e ficheiro configuration.php existente"}
\textcolor{ansi32}{}}

TASK [Simulação: Testa se o dbuser foi alterado no ficheiro configuration.php utilizando o módulo replace.] ******************************
\textcolor{ansi32}{ok: [ubuntu-20.04]}

TASK [Teste: O dbUser foi alterado?] *****************************************************************************************************
\textcolor{ansi32}{ok: [ubuntu-20.04] => {}
\textcolor{ansi32}{    "changed": false,}
\textcolor{ansi32}{    "msg": "SUCESSO: O dbUser foi alterado no ficheiro configuration.php"}
\textcolor{ansi32}{}}

TASK [Simulação: Testa se o dbDatabase foi alterado no ficheiro configuration.php utilizando o módulo replace.] **************************
\textcolor{ansi32}{ok: [ubuntu-20.04]}

TASK [Teste: a variável dbDatabase foi alterada?] ****************************************************************************************
\textcolor{ansi32}{ok: [ubuntu-20.04] => {}
\textcolor{ansi32}{    "changed": false,}
\textcolor{ansi32}{    "msg": "SUCESSO: O dbDatabase foi alterado no ficheiro configuration.php"}
\textcolor{ansi32}{}}

TASK [Simulação: Testa se o dbPassword foi alterado no ficheiro configuration.php utilizando o módulo replace.] **************************
\textcolor{ansi32}{ok: [ubuntu-20.04]}

TASK [Teste: O dbPassword foi alterado?] *************************************************************************************************
\textcolor{ansi32}{ok: [ubuntu-20.04] => {}
\textcolor{ansi32}{    "changed": false,}
\textcolor{ansi32}{    "msg": "SUCESSO: O dbPassword foi alterado no ficheiro configuration.php"}
\textcolor{ansi32}{}}

TASK [Simulação: Obtem informação sobre o ficheiro /etc/apache2/sites-available/leantime.conf] *******************************************
\textcolor{ansi32}{ok: [ubuntu-20.04]}

TASK [Teste: O ficheiro /etc/apache2/sites-available/leantime.conf existe e tem as permissões certas?] ***********************************
\textcolor{ansi31}{fatal: [ubuntu-20.04]: FAILED! => {}
\textcolor{ansi31}{    "assertion": "ficheiro.stat.exists",}
\textcolor{ansi31}{    "changed": false,}
\textcolor{ansi31}{    "evaluated_to": false,}
\textcolor{ansi31}{    "msg": "ERRO: Permissões incorrectas"}
\textcolor{ansi31}{}}

PLAY RECAP *******************************************************************************************************************************
\textcolor{ansi31}{ubuntu-20.04}               : \textcolor{ansi32}{ok=27  } changed=0    unreachable=0    \textcolor{ansi31}{failed=1   } skipped=0    rescued=0    ignored=0


\textbf{\textcolor{ansi31}{CRITICAL}} Ansible return code was \textbf{\textcolor{ansi36}{2}}, command was: ansible-playbook --inventory \textcolor{ansi35}{/home/admlocal/.cache/molecule/leantime_ansible/default/}\textcolor{ansi95}{inventory} --skip-tags molecule-notest,notest \textcolor{ansi35}{/home/admlocal/leantime_ansible/molecule/default/}\textcolor{ansi95}{verify.yml}
]0;admlocal@devOps: ~/leantime_ansibleadmlocal@devOps:~/leantime_ansible$ molecule verifyconverge
\textcolor{ansi34}{INFO    } default scenario test matrix: dependency, create, prepare, converge
\textcolor{ansi34}{INFO    } Performing prerun\textcolor{ansi33}{...}
\textcolor{ansi34}{INFO    } Using .cache/roles/nunomourinho.leantime_ansible symlink to current repository in order to enable Ansible to find the role using its expected full name.
\textcolor{ansi34}{INFO    } Added \textcolor{ansi33}{ANSIBLE_ROLES_PATH}=~\textcolor{ansi35}{/.ansible/}\textcolor{ansi95}{roles}:\textcolor{ansi35}{/usr/share/ansible/}\textcolor{ansi95}{roles}:\textcolor{ansi35}{/etc/ansible/}\textcolor{ansi95}{roles}:.\textcolor{ansi35}{/.cache/}\textcolor{ansi95}{roles}
\textcolor{ansi34}{INFO    } \textcolor{ansi2 ansi36}{Running }\textcolor{ansi2 ansi32}{default}\textcolor{ansi2 ansi36}{ > }\textcolor{ansi2 ansi32}{dependency}
\textcolor{ansi31}{WARNING } Skipping, missing the requirements file.
\textcolor{ansi31}{WARNING } Skipping, missing the requirements file.
\textcolor{ansi34}{INFO    } \textcolor{ansi2 ansi36}{Running }\textcolor{ansi2 ansi32}{default}\textcolor{ansi2 ansi36}{ > }\textcolor{ansi2 ansi32}{create}
\textcolor{ansi31}{WARNING } Skipping, instances already created.
\textcolor{ansi34}{INFO    } \textcolor{ansi2 ansi36}{Running }\textcolor{ansi2 ansi32}{default}\textcolor{ansi2 ansi36}{ > }\textcolor{ansi2 ansi32}{prepare}
\textcolor{ansi31}{WARNING } Skipping, instances already prepared.
\textcolor{ansi34}{INFO    } \textcolor{ansi2 ansi36}{Running }\textcolor{ansi2 ansi32}{default}\textcolor{ansi2 ansi36}{ > }\textcolor{ansi2 ansi32}{converge}

PLAY [Converge] **************************************************************************************************************************

TASK [Gathering Facts] *******************************************************************************************************************
\textcolor{ansi32}{ok: [ubuntu-20.04]}

TASK [Include leantime_ansible] **********************************************************************************************************

TASK [leantime_ansible : Atualiza a cache (equivalente a apt update)] ********************************************************************
\textcolor{ansi32}{ok: [ubuntu-20.04]}

TASK [leantime_ansible : Atualiza o sistema operativo] ***********************************************************************************
\textcolor{ansi32}{ok: [ubuntu-20.04]}

TASK [leantime_ansible : Instalar as dependencias necessária ao programa leantime] *******************************************************
\textcolor{ansi32}{ok: [ubuntu-20.04]}

TASK [leantime_ansible : Instala o serviço apache2 no arranque do sistema] ***************************************************************
\textcolor{ansi32}{ok: [ubuntu-20.04]}

TASK [leantime_ansible : Rotina Auxiliar: Procura o caminho do ficheiro php.ini do servidor apache2] *************************************
\textcolor{ansi32}{ok: [ubuntu-20.04]}

TASK [leantime_ansible : Ativa várias opções no ficheiro php.ini, utilizando o módulo lineinfile.] ***************************************
\textcolor{ansi32}{ok: [ubuntu-20.04] => (item={'regexp': '^file_uploads', 'line': 'file_uploads = 1'})}
\textcolor{ansi32}{ok: [ubuntu-20.04] => (item={'regexp': '^upload_max_filesize', 'line': 'upload_max_filesize = 1G'})}
\textcolor{ansi32}{ok: [ubuntu-20.04] => (item={'regexp': '^max_file_uploads', 'line': 'max_file_uploads = 20'})}
\textcolor{ansi32}{ok: [ubuntu-20.04] => (item={'regexp': '^post_max_size', 'line': 'post_max_size = 2G'})}
\textcolor{ansi32}{ok: [ubuntu-20.04] => (item={'regexp': '^memory_limit', 'line': 'memory_limit = 2G'})}
\textcolor{ansi32}{ok: [ubuntu-20.04] => (item={'regexp': '^max_input_time', 'line': 'max_input_time = 3600'})}

TASK [leantime_ansible : Cria a directoria temporária leantime e a directoria de apache leantime] ****************************************
\textcolor{ansi32}{ok: [ubuntu-20.04] => (item={'path': '/tmp/leantime'})}
\textcolor{ansi32}{ok: [ubuntu-20.04] => (item={'path': '/var/www/leantime'})}

TASK [leantime_ansible : Cria a base de dados de mysql para o leantime] ******************************************************************
\textcolor{ansi32}{ok: [ubuntu-20.04]}

TASK [leantime_ansible : Cria o utilizador para a base de dados leantime] ****************************************************************
\textbf{\textcolor{ansi35}{[WARNING]: Module did not set no_log for update_password}}
\textcolor{ansi32}{ok: [ubuntu-20.04]}

TASK [leantime_ansible : Faz o download do software leantime a partir do repositório git] ************************************************
\textcolor{ansi32}{ok: [ubuntu-20.04]}

TASK [leantime_ansible : Descomprime o software leantime] ********************************************************************************
\textcolor{ansi32}{ok: [ubuntu-20.04]}

TASK [leantime_ansible : Copia o ficheiro de configuração configuration.sample.php] ******************************************************
\textcolor{ansi32}{ok: [ubuntu-20.04]}

TASK [leantime_ansible : Altera as variáveis dbuser, dbpassword e dbdatabase no ficheiro configuration.php utilizando o módulo replace.] ***
\textcolor{ansi32}{ok: [ubuntu-20.04] => (item={'regexp': 'dbUser=""', 'line': 'dbUser="leantimeDBadmin"'})}
\textcolor{ansi32}{ok: [ubuntu-20.04] => (item={'regexp': 'dbDatabase=""', 'line': 'dbDatabase="leantime_database"'})}
\textcolor{ansi32}{ok: [ubuntu-20.04] => (item={'regexp': 'dbPassword=""', 'line': 'dbPassword="#S3gr3d0S3cr3t0#"'})}

TASK [leantime_ansible : Copia o template do site para a pásta /etc/apache2/sites-available] *********************************************
\textcolor{ansi33}{changed: [ubuntu-20.04]}

TASK [leantime_ansible : Cria um link simbólico para o ficheiro leantime.conf na pasta /etc/apache/sites-enabled] ************************
\textcolor{ansi33}{changed: [ubuntu-20.04]}

PLAY RECAP *******************************************************************************************************************************
\textcolor{ansi33}{ubuntu-20.04}               : \textcolor{ansi32}{ok=16  } \textcolor{ansi33}{changed=2   } unreachable=0    failed=0    skipped=0    rescued=0    ignored=0


]0;admlocal@devOps: ~/leantime_ansibleadmlocal@devOps:~/leantime_ansible$ molecule convergeidempotence
\textcolor{ansi34}{INFO    } default scenario test matrix: idempotence
\textcolor{ansi34}{INFO    } Performing prerun\textcolor{ansi33}{...}
\textcolor{ansi34}{INFO    } Using .cache/roles/nunomourinho.leantime_ansible symlink to current repository in order to enable Ansible to find the role using its expected full name.
\textcolor{ansi34}{INFO    } Added \textcolor{ansi33}{ANSIBLE_ROLES_PATH}=~\textcolor{ansi35}{/.ansible/}\textcolor{ansi95}{roles}:\textcolor{ansi35}{/usr/share/ansible/}\textcolor{ansi95}{roles}:\textcolor{ansi35}{/etc/ansible/}\textcolor{ansi95}{roles}:.\textcolor{ansi35}{/.cache/}\textcolor{ansi95}{roles}
\textcolor{ansi34}{INFO    } \textcolor{ansi2 ansi36}{Running }\textcolor{ansi2 ansi32}{default}\textcolor{ansi2 ansi36}{ > }\textcolor{ansi2 ansi32}{idempotence}

PLAY [Converge] **************************************************************************************************************************

TASK [Gathering Facts] *******************************************************************************************************************
\textcolor{ansi32}{ok: [ubuntu-20.04]}

TASK [Include leantime_ansible] **********************************************************************************************************

TASK [leantime_ansible : Atualiza a cache (equivalente a apt update)] ********************************************************************
\textcolor{ansi32}{ok: [ubuntu-20.04]}

TASK [leantime_ansible : Atualiza o sistema operativo] ***********************************************************************************
\textcolor{ansi32}{ok: [ubuntu-20.04]}

TASK [leantime_ansible : Instalar as dependencias necessária ao programa leantime] *******************************************************
\textcolor{ansi32}{ok: [ubuntu-20.04]}

TASK [leantime_ansible : Instala o serviço apache2 no arranque do sistema] ***************************************************************
\textcolor{ansi32}{ok: [ubuntu-20.04]}

TASK [leantime_ansible : Rotina Auxiliar: Procura o caminho do ficheiro php.ini do servidor apache2] *************************************
\textcolor{ansi32}{ok: [ubuntu-20.04]}

TASK [leantime_ansible : Ativa várias opções no ficheiro php.ini, utilizando o módulo lineinfile.] ***************************************
\textcolor{ansi32}{ok: [ubuntu-20.04] => (item={'regexp': '^file_uploads', 'line': 'file_uploads = 1'})}
\textcolor{ansi32}{ok: [ubuntu-20.04] => (item={'regexp': '^upload_max_filesize', 'line': 'upload_max_filesize = 1G'})}
\textcolor{ansi32}{ok: [ubuntu-20.04] => (item={'regexp': '^max_file_uploads', 'line': 'max_file_uploads = 20'})}
\textcolor{ansi32}{ok: [ubuntu-20.04] => (item={'regexp': '^post_max_size', 'line': 'post_max_size = 2G'})}
\textcolor{ansi32}{ok: [ubuntu-20.04] => (item={'regexp': '^memory_limit', 'line': 'memory_limit = 2G'})}
\textcolor{ansi32}{ok: [ubuntu-20.04] => (item={'regexp': '^max_input_time', 'line': 'max_input_time = 3600'})}

TASK [leantime_ansible : Cria a directoria temporária leantime e a directoria de apache leantime] ****************************************
\textcolor{ansi32}{ok: [ubuntu-20.04] => (item={'path': '/tmp/leantime'})}
\textcolor{ansi32}{ok: [ubuntu-20.04] => (item={'path': '/var/www/leantime'})}

TASK [leantime_ansible : Cria a base de dados de mysql para o leantime] ******************************************************************
\textcolor{ansi32}{ok: [ubuntu-20.04]}

TASK [leantime_ansible : Cria o utilizador para a base de dados leantime] ****************************************************************
\textbf{\textcolor{ansi35}{[WARNING]: Module did not set no_log for update_password}}
\textcolor{ansi32}{ok: [ubuntu-20.04]}

TASK [leantime_ansible : Faz o download do software leantime a partir do repositório git] ************************************************
\textcolor{ansi32}{ok: [ubuntu-20.04]}

TASK [leantime_ansible : Descomprime o software leantime] ********************************************************************************
\textcolor{ansi32}{ok: [ubuntu-20.04]}

TASK [leantime_ansible : Copia o ficheiro de configuração configuration.sample.php] ******************************************************
\textcolor{ansi32}{ok: [ubuntu-20.04]}

TASK [leantime_ansible : Altera as variáveis dbuser, dbpassword e dbdatabase no ficheiro configuration.php utilizando o módulo replace.] ***
\textcolor{ansi32}{ok: [ubuntu-20.04] => (item={'regexp': 'dbUser=""', 'line': 'dbUser="leantimeDBadmin"'})}
\textcolor{ansi32}{ok: [ubuntu-20.04] => (item={'regexp': 'dbDatabase=""', 'line': 'dbDatabase="leantime_database"'})}
\textcolor{ansi32}{ok: [ubuntu-20.04] => (item={'regexp': 'dbPassword=""', 'line': 'dbPassword="#S3gr3d0S3cr3t0#"'})}

TASK [leantime_ansible : Copia o template do site para a pásta /etc/apache2/sites-available] *********************************************
\textcolor{ansi32}{ok: [ubuntu-20.04]}

TASK [leantime_ansible : Cria um link simbólico para o ficheiro leantime.conf na pasta /etc/apache/sites-enabled] ************************
\textcolor{ansi32}{ok: [ubuntu-20.04]}

PLAY RECAP *******************************************************************************************************************************
\textcolor{ansi32}{ubuntu-20.04}               : \textcolor{ansi32}{ok=16  } changed=0    unreachable=0    failed=0    skipped=0    rescued=0    ignored=0


\textcolor{ansi34}{INFO    } Idempotence completed successfully.
]0;admlocal@devOps: ~/leantime_ansibleadmlocal@devOps:~/leantime_ansible$ molecule login
\textcolor{ansi34}{INFO    } \textcolor{ansi2 ansi36}{Running }\textcolor{ansi2 ansi32}{default}\textcolor{ansi2 ansi36}{ > }\textcolor{ansi2 ansi32}{login}
Welcome to Ubuntu 20.04.2 LTS (GNU/Linux 5.4.0-72-generic x86_64)

 * Documentation:  https://help.ubuntu.com
 * Management:     https://landscape.canonical.com
 * Support:        https://ubuntu.com/advantage

  System information as of Tue May 25 21:00:06 UTC 2021

  System load:  0.31              Processes:               135
  Usage of /:   7.8% of 38.71GB   Users logged in:         0
  Memory usage: 33%               IPv4 address for enp0s3: 10.0.2.15
  Swap usage:   0%                IPv4 address for enp0s8: 172.28.128.21

 * Super-optimized for small spaces - read how we shrank the memory
   footprint of MicroK8s to make it the smallest full K8s around.

   https://ubuntu.com/blog/microk8s-memory-optimisation




*** System restart required ***
Last login: Tue May 25 20:59:59 2021 from 10.0.2.2

]0;vagrant@ubuntu-20: ~vagrant@ubuntu-20:~$ sudo -i
]0;root@ubuntu-20: ~root@ubuntu-20:~# ls -alh /etc/apache2/sites-available/
total 24K
drwxr-xr-x 2 root root     4.0K May 25 20:59 \textbf{\textcolor{ansi34}{.}}
drwxr-xr-x 8 root root     4.0K May 20 21:19 \textbf{\textcolor{ansi34}{..}}
-rw-r--r-- 1 root root     1.4K Apr 13  2020 000-default.conf
-rw-r--r-- 1 root root     6.2K Apr 13  2020 default-ssl.conf
-rwxrwxrwx 1 root www-data  320 May 25 20:59 \textbf{\textcolor{ansi32}{leantime.conf}}
]0;root@ubuntu-20: ~root@ubuntu-20:~# ls -alh /etc/apache2/sites-available//////////e/n/a/b/l/e/d/
total 8.0K
drwxr-xr-x 2 root root 4.0K May 25 20:59 \textbf{\textcolor{ansi34}{.}}
drwxr-xr-x 8 root root 4.0K May 20 21:19 \textbf{\textcolor{ansi34}{..}}
lrwxrwxrwx 1 root root   35 May 20 21:19 \textbf{\textcolor{ansi36}{000-default.conf}} -> ../sites-available/000-default.conf
lrwxrwxrwx 1 root root   42 May 25 20:59 \textbf{\textcolor{ansi36}{leantime.conf}} -> \textbf{\textcolor{ansi32}{/etc/apache2/sites-available/leantime.conf}}
]0;root@ubuntu-20: ~root@ubuntu-20:~# stat -c "%a %n" &/etc/&apache2/sites-available/leantime.conf 
777 /etc/apache2/sites-available/leantime.conf
]0;root@ubuntu-20: ~root@ubuntu-20:~# stat -c "%a %n" /etc/apache2/sites-available/leantime.conf [1@e[1@n[1@a[1@b[1@l[1@e[1@d
777 /etc/apache2/sites-enabled/leantime.conf
]0;root@ubuntu-20: ~root@ubuntu-20:~# exit
logout
]0;vagrant@ubuntu-20: ~vagrant@ubuntu-20:~$ exit
logout
Connection to 127.0.0.1 closed.

]0;admlocal@devOps: ~/leantime_ansibleadmlocal@devOps:~/leantime_ansible$ nano molecule/default/
[?2004h[?1049h[?7h[?1h=[?1h=[?25l\textcolor{ansi1}{}\textbf{\textcolor{ansi37}{}}\textcolor{ansi1 ansi37 ansi41}{[ "molecule/default/" é uma pasta ]}\textcolor{ansi1}{}\textcolor{inv_background inv_foreground}{  GNU nano 4.8                                                  Novo buffer                                                               }
\textcolor{inv_background inv_foreground}{^G} Ajuda\textcolor{inv_background inv_foreground}{^O} Gravar\textcolor{inv_background inv_foreground}{^W} Procurar\textcolor{inv_background inv_foreground}{^K} Cortar txt    \textcolor{inv_background inv_foreground}{^J} Justificar    \textcolor{inv_background inv_foreground}{^C} Pos cursor    \textcolor{inv_background inv_foreground}{M-U} Desfazer     \textcolor{inv_background inv_foreground}{M-A} Marcar txt
\textcolor{inv_background inv_foreground}{^X} Sair\textcolor{inv_background inv_foreground}{^R} Carregar\textcolor{inv_background inv_foreground}{^\} Substituir    \textcolor{inv_background inv_foreground}{^U} Colar txt     \textcolor{inv_background inv_foreground}{^T} Ortografia    \textcolor{inv_background inv_foreground}{^_} Ir p/ linha   \textcolor{inv_background inv_foreground}{M-E} Refazer\textcolor{inv_background inv_foreground}{M-6} Copiar txt
[?12l[?25h[?25l\textcolor{inv_background inv_foreground}{Modificado}
v[?12l[?25h[?25l\textcolor{inv_background inv_foreground}{Gravar buffer modificado?                                                                                                                  S} Sim
\textcolor{inv_background inv_foreground}{ N} Não \textcolor{inv_background inv_foreground}{^C} Cancelar[?12l[?25h[?25l[?12l[?25h[?25l
[?12l[?25h[?1049l
[?1l>[?2004l]0;admlocal@devOps: ~/leantime_ansibleadmlocal@devOps:~/leantime_ansible$ nano molecule/default/verify.yml 
[?2004h[?1049h[?7h[?1h=[?1h=[?25l\textcolor{inv_background inv_foreground}{[ A ler... ]}\textcolor{inv_background inv_foreground}{[ 302 linhas lidas ]}\textcolor{inv_background inv_foreground}{  GNU nano 4.8                                          molecule/default/verify.yml                                                       }
\textcolor{inv_background inv_foreground}{^G} Ajuda\textcolor{inv_background inv_foreground}{^O} Gravar\textcolor{inv_background inv_foreground}{^W} Procurar\textcolor{inv_background inv_foreground}{^K} Cortar txt    \textcolor{inv_background inv_foreground}{^J} Justificar    \textcolor{inv_background inv_foreground}{^C} Pos cursor    \textcolor{inv_background inv_foreground}{M-U} Desfazer     \textcolor{inv_background inv_foreground}{M-A} Marcar txt
\textcolor{inv_background inv_foreground}{^X} Sair\textcolor{inv_background inv_foreground}{^R} Carregar\textcolor{inv_background inv_foreground}{^\} Substituir    \textcolor{inv_background inv_foreground}{^U} Colar txt     \textcolor{inv_background inv_foreground}{^T} Ortografia    \textcolor{inv_background inv_foreground}{^_} Ir p/ linha   \textcolor{inv_background inv_foreground}{M-E} Refazer\textcolor{inv_background inv_foreground}{M-6} Copiar txt
---
\textcolor{ansi36}{# Tutorial de inspiração para a infraestrutura conduzida por teste no site
# https://www.adictosaltrabajo.com/2020/05/08/ansible-testing-using-molecule-with-ansible-as-verifier/
}- name: "Infraestrutura conduzida por testes"hosts: allgather_facts: falsebecome: truetasks:- name: Variáveisinclude_vars:file: ../../vars/main.yml- name: "Simulação: Atualizar a cache do sistema"apt:update_cache: truecache_valid_time: 3600check_mode: trueregister: estado- name: "teste: a cache encontra-se actualizada?"assert:that:- not estado.changedsuccess_msg: "SUCESSO: A cache está atualizada"fail_msg: "ERRO: Existem actualizações pendentes"- name: "Atualizar o sistema operativo (equivalente a apt upgrade)"apt:upgrade: safecheck_mode: trueregister: estado- name: "teste: o sistema operativo encontra-se atualizado?"
[?12l[?25h[?25l  - name: "teste: o sistema operativo encontra-se atualizado?"
    assert:that: - not estado.changed    success_msg: "SUCESSO: O sistema operativo está atualizado"    fail_msg: "ERRO: Existem actualizações pendentes"
  - name: "Teste: a cache encontra-se atualizada?"  assert:    that:  - not estado.changedsuccess_msg: "SUCESSO: A cache está atualizada"fail_msg: "ERRO: Existem atualizações pendentes"
  - name: "Simulação: testa se as aplicações dependencia do software leantime se encontram instaladas"apt:pkg:  - mc  - screen- git    - apache2  - mysql-server- phpphp-mysql- php-ldap- php-cli- php-soap    - php-json  - graphviz- php-xml  - php-gd  - php-zip- libapache2-mod-php    - php-dev
[?12l[?25h[?25l- libapache2-mod-php    - php-dev  - libmcrypt-dev- gccmake- autoconf- libc-dev- pkg-config    - pwgen  - curl- unzipzip- php-mbstring- expect- net-tools    - python3-mysqldb  - python3-apt- python3-pycurlcheck_mode: trueregister: estado
  - name: "Teste: as dependencias encontra-se instaladas?"assert:that:not estado.changedsuccess_msg: "SUCESSO: As dependencias estavam instaladas"fail_msg: "ERRO: Faltam instalar algumas dependencias. O software pode estar parcialmente instalado"
\textcolor{ansi36}{  # Fonte de inspiração: https://docs.ansible.com/ansible/latest/collections/ansible/builtin/service_module.html}- name: "Simulação: Testar se o serviço apache2 se encontra instalado, iniciado e ativo no arranque"service:name: apache2state: startedenabled: truecheck_mode: true
[?12l[?25h[?25lenabled: trueO serviço apache2 encontra-se ativo no arranque no sistema, e está iniciado?"O serviço apache2 está correctamente instalado e inicia com o arranque do sistema"O serviço apache2 não se encontra correcta\textcolor{ansi36}{: https://www.mydailytutorials.com/using--find-module-search-filesfolder}Rotina Auxiliar> Procura o caminho do ficheiro php.ini do servidor apache2"find:paths: /etcpatterns: "php.ini"recurseregister: caminho_php_ini
\textcolor{ansi36}{  # Fonte: https://docs.ansible.com/ansible/latest/collections/ansible/builtin/lineinfile_module.html}- name: "Simulação: Ativa a opção file_uploads no ficheiro php.ini, utilizando o módulo lineinfile."ansible.builtin.lineinfile:path: "{{ caminho_php_ini.files[0].path }}"regexp: "{{ item.regexp  }}"line: "{{ item.line }}"with_items:- regexp: "^file_uploads"line: "file_uploads = 1"- regexp: "^upload_max_filesize"line: "upload_max_filesize = 1G"- regexp: "^max_file_uploads"line: "max_file_uploads = 20"- regexp: "^post_max_size"line: "post_max_size = 2G"- regexp: "^memory_limit"
[?12l[?25h[?25l78line: "post_max_size = 2G"- regexp: "^memory_limit"line: "memory_limit = 2G"- regexp: "^max_input_time"  line: "max_input_time = 3600"As linhas do php.ini encontram-se alteradas ?"ficheiro php.ini foi alterado com sucesso"Algumas alterações ao ficheiro php.ini não tiveram sucesso"\textcolor{ansi36}{O site deve existir na pasta /var/www/leantime. Este teste visa determinar se a pasta
  # existe e se temas permissões certas
  # Fonte de inspiração: https://docs.ansible.com/ansible/latest/collections/ansible/builtin/stat_module.html}- name: "Simulação: Obtem informação sobre a pastas /var/www/leantime"
    stat:    path: "/var/www/leantime"register: pasta
  - name: "Teste: A pasta /var/www/leantime existe e tem as permissões certas?"assert:  that:  - pasta.stat.exists- pasta.stat.isdir  - pasta.stat.mode == "0755"- pasta.stat.pw_name == "www-data"  - pasta.stat.gr_name == "www-data"success_msg: "SUCESSO: Permissões correctas no site leantime"fail_msg: "ERRO: Permissões incorrectas no site leantime ou site não existente"
\textcolor{ansi36}{  # Outro dos requesitos é a existência de uma base de dados para o site.
}[?12l[?25h[?25l\textcolor{ansi36}{  # Outro dos requesitos é a existência de uma base de dados para o site.
  # Esta base de dados deve ser criada, logo tem de ser testada para determinar se ela existe ou não.
  # Fonte de inspiração: https://serverfault.com/questions/173978/from-a-shell-script-how-can-i-check-whether-a-mysql-database-exists}- name: "Simulação: Verifica se é necessário criar a base de dados leantime_database"shell: "mysql -u root -e 'use leantime_database'"resultfailed_when: false  changed_when: false
  - name: "Teste: A base de dados leantime_database existe ?"assert:that:- result.rc == 0success_msg: "SUCESSO: A base de dados leantime_database já se encontra criada"fail_msg: "ERRO: Não existe a base de dados leantime_database"
\textcolor{ansi36}{  # Teste: O download do software leantime deve ser possível, a partir do site do leantime}- name: "Simulação e Teste: Verifica se o endereço git para a versão de leantime existe"uri:url: https://github.com/Leantime/leantime/releases/download/{{ versao_leantime }}/Leantime-{{ versao_leantime }}.zipregister: resultado\textcolor{ansi36}{Teste: Do ficheiro configuration.php no site do leantime. A existência deste ficheiro       com as permissões certas indica que o processo de unzip correu bemDocumentação oficial em: https://docs.ansible.com/ansible/latest/collections/ansible/builtin/stat_module.html}o ficheiro /var/www/leantime/config/configuration.php"/config/configuration.php"ficheiroO ficheiro /var/www/leantime/config/configuration.php existe e tem as permissões certas?"ficheiro.stat.exists
[?12l[?25h[?25lthat: - ficheiro.stat.exists - ficheiro.stat.isreg - ficheiro.stat.mode == "0755" - ficheiro.stat.pw_name == "www-data"    - ficheiro.stat.gr_name == "www-data"  success_msg: "SUCESSO: Permissões correctas e ficheiro configuration.php existente"  fail_msg: "ERRO: Permissões incorrectas ou ficheiro configuration.php não existente"
\textcolor{ansi36}{  # Teste: O ficheiro /var/www/leantime/config/configuration.php necessita de ser alterado.
  # As variáveis dbuser, dbDatabase, e dbPassword necessitam de estar configuradas.
  # Por simulação o sistema vai determinar se estas estão configuradas com o valor esperado.
  # Fonte: https://docs.ansible.com/ansible/latest/collections/ansible/builtin/replace_module.html}- name: "Simulação: Testa se o dbuser foi alterado no ficheiro configuration.php utilizando o módulo replace."ansible.builtin.replace:path: /var/www/leantime/config/configuration.phpregexp: 'dbUser=""' replace: 'dbUser="{{ login_leantime }}"'  check_mode: yesregister: conffailed_when: false
  - name: "Teste: O dbUser foi alterado?"
    assert: that: - conf is not changed    - conf is not failed  success_msg: "SUCESSO: O dbUser foi alterado no ficheiro configuration.php"fail_msg: "ERRO: O dbUser não foi alterado no ficheiro configuration.php"
\textcolor{ansi36}{  # Fonte: https://docs.ansible.com/ansible/latest/collections/ansible/builtin/replace_module.html}- name: "Simulação: Testa se o dbDatabase foi alterado no ficheiro configuration.php utilizando o módulo replace."ansible.builtin.replace:path: /var/www/leantime/config/configuration.php
[?12l[?25h[?25l    ansible.builtin.replace:path: /var/www/leantime/config/configuration.phpregexp: 'dbDatabase=""'replace: 'dbDatabase="leantime_database"'check_mode: yesregister: conffailed_when: false
  - name: "Teste: a variável dbDatabase foi alterada?"
    assert: that: - conf is not changed - conf is not failed    success_msg: "SUCESSO: O dbDatabase foi alterado no ficheiro configuration.php"  fail_msg: "ERRO: O dbDatabase não foi alterado no ficheiro configuration.php"
\textcolor{ansi36}{  # Fonte: https://docs.ansible.com/ansible/latest/collections/ansible/builtin/replace_module.html}- name: "Simulação: Testa se o dbPassword foi alterado no ficheiro configuration.php utilizando o módulo replace."ansible.builtin.replace:  path: /var/www/leantime/config/configuration.php  regexp: 'dbPassword=""'replace: 'dbPassword="{{ password_leantime }}"'  check_mode: yesregister: conffailed_when: false
  - name: "Teste: O dbPassword foi alterado?"assert:that:- conf is not changed- conf is not failed success_msg: "SUCESSO: O dbPassword foi alterado no ficheiro configuration.php"    fail_msg: "ERRO: O dbPassword não foi alterado no ficheiro configuration.php"
\textcolor{ansi36}{  # A configuração do site leantime necessita que seja criado um site através de um ficheiro de
}[?12l[?25h[?25l\textcolor{ansi36}{  # A configuração do site leantime necessita que seja criado um site através de um ficheiro de
  # configuração, dentro das pasta do apache2. Deve existir um ficheiro de configuração do
  # leantime em (/etc/apache2/sites-available}
\textcolor{ansi36}{  # Fonte: https://docs.ansible.com/ansible/latest/collections/ansible/builtin/stat_module.html}- name: "Simulação: Obtem informação sobre o ficheiro /etc/apache2/sites-available/leantime.conf"stat:path: "/etc/apache2/sites-available/leantime.conf"  register: ficheiro
  - name: "Teste: O ficheiro /etc/apache2/sites-available/leantime.conf existe e tem as permissões certas?"assert:that:  - ficheiro.stat.exists  - ficheiro.stat.isreg- ficheiro.stat.mode == "0755" - ficheiro.stat.pw_name == "root" - ficheiro.stat.gr_name == "www-data"  success_msg: "SUCESSO: Permissões correctas"fail_msg: "ERRO: Permissões incorrectas"
\textcolor{ansi36}{  # O site deve estar ativo no apache. Este teste vai verificar se o site se encontra ativo
  # https://docs.ansible.com/ansible/latest/collections/ansible/builtin/stat_module.html}- name: "Simulação: Obtem informação sobre o ficheiro /etc/apache2/sites-enabled/leantime.conf"stat:path: "/etc/apache2/sites-enabled/leantime.conf"  register: ficheiro
  - name: "Teste: O ficheiro /etc/apache2/sites-enabled/leantime.conf existe e tem as permissões certas?"assert:that:  - ficheiro.stat.exists  - ficheiro.stat.islnk- ficheiro.stat.mode == "0777" - ficheiro.stat.pw_name == "root"
[?12l[?25h[?25l77root"
[?12l[?25h[?25l78M- ficheiro.stat.islnk
[?12l[?25h[?25l78M- ficheiro.stat.exists
[?12l[?25h[?25l78Mthat:
[?12l[?25h[?25l78Massert:
[?12l[?25h[?25l78M- name: "Teste: O ficheiro /etc/apache2/sites-enabled/leantime.conf existe e tem as permissões certas?"
[?12l[?25h[?25l78M[?12l[?25h[?25l78Mregister: ficheiro
[?12l[?25h[?25l78Mpath: "/etc/apache2/sites-enabled/leantime.conf"
[?12l[?25h[?25l78Mstat:
[?12l[?25h[?25l78M- name: "Simulação: Obtem informação sobre o ficheiro /etc/apache2/sites-enabled/leantime.conf"
[?12l[?25h[?25l78M\textcolor{ansi36}{  # https://docs.ansible.com/ansible/latest/collections/ansible/builtin/stat_module.html
}[?12l[?25h[?25l78M\textcolor{ansi36}{  # O site deve estar ativo no apache. Este teste vai verificar se o site se encontra ativo
}[?12l[?25h[?25l78M[?12l[?25h[?25l78Mfail_msg: "ERRO: Permissões incorrectas"
[?12l[?25h[?25l78Msuccess_msg: "SUCESSO: Permissões correctas"
[?12l[?25h[?25l78M- ficheiro.stat.gr_name == "www-data"
[?12l[?25h[?25l78M- ficheiro.stat.pw_name == "root"
[?12l[?25h[?25l78M- ficheiro.stat.mode == "0755"
[?12l[?25h[?25l78M- ficheiro.stat.isreg
[?12l[?25h[?25l78M- ficheiro.stat.exists
[?12l[?25h[?25l78Mthat:
[?12l[?25h[?25l78Massert:
[?12l[?25h[?25l78M- name: "Teste: O ficheiro /etc/apache2/sites-available/leantime.conf existe e tem as permissões certas?"
[?12l[?25h[?25l78M[?12l[?25h[?25l78Mregister: ficheiro
[?12l[?25h[?25l78Mpath: "/etc/apache2/sites-available/leantime.conf"
[?12l[?25h[?25l[?12l[?25h[?25l[?12l[?25h[?25l[?12l[?25h[?25l[?12l[?25h[?25l[?12l[?25h[?25l[?12l[?25h[?25l[?12l[?25h[?25l[?12l[?25h[?25l[?12l[?25h[?25l[?12l[?25h[?25l\textcolor{inv_background inv_foreground}{Modificado}"[?12l[?25h[?25l"[?12l[?25h[?25l77"[?12l[?25h[?25l777"[?12l[?25h[?25l         \textcolor{inv_background inv_foreground}{M-D} Formato DOS\textcolor{inv_background inv_foreground}{M-A} Anexar\textcolor{inv_background inv_foreground}{B} Segurança\textcolor{inv_background inv_foreground}{C} Cancelar           \textcolor{inv_background inv_foreground}{M-M} Formato Mac\textcolor{inv_background inv_foreground}{M-P} Prepor\textcolor{inv_background inv_foreground}{^T} P/ ficheiros
\textcolor{inv_background inv_foreground}{Nome do ficheiro onde escrever: molecule/default/verify.yml                                                                               }[?12l[?25h[?25l \textcolor{inv_background inv_foreground}{[ A escrever... ]}\textcolor{inv_background inv_foreground}{          }\textcolor{inv_background inv_foreground}{[ 302 linhas escritas ]}\textcolor{inv_background inv_foreground}{^O} Gravar\textcolor{inv_background inv_foreground}{^W} Procurar      \textcolor{inv_background inv_foreground}{^K} Cortar txt    \textcolor{inv_background inv_foreground}{^J} Justificar    \textcolor{inv_background inv_foreground}{^C} Pos cursor\textcolor{inv_background inv_foreground}{U} Desfazer     \textcolor{inv_background inv_foreground}{M-A} Marcar txt\textcolor{inv_background inv_foreground}{X} Sair    \textcolor{inv_background inv_foreground}{^R} Carregar\textcolor{inv_background inv_foreground}{^\} Substituir    \textcolor{inv_background inv_foreground}{^U} Colar txt     \textcolor{inv_background inv_foreground}{^T} Ortografia    \textcolor{inv_background inv_foreground}{^_} Ir p/ linha   \textcolor{inv_background inv_foreground}{M-E} Refazer      \textcolor{inv_background inv_foreground}{M-6} Copiar txt
[?12l[?25h[?25l
[?12l[?25h[?1049l
[?1l>[?2004l]0;admlocal@devOps: ~/leantime_ansibleadmlocal@devOps:~/leantime_ansible$ nano molecule/default/verify.yml molecule loginverify
\textcolor{ansi34}{INFO    } default scenario test matrix: verify
\textcolor{ansi34}{INFO    } Performing prerun\textcolor{ansi33}{...}
\textcolor{ansi34}{INFO    } Using .cache/roles/nunomourinho.leantime_ansible symlink to current repository in order to enable Ansible to find the role using its expected full name.
\textcolor{ansi34}{INFO    } Added \textcolor{ansi33}{ANSIBLE_ROLES_PATH}=~\textcolor{ansi35}{/.ansible/}\textcolor{ansi95}{roles}:\textcolor{ansi35}{/usr/share/ansible/}\textcolor{ansi95}{roles}:\textcolor{ansi35}{/etc/ansible/}\textcolor{ansi95}{roles}:.\textcolor{ansi35}{/.cache/}\textcolor{ansi95}{roles}
\textcolor{ansi34}{INFO    } \textcolor{ansi2 ansi36}{Running }\textcolor{ansi2 ansi32}{default}\textcolor{ansi2 ansi36}{ > }\textcolor{ansi2 ansi32}{verify}
\textcolor{ansi34}{INFO    } Running Ansible Verifier

PLAY [Infraestrutura conduzida por testes] ***********************************************************************************************

TASK [Variáveis] *************************************************************************************************************************
\textcolor{ansi32}{ok: [ubuntu-20.04]}

TASK [Simulação: Atualizar a cache do sistema] *******************************************************************************************
\textcolor{ansi32}{ok: [ubuntu-20.04]}

TASK [teste: a cache encontra-se actualizada?] *******************************************************************************************
\textcolor{ansi32}{ok: [ubuntu-20.04] => {}
\textcolor{ansi32}{    "changed": false,}
\textcolor{ansi32}{    "msg": "SUCESSO: A cache está atualizada"}
\textcolor{ansi32}{}}

TASK [Atualizar o sistema operativo (equivalente a apt upgrade)] *************************************************************************
\textcolor{ansi32}{ok: [ubuntu-20.04]}

TASK [teste: o sistema operativo encontra-se atualizado?] ********************************************************************************
\textcolor{ansi32}{ok: [ubuntu-20.04] => {}
\textcolor{ansi32}{    "changed": false,}
\textcolor{ansi32}{    "msg": "SUCESSO: O sistema operativo está atualizado"}
\textcolor{ansi32}{}}

TASK [Teste: a cache encontra-se atualizada?] ********************************************************************************************
\textcolor{ansi32}{ok: [ubuntu-20.04] => {}
\textcolor{ansi32}{    "changed": false,}
\textcolor{ansi32}{    "msg": "SUCESSO: A cache está atualizada"}
\textcolor{ansi32}{}}

TASK [Simulação: testa se as aplicações dependencia do software leantime se encontram instaladas] ****************************************
\textcolor{ansi32}{ok: [ubuntu-20.04]}

TASK [Teste: as dependencias encontra-se instaladas?] ************************************************************************************
\textcolor{ansi32}{ok: [ubuntu-20.04] => {}
\textcolor{ansi32}{    "changed": false,}
\textcolor{ansi32}{    "msg": "SUCESSO: As dependencias estavam instaladas"}
\textcolor{ansi32}{}}

TASK [Simulação: Testar se o serviço apache2 se encontra instalado, iniciado e ativo no arranque] ****************************************
\textcolor{ansi32}{ok: [ubuntu-20.04]}

TASK [Teste: O serviço apache2 encontra-se ativo no arranque no sistema, e está iniciado?] ***********************************************
\textcolor{ansi32}{ok: [ubuntu-20.04] => {}
\textcolor{ansi32}{    "changed": false,}
\textcolor{ansi32}{    "msg": "SUCESSO: O serviço apache2 está correctamente instalado e inicia com o arranque do sistema"}
\textcolor{ansi32}{}}

TASK [Rotina Auxiliar> Procura o caminho do ficheiro php.ini do servidor apache2] ********************************************************
\textcolor{ansi32}{ok: [ubuntu-20.04]}

TASK [Simulação: Ativa a opção file_uploads no ficheiro php.ini, utilizando o módulo lineinfile.] ****************************************
\textcolor{ansi32}{ok: [ubuntu-20.04] => (item={'regexp': '^file_uploads', 'line': 'file_uploads = 1'})}
\textcolor{ansi32}{ok: [ubuntu-20.04] => (item={'regexp': '^upload_max_filesize', 'line': 'upload_max_filesize = 1G'})}
\textcolor{ansi32}{ok: [ubuntu-20.04] => (item={'regexp': '^max_file_uploads', 'line': 'max_file_uploads = 20'})}
\textcolor{ansi32}{ok: [ubuntu-20.04] => (item={'regexp': '^post_max_size', 'line': 'post_max_size = 2G'})}
\textcolor{ansi32}{ok: [ubuntu-20.04] => (item={'regexp': '^memory_limit', 'line': 'memory_limit = 2G'})}
\textcolor{ansi32}{ok: [ubuntu-20.04] => (item={'regexp': '^max_input_time', 'line': 'max_input_time = 3600'})}

TASK [Teste: As linhas do php.ini encontram-se alteradas ?] ******************************************************************************
\textcolor{ansi32}{ok: [ubuntu-20.04] => {}
\textcolor{ansi32}{    "changed": false,}
\textcolor{ansi32}{    "msg": "SUCESSO: O ficheiro php.ini foi alterado com sucesso"}
\textcolor{ansi32}{}}

TASK [Simulação: Obtem informação sobre a pastas /var/www/leantime] **********************************************************************
\textcolor{ansi32}{ok: [ubuntu-20.04]}

TASK [Teste: A pasta /var/www/leantime existe e tem as permissões certas?] ***************************************************************
\textcolor{ansi32}{ok: [ubuntu-20.04] => {}
\textcolor{ansi32}{    "changed": false,}
\textcolor{ansi32}{    "msg": "SUCESSO: Permissões correctas no site leantime"}
\textcolor{ansi32}{}}

TASK [Simulação: Verifica se é necessário criar a base de dados leantime_database] *******************************************************
\textcolor{ansi32}{ok: [ubuntu-20.04]}

TASK [Teste: A base de dados leantime_database existe ?] *********************************************************************************
\textcolor{ansi32}{ok: [ubuntu-20.04] => {}
\textcolor{ansi32}{    "changed": false,}
\textcolor{ansi32}{    "msg": "SUCESSO: A base de dados leantime_database já se encontra criada"}
\textcolor{ansi32}{}}

TASK [Simulação e Teste: Verifica se o endereço git para a versão de leantime existe] ****************************************************
\textcolor{ansi32}{ok: [ubuntu-20.04]}

TASK [Simulação: Obtem informação sobre o ficheiro /var/www/leantime/config/configuration.php] *******************************************
\textcolor{ansi32}{ok: [ubuntu-20.04]}

TASK [Teste: O ficheiro /var/www/leantime/config/configuration.php existe e tem as permissões certas?] ***********************************
\textcolor{ansi32}{ok: [ubuntu-20.04] => {}
\textcolor{ansi32}{    "changed": false,}
\textcolor{ansi32}{    "msg": "SUCESSO: Permissões correctas e ficheiro configuration.php existente"}
\textcolor{ansi32}{}}

TASK [Simulação: Testa se o dbuser foi alterado no ficheiro configuration.php utilizando o módulo replace.] ******************************
\textcolor{ansi32}{ok: [ubuntu-20.04]}

TASK [Teste: O dbUser foi alterado?] *****************************************************************************************************
\textcolor{ansi32}{ok: [ubuntu-20.04] => {}
\textcolor{ansi32}{    "changed": false,}
\textcolor{ansi32}{    "msg": "SUCESSO: O dbUser foi alterado no ficheiro configuration.php"}
\textcolor{ansi32}{}}

TASK [Simulação: Testa se o dbDatabase foi alterado no ficheiro configuration.php utilizando o módulo replace.] **************************
\textcolor{ansi32}{ok: [ubuntu-20.04]}

TASK [Teste: a variável dbDatabase foi alterada?] ****************************************************************************************
\textcolor{ansi32}{ok: [ubuntu-20.04] => {}
\textcolor{ansi32}{    "changed": false,}
\textcolor{ansi32}{    "msg": "SUCESSO: O dbDatabase foi alterado no ficheiro configuration.php"}
\textcolor{ansi32}{}}

TASK [Simulação: Testa se o dbPassword foi alterado no ficheiro configuration.php utilizando o módulo replace.] **************************
\textcolor{ansi32}{ok: [ubuntu-20.04]}

TASK [Teste: O dbPassword foi alterado?] *************************************************************************************************
\textcolor{ansi32}{ok: [ubuntu-20.04] => {}
\textcolor{ansi32}{    "changed": false,}
\textcolor{ansi32}{    "msg": "SUCESSO: O dbPassword foi alterado no ficheiro configuration.php"}
\textcolor{ansi32}{}}

TASK [Simulação: Obtem informação sobre o ficheiro /etc/apache2/sites-available/leantime.conf] *******************************************
\textcolor{ansi32}{ok: [ubuntu-20.04]}

TASK [Teste: O ficheiro /etc/apache2/sites-available/leantime.conf existe e tem as permissões certas?] ***********************************
\textcolor{ansi32}{ok: [ubuntu-20.04] => {}
\textcolor{ansi32}{    "changed": false,}
\textcolor{ansi32}{    "msg": "SUCESSO: Permissões correctas"}
\textcolor{ansi32}{}}

TASK [Simulação: Obtem informação sobre o ficheiro /etc/apache2/sites-enabled/leantime.conf] *********************************************
\textcolor{ansi32}{ok: [ubuntu-20.04]}

TASK [Teste: O ficheiro /etc/apache2/sites-enabled/leantime.conf existe e tem as permissões certas?] *************************************
\textcolor{ansi32}{ok: [ubuntu-20.04] => {}
\textcolor{ansi32}{    "changed": false,}
\textcolor{ansi32}{    "msg": "SUCESSO: Permissões correctas"}
\textcolor{ansi32}{}}

PLAY RECAP *******************************************************************************************************************************
\textcolor{ansi32}{ubuntu-20.04}               : \textcolor{ansi32}{ok=30  } changed=0    unreachable=0    failed=0    skipped=0    rescued=0    ignored=0

\textcolor{ansi34}{INFO    } Verifier completed successfully.
]0;admlocal@devOps: ~/leantime_ansibleadmlocal@devOps:~/leantime_ansible$ molecule idempotence
\textcolor{ansi34}{INFO    } default scenario test matrix: idempotence
\textcolor{ansi34}{INFO    } Performing prerun\textcolor{ansi33}{...}
\textcolor{ansi34}{INFO    } Using .cache/roles/nunomourinho.leantime_ansible symlink to current repository in order to enable Ansible to find the role using its expected full name.
\textcolor{ansi34}{INFO    } Added \textcolor{ansi33}{ANSIBLE_ROLES_PATH}=~\textcolor{ansi35}{/.ansible/}\textcolor{ansi95}{roles}:\textcolor{ansi35}{/usr/share/ansible/}\textcolor{ansi95}{roles}:\textcolor{ansi35}{/etc/ansible/}\textcolor{ansi95}{roles}:.\textcolor{ansi35}{/.cache/}\textcolor{ansi95}{roles}
\textcolor{ansi34}{INFO    } \textcolor{ansi2 ansi36}{Running }\textcolor{ansi2 ansi32}{default}\textcolor{ansi2 ansi36}{ > }\textcolor{ansi2 ansi32}{idempotence}

PLAY [Converge] **************************************************************************************************************************

TASK [Gathering Facts] *******************************************************************************************************************
\textcolor{ansi32}{ok: [ubuntu-20.04]}

TASK [Include leantime_ansible] **********************************************************************************************************

TASK [leantime_ansible : Atualiza a cache (equivalente a apt update)] ********************************************************************
\textcolor{ansi32}{ok: [ubuntu-20.04]}

TASK [leantime_ansible : Atualiza o sistema operativo] ***********************************************************************************
\textcolor{ansi32}{ok: [ubuntu-20.04]}

TASK [leantime_ansible : Instalar as dependencias necessária ao programa leantime] *******************************************************
\textcolor{ansi32}{ok: [ubuntu-20.04]}

TASK [leantime_ansible : Instala o serviço apache2 no arranque do sistema] ***************************************************************
\textcolor{ansi32}{ok: [ubuntu-20.04]}

TASK [leantime_ansible : Rotina Auxiliar: Procura o caminho do ficheiro php.ini do servidor apache2] *************************************
\textcolor{ansi32}{ok: [ubuntu-20.04]}

TASK [leantime_ansible : Ativa várias opções no ficheiro php.ini, utilizando o módulo lineinfile.] ***************************************
\textcolor{ansi32}{ok: [ubuntu-20.04] => (item={'regexp': '^file_uploads', 'line': 'file_uploads = 1'})}
\textcolor{ansi32}{ok: [ubuntu-20.04] => (item={'regexp': '^upload_max_filesize', 'line': 'upload_max_filesize = 1G'})}
\textcolor{ansi32}{ok: [ubuntu-20.04] => (item={'regexp': '^max_file_uploads', 'line': 'max_file_uploads = 20'})}
\textcolor{ansi32}{ok: [ubuntu-20.04] => (item={'regexp': '^post_max_size', 'line': 'post_max_size = 2G'})}
\textcolor{ansi32}{ok: [ubuntu-20.04] => (item={'regexp': '^memory_limit', 'line': 'memory_limit = 2G'})}
\textcolor{ansi32}{ok: [ubuntu-20.04] => (item={'regexp': '^max_input_time', 'line': 'max_input_time = 3600'})}

TASK [leantime_ansible : Cria a directoria temporária leantime e a directoria de apache leantime] ****************************************
\textcolor{ansi32}{ok: [ubuntu-20.04] => (item={'path': '/tmp/leantime'})}
\textcolor{ansi32}{ok: [ubuntu-20.04] => (item={'path': '/var/www/leantime'})}

TASK [leantime_ansible : Cria a base de dados de mysql para o leantime] ******************************************************************
\textcolor{ansi32}{ok: [ubuntu-20.04]}

TASK [leantime_ansible : Cria o utilizador para a base de dados leantime] ****************************************************************
\textbf{\textcolor{ansi35}{[WARNING]: Module did not set no_log for update_password}}
\textcolor{ansi32}{ok: [ubuntu-20.04]}

TASK [leantime_ansible : Faz o download do software leantime a partir do repositório git] ************************************************
\textcolor{ansi32}{ok: [ubuntu-20.04]}

TASK [leantime_ansible : Descomprime o software leantime] ********************************************************************************
\textcolor{ansi32}{ok: [ubuntu-20.04]}

TASK [leantime_ansible : Copia o ficheiro de configuração configuration.sample.php] ******************************************************
\textcolor{ansi32}{ok: [ubuntu-20.04]}

TASK [leantime_ansible : Altera as variáveis dbuser, dbpassword e dbdatabase no ficheiro configuration.php utilizando o módulo replace.] ***
\textcolor{ansi32}{ok: [ubuntu-20.04] => (item={'regexp': 'dbUser=""', 'line': 'dbUser="leantimeDBadmin"'})}
\textcolor{ansi32}{ok: [ubuntu-20.04] => (item={'regexp': 'dbDatabase=""', 'line': 'dbDatabase="leantime_database"'})}
\textcolor{ansi32}{ok: [ubuntu-20.04] => (item={'regexp': 'dbPassword=""', 'line': 'dbPassword="#S3gr3d0S3cr3t0#"'})}

TASK [leantime_ansible : Copia o template do site para a pásta /etc/apache2/sites-available] *********************************************
\textcolor{ansi32}{ok: [ubuntu-20.04]}

TASK [leantime_ansible : Cria um link simbólico para o ficheiro leantime.conf na pasta /etc/apache/sites-enabled] ************************
\textcolor{ansi32}{ok: [ubuntu-20.04]}

PLAY RECAP *******************************************************************************************************************************
\textcolor{ansi32}{ubuntu-20.04}               : \textcolor{ansi32}{ok=16  } changed=0    unreachable=0    failed=0    skipped=0    rescued=0    ignored=0


\textcolor{ansi34}{INFO    } Idempotence completed successfully.
]0;admlocal@devOps: ~/leantime_ansibleadmlocal@devOps:~/leantime_ansible$ exit
exit

\end{Verbatim}
\end{document}
