\documentclass{scrartcl}
\usepackage[utf8]{inputenc}
\usepackage{fancyvrb}
\usepackage[usenames,dvipsnames]{xcolor}
% \definecolor{red-sd}{HTML}{7ed2d2}

\title{}

\fvset{commandchars=\\\{\}}

\begin{document}

\begin{Verbatim}
]0;admlocal@devOps: ~/leantime_ansibleadmlocal@devOps:~/leantime_ansible$ nano molecule/default/verify.yml 
[?2004h[?1049h[?7h[?1h=[?1h=[?25l\textcolor{inv_background inv_foreground}{[ A ler... ]}\textcolor{inv_background inv_foreground}{[ 177 linhas lidas ]}\textcolor{inv_background inv_foreground}{  GNU nano 4.8                                          molecule/default/verify.yml                                                       }
\textcolor{inv_background inv_foreground}{^G} Ajuda\textcolor{inv_background inv_foreground}{^O} Gravar\textcolor{inv_background inv_foreground}{^W} Procurar\textcolor{inv_background inv_foreground}{^K} Cortar txt    \textcolor{inv_background inv_foreground}{^J} Justificar    \textcolor{inv_background inv_foreground}{^C} Pos cursor    \textcolor{inv_background inv_foreground}{M-U} Desfazer     \textcolor{inv_background inv_foreground}{M-A} Marcar txt
\textcolor{inv_background inv_foreground}{^X} Sair\textcolor{inv_background inv_foreground}{^R} Carregar\textcolor{inv_background inv_foreground}{^\} Substituir    \textcolor{inv_background inv_foreground}{^U} Colar txt     \textcolor{inv_background inv_foreground}{^T} Ortografia    \textcolor{inv_background inv_foreground}{^_} Ir p/ linha   \textcolor{inv_background inv_foreground}{M-E} Refazer\textcolor{inv_background inv_foreground}{M-6} Copiar txt
---
\textcolor{ansi36}{# Tutorial de inspiração para a infraestrutura conduzida por teste no site
# https://www.adictosaltrabajo.com/2020/05/08/ansible-testing-using-molecule-with-ansible-as-verifier/
}- name: "Infraestrutura conduzida por testes"hosts: allgather_facts: falsebecome: truetasks:- name: "Simulação: Atualizar a cache do sistema"apt:update_cache: truecache_valid_time: 3600check_mode: trueregister: estado- name: "teste: a cache encontra-se actualizada?"assert:that:- not estado.changedsuccess_msg: "SUCESSO: A cache está atualizada"fail_msg: "ERRO: Existem actualizações pendentes"- name: "Atualizar o sistema operativo (equivalente a apt upgrade)"apt:upgrade: safecheck_mode: trueregister: estado- name: "teste: o sistema operativo encontra-se atualizado?"assert:that:- not estado.changedsuccess_msg: "SUCESSO: O sistema operativo está atualizado"
[?12l[?25h[?25l - not estado.changed success_msg: "SUCESSO: O sistema operativo está atualizado" fail_msg: "ERRO: Existem actualizações pendentes"
  - name: "Teste: a cache encontra-se atualizada?"  assert:    that:    - not estado.changedsuccess_msg: "SUCESSO: A cache está atualizada"    fail_msg: "ERRO: Existem atualizações pendentes"
  - name: "Simulação: testa se as aplicações dependencia do software leantime se encontram instaladas"apt:pkg:  - mc  - screen- git    - apache2  - mysql-server- phpphp-mysql- php-ldap- php-cli- php-soap    - php-json  - graphviz- php-xml  - php-gd  - php-zip- libapache2-mod-php    - php-dev  - libmcrypt-dev- gccmake- autoconf
[?12l[?25h[?25l78- make- autoconf- libc-dev- pkg-config- pwgen- curl- unzip- zip- php-mbstring- expect- net-tools- python3-mysqldb- python3-aptpython3-pycurlcheck_mode: trueregister: estados dependencias encontra-se instaladas?"s dependencias estavam instaladas"Faltam instalar algumas dependencias. O software pode estar parcialmente instalado"
\textcolor{ansi36}{  # Fonte de inspiração: https://docs.ansible.com/ansible/latest/collections/ansible/builtin/service_module.html}- name: "Simulação: Testar se o serviço apache2 se encontra instalado, iniciado e ativo no arranque"service:name: apache2state: startedenabled: truecheck_mode: trueregister: estado
  - name: "Teste: O serviço apache2 encontra-se ativo no arranque no sistema, e está iniciado?"assert:
[?12l[?25h[?25l  - name: "Teste: O serviço apache2 encontra-se ativo no arranque no sistema, e está iniciado?"assert:that:not estado.changedsuccess_msg: "SUCESSO: O serviço apache2 está correctamente instalado e inicia com o arranque do sistema"fail_msg: "ERRO: O serviço apache2 não se encontra correctamente instalado"
\textcolor{ansi36}{  # Fonte: https://www.mydailytutorials.com/using-ansible-find-module-search-filesfolder}- name: "Rotina Auxiliar> Procura o caminho do ficheiro php.ini do servidor apache2"find:paths: /etcpatterns: "php.ini"recurse: trueregister: caminho_php_ini
\textcolor{ansi36}{  # Fonte: https://docs.ansible.com/ansible/latest/collections/ansible/builtin/lineinfile_module.html}- name: "Simulação: Ativa a opção file_uploads no ficheiro php.ini, utilizando o módulo lineinfile."  ansible.builtin.lineinfile:  path: "{{ caminho_php_ini.files[0].path }}"regexp: "{{ item.regexp  }}"line: "{{ item.line }}"with_items:- regexp: "^file_uploads"line: "file_uploads = 1" - regexp: "^upload_max_filesize" line: "upload_max_filesize = 1G"  - regexp: "^max_file_uploads"  line: "max_file_uploads = 20"- regexp: "^post_max_size"  line: "post_max_size = 2G"  - regexp: "^memory_limit"    line: "memory_limit = 2G"- regexp: "^max_input_time" line: "max_input_time = 3600"check_mode: true
[?12l[?25h[?25l78line: "max_input_time = 3600"check_mode: trueregister: estadoAs linhas do php.ini encontram-se alteradas ?"ficheiro php.ini foi alterado com sucesso"Algumas alterações ao ficheiro php.ini não tiveram sucesso"\textcolor{ansi36}{O site deve existir na pasta /var/www/leantime. Este teste visa determinar se a pasta
  # existe e se temas permissões certas
  # Fonte de inspiração: https://docs.ansible.com/ansible/latest/collections/ansible/builtin/stat_module.html}- name: "Simulação: Obtem informação sobre a pastas /var/www/leantime"
    stat:    path: "/var/www/leantime"register: pasta
  - name: "Teste: A pasta /var/www/leantime existe e tem as permissões certas?"assert:  that:  - pasta.stat.exists- pasta.stat.isdir  - pasta.stat.mode == "0755"- pasta.stat.pw_name == "www-data"  - pasta.stat.gr_name == "www-data"success_msg: "SUCESSO: Permissões correctas no site leantime"fail_msg: "ERRO: Permissões incorrectas no site leantime ou site não existente"
\textcolor{ansi36}{  # Outro dos requesitos é a existência de uma base de dados para o site.
  # Esta base de dados deve ser criada, logo tem de ser testada para determinar se ela existe ou não.
  # Fonte de inspiração: https://serverfault.com/questions/173978/from-a-shell-script-how-can-i-check-whether-a-mysql-database-exists}- name: "Simulação: Verifica se é necessário criar a base de dados leantime_database"shell: "mysql -u root -e 'use leantime_database'"
[?12l[?25h[?25l  - name: "Simulação: Verifica se é necessário criar a base de dados leantime_database"shell: "mysql -u root -e 'use leantime_database'"resultfailed_when: false  changed_when: false
  - name: "Teste: A base de dados leantime_database existe ?"assert:that:- result.rc == 0success_msg: "SUCESSO: A base de dados leantime_database já se encontra criada" fail_msg: "ERRO: Não existe a base de dados leantime_database"
[?12l[?25h[?25l[?12l[?25h[?25l[?12l[?25h[?25l[?12l[?25h[?25l[?12l[?25h[?25l[?12l[?25h[?25l[?12l[?25h[?25l[?12l[?25h[?25l[?12l[?25h[?25l[?12l[?25h[?25l[?12l[?25h[?25l[?12l[?25h[?25l\textcolor{inv_background inv_foreground}{Modificado}
[?12l[?25h[?25l[?12l[?25h[?25l [?12l[?25h[?25l [?12l[?25h[?25l
\textcolor{ansi36}{  #}[?12l[?25h[?25l\textcolor{ansi36}{ }[?12l[?25h[?25l\textcolor{ansi36}{T}[?12l[?25h[?25l\textcolor{ansi36}{e}[?12l[?25h[?25l\textcolor{ansi36}{s}[?12l[?25h[?25l\textcolor{ansi36}{t
}[?12l[?25h[?25l\textcolor{ansi36}{e}[?12l[?25h[?25l\textcolor{ansi36}{:}[?12l[?25h[?25l\textcolor{ansi36}{ }[?12l[?25h[?25l\textcolor{ansi36}{O}[?12l[?25h[?25l\textcolor{ansi36}{ }[?12l[?25h[?25l\textcolor{ansi36}{d}[?12l[?25h[?25l\textcolor{ansi36}{o}[?12l[?25h[?25l\textcolor{ansi36}{w}[?12l[?25h[?25l\textcolor{ansi36}{n}[?12l[?25h[?25l\textcolor{ansi36}{l}[?12l[?25h[?25l\textcolor{ansi36}{o}[?12l[?25h[?25l\textcolor{ansi36}{a}[?12l[?25h[?25l\textcolor{ansi36}{d}[?12l[?25h[?25l\textcolor{ansi36}{ }[?12l[?25h[?25l\textcolor{ansi36}{d}[?12l[?25h[?25l\textcolor{ansi36}{o}[?12l[?25h[?25l\textcolor{ansi36}{ }[?12l[?25h[?25l\textcolor{ansi36}{s}[?12l[?25h[?25l\textcolor{ansi36}{o}[?12l[?25h[?25l\textcolor{ansi36}{f}[?12l[?25h[?25l\textcolor{ansi36}{t}[?12l[?25h[?25l\textcolor{ansi36}{w}[?12l[?25h[?25l\textcolor{ansi36}{a}[?12l[?25h[?25l\textcolor{ansi36}{r}[?12l[?25h[?25l\textcolor{ansi36}{e}[?12l[?25h[?25l\textcolor{ansi36}{ }[?12l[?25h[?25l\textcolor{ansi36}{l}[?12l[?25h[?25l\textcolor{ansi36}{e}[?12l[?25h[?25l\textcolor{ansi36}{a}[?12l[?25h[?25l\textcolor{ansi36}{n}[?12l[?25h[?25l\textcolor{ansi36}{t}[?12l[?25h[?25l\textcolor{ansi36}{i}[?12l[?25h[?25l\textcolor{ansi36}{m}[?12l[?25h[?25l\textcolor{ansi36}{e}[?12l[?25h[?25l\textcolor{ansi36}{ }[?12l[?25h[?25l\textcolor{ansi36}{d}[?12l[?25h[?25l\textcolor{ansi36}{e}[?12l[?25h[?25l\textcolor{ansi36}{v}[?12l[?25h[?25l\textcolor{ansi36}{e}[?12l[?25h[?25l\textcolor{ansi36}{ }[?12l[?25h[?25l\textcolor{ansi36}{s}[?12l[?25h[?25l\textcolor{ansi36}{e}[?12l[?25h[?25l\textcolor{ansi36}{r}[?12l[?25h[?25l\textcolor{ansi36}{ }[?12l[?25h[?25l\textcolor{ansi36}{p}[?12l[?25h[?25l\textcolor{ansi36}{o}[?12l[?25h[?25l\textcolor{ansi36}{s}[?12l[?25h[?25l\textcolor{ansi36}{s}[?12l[?25h[?25l\textcolor{ansi36}{í}[?12l[?25h[?25l\textcolor{ansi36}{v}[?12l[?25h[?25l\textcolor{ansi36}{e}[?12l[?25h[?25l\textcolor{ansi36}{l}[?12l[?25h[?25l\textcolor{ansi36}{,}[?12l[?25h[?25l\textcolor{ansi36}{ }[?12l[?25h[?25l\textcolor{ansi36}{a}[?12l[?25h[?25l\textcolor{ansi36}{ }[?12l[?25h[?25l\textcolor{ansi36}{p}[?12l[?25h[?25l\textcolor{ansi36}{a}[?12l[?25h[?25l\textcolor{ansi36}{r}[?12l[?25h[?25l\textcolor{ansi36}{t}[?12l[?25h[?25l\textcolor{ansi36}{i}[?12l[?25h[?25l\textcolor{ansi36}{r}[?12l[?25h[?25l\textcolor{ansi36}{ }[?12l[?25h[?25l\textcolor{ansi36}{d}[?12l[?25h[?25l\textcolor{ansi36}{o}[?12l[?25h[?25l\textcolor{ansi36}{ }[?12l[?25h[?25l\textcolor{ansi36}{s}[?12l[?25h[?25l\textcolor{ansi36}{i}[?12l[?25h[?25l\textcolor{ansi36}{t}[?12l[?25h[?25l\textcolor{ansi36}{e}[?12l[?25h[?25l\textcolor{ansi36}{ }[?12l[?25h[?25l\textcolor{ansi36}{d}[?12l[?25h[?25l\textcolor{ansi36}{o}[?12l[?25h[?25l\textcolor{ansi36}{ }[?12l[?25h[?25l\textcolor{ansi36}{l}[?12l[?25h[?25l\textcolor{ansi36}{e}[?12l[?25h[?25l\textcolor{ansi36}{a}[?12l[?25h[?25l\textcolor{ansi36}{n}[?12l[?25h[?25l\textcolor{ansi36}{t}[?12l[?25h[?25l\textcolor{ansi36}{i}[?12l[?25h[?25l\textcolor{ansi36}{m}[?12l[?25h[?25l\textcolor{ansi36}{e}[?12l[?25h[?25l
[?12l[?25h[?25l [?12l[?25h[?25l [?12l[?25h[?25l-[?12l[?25h[?25l [?12l[?25h[?25ln[?12l[?25h[?25la[?12l[?25h[?25lm[?12l[?25h[?25le[?12l[?25h[?25l:[?12l[?25h[?25l [?12l[?25h[?25l"[?12l[?25h[?25lS[?12l[?25h[?25li[?12l[?25h[?25lm[?12l[?25h[?25lu[?12l[?25h[?25ll[?12l[?25h[?25la[?12l[?25h[?25lç[?12l[?25h[?25lã[?12l[?25h[?25lo[?12l[?25h[?25l [?12l[?25h[?25le[?12l[?25h[?25l [?12l[?25h[?25lT[?12l[?25h[?25le[?12l[?25h[?25ls[?12l[?25h[?25lt[?12l[?25h[?25le[?12l[?25h[?25l:[?12l[?25h[?25l [?12l[?25h[?25lV[?12l[?25h[?25le[?12l[?25h[?25lr[?12l[?25h[?25li[?12l[?25h[?25lf[?12l[?25h[?25li[?12l[?25h[?25lc[?12l[?25h[?25la[?12l[?25h[?25l [?12l[?25h[?25ls[?12l[?25h[?25le[?12l[?25h[?25l [?12l[?25h[?25lo[?12l[?25h[?25l [?12l[?25h[?25le[?12l[?25h[?25ln[?12l[?25h[?25ld[?12l[?25h[?25le[?12l[?25h[?25lr[?12l[?25h[?25le[?12l[?25h[?25lç[?12l[?25h[?25lo[?12l[?25h[?25l [?12l[?25h[?25lg[?12l[?25h[?25li[?12l[?25h[?25lt[?12l[?25h[?25l [?12l[?25h[?25lp[?12l[?25h[?25la[?12l[?25h[?25lr[?12l[?25h[?25la[?12l[?25h[?25l [?12l[?25h[?25la[?12l[?25h[?25l v[?12l[?25h[?25le[?12l[?25h[?25lr[?12l[?25h[?25ls[?12l[?25h[?25lã[?12l[?25h[?25lo[?12l[?25h[?25l [?12l[?25h[?25ld[?12l[?25h[?25le[?12l[?25h[?25l [?12l[?25h[?25ll[?12l[?25h[?25le[?12l[?25h[?25la[?12l[?25h[?25ln[?12l[?25h[?25lt[?12l[?25h[?25li[?12l[?25h[?25lm[?12l[?25h[?25le[?12l[?25h[?25l [?12l[?25h[?25le[?12l[?25h[?25lx[?12l[?25h[?25li[?12l[?25h[?25ls[?12l[?25h[?25lt[?12l[?25h[?25le[?12l[?25h[?25l"[?12l[?25h[?25l
[?12l[?25h[?25l [?12l[?25h[?25l [?12l[?25h[?25l [?12l[?25h[?25l [?12l[?25h[?25lu[?12l[?25h[?25lr[?12l[?25h[?25li[?12l[?25h[?25l:[?12l[?25h[?25l
[?12l[?25h[?25l [?12l[?25h[?25l [?12l[?25h[?25l [?12l[?25h[?25l [?12l[?25h[?25l [?12l[?25h[?25l [?12l[?25h[?25lu[?12l[?25h[?25lr[?12l[?25h[?25ll[?12l[?25h[?25l:[?12l[?25h[?25l [?12l[?25h[?25lh[?12l[?25h[?25lt[?12l[?25h[?25lt[?12l[?25h[?25lp[?12l[?25h[?25ls[?12l[?25h[?25l:[?12l[?25h[?25l/[?12l[?25h[?25l/[?12l[?25h[?25lg[?12l[?25h[?25li[?12l[?25h[?25lt[?12l[?25h[?25lh[?12l[?25h[?25lu[?12l[?25h[?25lb[?12l[?25h[?25l.[?12l[?25h[?25lc[?12l[?25h[?25lo[?12l[?25h[?25lm[?12l[?25h[?25l/[?12l[?25h[?25lL[?12l[?25h[?25le[?12l[?25h[?25la[?12l[?25h[?25ln[?12l[?25h[?25lt[?12l[?25h[?25li[?12l[?25h[?25lm[?12l[?25h[?25le[?12l[?25h[?25l/[?12l[?25h[?25ll[?12l[?25h[?25le[?12l[?25h[?25la[?12l[?25h[?25ln[?12l[?25h[?25lt[?12l[?25h[?25li[?12l[?25h[?25lm[?12l[?25h[?25le[?12l[?25h[?25l/[?12l[?25h[?25lr[?12l[?25h[?25le[?12l[?25h[?25ll[?12l[?25h[?25le[?12l[?25h[?25la[?12l[?25h[?25ls[?12l[?25h[?25le[?12l[?25h[?25ls[?12l[?25h[?25l/[?12l[?25h[?25ld[?12l[?25h[?25lo[?12l[?25h[?25lw[?12l[?25h[?25ln[?12l[?25h[?25ll[?12l[?25h[?25lo[?12l[?25h[?25la[?12l[?25h[?25ld[?12l[?25h[?25l/[?12l[?25h[?25l{[?12l[?25h[?25l{[?12l[?25h[?25l [?12l[?25h[?25lv[?12l[?25h[?25le[?12l[?25h[?25lr[?12l[?25h[?25ls[?12l[?25h[?25la[?12l[?25h[?25lo[?12l[?25h[?25l_[?12l[?25h[?25ll[?12l[?25h[?25le[?12l[?25h[?25la[?12l[?25h[?25ln[?12l[?25h[?25lt[?12l[?25h[?25li[?12l[?25h[?25lm[?12l[?25h[?25le[?12l[?25h[?25l [?12l[?25h[?25l}[?12l[?25h[?25l}[?12l[?25h[?25l/[?12l[?25h[?25lL[?12l[?25h[?25le[?12l[?25h[?25la[?12l[?25h[?25ln[?12l[?25h[?25lt[?12l[?25h[?25li[?12l[?25h[?25lm[?12l[?25h[?25le[?12l[?25h[?25l-[?12l[?25h[?25l{[?12l[?25h[?25l{[?12l[?25h[?25l [?12l[?25h[?25lv[?12l[?25h[?25le[?12l[?25h[?25lr[?12l[?25h[?25ls[?12l[?25h[?25la[?12l[?25h[?25lo[?12l[?25h[?25l_[?12l[?25h[?25ll[?12l[?25h[?25le[?12l[?25h[?25la[?12l[?25h[?25ln[?12l[?25h[?25lt[?12l[?25h[?25li[?12l[?25h[?25lm[?12l[?25h[?25le[?12l[?25h[?25l [?12l[?25h[?25l}[?12l[?25h[?25l}[?12l[?25h[?25l.[?12l[?25h[?25lz[?12l[?25h[?25li[?12l[?25h[?25lp[?12l[?25h[?25l
[?12l[?25h[?25l [?12l[?25h[?25l [?12l[?25h[?25l [?12l[?25h[?25l [?12l[?25h[?25lr[?12l[?25h[?25le[?12l[?25h[?25lg[?12l[?25h[?25li[?12l[?25h[?25ls[?12l[?25h[?25lt[?12l[?25h[?25le[?12l[?25h[?25lr[?12l[?25h[?25l:[?12l[?25h[?25l [?12l[?25h[?25lr[?12l[?25h[?25le[?12l[?25h[?25ls[?12l[?25h[?25lu[?12l[?25h[?25ll[?12l[?25h[?25lt[?12l[?25h[?25la[?12l[?25h[?25ld[?12l[?25h[?25lo[?12l[?25h[?25l
  [?12l[?25h[?25l
[?12l[?25h[?25l
[?12l[?25h[?25l
[?12l[?25h[?25l[?12l[?25h[?25l line: "max_input_time = 3600"check_mode: trueestado
  - name: "Teste: As linhas do php.ini encontram-se alteradas ?"assert:    that:  - not estado.changedsuccess_msg: "SUCESSO: O ficheiro php.ini foi alterado com sucesso"fail_msg: "ERRO: Algumas alterações ao ficheiro php.ini não tiveram sucesso"
\textcolor{ansi36}{  # O site deve existir na pasta /var/www/leantime. Este teste visa determinar se a pasta
  # existe e se temas permissões certasFonte de inspiração: https://docs.ansible.com/ansible/latest/collections/ansible/builtin/stat_module.html}: Obtem informação sobre a pastas /var/www/leantime"stat:path: "/var/www/leantime"pasta- name: "Teste: A pasta /var/www/leantime existe e tem as permissões certas?"assert:that:- pasta.stat.exists- pasta.stat.isdir- pasta.stat.mode == "0755"- pasta.stat.pw_name == "www-data"- pasta.stat.gr_name == "www-data"success_msg: "SUCESSO: Permissões correctas no site leantime"fail_msg: "ERRO: Permissões incorrectas no site leantime ou site não existente"
\textcolor{ansi36}{  # Outro dos requesitos é a existência de uma base de dados para o site.
  # Esta base de dados deve ser criada, logo tem de ser testada para determinar se ela existe ou não.
  # Fonte de inspiração: https://serverfault.com/questions/173978/from-a-shell-script-how-can-i-check-whether-a-mysql-database-exists}- name: "Simulação: Verifica se é necessário criar a base de dados leantime_database"shell: "mysql -u root -e 'use leantime_database'"[?12l[?25h[?25lO serviço apache2 encontra-se ativo no arranque no sistema, e está iniciado?"serviço apache2 está correctamente instalado e inicia com o arranque do sistema"O serviço apache2 não se encontra correctamente instalado"\textcolor{ansi36}{Fonte: https://www.mydailytutorials.com/using-ansible-find-module-search-filesfolder}
  - name: "Rotina Auxiliar> Procura o caminho do ficheiro php.ini do servidor apache2"find:paths: /etcpatterns: "php.ini"recurse: true
    register: caminho_php_ini
\textcolor{ansi36}{  # Fonte: https://docs.ansible.com/ansible/latest/collections/ansible/builtin/lineinfile_module.html}- name: "Simulação: Ativa a opção file_uploads no ficheiro php.ini, utilizando o módulo lineinfile."ansible.builtin.lineinfile:path: "{{ caminho_php_ini.files[0].path }}"    regexp: "{{ item.regexp  }}"  line: "{{ item.line }}"with_items:- regexp: "^file_uploads"line: "file_uploads = 1"- regexp: "^upload_max_filesize"line: "upload_max_filesize = 1G"- regexp: "^max_file_uploads"  line: "max_file_uploads = 20"- regexp: "^post_max_size"line: "post_max_size = 2G" - regexp: "^memory_limit" line: "memory_limit = 2G" - regexp: "^max_input_time" line: "max_input_time = 3600"check_mode: true[?12l[?25h[?25l- make- autoconf- libc-dev- pkg-config- pwgen- curl- unzip- zip- php-mbstring- expect- net-tools- python3-mysqldb- python3-apt- python3-pycurlcheck_mode: trueregister: estadoas dependencias encontra-se instaladas?"As dependencias estavam instaladas"Faltam instalar algumas dependencias. O software pode estar parcialmente instalado"\textcolor{ansi36}{ de inspiração: https://docs.ansible.com//latest/collections/ansible/builtin/service_module.html}Simulação: Testar se o serviço apache2 se encontra instalado, iniciado e ativo no arranque"service:name: apache2state: startedenabledcheck_mode: trueregister: estado
Teste: O serviço apache2 encontra-se ativo no arranque no sistema, e está iniciado?"assert:[?12l[?25h[?25lnot estado.changedsuccess_msg: "SUCESSO: O sistema operativo está atualizado"fail_msg: "ERRO: Existem actualizações pendentes"
  - name: "Teste: a cache encontra-se atualizada?"assert:that:not estado.changedsuccess_msg: "SUCESSO: A cache está atualizada"fail_msg: "ERRO: Existem atualizações pendentes"
  - name: "Simulação: testa se as aplicações dependencia do software leantime se encontram instaladas"apt:pkg:  - mc  - screen- git    - apache2  - mysql-server- phpphp-mysql- php-ldap- php-cli- php-soap - php-json    - graphviz  - php-xml- php-gd- php-zip- libapache2-mod-php  - php-dev  - libmcrypt-dev- gcc    - make  - autoconf[?12l[?25h[?25l
---
\textcolor{ansi36}{# Tutorial de inspiração para a infraestrutura conduzida por teste no site
# https://www.adictosaltrabajo.com/2020/05/08/ansible-testing-using-molecule-with-ansible-as-verifier/
}- name: "Infraestrutura conduzida por testes"hosts: allgather_facts: falsebecome: true
  tasks:- name: "Simulação: Atualizar a cache do sistema"  apt:  update_cache: truecache_valid_time: 3600check_mode: trueregister: estado
  - name: "teste: a cache encontra-se actualizada?"assert:that:not estado.changedsuccess_msg: "SUCESSO: A cache está atualizada"fail_msg: "ERRO: Existem actualizações pendentes"
  - name: "Atualizar o sistema operativo (equivalente a apt upgrade)"apt:upgrade: safecheck_mode: trueregister: estado
  - name: "teste: o sistema operativo encontra-se atualizado?"assert:that:not estado.changedsuccess_msg: "SUCESSO: O sistema operativo está atualizado"[?12l[?25h[?25l
[?12l[?25h[?25l-[?12l[?25h[?25l [?12l[?25h[?25ln[?12l[?25h[?25la[?12l[?25h[?25lm[?12l[?25h[?25le[?12l[?25h[?25l:[?12l[?25h[?25l [?12l[?25h[?25lV[?12l[?25h[?25la[?12l[?25h[?25lr[?12l[?25h[?25li[?12l[?25h[?25lá[?12l[?25h[?25lv[?12l[?25h[?25le[?12l[?25h[?25li[?12l[?25h[?25ls[?12l[?25h[?25l78
M[?12l[?25h[?25l [?12l[?25h[?25l [?12l[?25h[?25l [?12l[?25h[?25l [?12l[?25h[?25li[?12l[?25h[?25ln[?12l[?25h[?25lc[?12l[?25h[?25ll[?12l[?25h[?25lu[?12l[?25h[?25ld[?12l[?25h[?25le[?12l[?25h[?25l_[?12l[?25h[?25lv[?12l[?25h[?25la[?12l[?25h[?25lr[?12l[?25h[?25ls[?12l[?25h[?25l:[?12l[?25h[?25l78
M[?12l[?25h[?25l [?12l[?25h[?25l [?12l[?25h[?25l [?12l[?25h[?25l [?12l[?25h[?25l [?12l[?25h[?25l [?12l[?25h[?25lf[?12l[?25h[?25li[?12l[?25h[?25ll[?12l[?25h[?25le[?12l[?25h[?25l:[?12l[?25h[?25l [?12l[?25h[?25l.[?12l[?25h[?25l.[?12l[?25h[?25l/[?12l[?25h[?25l.[?12l[?25h[?25l.[?12l[?25h[?25l/[?12l[?25h[?25lv[?12l[?25h[?25la[?12l[?25h[?25lr[?12l[?25h[?25ls[?12l[?25h[?25l/[?12l[?25h[?25lm[?12l[?25h[?25la[?12l[?25h[?25li[?12l[?25h[?25ln[?12l[?25h[?25l.[?12l[?25h[?25ly[?12l[?25h[?25lm[?12l[?25h[?25ll[?12l[?25h[?25l78
M[?12l[?25h[?25l78M[?12l[?25h[?25l
- name: "teste: o sistema operativo encontra-se atualizado?"
[?12l[?25h[?25l - name: Variáveis
 [?12l[?25h[?25l  - name: Variáveis
  [?12l[?25h[?25l[?12l[?25h[?25l[?12l[?25h[?25l
[?12l[?25h[?25l         \textcolor{inv_background inv_foreground}{M-D} Formato DOS\textcolor{inv_background inv_foreground}{M-A} Anexar\textcolor{inv_background inv_foreground}{B} Segurança\textcolor{inv_background inv_foreground}{C} Cancelar           \textcolor{inv_background inv_foreground}{M-M} Formato Mac\textcolor{inv_background inv_foreground}{M-P} Prepor\textcolor{inv_background inv_foreground}{^T} P/ ficheiros
\textcolor{inv_background inv_foreground}{Nome do ficheiro onde escrever: molecule/default/verify.yml                                                                               }[?12l[?25h[?25l \textcolor{inv_background inv_foreground}{[ A escrever... ]}\textcolor{inv_background inv_foreground}{          }\textcolor{inv_background inv_foreground}{[ 189 linhas escritas ]}\textcolor{inv_background inv_foreground}{^O} Gravar\textcolor{inv_background inv_foreground}{^W} Procurar      \textcolor{inv_background inv_foreground}{^K} Cortar txt    \textcolor{inv_background inv_foreground}{^J} Justificar    \textcolor{inv_background inv_foreground}{^C} Pos cursor\textcolor{inv_background inv_foreground}{U} Desfazer     \textcolor{inv_background inv_foreground}{M-A} Marcar txt\textcolor{inv_background inv_foreground}{X} Sair    \textcolor{inv_background inv_foreground}{^R} Carregar\textcolor{inv_background inv_foreground}{^\} Substituir    \textcolor{inv_background inv_foreground}{^U} Colar txt     \textcolor{inv_background inv_foreground}{^T} Ortografia    \textcolor{inv_background inv_foreground}{^_} Ir p/ linha   \textcolor{inv_background inv_foreground}{M-E} Refazer      \textcolor{inv_background inv_foreground}{M-6} Copiar txt
[?12l[?25h[?25l  - name: "teste: o sistema operativo encontra-se atualizado?"
    assert:that: - not estado.changed    success_msg: "SUCESSO: O sistema operativo está atualizado"    fail_msg: "ERRO: Existem actualizações pendentes"
  - name: "Teste: a cache encontra-se atualizada?"  assert:    that:  - not estado.changedsuccess_msg: "SUCESSO: A cache está atualizada"fail_msg: "ERRO: Existem atualizações pendentes"
  - name: "Simulação: testa se as aplicações dependencia do software leantime se encontram instaladas"apt:pkg:  - mc  - screen- git    - apache2  - mysql-server- phpphp-mysql- php-ldap- php-cli- php-soap    - php-json  - graphviz- php-xml  - php-gd  - php-zip- libapache2-mod-php    - php-dev[?12l[?25h[?25l- libapache2-mod-php    - php-dev  - libmcrypt-dev- gccmake- autoconf- libc-dev- pkg-config    - pwgen  - curl- unzipzip- php-mbstring- expect- net-tools    - python3-mysqldb  - python3-apt- python3-pycurlcheck_mode: trueregister: estado
  - name: "Teste: as dependencias encontra-se instaladas?"assert:that:not estado.changedsuccess_msg: "SUCESSO: As dependencias estavam instaladas"fail_msg: "ERRO: Faltam instalar algumas dependencias. O software pode estar parcialmente instalado"
\textcolor{ansi36}{  # Fonte de inspiração: https://docs.ansible.com/ansible/latest/collections/ansible/builtin/service_module.html}- name: "Simulação: Testar se o serviço apache2 se encontra instalado, iniciado e ativo no arranque"service:name: apache2state: startedenabled: truecheck_mode: true[?12l[?25h[?25lenabled: trueO serviço apache2 encontra-se ativo no arranque no sistema, e está iniciado?"O serviço apache2 está correctamente instalado e inicia com o arranque do sistema"O serviço apache2 não se encontra correcta\textcolor{ansi36}{: https://www.mydailytutorials.com/using--find-module-search-filesfolder}Rotina Auxiliar> Procura o caminho do ficheiro php.ini do servidor apache2"find:paths: /etcpatterns: "php.ini"recurseregister: caminho_php_ini
\textcolor{ansi36}{  # Fonte: https://docs.ansible.com/ansible/latest/collections/ansible/builtin/lineinfile_module.html}- name: "Simulação: Ativa a opção file_uploads no ficheiro php.ini, utilizando o módulo lineinfile."ansible.builtin.lineinfile:path: "{{ caminho_php_ini.files[0].path }}"regexp: "{{ item.regexp  }}"line: "{{ item.line }}"with_items:- regexp: "^file_uploads"line: "file_uploads = 1"- regexp: "^upload_max_filesize"line: "upload_max_filesize = 1G"- regexp: "^max_file_uploads"line: "max_file_uploads = 20"- regexp: "^post_max_size"line: "post_max_size = 2G"- regexp: "^memory_limit"[?12l[?25h[?25lline: "post_max_size = 2G"- regexp: "^memory_limit"line: "memory_limit = 2G"- regexp: "^max_input_time"  line: "max_input_time = 3600"As linhas do php.ini encontram-se alteradas ?"ficheiro php.ini foi alterado com sucesso"Algumas alterações ao ficheiro php.ini não tiveram sucesso"\textcolor{ansi36}{O site deve existir na pasta /var/www/leantime. Este teste visa determinar se a pasta
  # existe e se temas permissões certas
  # Fonte de inspiração: https://docs.ansible.com/ansible/latest/collections/ansible/builtin/stat_module.html}- name: "Simulação: Obtem informação sobre a pastas /var/www/leantime"
    stat:    path: "/var/www/leantime"register: pasta
  - name: "Teste: A pasta /var/www/leantime existe e tem as permissões certas?"assert:  that:  - pasta.stat.exists- pasta.stat.isdir  - pasta.stat.mode == "0755"- pasta.stat.pw_name == "www-data"  - pasta.stat.gr_name == "www-data"success_msg: "SUCESSO: Permissões correctas no site leantime"fail_msg: "ERRO: Permissões incorrectas no site leantime ou site não existente"
\textcolor{ansi36}{  # Outro dos requesitos é a existência de uma base de dados para o site.}[?12l[?25h[?25l
\textcolor{ansi36}{  # Outro dos requesitos é a existência de uma base de dados para o site.
  # Esta base de dados deve ser criada, logo tem de ser testada para determinar se ela existe ou não.
  # Fonte de inspiração: https://serverfault.com/questions/173978/from-a-shell-script-how-can-i-check-whether-a-mysql-database-exists}- name: "Simulação: Verifica se é necessário criar a base de dados leantime_database"shell: "mysql -u root -e 'use leantime_database'"resultfailed_when: false  changed_when: false
  - name: "Teste: A base de dados leantime_database existe ?"assert:that:- result.rc == 0success_msg: "SUCESSO: A base de dados leantime_database já se encontra criada" fail_msg: "ERRO: Não existe a base de dados leantime_database"
\textcolor{ansi36}{Teste: O download do software leantime deve ser possível, a partir do site do leantime} e Teste: Verifica se o endereço git para a versão de leantime existe"uri:url: https://github.com/Leantime/leantime/releases/download/{{ versao_leantime }}/Leantime-{{ versao_leantime }}.zipresultado
  [?12l[?25h[?25l
[?12l[?25h[?25l[?12l[?25h[?25l[?12l[?25h[?25l[?12l[?25h[?25l[?12l[?25h[?25l[?12l[?25h[?25l[?12l[?25h[?25l
[?12l[?25h[?1049l
[?1l>[?2004l]0;admlocal@devOps: ~/leantime_ansibleadmlocal@devOps:~/leantime_ansible$ nano vasars/main.yml 
[?2004h[?1049h[?7h[?1h=[?1h=[?25l\textcolor{inv_background inv_foreground}{[ A ler... ]}\textcolor{inv_background inv_foreground}{[ 4 linhas lidas ]}\textcolor{inv_background inv_foreground}{  GNU nano 4.8                                                 vars/main.yml                                                              }
\textcolor{inv_background inv_foreground}{^G} Ajuda\textcolor{inv_background inv_foreground}{^O} Gravar\textcolor{inv_background inv_foreground}{^W} Procurar\textcolor{inv_background inv_foreground}{^K} Cortar txt    \textcolor{inv_background inv_foreground}{^J} Justificar    \textcolor{inv_background inv_foreground}{^C} Pos cursor    \textcolor{inv_background inv_foreground}{M-U} Desfazer     \textcolor{inv_background inv_foreground}{M-A} Marcar txt
\textcolor{inv_background inv_foreground}{^X} Sair\textcolor{inv_background inv_foreground}{^R} Carregar\textcolor{inv_background inv_foreground}{^\} Substituir    \textcolor{inv_background inv_foreground}{^U} Colar txt     \textcolor{inv_background inv_foreground}{^T} Ortografia    \textcolor{inv_background inv_foreground}{^_} Ir p/ linha   \textcolor{inv_background inv_foreground}{M-E} Refazer\textcolor{inv_background inv_foreground}{M-6} Copiar txt
--- login_leantime: leantimeDBadminpassword_leantime: "#S3gr3d0S3cr3t0#"
[?12l[?25h[?25l[?12l[?25h[?25l[?12l[?25h[?25l[?12l[?25h[?25l[?12l[?25h[?25l\textcolor{inv_background inv_foreground}{Modificado}
[?12l[?25h[?25l [?12l[?25h[?25l [?12l[?25h[?25l [?12l[?25h[?25l [?12l[?25h[?25lv[?12l[?25h[?25le[?12l[?25h[?25lr[?12l[?25h[?25ls[?12l[?25h[?25la[?12l[?25h[?25lo[?12l[?25h[?25l?[?12l[?25h[?25ll[?12l[?25h[?25le[?12l[?25h[?25la[?12l[?25h[?25ln[?12l[?25h[?25lt[?12l[?25h[?25li[?12l[?25h[?25lm[?12l[?25h[?25le[?12l[?25h[?25l[?12l[?25h[?25l
[?12l[?25h[?25l[?12l[?25h[?25l[?12l[?25h[?25l[?12l[?25h[?25l[?12l[?25h[?25l[?12l[?25h[?25l[?12l[?25h[?25l[?12l[?25h[?25l_leantime[?12l[?25h[?25l[?12l[?25h[?25l:[?12l[?25h[?25l [?12l[?25h[?25lv[?12l[?25h[?25l2[?12l[?25h[?25l.[?12l[?25h[?25l1[?12l[?25h[?25l.[?12l[?25h[?25l7[?12l[?25h[?25l
[?12l[?25h[?25l[?12l[?25h[?25l         \textcolor{inv_background inv_foreground}{M-D} Formato DOS\textcolor{inv_background inv_foreground}{M-A} Anexar\textcolor{inv_background inv_foreground}{B} Segurança\textcolor{inv_background inv_foreground}{C} Cancelar           \textcolor{inv_background inv_foreground}{M-M} Formato Mac\textcolor{inv_background inv_foreground}{M-P} Prepor\textcolor{inv_background inv_foreground}{^T} P/ ficheiros
\textcolor{inv_background inv_foreground}{Nome do ficheiro onde escrever: vars/main.yml                                                                                             }[?12l[?25h[?25l \textcolor{inv_background inv_foreground}{[ A escrever... ]}\textcolor{inv_background inv_foreground}{          }\textcolor{inv_background inv_foreground}{[ 5 linhas escritas ]}\textcolor{inv_background inv_foreground}{^O} Gravar\textcolor{inv_background inv_foreground}{^W} Procurar      \textcolor{inv_background inv_foreground}{^K} Cortar txt    \textcolor{inv_background inv_foreground}{^J} Justificar    \textcolor{inv_background inv_foreground}{^C} Pos cursor\textcolor{inv_background inv_foreground}{U} Desfazer     \textcolor{inv_background inv_foreground}{M-A} Marcar txt\textcolor{inv_background inv_foreground}{X} Sair    \textcolor{inv_background inv_foreground}{^R} Carregar\textcolor{inv_background inv_foreground}{^\} Substituir    \textcolor{inv_background inv_foreground}{^U} Colar txt     \textcolor{inv_background inv_foreground}{^T} Ortografia    \textcolor{inv_background inv_foreground}{^_} Ir p/ linha   \textcolor{inv_background inv_foreground}{M-E} Refazer      \textcolor{inv_background inv_foreground}{M-6} Copiar txt
[?12l[?25h[?25l[?12l[?25h[?1049l
[?1l>[?2004l]0;admlocal@devOps: ~/leantime_ansibleadmlocal@devOps:~/leantime_ansible$ molecule lint
\textcolor{ansi34}{INFO    } default scenario test matrix: dependency, lint
\textcolor{ansi34}{INFO    } Performing prerun\textcolor{ansi33}{...}
\textcolor{ansi34}{INFO    } Using .cache/roles/nunomourinho.leantime_ansible symlink to current repository in order to enable Ansible to find the role using its expected full name.
\textcolor{ansi34}{INFO    } Added \textcolor{ansi33}{ANSIBLE_ROLES_PATH}=~\textcolor{ansi35}{/.ansible/}\textcolor{ansi95}{roles}:\textcolor{ansi35}{/usr/share/ansible/}\textcolor{ansi95}{roles}:\textcolor{ansi35}{/etc/ansible/}\textcolor{ansi95}{roles}:.\textcolor{ansi35}{/.cache/}\textcolor{ansi95}{roles}
\textcolor{ansi34}{INFO    } \textcolor{ansi2 ansi36}{Running }\textcolor{ansi2 ansi32}{default}\textcolor{ansi2 ansi36}{ > }\textcolor{ansi2 ansi32}{dependency}
\textcolor{ansi31}{WARNING } Skipping, missing the requirements file.
\textcolor{ansi31}{WARNING } Skipping, missing the requirements file.
\textcolor{ansi34}{INFO    } \textcolor{ansi2 ansi36}{Running }\textcolor{ansi2 ansi32}{default}\textcolor{ansi2 ansi36}{ > }\textcolor{ansi2 ansi32}{lint}
COMMAND: set -e
yamllint .
ansible-lint

./molecule/default/verify.yml
  189:1     error    too many blank lines (2 > 0)  (empty-lines)

\textbf{\textcolor{ansi31}{CRITICAL}} Lint failed with error code \textbf{\textcolor{ansi36}{1}}
]0;admlocal@devOps: ~/leantime_ansibleadmlocal@devOps:~/leantime_ansible$ molecule lintnano vars/main.yml [14@molecule/default/verifyexitnano molecule/default/verify.yml 
[?2004h[?1049h[?7h[?1h=[?1h=[?25l\textcolor{inv_background inv_foreground}{[ A ler... ]}\textcolor{inv_background inv_foreground}{[ 189 linhas lidas ]}\textcolor{inv_background inv_foreground}{  GNU nano 4.8                                          molecule/default/verify.yml                                                       }
\textcolor{inv_background inv_foreground}{^G} Ajuda\textcolor{inv_background inv_foreground}{^O} Gravar\textcolor{inv_background inv_foreground}{^W} Procurar\textcolor{inv_background inv_foreground}{^K} Cortar txt    \textcolor{inv_background inv_foreground}{^J} Justificar    \textcolor{inv_background inv_foreground}{^C} Pos cursor    \textcolor{inv_background inv_foreground}{M-U} Desfazer     \textcolor{inv_background inv_foreground}{M-A} Marcar txt
\textcolor{inv_background inv_foreground}{^X} Sair\textcolor{inv_background inv_foreground}{^R} Carregar\textcolor{inv_background inv_foreground}{^\} Substituir    \textcolor{inv_background inv_foreground}{^U} Colar txt     \textcolor{inv_background inv_foreground}{^T} Ortografia    \textcolor{inv_background inv_foreground}{^_} Ir p/ linha   \textcolor{inv_background inv_foreground}{M-E} Refazer\textcolor{inv_background inv_foreground}{M-6} Copiar txt
---
\textcolor{ansi36}{# Tutorial de inspiração para a infraestrutura conduzida por teste no site
# https://www.adictosaltrabajo.com/2020/05/08/ansible-testing-using-molecule-with-ansible-as-verifier/
}- name: "Infraestrutura conduzida por testes"hosts: allgather_facts: falsebecome: truetasks:- name: Variáveisinclude_vars:file: ../../vars/main.yml- name: "Simulação: Atualizar a cache do sistema"apt:update_cache: truecache_valid_time: 3600check_mode: trueregister: estado- name: "teste: a cache encontra-se actualizada?"assert:that:- not estado.changedsuccess_msg: "SUCESSO: A cache está atualizada"fail_msg: "ERRO: Existem actualizações pendentes"- name: "Atualizar o sistema operativo (equivalente a apt upgrade)"apt:upgrade: safecheck_mode: trueregister: estado- name: "teste: o sistema operativo encontra-se atualizado?"
[?12l[?25h[?25l  - name: "teste: o sistema operativo encontra-se atualizado?"
    assert:that: - not estado.changed    success_msg: "SUCESSO: O sistema operativo está atualizado"    fail_msg: "ERRO: Existem actualizações pendentes"
  - name: "Teste: a cache encontra-se atualizada?"  assert:    that:  - not estado.changedsuccess_msg: "SUCESSO: A cache está atualizada"fail_msg: "ERRO: Existem atualizações pendentes"
  - name: "Simulação: testa se as aplicações dependencia do software leantime se encontram instaladas"apt:pkg:  - mc  - screen- git    - apache2  - mysql-server- phpphp-mysql- php-ldap- php-cli- php-soap    - php-json  - graphviz- php-xml  - php-gd  - php-zip- libapache2-mod-php    - php-dev
[?12l[?25h[?25l- libapache2-mod-php    - php-dev  - libmcrypt-dev- gccmake- autoconf- libc-dev- pkg-config    - pwgen  - curl- unzipzip- php-mbstring- expect- net-tools    - python3-mysqldb  - python3-apt- python3-pycurlcheck_mode: trueregister: estado
  - name: "Teste: as dependencias encontra-se instaladas?"assert:that:not estado.changedsuccess_msg: "SUCESSO: As dependencias estavam instaladas"fail_msg: "ERRO: Faltam instalar algumas dependencias. O software pode estar parcialmente instalado"
\textcolor{ansi36}{  # Fonte de inspiração: https://docs.ansible.com/ansible/latest/collections/ansible/builtin/service_module.html}- name: "Simulação: Testar se o serviço apache2 se encontra instalado, iniciado e ativo no arranque"service:name: apache2state: startedenabled: truecheck_mode: true
[?12l[?25h[?25lenabled: trueO serviço apache2 encontra-se ativo no arranque no sistema, e está iniciado?"O serviço apache2 está correctamente instalado e inicia com o arranque do sistema"O serviço apache2 não se encontra correcta\textcolor{ansi36}{: https://www.mydailytutorials.com/using--find-module-search-filesfolder}Rotina Auxiliar> Procura o caminho do ficheiro php.ini do servidor apache2"find:paths: /etcpatterns: "php.ini"recurseregister: caminho_php_ini
\textcolor{ansi36}{  # Fonte: https://docs.ansible.com/ansible/latest/collections/ansible/builtin/lineinfile_module.html}- name: "Simulação: Ativa a opção file_uploads no ficheiro php.ini, utilizando o módulo lineinfile."ansible.builtin.lineinfile:path: "{{ caminho_php_ini.files[0].path }}"regexp: "{{ item.regexp  }}"line: "{{ item.line }}"with_items:- regexp: "^file_uploads"line: "file_uploads = 1"- regexp: "^upload_max_filesize"line: "upload_max_filesize = 1G"- regexp: "^max_file_uploads"line: "max_file_uploads = 20"- regexp: "^post_max_size"line: "post_max_size = 2G"- regexp: "^memory_limit"
[?12l[?25h[?25l78line: "post_max_size = 2G"- regexp: "^memory_limit"line: "memory_limit = 2G"- regexp: "^max_input_time"  line: "max_input_time = 3600"As linhas do php.ini encontram-se alteradas ?"ficheiro php.ini foi alterado com sucesso"Algumas alterações ao ficheiro php.ini não tiveram sucesso"\textcolor{ansi36}{O site deve existir na pasta /var/www/leantime. Este teste visa determinar se a pasta
  # existe e se temas permissões certas
  # Fonte de inspiração: https://docs.ansible.com/ansible/latest/collections/ansible/builtin/stat_module.html}- name: "Simulação: Obtem informação sobre a pastas /var/www/leantime"
    stat:    path: "/var/www/leantime"register: pasta
  - name: "Teste: A pasta /var/www/leantime existe e tem as permissões certas?"assert:  that:  - pasta.stat.exists- pasta.stat.isdir  - pasta.stat.mode == "0755"- pasta.stat.pw_name == "www-data"  - pasta.stat.gr_name == "www-data"success_msg: "SUCESSO: Permissões correctas no site leantime"fail_msg: "ERRO: Permissões incorrectas no site leantime ou site não existente"
\textcolor{ansi36}{  # Outro dos requesitos é a existência de uma base de dados para o site.
}[?12l[?25h[?25l\textcolor{ansi36}{  # Outro dos requesitos é a existência de uma base de dados para o site.
  # Esta base de dados deve ser criada, logo tem de ser testada para determinar se ela existe ou não.
  # Fonte de inspiração: https://serverfault.com/questions/173978/from-a-shell-script-how-can-i-check-whether-a-mysql-database-exists}- name: "Simulação: Verifica se é necessário criar a base de dados leantime_database"shell: "mysql -u root -e 'use leantime_database'"resultfailed_when: false  changed_when: false
  - name: "Teste: A base de dados leantime_database existe ?"assert:that:- result.rc == 0success_msg: "SUCESSO: A base de dados leantime_database já se encontra criada" fail_msg: "ERRO: Não existe a base de dados leantime_database"
\textcolor{ansi36}{Teste: O download do software leantime deve ser possível, a partir do site do leantime} e Teste: Verifica se o endereço git para a versão de leantime existe"uri:url: https://github.com/Leantime/leantime/releases/download/{{ versao_leantime }}/Leantime-{{ versao_leantime }}.zipresultado
[?12l[?25h[?25l[?12l[?25h[?25l[?12l[?25h[?25l[?12l[?25h[?25l[?12l[?25h[?25l[?12l[?25h[?25l[?12l[?25h[?25l[?12l[?25h[?25l[?12l[?25h[?25l[?12l[?25h[?25l[?12l[?25h[?25l[?12l[?25h[?25l[?12l[?25h[?25l[?12l[?25h[?25l[?12l[?25h[?25l\textcolor{inv_background inv_foreground}{Modificado}
[?12l[?25h[?25l[?12l[?25h[?25l
[?12l[?25h[?25l         \textcolor{inv_background inv_foreground}{M-D} Formato DOS\textcolor{inv_background inv_foreground}{M-A} Anexar\textcolor{inv_background inv_foreground}{B} Segurança\textcolor{inv_background inv_foreground}{C} Cancelar           \textcolor{inv_background inv_foreground}{M-M} Formato Mac\textcolor{inv_background inv_foreground}{M-P} Prepor\textcolor{inv_background inv_foreground}{^T} P/ ficheiros
\textcolor{inv_background inv_foreground}{Nome do ficheiro onde escrever: molecule/default/verify.yml                                                                               }[?12l[?25h[?25l \textcolor{inv_background inv_foreground}{[ A escrever... ]}\textcolor{inv_background inv_foreground}{          }\textcolor{inv_background inv_foreground}{[ 187 linhas escritas ]}\textcolor{inv_background inv_foreground}{^O} Gravar\textcolor{inv_background inv_foreground}{^W} Procurar      \textcolor{inv_background inv_foreground}{^K} Cortar txt    \textcolor{inv_background inv_foreground}{^J} Justificar    \textcolor{inv_background inv_foreground}{^C} Pos cursor\textcolor{inv_background inv_foreground}{U} Desfazer     \textcolor{inv_background inv_foreground}{M-A} Marcar txt\textcolor{inv_background inv_foreground}{X} Sair    \textcolor{inv_background inv_foreground}{^R} Carregar\textcolor{inv_background inv_foreground}{^\} Substituir    \textcolor{inv_background inv_foreground}{^U} Colar txt     \textcolor{inv_background inv_foreground}{^T} Ortografia    \textcolor{inv_background inv_foreground}{^_} Ir p/ linha   \textcolor{inv_background inv_foreground}{M-E} Refazer      \textcolor{inv_background inv_foreground}{M-6} Copiar txt
[?12l[?25h[?25l[?12l[?25h[?1049l
[?1l>[?2004l]0;admlocal@devOps: ~/leantime_ansibleadmlocal@devOps:~/leantime_ansible$ nano molecule/default/verify.yml molecule lint
\textcolor{ansi34}{INFO    } default scenario test matrix: dependency, lint
\textcolor{ansi34}{INFO    } Performing prerun\textcolor{ansi33}{...}
\textcolor{ansi34}{INFO    } Using .cache/roles/nunomourinho.leantime_ansible symlink to current repository in order to enable Ansible to find the role using its expected full name.
\textcolor{ansi34}{INFO    } Added \textcolor{ansi33}{ANSIBLE_ROLES_PATH}=~\textcolor{ansi35}{/.ansible/}\textcolor{ansi95}{roles}:\textcolor{ansi35}{/usr/share/ansible/}\textcolor{ansi95}{roles}:\textcolor{ansi35}{/etc/ansible/}\textcolor{ansi95}{roles}:.\textcolor{ansi35}{/.cache/}\textcolor{ansi95}{roles}
\textcolor{ansi34}{INFO    } \textcolor{ansi2 ansi36}{Running }\textcolor{ansi2 ansi32}{default}\textcolor{ansi2 ansi36}{ > }\textcolor{ansi2 ansi32}{dependency}
\textcolor{ansi31}{WARNING } Skipping, missing the requirements file.
\textcolor{ansi31}{WARNING } Skipping, missing the requirements file.
\textcolor{ansi34}{INFO    } \textcolor{ansi2 ansi36}{Running }\textcolor{ansi2 ansi32}{default}\textcolor{ansi2 ansi36}{ > }\textcolor{ansi2 ansi32}{lint}
COMMAND: set -e
yamllint .
ansible-lint

Loading custom .yamllint config file, this extends our internal yamllint config.
WARNING  Listing 1 violation(s) that are fatal
\textcolor{ansi91}{yaml}\textcolor{ansi2}{:} \textcolor{ansi31}{too many blank lines (1 > 0)} \textcolor{ansi2 ansi91}{(empty-lines)}
\textcolor{ansi34}{.ansible-lint}:3

You can skip specific rules or tags by adding them to your configuration file:
\textcolor{ansi2}{# .ansible-lint}
\textcolor{ansi94}{warn_list}:  \textcolor{ansi2}{# or 'skip_list' to silence them completely}
  - yaml  \textcolor{ansi2}{# Violations reported by yamllint}
Finished with \textbf{\textcolor{ansi36}{1}} failure\textcolor{ansi1}{(}s\textcolor{ansi1}{)}, \textbf{\textcolor{ansi36}{0}} warning\textcolor{ansi1}{(}s\textcolor{ansi1}{)} on \textbf{\textcolor{ansi36}{40}} files.
\textbf{\textcolor{ansi31}{CRITICAL}} Lint failed with error code \textbf{\textcolor{ansi36}{2}}
]0;admlocal@devOps: ~/leantime_ansibleadmlocal@devOps:~/leantime_ansible$ molecule lintnano molecule/default/verify.yml 
[?2004h[?1049h[?7h[?1h=[?1h=[?25l\textcolor{inv_background inv_foreground}{[ A ler... ]}\textcolor{inv_background inv_foreground}{[ 187 linhas lidas ]}\textcolor{inv_background inv_foreground}{  GNU nano 4.8                                          molecule/default/verify.yml                                                       }
\textcolor{inv_background inv_foreground}{^G} Ajuda\textcolor{inv_background inv_foreground}{^O} Gravar\textcolor{inv_background inv_foreground}{^W} Procurar\textcolor{inv_background inv_foreground}{^K} Cortar txt    \textcolor{inv_background inv_foreground}{^J} Justificar    \textcolor{inv_background inv_foreground}{^C} Pos cursor    \textcolor{inv_background inv_foreground}{M-U} Desfazer     \textcolor{inv_background inv_foreground}{M-A} Marcar txt
\textcolor{inv_background inv_foreground}{^X} Sair\textcolor{inv_background inv_foreground}{^R} Carregar\textcolor{inv_background inv_foreground}{^\} Substituir    \textcolor{inv_background inv_foreground}{^U} Colar txt     \textcolor{inv_background inv_foreground}{^T} Ortografia    \textcolor{inv_background inv_foreground}{^_} Ir p/ linha   \textcolor{inv_background inv_foreground}{M-E} Refazer\textcolor{inv_background inv_foreground}{M-6} Copiar txt
---
\textcolor{ansi36}{# Tutorial de inspiração para a infraestrutura conduzida por teste no site
# https://www.adictosaltrabajo.com/2020/05/08/ansible-testing-using-molecule-with-ansible-as-verifier/
}- name: "Infraestrutura conduzida por testes"hosts: allgather_facts: falsebecome: truetasks:- name: Variáveisinclude_vars:file: ../../vars/main.yml- name: "Simulação: Atualizar a cache do sistema"apt:update_cache: truecache_valid_time: 3600check_mode: trueregister: estado- name: "teste: a cache encontra-se actualizada?"assert:that:- not estado.changedsuccess_msg: "SUCESSO: A cache está atualizada"fail_msg: "ERRO: Existem actualizações pendentes"- name: "Atualizar o sistema operativo (equivalente a apt upgrade)"apt:upgrade: safecheck_mode: trueregister: estado- name: "teste: o sistema operativo encontra-se atualizado?"
[?12l[?25h[?25l         In. parág\textcolor{inv_background inv_foreground}{^Y} Prim.linha\textcolor{inv_background inv_foreground}{^T} Ir para txt\textcolor{inv_background inv_foreground}{C} Cancelar           \textcolor{inv_background inv_foreground}{O} Fim parág\textcolor{inv_background inv_foreground}{V} Últ. linha
\textcolor{inv_background inv_foreground}{Insira nº da linha, nº da coluna:                                                                                                         }[?12l[?25h[?25l\textcolor{inv_background inv_foreground}{1
}[?12l[?25h[?25l\textcolor{inv_background inv_foreground}{8
}[?12l[?25h[?25l\textcolor{inv_background inv_foreground}{9
}[?12l[?25h[?25l
\textcolor{inv_background inv_foreground}{^O} GravarProcurar      \textcolor{inv_background inv_foreground}{^K} Cortar txt    \textcolor{inv_background inv_foreground}{^J} Justificar    \textcolor{inv_background inv_foreground}{^C} Pos cursor    \textcolor{inv_background inv_foreground}{M-U} Desfazer     \textcolor{inv_background inv_foreground}{M-A} Marcar txt\textcolor{inv_background inv_foreground}{X} Sair    \textcolor{inv_background inv_foreground}{^R} Carregar\textcolor{inv_background inv_foreground}{\} Substituir    \textcolor{inv_background inv_foreground}{^U} Colar txt\textcolor{inv_background inv_foreground}{T} Ortografia    \textcolor{inv_background inv_foreground}{^_} Ir p/ linha   \textcolor{inv_background inv_foreground}{M-E} Refazer\textcolor{inv_background inv_foreground}{M-6} Copiar txt
78shell: "mysql -u root -e 'use leantime_database'"register: resultfailed_when: falsechanged_when: falseTA base de dados leantime_database existe ?"result.rc == 0base de dados leantime_database já se encontra criada"Não existe a base de dados leantime_database"
[?12l[?25h[?25l[?12l[?25h[?25l[?12l[?25h[?25l
[?12l[?25h[?25l[?12l[?25h[?25l[?12l[?25h[?25l[?12l[?25h[?25l[?12l[?25h[?25l[?12l[?25h[?25l[?12l[?25h[?25l[?12l[?25h[?25l[?12l[?25h[?25l[?12l[?25h[?25l[?12l[?25h[?25l[?12l[?25h[?25l[?12l[?25h[?25l[?12l[?25h[?25l[?12l[?25h[?25l[?12l[?25h[?25l[?12l[?25h[?25l[?12l[?25h[?25l[?12l[?25h[?25l[?12l[?25h[?25l[?12l[?25h[?25l
[?12l[?25h[?25l[?12l[?25h[?25l[?12l[?25h[?25l[?12l[?25h[?25l[?12l[?25h[?25l[?12l[?25h[?25l[?12l[?25h[?25l[?12l[?25h[?25l[?12l[?25h[?25l[?12l[?25h[?25l[?12l[?25h[?25l[?12l[?25h[?25l
[?12l[?25h[?25l[?12l[?25h[?25l[?12l[?25h[?25l[?12l[?25h[?25l[?12l[?25h[?25l[?12l[?25h[?25l[?12l[?25h[?25l[?12l[?25h[?25l[?12l[?25h[?25l[?12l[?25h[?25l[?12l[?25h[?25l[?12l[?25h[?25l         \textcolor{inv_background inv_foreground}{M-D} Formato DOS\textcolor{inv_background inv_foreground}{M-A} Anexar\textcolor{inv_background inv_foreground}{B} Segurança\textcolor{inv_background inv_foreground}{C} Cancelar           \textcolor{inv_background inv_foreground}{M-M} Formato Mac\textcolor{inv_background inv_foreground}{M-P} Prepor\textcolor{inv_background inv_foreground}{^T} P/ ficheiros
\textcolor{inv_background inv_foreground}{Nome do ficheiro onde escrever: molecule/default/verify.yml                                                                               }[?12l[?25h[?25l \textcolor{inv_background inv_foreground}{[ A escrever... ]}\textcolor{inv_background inv_foreground}{[ 187 linhas escritas ]}\textcolor{inv_background inv_foreground}{^O} Gravar\textcolor{inv_background inv_foreground}{^W} Procurar      \textcolor{inv_background inv_foreground}{^K} Cortar txt    \textcolor{inv_background inv_foreground}{^J} Justificar    \textcolor{inv_background inv_foreground}{^C} Pos cursor\textcolor{inv_background inv_foreground}{U} Desfazer     \textcolor{inv_background inv_foreground}{M-A} Marcar txt\textcolor{inv_background inv_foreground}{X} Sair    \textcolor{inv_background inv_foreground}{^R} Carregar\textcolor{inv_background inv_foreground}{^\} Substituir    \textcolor{inv_background inv_foreground}{^U} Colar txt     \textcolor{inv_background inv_foreground}{^T} Ortografia    \textcolor{inv_background inv_foreground}{^_} Ir p/ linha   \textcolor{inv_background inv_foreground}{M-E} Refazer      \textcolor{inv_background inv_foreground}{M-6} Copiar txt
[?12l[?25h[?25l
[?12l[?25h[?1049l
[?1l>[?2004l]0;admlocal@devOps: ~/leantime_ansibleadmlocal@devOps:~/leantime_ansible$ nano molecule/default/verify.yml molecule lint
\textcolor{ansi34}{INFO    } default scenario test matrix: dependency, lint
\textcolor{ansi34}{INFO    } Performing prerun\textcolor{ansi33}{...}
\textcolor{ansi34}{INFO    } Using .cache/roles/nunomourinho.leantime_ansible symlink to current repository in order to enable Ansible to find the role using its expected full name.
\textcolor{ansi34}{INFO    } Added \textcolor{ansi33}{ANSIBLE_ROLES_PATH}=~\textcolor{ansi35}{/.ansible/}\textcolor{ansi95}{roles}:\textcolor{ansi35}{/usr/share/ansible/}\textcolor{ansi95}{roles}:\textcolor{ansi35}{/etc/ansible/}\textcolor{ansi95}{roles}:.\textcolor{ansi35}{/.cache/}\textcolor{ansi95}{roles}
\textcolor{ansi34}{INFO    } \textcolor{ansi2 ansi36}{Running }\textcolor{ansi2 ansi32}{default}\textcolor{ansi2 ansi36}{ > }\textcolor{ansi2 ansi32}{dependency}
\textcolor{ansi31}{WARNING } Skipping, missing the requirements file.
\textcolor{ansi31}{WARNING } Skipping, missing the requirements file.
\textcolor{ansi34}{INFO    } \textcolor{ansi2 ansi36}{Running }\textcolor{ansi2 ansi32}{default}\textcolor{ansi2 ansi36}{ > }\textcolor{ansi2 ansi32}{lint}
COMMAND: set -e
yamllint .
ansible-lint

Loading custom .yamllint config file, this extends our internal yamllint config.
WARNING  Listing 1 violation(s) that are fatal
\textcolor{ansi91}{yaml}\textcolor{ansi2}{:} \textcolor{ansi31}{too many blank lines (1 > 0)} \textcolor{ansi2 ansi91}{(empty-lines)}
\textcolor{ansi34}{.ansible-lint}:3

You can skip specific rules or tags by adding them to your configuration file:
\textcolor{ansi2}{# .ansible-lint}
\textcolor{ansi94}{warn_list}:  \textcolor{ansi2}{# or 'skip_list' to silence them completely}
  - yaml  \textcolor{ansi2}{# Violations reported by yamllint}
Finished with \textbf{\textcolor{ansi36}{1}} failure\textcolor{ansi1}{(}s\textcolor{ansi1}{)}, \textbf{\textcolor{ansi36}{0}} warning\textcolor{ansi1}{(}s\textcolor{ansi1}{)} on \textbf{\textcolor{ansi36}{40}} files.
\textbf{\textcolor{ansi31}{CRITICAL}} Lint failed with error code \textbf{\textcolor{ansi36}{2}}
]0;admlocal@devOps: ~/leantime_ansibleadmlocal@devOps:~/leantime_ansible$ nano .ansible-lint 
[?2004h[?1049h[?7h[?1h=[?1h=[?25l\textcolor{inv_background inv_foreground}{[ A ler... ]}\textcolor{inv_background inv_foreground}{[ 3 linhas lidas ]}\textcolor{inv_background inv_foreground}{  GNU nano 4.8                                                 .ansible-lint                                                              }
\textcolor{inv_background inv_foreground}{^G} Ajuda\textcolor{inv_background inv_foreground}{^O} Gravar\textcolor{inv_background inv_foreground}{^W} Procurar\textcolor{inv_background inv_foreground}{^K} Cortar txt    \textcolor{inv_background inv_foreground}{^J} Justificar    \textcolor{inv_background inv_foreground}{^C} Pos cursor    \textcolor{inv_background inv_foreground}{M-U} Desfazer     \textcolor{inv_background inv_foreground}{M-A} Marcar txt
\textcolor{inv_background inv_foreground}{^X} Sair\textcolor{inv_background inv_foreground}{^R} Carregar\textcolor{inv_background inv_foreground}{^\} Substituir    \textcolor{inv_background inv_foreground}{^U} Colar txt     \textcolor{inv_background inv_foreground}{^T} Ortografia    \textcolor{inv_background inv_foreground}{^_} Ir p/ linha   \textcolor{inv_background inv_foreground}{M-E} Refazer\textcolor{inv_background inv_foreground}{M-6} Copiar txt
[?12l[?25h[?25l[?12l[?25h[?25l[?12l[?25h[?25l[?12l[?25h[?25l[?12l[?25h[?25l[?12l[?25h[?25l[?12l[?25h[?25l[?12l[?25h[?25l[?12l[?25h[?25l[?12l[?25h[?25l[?12l[?25h[?25l[?12l[?25h[?25l[?12l[?25h[?25l[?12l[?25h[?25l[?12l[?25h[?25l[?12l[?25h[?25l[?12l[?25h[?25l\textcolor{inv_background inv_foreground}{Modificado}
[?12l[?25h[?25l[?12l[?25h[?25l
[?12l[?25h[?25l[?12l[?25h[?25l[?12l[?25h[?25l
[?12l[?25h[?25l
[?12l[?25h[?25l[?12l[?25h[?25l
[?12l[?25h[?25l[?12l[?25h[?25l[?12l[?25h[?25l[?12l[?25h[?25l[?12l[?25h[?25l[?12l[?25h[?25l[?12l[?25h[?25l[?12l[?25h[?25l[?12l[?25h[?25l[?12l[?25h[?25l[?12l[?25h[?25l[?12l[?25h[?25l         \textcolor{inv_background inv_foreground}{M-D} Formato DOS\textcolor{inv_background inv_foreground}{M-A} Anexar\textcolor{inv_background inv_foreground}{B} Segurança\textcolor{inv_background inv_foreground}{C} Cancelar           \textcolor{inv_background inv_foreground}{M-M} Formato Mac\textcolor{inv_background inv_foreground}{M-P} Prepor\textcolor{inv_background inv_foreground}{^T} P/ ficheiros
\textcolor{inv_background inv_foreground}{Nome do ficheiro onde escrever: .ansible-lint                                                                                             }[?12l[?25h[?25l \textcolor{inv_background inv_foreground}{[ A escrever... ]}\textcolor{inv_background inv_foreground}{          }\textcolor{inv_background inv_foreground}{[ 2 linhas escritas ]}\textcolor{inv_background inv_foreground}{^O} Gravar\textcolor{inv_background inv_foreground}{^W} Procurar      \textcolor{inv_background inv_foreground}{^K} Cortar txt    \textcolor{inv_background inv_foreground}{^J} Justificar    \textcolor{inv_background inv_foreground}{^C} Pos cursor\textcolor{inv_background inv_foreground}{U} Desfazer     \textcolor{inv_background inv_foreground}{M-A} Marcar txt\textcolor{inv_background inv_foreground}{X} Sair    \textcolor{inv_background inv_foreground}{^R} Carregar\textcolor{inv_background inv_foreground}{^\} Substituir    \textcolor{inv_background inv_foreground}{^U} Colar txt     \textcolor{inv_background inv_foreground}{^T} Ortografia    \textcolor{inv_background inv_foreground}{^_} Ir p/ linha   \textcolor{inv_background inv_foreground}{M-E} Refazer      \textcolor{inv_background inv_foreground}{M-6} Copiar txt
[?12l[?25h[?25l[?12l[?25h[?1049l
[?1l>[?2004l]0;admlocal@devOps: ~/leantime_ansibleadmlocal@devOps:~/leantime_ansible$ nano .ansible-lint molecule lint
\textcolor{ansi34}{INFO    } default scenario test matrix: dependency, lint
\textcolor{ansi34}{INFO    } Performing prerun\textcolor{ansi33}{...}
\textcolor{ansi34}{INFO    } Using .cache/roles/nunomourinho.leantime_ansible symlink to current repository in order to enable Ansible to find the role using its expected full name.
\textcolor{ansi34}{INFO    } Added \textcolor{ansi33}{ANSIBLE_ROLES_PATH}=~\textcolor{ansi35}{/.ansible/}\textcolor{ansi95}{roles}:\textcolor{ansi35}{/usr/share/ansible/}\textcolor{ansi95}{roles}:\textcolor{ansi35}{/etc/ansible/}\textcolor{ansi95}{roles}:.\textcolor{ansi35}{/.cache/}\textcolor{ansi95}{roles}
\textcolor{ansi34}{INFO    } \textcolor{ansi2 ansi36}{Running }\textcolor{ansi2 ansi32}{default}\textcolor{ansi2 ansi36}{ > }\textcolor{ansi2 ansi32}{dependency}
\textcolor{ansi31}{WARNING } Skipping, missing the requirements file.
\textcolor{ansi31}{WARNING } Skipping, missing the requirements file.
\textcolor{ansi34}{INFO    } \textcolor{ansi2 ansi36}{Running }\textcolor{ansi2 ansi32}{default}\textcolor{ansi2 ansi36}{ > }\textcolor{ansi2 ansi32}{lint}
COMMAND: set -e
yamllint .
ansible-lint

Loading custom .yamllint config file, this extends our internal yamllint config.
]0;admlocal@devOps: ~/leantime_ansibleadmlocal@devOps:~/leantime_ansible$ molecule verify
\textcolor{ansi34}{INFO    } default scenario test matrix: verify
\textcolor{ansi34}{INFO    } Performing prerun\textcolor{ansi33}{...}
\textcolor{ansi34}{INFO    } Using .cache/roles/nunomourinho.leantime_ansible symlink to current repository in order to enable Ansible to find the role using its expected full name.
\textcolor{ansi34}{INFO    } Added \textcolor{ansi33}{ANSIBLE_ROLES_PATH}=~\textcolor{ansi35}{/.ansible/}\textcolor{ansi95}{roles}:\textcolor{ansi35}{/usr/share/ansible/}\textcolor{ansi95}{roles}:\textcolor{ansi35}{/etc/ansible/}\textcolor{ansi95}{roles}:.\textcolor{ansi35}{/.cache/}\textcolor{ansi95}{roles}
\textcolor{ansi34}{INFO    } \textcolor{ansi2 ansi36}{Running }\textcolor{ansi2 ansi32}{default}\textcolor{ansi2 ansi36}{ > }\textcolor{ansi2 ansi32}{verify}
\textcolor{ansi34}{INFO    } Running Ansible Verifier

PLAY [Infraestrutura conduzida por testes] ***********************************************************************************************

TASK [Variáveis] *************************************************************************************************************************
\textcolor{ansi32}{ok: [ubuntu-20.04]}

TASK [Simulação: Atualizar a cache do sistema] *******************************************************************************************
\textcolor{ansi32}{ok: [ubuntu-20.04]}

TASK [teste: a cache encontra-se actualizada?] *******************************************************************************************
\textcolor{ansi32}{ok: [ubuntu-20.04] => {}
\textcolor{ansi32}{    "changed": false,}
\textcolor{ansi32}{    "msg": "SUCESSO: A cache está atualizada"}
\textcolor{ansi32}{}}

TASK [Atualizar o sistema operativo (equivalente a apt upgrade)] *************************************************************************
\textcolor{ansi32}{ok: [ubuntu-20.04]}

TASK [teste: o sistema operativo encontra-se atualizado?] ********************************************************************************
\textcolor{ansi32}{ok: [ubuntu-20.04] => {}
\textcolor{ansi32}{    "changed": false,}
\textcolor{ansi32}{    "msg": "SUCESSO: O sistema operativo está atualizado"}
\textcolor{ansi32}{}}

TASK [Teste: a cache encontra-se atualizada?] ********************************************************************************************
\textcolor{ansi32}{ok: [ubuntu-20.04] => {}
\textcolor{ansi32}{    "changed": false,}
\textcolor{ansi32}{    "msg": "SUCESSO: A cache está atualizada"}
\textcolor{ansi32}{}}

TASK [Simulação: testa se as aplicações dependencia do software leantime se encontram instaladas] ****************************************
\textcolor{ansi32}{ok: [ubuntu-20.04]}

TASK [Teste: as dependencias encontra-se instaladas?] ************************************************************************************
\textcolor{ansi32}{ok: [ubuntu-20.04] => {}
\textcolor{ansi32}{    "changed": false,}
\textcolor{ansi32}{    "msg": "SUCESSO: As dependencias estavam instaladas"}
\textcolor{ansi32}{}}

TASK [Simulação: Testar se o serviço apache2 se encontra instalado, iniciado e ativo no arranque] ****************************************
\textcolor{ansi32}{ok: [ubuntu-20.04]}

TASK [Teste: O serviço apache2 encontra-se ativo no arranque no sistema, e está iniciado?] ***********************************************
\textcolor{ansi32}{ok: [ubuntu-20.04] => {}
\textcolor{ansi32}{    "changed": false,}
\textcolor{ansi32}{    "msg": "SUCESSO: O serviço apache2 está correctamente instalado e inicia com o arranque do sistema"}
\textcolor{ansi32}{}}

TASK [Rotina Auxiliar> Procura o caminho do ficheiro php.ini do servidor apache2] ********************************************************
\textcolor{ansi32}{ok: [ubuntu-20.04]}

TASK [Simulação: Ativa a opção file_uploads no ficheiro php.ini, utilizando o módulo lineinfile.] ****************************************
\textcolor{ansi32}{ok: [ubuntu-20.04] => (item={'regexp': '^file_uploads', 'line': 'file_uploads = 1'})}
\textcolor{ansi32}{ok: [ubuntu-20.04] => (item={'regexp': '^upload_max_filesize', 'line': 'upload_max_filesize = 1G'})}
\textcolor{ansi32}{ok: [ubuntu-20.04] => (item={'regexp': '^max_file_uploads', 'line': 'max_file_uploads = 20'})}
\textcolor{ansi32}{ok: [ubuntu-20.04] => (item={'regexp': '^post_max_size', 'line': 'post_max_size = 2G'})}
\textcolor{ansi32}{ok: [ubuntu-20.04] => (item={'regexp': '^memory_limit', 'line': 'memory_limit = 2G'})}
\textcolor{ansi32}{ok: [ubuntu-20.04] => (item={'regexp': '^max_input_time', 'line': 'max_input_time = 3600'})}

TASK [Teste: As linhas do php.ini encontram-se alteradas ?] ******************************************************************************
\textcolor{ansi32}{ok: [ubuntu-20.04] => {}
\textcolor{ansi32}{    "changed": false,}
\textcolor{ansi32}{    "msg": "SUCESSO: O ficheiro php.ini foi alterado com sucesso"}
\textcolor{ansi32}{}}

TASK [Simulação: Obtem informação sobre a pastas /var/www/leantime] **********************************************************************
\textcolor{ansi32}{ok: [ubuntu-20.04]}

TASK [Teste: A pasta /var/www/leantime existe e tem as permissões certas?] ***************************************************************
\textcolor{ansi32}{ok: [ubuntu-20.04] => {}
\textcolor{ansi32}{    "changed": false,}
\textcolor{ansi32}{    "msg": "SUCESSO: Permissões correctas no site leantime"}
\textcolor{ansi32}{}}

TASK [Simulação: Verifica se é necessário criar a base de dados leantime_database] *******************************************************
\textcolor{ansi32}{ok: [ubuntu-20.04]}

TASK [Teste: A base de dados leantime_database existe ?] *********************************************************************************
\textcolor{ansi32}{ok: [ubuntu-20.04] => {}
\textcolor{ansi32}{    "changed": false,}
\textcolor{ansi32}{    "msg": "SUCESSO: A base de dados leantime_database já se encontra criada"}
\textcolor{ansi32}{}}

TASK [Simulação e Teste: Verifica se o endereço git para a versão de leantime existe] ****************************************************
\textcolor{ansi32}{ok: [ubuntu-20.04]}

PLAY RECAP *******************************************************************************************************************************
\textcolor{ansi32}{ubuntu-20.04}               : \textcolor{ansi32}{ok=18  } changed=0    unreachable=0    failed=0    skipped=0    rescued=0    ignored=0

\textcolor{ansi34}{INFO    } Verifier completed successfully.
]0;admlocal@devOps: ~/leantime_ansibleadmlocal@devOps:~/leantime_ansible$ exit
exit

\end{Verbatim}
\end{document}
