\documentclass{scrartcl}
\usepackage[utf8]{inputenc}
\usepackage{fancyvrb}
\usepackage[usenames,dvipsnames]{xcolor}
% \definecolor{red-sd}{HTML}{7ed2d2}

\title{}

\fvset{commandchars=\\\{\}}

\begin{document}

\begin{Verbatim}
]0;admlocal@devOps: ~/leantime_ansibleadmlocal@devOps:~/leantime_ansible$ nano tasks&/tasks.yml 
[?2004h[?1049h[?7h[?1h=[?1h=[?25l\textcolor{inv_background inv_foreground}{[ A ler... ]}\textcolor{inv_background inv_foreground}{[ 93 linhas lidas ]}\textcolor{inv_background inv_foreground}{  GNU nano 4.8                                                tasks/tasks.yml                                                             }
\textcolor{inv_background inv_foreground}{^G} Ajuda\textcolor{inv_background inv_foreground}{^O} Gravar\textcolor{inv_background inv_foreground}{^W} Procurar\textcolor{inv_background inv_foreground}{^K} Cortar txt    \textcolor{inv_background inv_foreground}{^J} Justificar    \textcolor{inv_background inv_foreground}{^C} Pos cursor    \textcolor{inv_background inv_foreground}{M-U} Desfazer     \textcolor{inv_background inv_foreground}{M-A} Marcar txt
\textcolor{inv_background inv_foreground}{^X} Sair\textcolor{inv_background inv_foreground}{^R} Carregar\textcolor{inv_background inv_foreground}{^\} Substituir    \textcolor{inv_background inv_foreground}{^U} Colar txt     \textcolor{inv_background inv_foreground}{^T} Ortografia    \textcolor{inv_background inv_foreground}{^_} Ir p/ linha   \textcolor{inv_background inv_foreground}{M-E} Refazer\textcolor{inv_background inv_foreground}{M-6} Copiar txt
---- name: Atualiza a cache (equivalente a apt update)apt:update_cache: truecache_valid_time: 3600- name: "Atualiza o sistema operativo"apt:upgrade: safe- name: "Instalar as dependencias necessária ao programa leantime"apt:pkg:- mc- screen- git- apache2- mysql-server- php- php-mysql- php-ldap- php-cli- php-soap- php-json- graphviz- php-xml- php-gd- php-zip- libapache2-mod-php- php-dev- libmcrypt-dev- gcc- make- autoconf- libc-dev
[?12l[?25h[?25l - autoconf    - libc-dev  - pkg-config- pwgen- curl- unzip    - zip  - php-mbstring- expect- net-tools    - python3-mysqldb  - python3-apt- python3-pycurl
\textcolor{ansi36}{   # Fonte de inspiração: https://docs.ansible.com/ansible/latest/collections/ansible/builtin/service_module.html}- name: "Instala o serviço apache2 no arranque do sistema"service:name: apache2state: startedenabled: true
\textcolor{ansi36}{   # Fonte: https://www.mydailytutorials.com/using-ansible-find-module-search-filesfolder}- name: "Rotina Auxiliar: Procura o caminho do ficheiro php.ini do servidor apache2"find:paths: /etcpatterns: "php.ini"recurse: trueregister: caminho_php_ini
\textcolor{ansi36}{   # Fonte: https://docs.ansible.com/ansible/latest/collections/ansible/builtin/lineinfile_module.html}- name: "Ativa várias opções no ficheiro php.ini, utilizando o módulo lineinfile."ansible.builtin.lineinfile:path: "{{ caminho_php_ini.files[0].path }}"regexp: "{{ item.regexp  }}"line: "{{ item.line }}"
[?12l[?25h[?25lregexp: "{{ item.regexp  }}"line: "{{ item.line }}"with_items:regexp: "^file_uploads"  line: "file_uploads = 1"regexp: "^upload_max_filesize"  line: "upload_max_filesize = 1G"regexp: "^max_file_uploads"  line: "max_file_uploads = 20"regexp: "^post_max_size"  line: "post_max_size = 2G"regexp: "^memory_limit"  line: "memory_limit = 2G"- regexp: "^max_input_time" line: "max_input_time = 3600"
\textcolor{ansi36}{   # Fonte: https://docs.ansible.com/ansible/2.7/modules/file_module.html}- name: Cria a directoria temporária leantime e a directoria de apache leantimefile:path: "{{ item.path }}"state: directory mode: 0755    owner: www-data  group: www-datawith_items:- path: "/tmp/leantime"- path: "/var/www/leantime"
[?12l[?25h[?25l[?12l[?25h[?25l\textcolor{inv_background inv_foreground}{Modificado}
[?12l[?25h[?25l [?12l[?25h[?25l [?12l[?25h[?25l [?12l[?25h[?25l
\textcolor{ansi36}{   #}[?12l[?25h[?25l\textcolor{ansi36}{ }[?12l[?25h[?25l\textcolor{ansi36}{F}[?12l[?25h[?25l\textcolor{ansi36}{o}[?12l[?25h[?25l\textcolor{ansi36}{n}[?12l[?25h[?25l\textcolor{ansi36}{t}[?12l[?25h[?25l\textcolor{ansi36}{e}[?12l[?25h[?25l\textcolor{ansi36}{:}[?12l[?25h[?25l\textcolor{ansi36}{ }[?12l[?25h[?25l\textcolor{ansi36}{A}[?12l[?25h[?25l\textcolor{ansi36}{n}[?12l[?25h[?25l\textcolor{ansi36}{s}[?12l[?25h[?25l\textcolor{ansi36}{i}[?12l[?25h[?25l\textcolor{ansi36}{b}[?12l[?25h[?25l\textcolor{ansi36}{l}[?12l[?25h[?25l\textcolor{ansi36}{e}[?12l[?25h[?25l\textcolor{ansi36}{ }[?12l[?25h[?25l\textcolor{ansi36}{f}[?12l[?25h[?25l\textcolor{ansi36}{o
}[?12l[?25h[?25l\textcolor{ansi36}{r}[?12l[?25h[?25l\textcolor{ansi36}{ }[?12l[?25h[?25l\textcolor{ansi36}{D}[?12l[?25h[?25l\textcolor{ansi36}{e}[?12l[?25h[?25l\textcolor{ansi36}{v}[?12l[?25h[?25l\textcolor{ansi36}{O}[?12l[?25h[?25l\textcolor{ansi36}{p}[?12l[?25h[?25l\textcolor{ansi36}{s}[?12l[?25h[?25l\textcolor{ansi36}{ }[?12l[?25h[?25l\textcolor{ansi36}{-}[?12l[?25h[?25l\textcolor{ansi36}{ }[?12l[?25h[?25l\textcolor{ansi36}{S}[?12l[?25h[?25l\textcolor{ansi36}{e}[?12l[?25h[?25l\textcolor{ansi36}{r}[?12l[?25h[?25l\textcolor{ansi36}{v}[?12l[?25h[?25l\textcolor{ansi36}{e}[?12l[?25h[?25l\textcolor{ansi36}{r}[?12l[?25h[?25l\textcolor{ansi36}{ }[?12l[?25h[?25l\textcolor{ansi36}{a}[?12l[?25h[?25l\textcolor{ansi36}{n}[?12l[?25h[?25l\textcolor{ansi36}{d}[?12l[?25h[?25l\textcolor{ansi36}{ }[?12l[?25h[?25l\textcolor{ansi36}{c}[?12l[?25h[?25l\textcolor{ansi36}{o}[?12l[?25h[?25l\textcolor{ansi36}{n}[?12l[?25h[?25l\textcolor{ansi36}{f}[?12l[?25h[?25l\textcolor{ansi36}{i}[?12l[?25h[?25l\textcolor{ansi36}{g}[?12l[?25h[?25l\textcolor{ansi36}{u}[?12l[?25h[?25l\textcolor{ansi36}{r}[?12l[?25h[?25l\textcolor{ansi36}{a}[?12l[?25h[?25l\textcolor{ansi36}{t}[?12l[?25h[?25l\textcolor{ansi36}{i}[?12l[?25h[?25l\textcolor{ansi36}{o}[?12l[?25h[?25l\textcolor{ansi36}{n}[?12l[?25h[?25l\textcolor{ansi36}{ }[?12l[?25h[?25l\textcolor{ansi36}{m}[?12l[?25h[?25l\textcolor{ansi36}{a}[?12l[?25h[?25l\textcolor{ansi36}{n}[?12l[?25h[?25l\textcolor{ansi36}{a}[?12l[?25h[?25l\textcolor{ansi36}{g}[?12l[?25h[?25l\textcolor{ansi36}{e}[?12l[?25h[?25l\textcolor{ansi36}{m}[?12l[?25h[?25l\textcolor{ansi36}{e}[?12l[?25h[?25l\textcolor{ansi36}{n}[?12l[?25h[?25l\textcolor{ansi36}{t}[?12l[?25h[?25l\textcolor{ansi36}{ }[?12l[?25h[?25l\textcolor{ansi36}{f}[?12l[?25h[?25l\textcolor{ansi36}{o}[?12l[?25h[?25l\textcolor{ansi36}{r}[?12l[?25h[?25l\textcolor{ansi36}{ }[?12l[?25h[?25l\textcolor{ansi36}{h}[?12l[?25h[?25l\textcolor{ansi36}{u}[?12l[?25h[?25l\textcolor{ansi36}{m}[?12l[?25h[?25l\textcolor{ansi36}{a}[?12l[?25h[?25l\textcolor{ansi36}{n}[?12l[?25h[?25l\textcolor{ansi36}{s}[?12l[?25h[?25l\textcolor{ansi36}{ }[?12l[?25h[?25l\textcolor{ansi36}{d}[?12l[?25h[?25l\textcolor{ansi36}{e}[?12l[?25h[?25l\textcolor{ansi36}{ }[?12l[?25h[?25l\textcolor{ansi36}{J}[?12l[?25h[?25l\textcolor{ansi36}{e}[?12l[?25h[?25l\textcolor{ansi36}{f}[?12l[?25h[?25l\textcolor{ansi36}{f}[?12l[?25h[?25l\textcolor{ansi36}{ }[?12l[?25h[?25l\textcolor{ansi36}{G}[?12l[?25h[?25l\textcolor{ansi36}{e}[?12l[?25h[?25l\textcolor{ansi36}{e}[?12l[?25h[?25l\textcolor{ansi36}{r}[?12l[?25h[?25l\textcolor{ansi36}{l}[?12l[?25h[?25l\textcolor{ansi36}{i}[?12l[?25h[?25l\textcolor{ansi36}{n}[?12l[?25h[?25l\textcolor{ansi36}{g}[?12l[?25h[?25l\textcolor{ansi36}{,}[?12l[?25h[?25l\textcolor{ansi36}{ }[?12l[?25h[?25l\textcolor{ansi36}{p}[?12l[?25h[?25l\textcolor{ansi36}{á}[?12l[?25h[?25l\textcolor{ansi36}{g}[?12l[?25h[?25l\textcolor{ansi36}{i}[?12l[?25h[?25l\textcolor{ansi36}{n}[?12l[?25h[?25l\textcolor{ansi36}{a}[?12l[?25h[?25l\textcolor{ansi36}{ }[?12l[?25h[?25l\textcolor{ansi36}{9}[?12l[?25h[?25l\textcolor{ansi36}{8}[?12l[?25h[?25l
[?12l[?25h[?25l [?12l[?25h[?25l [?12l[?25h[?25l [?12l[?25h[?25l-[?12l[?25h[?25l [?12l[?25h[?25ln[?12l[?25h[?25la[?12l[?25h[?25lm[?12l[?25h[?25le[?12l[?25h[?25l:[?12l[?25h[?25l [?12l[?25h[?25l"[?12l[?25h[?25lC[?12l[?25h[?25lr[?12l[?25h[?25li[?12l[?25h[?25la[?12l[?25h[?25l [?12l[?25h[?25la[?12l[?25h[?25l [?12l[?25h[?25lb[?12l[?25h[?25la[?12l[?25h[?25ls[?12l[?25h[?25le[?12l[?25h[?25l [?12l[?25h[?25ld[?12l[?25h[?25le[?12l[?25h[?25l [?12l[?25h[?25ld[?12l[?25h[?25la[?12l[?25h[?25ld[?12l[?25h[?25lo[?12l[?25h[?25ls[?12l[?25h[?25l [?12l[?25h[?25ld[?12l[?25h[?25le[?12l[?25h[?25l [?12l[?25h[?25lm[?12l[?25h[?25ly[?12l[?25h[?25ls[?12l[?25h[?25lq[?12l[?25h[?25ll[?12l[?25h[?25l [?12l[?25h[?25lp[?12l[?25h[?25la[?12l[?25h[?25lr[?12l[?25h[?25la[?12l[?25h[?25l [?12l[?25h[?25lo[?12l[?25h[?25l [?12l[?25h[?25ll[?12l[?25h[?25le[?12l[?25h[?25la[?12l[?25h[?25ln[?12l[?25h[?25lt[?12l[?25h[?25li[?12l[?25h[?25lm[?12l[?25h[?25le[?12l[?25h[?25l"[?12l[?25h[?25l
[?12l[?25h[?25l [?12l[?25h[?25l [?12l[?25h[?25l [?12l[?25h[?25l [?12l[?25h[?25l [?12l[?25h[?25lm[?12l[?25h[?25ly[?12l[?25h[?25ls[?12l[?25h[?25lq[?12l[?25h[?25ll[?12l[?25h[?25l_[?12l[?25h[?25ld[?12l[?25h[?25lb[?12l[?25h[?25l:[?12l[?25h[?25l [?12l[?25h[?25l"[?12l[?25h[?25ld[?12l[?25h[?25lb[?12l[?25h[?25l=[?12l[?25h[?25ll[?12l[?25h[?25le[?12l[?25h[?25la[?12l[?25h[?25ln[?12l[?25h[?25lt[?12l[?25h[?25li[?12l[?25h[?25lm[?12l[?25h[?25le[?12l[?25h[?25l_[?12l[?25h[?25ld[?12l[?25h[?25la[?12l[?25h[?25lt[?12l[?25h[?25la[?12l[?25h[?25lb[?12l[?25h[?25la[?12l[?25h[?25ls[?12l[?25h[?25le[?12l[?25h[?25l [?12l[?25h[?25ls[?12l[?25h[?25lt[?12l[?25h[?25la[?12l[?25h[?25lt[?12l[?25h[?25le[?12l[?25h[?25l=[?12l[?25h[?25lp[?12l[?25h[?25lr[?12l[?25h[?25le[?12l[?25h[?25ls[?12l[?25h[?25le[?12l[?25h[?25ln[?12l[?25h[?25lt[?12l[?25h[?25l"[?12l[?25h[?25l
[?12l[?25h[?25l[?12l[?25h[?25l [?12l[?25h[?25l [?12l[?25h[?25l [?12l[?25h[?25l-[?12l[?25h[?25l [?12l[?25h[?25ln[?12l[?25h[?25la[?12l[?25h[?25lm[?12l[?25h[?25le[?12l[?25h[?25l:[?12l[?25h[?25l [?12l[?25h[?25l"[?12l[?25h[?25lC[?12l[?25h[?25lr[?12l[?25h[?25li[?12l[?25h[?25la[?12l[?25h[?25l [?12l[?25h[?25lo[?12l[?25h[?25l [?12l[?25h[?25lu[?12l[?25h[?25lt[?12l[?25h[?25li[?12l[?25h[?25ll[?12l[?25h[?25li[?12l[?25h[?25lz[?12l[?25h[?25la[?12l[?25h[?25ld[?12l[?25h[?25lo[?12l[?25h[?25lr[?12l[?25h[?25l [?12l[?25h[?25lp[?12l[?25h[?25la[?12l[?25h[?25lr[?12l[?25h[?25la[?12l[?25h[?25l [?12l[?25h[?25la[?12l[?25h[?25l [?12l[?25h[?25lb[?12l[?25h[?25la[?12l[?25h[?25ls[?12l[?25h[?25le[?12l[?25h[?25l [?12l[?25h[?25ld[?12l[?25h[?25le[?12l[?25h[?25l [?12l[?25h[?25ld[?12l[?25h[?25la[?12l[?25h[?25ld[?12l[?25h[?25lo[?12l[?25h[?25ls[?12l[?25h[?25l [?12l[?25h[?25ll[?12l[?25h[?25le[?12l[?25h[?25la[?12l[?25h[?25ln[?12l[?25h[?25lt[?12l[?25h[?25li[?12l[?25h[?25lm[?12l[?25h[?25le[?12l[?25h[?25l"[?12l[?25h[?25l
[?12l[?25h[?25l [?12l[?25h[?25l [?12l[?25h[?25l [?12l[?25h[?25l [?12l[?25h[?25l [?12l[?25h[?25lm[?12l[?25h[?25ly[?12l[?25h[?25ls[?12l[?25h[?25lq[?12l[?25h[?25ll[?12l[?25h[?25l_[?12l[?25h[?25lu[?12l[?25h[?25ls[?12l[?25h[?25le[?12l[?25h[?25lr[?12l[?25h[?25l:[?12l[?25h[?25l
[?12l[?25h[?25l [?12l[?25h[?25l [?12l[?25h[?25l [?12l[?25h[?25l [?12l[?25h[?25l [?12l[?25h[?25l [?12l[?25h[?25l [?12l[?25h[?25ln[?12l[?25h[?25la[?12l[?25h[?25lm[?12l[?25h[?25le[?12l[?25h[?25l:[?12l[?25h[?25l [?12l[?25h[?25ll[?12l[?25h[?25le[?12l[?25h[?25la[?12l[?25h[?25ln[?12l[?25h[?25lt[?12l[?25h[?25li[?12l[?25h[?25lm[?12l[?25h[?25le[?12l[?25h[?25lD[?12l[?25h[?25lB[?12l[?25h[?25la[?12l[?25h[?25ld[?12l[?25h[?25lm[?12l[?25h[?25li[?12l[?25h[?25ln[?12l[?25h[?25l78

[?12l[?25h[?25l [?12l[?25h[?25l [?12l[?25h[?25l [?12l[?25h[?25l [?12l[?25h[?25l [?12l[?25h[?25l [?12l[?25h[?25l [?12l[?25h[?25lp[?12l[?25h[?25la[?12l[?25h[?25ls[?12l[?25h[?25ls[?12l[?25h[?25lw[?12l[?25h[?25lo[?12l[?25h[?25lr[?12l[?25h[?25ld[?12l[?25h[?25l:[?12l[?25h[?25l [?12l[?25h[?25l"[?12l[?25h[?25l#[?12l[?25h[?25lS[?12l[?25h[?25l3[?12l[?25h[?25lg[?12l[?25h[?25lr[?12l[?25h[?25l3[?12l[?25h[?25ld[?12l[?25h[?25l0[?12l[?25h[?25lS[?12l[?25h[?25l3[?12l[?25h[?25lc[?12l[?25h[?25lr[?12l[?25h[?25l3[?12l[?25h[?25lt[?12l[?25h[?25l0[?12l[?25h[?25l#[?12l[?25h[?25l"[?12l[?25h[?25l78

[?12l[?25h[?25l [?12l[?25h[?25l [?12l[?25h[?25l [?12l[?25h[?25l [?12l[?25h[?25l [?12l[?25h[?25l [?12l[?25h[?25l [?12l[?25h[?25lp[?12l[?25h[?25lr[?12l[?25h[?25li[?12l[?25h[?25lv[?12l[?25h[?25l:[?12l[?25h[?25l [?12l[?25h[?25l"[?12l[?25h[?25ll[?12l[?25h[?25le[?12l[?25h[?25la[?12l[?25h[?25ln[?12l[?25h[?25lt[?12l[?25h[?25li[?12l[?25h[?25lm[?12l[?25h[?25le[?12l[?25h[?25l_[?12l[?25h[?25ld[?12l[?25h[?25la[?12l[?25h[?25lt[?12l[?25h[?25la[?12l[?25h[?25lb[?12l[?25h[?25la[?12l[?25h[?25ls[?12l[?25h[?25le[?12l[?25h[?25l.[?12l[?25h[?25l*[?12l[?25h[?25l:[?12l[?25h[?25lA[?12l[?25h[?25lL[?12l[?25h[?25lL[?12l[?25h[?25l"[?12l[?25h[?25l78

[?12l[?25h[?25l [?12l[?25h[?25l [?12l[?25h[?25l [?12l[?25h[?25l [?12l[?25h[?25l [?12l[?25h[?25l [?12l[?25h[?25l [?12l[?25h[?25lh[?12l[?25h[?25lo[?12l[?25h[?25ls[?12l[?25h[?25lt[?12l[?25h[?25l:[?12l[?25h[?25l [?12l[?25h[?25ll[?12l[?25h[?25lo[?12l[?25h[?25lc[?12l[?25h[?25la[?12l[?25h[?25ll[?12l[?25h[?25lh[?12l[?25h[?25lo[?12l[?25h[?25ls[?12l[?25h[?25lt[?12l[?25h[?25l78

[?12l[?25h[?25l [?12l[?25h[?25l [?12l[?25h[?25l [?12l[?25h[?25l [?12l[?25h[?25l [?12l[?25h[?25l [?12l[?25h[?25l [?12l[?25h[?25ls[?12l[?25h[?25lt[?12l[?25h[?25la[?12l[?25h[?25lt[?12l[?25h[?25le[?12l[?25h[?25l:[?12l[?25h[?25l [?12l[?25h[?25lp[?12l[?25h[?25lr[?12l[?25h[?25le[?12l[?25h[?25ls[?12l[?25h[?25le[?12l[?25h[?25ln[?12l[?25h[?25lt[?12l[?25h[?25l78

[?12l[?25h[?25l78
[?12l[?25h[?25l         \textcolor{inv_background inv_foreground}{M-D} Formato DOS\textcolor{inv_background inv_foreground}{M-A} Anexar\textcolor{inv_background inv_foreground}{B} Segurança\textcolor{inv_background inv_foreground}{C} Cancelar           \textcolor{inv_background inv_foreground}{M-M} Formato Mac\textcolor{inv_background inv_foreground}{M-P} Prepor\textcolor{inv_background inv_foreground}{^T} P/ ficheiros
[?12l[?25h[?25l[?12l[?25h[?25l[?12l[?25h[?25l[?12l[?25h[?25l[?12l[?25h[?25l[?12l[?25h[?25l[?12l[?25h[?25l[?12l[?25h[?25l[?12l[?25h[?25l[?12l[?25h[?25l[?12l[?25h[?25l78
[?12l[?25h[?25l[?12l[?25h[?25l[?12l[?25h[?25l[?12l[?25h[?25l\textcolor{inv_background inv_foreground}{Modificado}
[?12l[?25h[?25l[?12l[?25h[?25l         \textcolor{inv_background inv_foreground}{M-D} Formato DOS\textcolor{inv_background inv_foreground}{M-A} Anexar\textcolor{inv_background inv_foreground}{B} Segurança\textcolor{inv_background inv_foreground}{C} Cancelar           \textcolor{inv_background inv_foreground}{M-M} Formato Mac\textcolor{inv_background inv_foreground}{M-P} Prepor\textcolor{inv_background inv_foreground}{^T} P/ ficheiros
\textcolor{inv_background inv_foreground}{Nome do ficheiro onde escrever: tasks/tasks.yml                                                                                           }[?12l[?25h[?25l \textcolor{inv_background inv_foreground}{[ A escrever... ]}\textcolor{inv_background inv_foreground}{          }\textcolor{inv_background inv_foreground}{[ 105 linhas escritas ]}\textcolor{inv_background inv_foreground}{^O} Gravar\textcolor{inv_background inv_foreground}{^W} Procurar      \textcolor{inv_background inv_foreground}{^K} Cortar txt    \textcolor{inv_background inv_foreground}{^J} Justificar    \textcolor{inv_background inv_foreground}{^C} Pos cursor\textcolor{inv_background inv_foreground}{U} Desfazer     \textcolor{inv_background inv_foreground}{M-A} Marcar txt\textcolor{inv_background inv_foreground}{X} Sair    \textcolor{inv_background inv_foreground}{^R} Carregar\textcolor{inv_background inv_foreground}{^\} Substituir    \textcolor{inv_background inv_foreground}{^U} Colar txt     \textcolor{inv_background inv_foreground}{^T} Ortografia    \textcolor{inv_background inv_foreground}{^_} Ir p/ linha   \textcolor{inv_background inv_foreground}{M-E} Refazer      \textcolor{inv_background inv_foreground}{M-6} Copiar txt
[?12l[?25h[?25l[?12l[?25h[?1049l
[?1l>[?2004l]0;admlocal@devOps: ~/leantime_ansibleadmlocal@devOps:~/leantime_ansible$ anano vars/main.yml 
[?2004h[?1049h[?7h[?1h=[?1h=[?25l\textcolor{inv_background inv_foreground}{[ A ler... ]}\textcolor{inv_background inv_foreground}{[ 2 linhas lidas ]}\textcolor{inv_background inv_foreground}{  GNU nano 4.8                                                 vars/main.yml                                                              }
\textcolor{inv_background inv_foreground}{^G} Ajuda\textcolor{inv_background inv_foreground}{^O} Gravar\textcolor{inv_background inv_foreground}{^W} Procurar\textcolor{inv_background inv_foreground}{^K} Cortar txt    \textcolor{inv_background inv_foreground}{^J} Justificar    \textcolor{inv_background inv_foreground}{^C} Pos cursor    \textcolor{inv_background inv_foreground}{M-U} Desfazer     \textcolor{inv_background inv_foreground}{M-A} Marcar txt
\textcolor{inv_background inv_foreground}{^X} Sair\textcolor{inv_background inv_foreground}{^R} Carregar\textcolor{inv_background inv_foreground}{^\} Substituir    \textcolor{inv_background inv_foreground}{^U} Colar txt     \textcolor{inv_background inv_foreground}{^T} Ortografia    \textcolor{inv_background inv_foreground}{^_} Ir p/ linha   \textcolor{inv_background inv_foreground}{M-E} Refazer\textcolor{inv_background inv_foreground}{M-6} Copiar txt
---
}[?12l[?25h[?25l[?12l[?25h[?25l\textcolor{inv_background inv_foreground}{Modificado}
[?12l[?25h[?25l         \textcolor{inv_background inv_foreground}{M-D} Formato DOS\textcolor{inv_background inv_foreground}{M-A} Anexar\textcolor{inv_background inv_foreground}{B} Segurança\textcolor{inv_background inv_foreground}{C} Cancelar           \textcolor{inv_background inv_foreground}{M-M} Formato Mac\textcolor{inv_background inv_foreground}{M-P} Prepor\textcolor{inv_background inv_foreground}{^T} P/ ficheiros
\textcolor{inv_background inv_foreground}{Nome do ficheiro onde escrever: vars/main.yml                                                                                             }[?12l[?25h[?25l \textcolor{inv_background inv_foreground}{[ A escrever... ]}\textcolor{inv_background inv_foreground}{          }\textcolor{inv_background inv_foreground}{[ 1 linha escrita ]}\textcolor{inv_background inv_foreground}{^O} Gravar\textcolor{inv_background inv_foreground}{^W} Procurar      \textcolor{inv_background inv_foreground}{^K} Cortar txt    \textcolor{inv_background inv_foreground}{^J} Justificar    \textcolor{inv_background inv_foreground}{^C} Pos cursor\textcolor{inv_background inv_foreground}{U} Desfazer     \textcolor{inv_background inv_foreground}{M-A} Marcar txt\textcolor{inv_background inv_foreground}{X} Sair    \textcolor{inv_background inv_foreground}{^R} Carregar\textcolor{inv_background inv_foreground}{^\} Substituir    \textcolor{inv_background inv_foreground}{^U} Colar txt     \textcolor{inv_background inv_foreground}{^T} Ortografia    \textcolor{inv_background inv_foreground}{^_} Ir p/ linha   \textcolor{inv_background inv_foreground}{M-E} Refazer      \textcolor{inv_background inv_foreground}{M-6} Copiar txt
[?12l[?25h[?25l\textcolor{inv_background inv_foreground}{Modificado}
[?12l[?25h[?25l [?12l[?25h[?25l [?12l[?25h[?25l [?12l[?25h[?25l [?12l[?25h[?25ll[?12l[?25h[?25lo[?12l[?25h[?25lg[?12l[?25h[?25li[?12l[?25h[?25ln[?12l[?25h[?25l_[?12l[?25h[?25ll[?12l[?25h[?25le[?12l[?25h[?25la[?12l[?25h[?25ln[?12l[?25h[?25lt[?12l[?25h[?25li[?12l[?25h[?25lm[?12l[?25h[?25le[?12l[?25h[?25l:[?12l[?25h[?25l [?12l[?25h[?25ll[?12l[?25h[?25le[?12l[?25h[?25la[?12l[?25h[?25ln[?12l[?25h[?25lt
[?12l[?25h[?25li[?12l[?25h[?25lm[?12l[?25h[?25le[?12l[?25h[?25lD[?12l[?25h[?25lB[?12l[?25h[?25la[?12l[?25h[?25ld[?12l[?25h[?25lm[?12l[?25h[?25li[?12l[?25h[?25ln[?12l[?25h[?25l
[?12l[?25h[?25l [?12l[?25h[?25l [?12l[?25h[?25l [?12l[?25h[?25l [?12l[?25h[?25lp[?12l[?25h[?25la[?12l[?25h[?25ls[?12l[?25h[?25ls[?12l[?25h[?25lw[?12l[?25h[?25lo[?12l[?25h[?25lr[?12l[?25h[?25ld[?12l[?25h[?25l_[?12l[?25h[?25ll[?12l[?25h[?25le[?12l[?25h[?25la[?12l[?25h[?25ln[?12l[?25h[?25lt[?12l[?25h[?25li[?12l[?25h[?25lm[?12l[?25h[?25le[?12l[?25h[?25l:[?12l[?25h[?25l [?12l[?25h[?25l"[?12l[?25h[?25l#[?12l[?25h[?25lS[?12l[?25h[?25l3[?12l[?25h[?25lg[?12l[?25h[?25lr[?12l[?25h[?25l3[?12l[?25h[?25ld[?12l[?25h[?25l0[?12l[?25h[?25lS[?12l[?25h[?25l3[?12l[?25h[?25lc[?12l[?25h[?25lr[?12l[?25h[?25l3[?12l[?25h[?25lt[?12l[?25h[?25l0[?12l[?25h[?25l#[?12l[?25h[?25l"[?12l[?25h[?25l
[?12l[?25h[?25l         \textcolor{inv_background inv_foreground}{M-D} Formato DOS\textcolor{inv_background inv_foreground}{M-A} Anexar\textcolor{inv_background inv_foreground}{B} Segurança\textcolor{inv_background inv_foreground}{C} Cancelar           \textcolor{inv_background inv_foreground}{M-M} Formato Mac\textcolor{inv_background inv_foreground}{M-P} Prepor\textcolor{inv_background inv_foreground}{^T} P/ ficheiros
\textcolor{inv_background inv_foreground}{Nome do ficheiro onde escrever: vars/main.yml                                                                                             }[?12l[?25h[?25l \textcolor{inv_background inv_foreground}{[ A escrever... ]}\textcolor{inv_background inv_foreground}{          }\textcolor{inv_background inv_foreground}{[ 5 linhas escritas ]}\textcolor{inv_background inv_foreground}{^O} Gravar\textcolor{inv_background inv_foreground}{^W} Procurar      \textcolor{inv_background inv_foreground}{^K} Cortar txt    \textcolor{inv_background inv_foreground}{^J} Justificar    \textcolor{inv_background inv_foreground}{^C} Pos cursor\textcolor{inv_background inv_foreground}{U} Desfazer     \textcolor{inv_background inv_foreground}{M-A} Marcar txt\textcolor{inv_background inv_foreground}{X} Sair    \textcolor{inv_background inv_foreground}{^R} Carregar\textcolor{inv_background inv_foreground}{^\} Substituir    \textcolor{inv_background inv_foreground}{^U} Colar txt     \textcolor{inv_background inv_foreground}{^T} Ortografia    \textcolor{inv_background inv_foreground}{^_} Ir p/ linha   \textcolor{inv_background inv_foreground}{M-E} Refazer      \textcolor{inv_background inv_foreground}{M-6} Copiar txt
[?12l[?25h[?25l[?12l[?25h[?1049l
[?1l>[?2004l]0;admlocal@devOps: ~/leantime_ansibleadmlocal@devOps:~/leantime_ansible$ nano vars/main.yml [2@tasks/tasks
[?2004h[?1049h[?7h[?1h=[?1h=[?25l\textcolor{inv_background inv_foreground}{[ A ler... ]}\textcolor{inv_background inv_foreground}{[ 105 linhas lidas ]}\textcolor{inv_background inv_foreground}{  GNU nano 4.8                                                tasks/tasks.yml                                                             }
\textcolor{inv_background inv_foreground}{^G} Ajuda\textcolor{inv_background inv_foreground}{^O} Gravar\textcolor{inv_background inv_foreground}{^W} Procurar\textcolor{inv_background inv_foreground}{^K} Cortar txt    \textcolor{inv_background inv_foreground}{^J} Justificar    \textcolor{inv_background inv_foreground}{^C} Pos cursor    \textcolor{inv_background inv_foreground}{M-U} Desfazer     \textcolor{inv_background inv_foreground}{M-A} Marcar txt
\textcolor{inv_background inv_foreground}{^X} Sair\textcolor{inv_background inv_foreground}{^R} Carregar\textcolor{inv_background inv_foreground}{^\} Substituir    \textcolor{inv_background inv_foreground}{^U} Colar txt     \textcolor{inv_background inv_foreground}{^T} Ortografia    \textcolor{inv_background inv_foreground}{^_} Ir p/ linha   \textcolor{inv_background inv_foreground}{M-E} Refazer\textcolor{inv_background inv_foreground}{M-6} Copiar txt
---- name: Atualiza a cache (equivalente a apt update)apt:update_cache: truecache_valid_time: 3600- name: "Atualiza o sistema operativo"apt:upgrade: safe- name: "Instalar as dependencias necessária ao programa leantime"apt:pkg:- mc- screen- git- apache2- mysql-server- php- php-mysql- php-ldap- php-cli- php-soap- php-json- graphviz- php-xml- php-gd- php-zip- libapache2-mod-php- php-dev- libmcrypt-dev- gcc- make- autoconf- libc-dev
[?12l[?25h[?25l - autoconf    - libc-dev  - pkg-config- pwgen- curl- unzip    - zip  - php-mbstring- expect- net-tools    - python3-mysqldb  - python3-apt- python3-pycurl
\textcolor{ansi36}{   # Fonte de inspiração: https://docs.ansible.com/ansible/latest/collections/ansible/builtin/service_module.html}- name: "Instala o serviço apache2 no arranque do sistema"service:name: apache2state: startedenabled: true
\textcolor{ansi36}{   # Fonte: https://www.mydailytutorials.com/using-ansible-find-module-search-filesfolder}- name: "Rotina Auxiliar: Procura o caminho do ficheiro php.ini do servidor apache2"find:paths: /etcpatterns: "php.ini"recurse: trueregister: caminho_php_ini
\textcolor{ansi36}{   # Fonte: https://docs.ansible.com/ansible/latest/collections/ansible/builtin/lineinfile_module.html}- name: "Ativa várias opções no ficheiro php.ini, utilizando o módulo lineinfile."ansible.builtin.lineinfile:path: "{{ caminho_php_ini.files[0].path }}"regexp: "{{ item.regexp  }}"line: "{{ item.line }}"
[?12l[?25h[?25lregexp: "{{ item.regexp  }}"line: "{{ item.line }}"with_items:regexp: "^file_uploads"  line: "file_uploads = 1"regexp: "^upload_max_filesize"  line: "upload_max_filesize = 1G"regexp: "^max_file_uploads"  line: "max_file_uploads = 20"regexp: "^post_max_size"  line: "post_max_size = 2G"regexp: "^memory_limit"  line: "memory_limit = 2G"- regexp: "^max_input_time" line: "max_input_time = 3600"
\textcolor{ansi36}{   # Fonte: https://docs.ansible.com/ansible/2.7/modules/file_module.html}- name: Cria a directoria temporária leantime e a directoria de apache leantimefile:path: "{{ item.path }}"state: directory mode: 0755    owner: www-data  group: www-datawith_items:- path: "/tmp/leantime"- path: "/var/www/leantime"
\textcolor{ansi36}{   # Fonte: Ansible for DevOps - Server and configuration management for humans de Jeff Geerling, página 98
}   - name: "Cria a base de dados de mysql para o leantime"  mysql_db: "db=leantime_database state=present"
- name: "Cria o utilizador para a base de dados leantime"mysql_user:name: leantimeDBadmin
[?12l[?25h[?25lmysql_user:name: leantimeDBadmin  password: "#S3gr3d0S3cr3t0#"priv: "leantime_database.*:ALL"host: localhoststate: present
[?12l[?25h[?25l\textcolor{inv_background inv_foreground}{Modificado}
[?12l[?25h[?25l
[?12l[?25h[?25l\textcolor{inv_background inv_foreground}{Gravar buffer modificado?                                                                                                                  S} Sim
\textcolor{inv_background inv_foreground}{ N} Não \textcolor{inv_background inv_foreground}{^C} Cancelar[?12l[?25h[?25l
[?12l[?25h[?1049l
[?1l>[?2004l]0;admlocal@devOps: ~/leantime_ansibleadmlocal@devOps:~/leantime_ansible$ nano tasks/tasks.yml 
[?2004h[?1049h[?7h[?1h=[?1h=[?25l\textcolor{inv_background inv_foreground}{[ A ler... ]}\textcolor{inv_background inv_foreground}{[ 105 linhas lidas ]}\textcolor{inv_background inv_foreground}{  GNU nano 4.8                                                tasks/tasks.yml                                                             }
\textcolor{inv_background inv_foreground}{^G} Ajuda\textcolor{inv_background inv_foreground}{^O} Gravar\textcolor{inv_background inv_foreground}{^W} Procurar\textcolor{inv_background inv_foreground}{^K} Cortar txt    \textcolor{inv_background inv_foreground}{^J} Justificar    \textcolor{inv_background inv_foreground}{^C} Pos cursor    \textcolor{inv_background inv_foreground}{M-U} Desfazer     \textcolor{inv_background inv_foreground}{M-A} Marcar txt
\textcolor{inv_background inv_foreground}{^X} Sair\textcolor{inv_background inv_foreground}{^R} Carregar\textcolor{inv_background inv_foreground}{^\} Substituir    \textcolor{inv_background inv_foreground}{^U} Colar txt     \textcolor{inv_background inv_foreground}{^T} Ortografia    \textcolor{inv_background inv_foreground}{^_} Ir p/ linha   \textcolor{inv_background inv_foreground}{M-E} Refazer\textcolor{inv_background inv_foreground}{M-6} Copiar txt
---- name: Atualiza a cache (equivalente a apt update)apt:update_cache: truecache_valid_time: 3600- name: "Atualiza o sistema operativo"apt:upgrade: safe- name: "Instalar as dependencias necessária ao programa leantime"apt:pkg:- mc- screen- git- apache2- mysql-server- php- php-mysql- php-ldap- php-cli- php-soap- php-json- graphviz- php-xml- php-gd- php-zip- libapache2-mod-php- php-dev- libmcrypt-dev- gcc- make- autoconf- libc-dev
[?12l[?25h[?25l - autoconf    - libc-dev  - pkg-config- pwgen- curl- unzip    - zip  - php-mbstring- expect- net-tools    - python3-mysqldb  - python3-apt- python3-pycurl
\textcolor{ansi36}{   # Fonte de inspiração: https://docs.ansible.com/ansible/latest/collections/ansible/builtin/service_module.html}- name: "Instala o serviço apache2 no arranque do sistema"service:name: apache2state: startedenabled: true
\textcolor{ansi36}{   # Fonte: https://www.mydailytutorials.com/using-ansible-find-module-search-filesfolder}- name: "Rotina Auxiliar: Procura o caminho do ficheiro php.ini do servidor apache2"find:paths: /etcpatterns: "php.ini"recurse: trueregister: caminho_php_ini
\textcolor{ansi36}{   # Fonte: https://docs.ansible.com/ansible/latest/collections/ansible/builtin/lineinfile_module.html}- name: "Ativa várias opções no ficheiro php.ini, utilizando o módulo lineinfile."ansible.builtin.lineinfile:path: "{{ caminho_php_ini.files[0].path }}"regexp: "{{ item.regexp  }}"line: "{{ item.line }}"
[?12l[?25h[?25lregexp: "{{ item.regexp  }}"line: "{{ item.line }}"with_items:regexp: "^file_uploads"  line: "file_uploads = 1"regexp: "^upload_max_filesize"  line: "upload_max_filesize = 1G"regexp: "^max_file_uploads"  line: "max_file_uploads = 20"regexp: "^post_max_size"  line: "post_max_size = 2G"regexp: "^memory_limit"  line: "memory_limit = 2G"- regexp: "^max_input_time" line: "max_input_time = 3600"
\textcolor{ansi36}{   # Fonte: https://docs.ansible.com/ansible/2.7/modules/file_module.html}- name: Cria a directoria temporária leantime e a directoria de apache leantimefile:path: "{{ item.path }}"state: directory mode: 0755    owner: www-data  group: www-datawith_items:- path: "/tmp/leantime"- path: "/var/www/leantime"
\textcolor{ansi36}{   # Fonte: Ansible for DevOps - Server and configuration management for humans de Jeff Geerling, página 98
}   - name: "Cria a base de dados de mysql para o leantime"  mysql_db: "db=leantime_database state=present"
- name: "Cria o utilizador para a base de dados leantime"mysql_user:name: leantimeDBadmin
[?12l[?25h[?25lmysql_user:name: leantimeDBadmin  password: "#S3gr3d0S3cr3t0#"priv: "leantime_database.*:ALL"host: localhoststate: present
[?12l[?25h[?25l\textcolor{inv_background inv_foreground}{Modificado}
[?12l[?25h[?25l
[?12l[?25h[?25l [?12l[?25h[?25ll[?12l[?25h[?25lo[?12l[?25h[?25lg[?12l[?25h[?25li[?12l[?25h[?25ln[?12l[?25h[?25l_[?12l[?25h[?25ll[?12l[?25h[?25le[?12l[?25h[?25la[?12l[?25h[?25ln[?12l[?25h[?25lt[?12l[?25h[?25li[?12l[?25h[?25lm[?12l[?25h[?25le[?12l[?25h[?25l [?12l[?25h[?25l}[?12l[?25h[?25l}[?12l[?25h[?25l"[?12l[?25h[?25l78
[?12l[?25h[?25l
[?12l[?25h[?25l
[?12l[?25h[?25l\textcolor{inv_background inv_foreground}{[ Entrada literal ]}[?1l>[?12l[?25h
[?25l[?1h=[?1h= [?12l[?25h[?25l [?12l[?25h[?25l [?12l[?25h[?25l [?12l[?25h[?25l [?12l[?25h[?25l [?12l[?25h[?25l [?12l[?25h[?25ln[?12l[?25h[?25la[?12l[?25h[?25lm[?12l[?25h[?25le[?12l[?25h[?25l:[?12l[?25h[?25l [?12l[?25h[?25l"[?12l[?25h[?25l{[?12l[?25h[?25l{[?12l[?25h[?25l [?12l[?25h[?25ll[?12l[?25h[?25lo[?12l[?25h[?25lg[?12l[?25h[?25li[?12l[?25h[?25ln[?12l[?25h[?25l_[?12l[?25h[?25ll[?12l[?25h[?25le[?12l[?25h[?25la[?12l[?25h[?25ln[?12l[?25h[?25lt[?12l[?25h[?25li[?12l[?25h[?25lm[?12l[?25h[?25le[?12l[?25h[?25l [?12l[?25h[?25l}[?12l[?25h[?25l}[?12l[?25h[?25l"[?12l[?25h[?25l78
M[?12l[?25h[?25l [?12l[?25h[?25l [?12l[?25h[?25l [?12l[?25h[?25l [?12l[?25h[?25l [?12l[?25h[?25l [?12l[?25h[?25l [?12l[?25h[?25lp[?12l[?25h[?25la[?12l[?25h[?25ls[?12l[?25h[?25ls[?12l[?25h[?25lw[?12l[?25h[?25lo[?12l[?25h[?25lr[?12l[?25h[?25ld[?12l[?25h[?25l:[?12l[?25h[?25l [?12l[?25h[?25l"[?12l[?25h[?25l{[?12l[?25h[?25l{[?12l[?25h[?25l [?12l[?25h[?25lp[?12l[?25h[?25la[?12l[?25h[?25ls[?12l[?25h[?25ls[?12l[?25h[?25lw[?12l[?25h[?25lo[?12l[?25h[?25lr[?12l[?25h[?25ld[?12l[?25h[?25l_[?12l[?25h[?25ll[?12l[?25h[?25le[?12l[?25h[?25la[?12l[?25h[?25ln[?12l[?25h[?25lt[?12l[?25h[?25li[?12l[?25h[?25lm[?12l[?25h[?25le[?12l[?25h[?25l [?12l[?25h[?25l}[?12l[?25h[?25l}[?12l[?25h[?25l"[?12l[?25h[?25l78
M[?12l[?25h[?25l
[?12l[?25h[?25l[?12l[?25h[?25l[?12l[?25h[?25l78
M- name: "Cria o utilizador para a base de dados leantime"[?12l[?25h[?25l78
M[?12l[?25h[?25l78Mmysql_db: "db=leantime_database state=present"[?12l[?25h[?25l78
M- name: "Cria a base de dados de mysql para o leantime"[?12l[?25h[?25l78
M\textcolor{ansi36}{   # Fonte: Ansible for DevOps - Server and configuration management for humans de Jeff Geerling, página 98}[?12l[?25h[?25l78
M[?12l[?25h[?25l78M- path: "/var/www/leantime"[?12l[?25h[?25l78
M- path: "/tmp/leantime"[?12l[?25h[?25l78
Mwith_items:[?12l[?25h[?25l[?12l[?25h[?25l[?12l[?25h[?25l
[?12l[?25h[?25l[?12l[?25h[?25l[?12l[?25h[?25l[?12l[?25h[?25l
[?12l[?25h[?25l[?12l[?25h[?25l[?12l[?25h[?25l[?12l[?25h[?25l[?12l[?25h[?25l[?12l[?25h[?25l[?12l[?25h[?25l[?12l[?25h[?25l
[?12l[?25h[?25l[?12l[?25h[?25l         \textcolor{inv_background inv_foreground}{M-D} Formato DOS\textcolor{inv_background inv_foreground}{M-A} Anexar\textcolor{inv_background inv_foreground}{B} Segurança\textcolor{inv_background inv_foreground}{C} Cancelar           \textcolor{inv_background inv_foreground}{M-M} Formato Mac\textcolor{inv_background inv_foreground}{M-P} Prepor\textcolor{inv_background inv_foreground}{^T} P/ ficheiros
\textcolor{inv_background inv_foreground}{Nome do ficheiro onde escrever: tasks/tasks.yml                                                                                           }[?12l[?25h[?25l \textcolor{inv_background inv_foreground}{[ A escrever... ]}\textcolor{inv_background inv_foreground}{          }\textcolor{inv_background inv_foreground}{[ 105 linhas escritas ]}\textcolor{inv_background inv_foreground}{^O} Gravar\textcolor{inv_background inv_foreground}{^W} Procurar      \textcolor{inv_background inv_foreground}{^K} Cortar txt    \textcolor{inv_background inv_foreground}{^J} Justificar    \textcolor{inv_background inv_foreground}{^C} Pos cursor\textcolor{inv_background inv_foreground}{U} Desfazer     \textcolor{inv_background inv_foreground}{M-A} Marcar txt\textcolor{inv_background inv_foreground}{X} Sair    \textcolor{inv_background inv_foreground}{^R} Carregar\textcolor{inv_background inv_foreground}{^\} Substituir    \textcolor{inv_background inv_foreground}{^U} Colar txt     \textcolor{inv_background inv_foreground}{^T} Ortografia    \textcolor{inv_background inv_foreground}{^_} Ir p/ linha   \textcolor{inv_background inv_foreground}{M-E} Refazer      \textcolor{inv_background inv_foreground}{M-6} Copiar txt
[?12l[?25h[?25l[?12l[?25h[?1049l
[?1l>[?2004l

Use "fg" para retornar ao Nano.

[1]+  Interrompido            nano tasks/tasks.yml
]0;admlocal@devOps: ~/leantime_ansibleadmlocal@devOps:~/leantime_ansible$ molecule lint
\textcolor{ansi34}{INFO    } default scenario test matrix: dependency, lint
\textcolor{ansi34}{INFO    } Performing prerun\textcolor{ansi33}{...}
\textcolor{ansi34}{INFO    } Using .cache/roles/nunomourinho.leantime_ansible symlink to current repository in order to enable Ansible to find the role using its expected full name.
\textcolor{ansi34}{INFO    } Added \textcolor{ansi33}{ANSIBLE_ROLES_PATH}=~\textcolor{ansi35}{/.ansible/}\textcolor{ansi95}{roles}:\textcolor{ansi35}{/usr/share/ansible/}\textcolor{ansi95}{roles}:\textcolor{ansi35}{/etc/ansible/}\textcolor{ansi95}{roles}:.\textcolor{ansi35}{/.cache/}\textcolor{ansi95}{roles}
\textcolor{ansi34}{INFO    } \textcolor{ansi2 ansi36}{Running }\textcolor{ansi2 ansi32}{default}\textcolor{ansi2 ansi36}{ > }\textcolor{ansi2 ansi32}{dependency}
\textcolor{ansi31}{WARNING } Skipping, missing the requirements file.
\textcolor{ansi31}{WARNING } Skipping, missing the requirements file.
\textcolor{ansi34}{INFO    } \textcolor{ansi2 ansi36}{Running }\textcolor{ansi2 ansi32}{default}\textcolor{ansi2 ansi36}{ > }\textcolor{ansi2 ansi32}{lint}
COMMAND: set -e
yamllint .
ansible-lint

./vars/main.yml
  5:1       error    too many blank lines (1 > 0)  (empty-lines)

\textbf{\textcolor{ansi31}{CRITICAL}} Lint failed with error code \textbf{\textcolor{ansi36}{1}}
]0;admlocal@devOps: ~/leantime_ansibleadmlocal@devOps:~/leantime_ansible$ molecule lintnano tasks/tasks.yml vars/main.yml 
[?2004h[?1049h[?7h[?1h=[?1h=[?25l\textcolor{inv_background inv_foreground}{[ A ler... ]}\textcolor{inv_background inv_foreground}{[ 5 linhas lidas ]}\textcolor{inv_background inv_foreground}{  GNU nano 4.8                                                 vars/main.yml                                                              }
\textcolor{inv_background inv_foreground}{^G} Ajuda\textcolor{inv_background inv_foreground}{^O} Gravar\textcolor{inv_background inv_foreground}{^W} Procurar\textcolor{inv_background inv_foreground}{^K} Cortar txt    \textcolor{inv_background inv_foreground}{^J} Justificar    \textcolor{inv_background inv_foreground}{^C} Pos cursor    \textcolor{inv_background inv_foreground}{M-U} Desfazer     \textcolor{inv_background inv_foreground}{M-A} Marcar txt
\textcolor{inv_background inv_foreground}{^X} Sair\textcolor{inv_background inv_foreground}{^R} Carregar\textcolor{inv_background inv_foreground}{^\} Substituir    \textcolor{inv_background inv_foreground}{^U} Colar txt     \textcolor{inv_background inv_foreground}{^T} Ortografia    \textcolor{inv_background inv_foreground}{^_} Ir p/ linha   \textcolor{inv_background inv_foreground}{M-E} Refazer\textcolor{inv_background inv_foreground}{M-6} Copiar txt
--- login_leantime: leantimeDBadminpassword_leantime: "#S3gr3d0S3cr3t0#"
[?12l[?25h[?25l[?12l[?25h[?25l[?12l[?25h[?25l[?12l[?25h[?25l[?12l[?25h[?25l[?12l[?25h[?25l[?12l[?25h[?25l[?12l[?25h[?25l[?12l[?25h[?25l[?12l[?25h[?25l[?12l[?25h[?25l[?12l[?25h[?25l[?12l[?25h[?25l[?12l[?25h[?25l[?12l[?25h[?25l[?12l[?25h[?25l[?12l[?25h[?25l[?12l[?25h[?25l[?12l[?25h[?25l\textcolor{inv_background inv_foreground}{Modificado}
[?12l[?25h[?25l[?12l[?25h[?25l         \textcolor{inv_background inv_foreground}{M-D} Formato DOS\textcolor{inv_background inv_foreground}{M-A} Anexar\textcolor{inv_background inv_foreground}{B} Segurança\textcolor{inv_background inv_foreground}{C} Cancelar           \textcolor{inv_background inv_foreground}{M-M} Formato Mac\textcolor{inv_background inv_foreground}{M-P} Prepor\textcolor{inv_background inv_foreground}{^T} P/ ficheiros
\textcolor{inv_background inv_foreground}{Nome do ficheiro onde escrever: vars/main.yml                                                                                             }[?12l[?25h[?25l \textcolor{inv_background inv_foreground}{[ A escrever... ]}\textcolor{inv_background inv_foreground}{          }\textcolor{inv_background inv_foreground}{[ 4 linhas escritas ]}\textcolor{inv_background inv_foreground}{^O} Gravar\textcolor{inv_background inv_foreground}{^W} Procurar      \textcolor{inv_background inv_foreground}{^K} Cortar txt    \textcolor{inv_background inv_foreground}{^J} Justificar    \textcolor{inv_background inv_foreground}{^C} Pos cursor\textcolor{inv_background inv_foreground}{U} Desfazer     \textcolor{inv_background inv_foreground}{M-A} Marcar txt\textcolor{inv_background inv_foreground}{X} Sair    \textcolor{inv_background inv_foreground}{^R} Carregar\textcolor{inv_background inv_foreground}{^\} Substituir    \textcolor{inv_background inv_foreground}{^U} Colar txt     \textcolor{inv_background inv_foreground}{^T} Ortografia    \textcolor{inv_background inv_foreground}{^_} Ir p/ linha   \textcolor{inv_background inv_foreground}{M-E} Refazer      \textcolor{inv_background inv_foreground}{M-6} Copiar txt
[?12l[?25h[?25l
[?12l[?25h[?1049l
[?1l>[?2004l]0;admlocal@devOps: ~/leantime_ansibleadmlocal@devOps:~/leantime_ansible$ nano vars/main.yml molecule lint
\textcolor{ansi34}{INFO    } default scenario test matrix: dependency, lint
\textcolor{ansi34}{INFO    } Performing prerun\textcolor{ansi33}{...}
\textcolor{ansi34}{INFO    } Using .cache/roles/nunomourinho.leantime_ansible symlink to current repository in order to enable Ansible to find the role using its expected full name.
\textcolor{ansi34}{INFO    } Added \textcolor{ansi33}{ANSIBLE_ROLES_PATH}=~\textcolor{ansi35}{/.ansible/}\textcolor{ansi95}{roles}:\textcolor{ansi35}{/usr/share/ansible/}\textcolor{ansi95}{roles}:\textcolor{ansi35}{/etc/ansible/}\textcolor{ansi95}{roles}:.\textcolor{ansi35}{/.cache/}\textcolor{ansi95}{roles}
\textcolor{ansi34}{INFO    } \textcolor{ansi2 ansi36}{Running }\textcolor{ansi2 ansi32}{default}\textcolor{ansi2 ansi36}{ > }\textcolor{ansi2 ansi32}{dependency}
\textcolor{ansi31}{WARNING } Skipping, missing the requirements file.
\textcolor{ansi31}{WARNING } Skipping, missing the requirements file.
\textcolor{ansi34}{INFO    } \textcolor{ansi2 ansi36}{Running }\textcolor{ansi2 ansi32}{default}\textcolor{ansi2 ansi36}{ > }\textcolor{ansi2 ansi32}{lint}
COMMAND: set -e
yamllint .
ansible-lint

Loading custom .yamllint config file, this extends our internal yamllint config.
WARNING  Listing 1 violation(s) that are fatal
\textcolor{ansi91}{command-instead-of-shell}\textcolor{ansi2}{:} \textcolor{ansi31}{Use shell only when shell functionality is required}
\textcolor{ansi34}{molecule/default/verify.yml}:166 \textcolor{ansi2}{Task/Handler: Simulação: Verifica se é necessário criar a base de dados leantime_database}

You can skip specific rules or tags by adding them to your configuration file:
\textcolor{ansi2}{# .ansible-lint}
\textcolor{ansi94}{warn_list}:  \textcolor{ansi2}{# or 'skip_list' to silence them completely}
  - command-instead-of-shell  \textcolor{ansi2}{# Use shell only when shell functionality is required}
Finished with \textbf{\textcolor{ansi36}{1}} failure\textcolor{ansi1}{(}s\textcolor{ansi1}{)}, \textbf{\textcolor{ansi36}{0}} warning\textcolor{ansi1}{(}s\textcolor{ansi1}{)} on \textbf{\textcolor{ansi36}{38}} files.
\textbf{\textcolor{ansi31}{CRITICAL}} Lint failed with error code \textbf{\textcolor{ansi36}{2}}
]0;admlocal@devOps: ~/leantime_ansibleadmlocal@devOps:~/leantime_ansible$ nano .ansible-lint
[?2004h[?1049h[?7h[?1h=[?1h=[?25l\textcolor{inv_background inv_foreground}{[ Novo ficheiro ]}\textcolor{inv_background inv_foreground}{  GNU nano 4.8                                                 .ansible-lint                                                              }
\textcolor{inv_background inv_foreground}{^G} Ajuda\textcolor{inv_background inv_foreground}{^O} Gravar\textcolor{inv_background inv_foreground}{^W} Procurar\textcolor{inv_background inv_foreground}{^K} Cortar txt    \textcolor{inv_background inv_foreground}{^J} Justificar    \textcolor{inv_background inv_foreground}{^C} Pos cursor    \textcolor{inv_background inv_foreground}{M-U} Desfazer     \textcolor{inv_background inv_foreground}{M-A} Marcar txt
\textcolor{inv_background inv_foreground}{^X} Sair\textcolor{inv_background inv_foreground}{^R} Carregar\textcolor{inv_background inv_foreground}{^\} Substituir    \textcolor{inv_background inv_foreground}{^U} Colar txt     \textcolor{inv_background inv_foreground}{^T} Ortografia    \textcolor{inv_background inv_foreground}{^_} Ir p/ linha   \textcolor{inv_background inv_foreground}{M-E} Refazer\textcolor{inv_background inv_foreground}{M-6} Copiar txt
[?12l[?25h[?25l\textcolor{inv_background inv_foreground}{Modificado
}
warn_list:  # or 'skip_list' to silence them completely- command-instead-of-shell [?12l[?25h[?25l
[?12l[?25h[?25l         \textcolor{inv_background inv_foreground}{M-D} Formato DOS\textcolor{inv_background inv_foreground}{M-A} Anexar\textcolor{inv_background inv_foreground}{B} Segurança\textcolor{inv_background inv_foreground}{C} Cancelar           \textcolor{inv_background inv_foreground}{M-M} Formato Mac\textcolor{inv_background inv_foreground}{M-P} Prepor\textcolor{inv_background inv_foreground}{^T} P/ ficheiros
\textcolor{inv_background inv_foreground}{Nome do ficheiro onde escrever: .ansible-lint                                                                                             }[?12l[?25h[?25l \textcolor{inv_background inv_foreground}{[ A escrever em FIFO... ]}    \textcolor{inv_background inv_foreground}{[ A escrever... ]}\textcolor{inv_background inv_foreground}{          }\textcolor{inv_background inv_foreground}{[ 3 linhas escritas ]}\textcolor{inv_background inv_foreground}{^O} Gravar\textcolor{inv_background inv_foreground}{^W} Procurar      \textcolor{inv_background inv_foreground}{^K} Cortar txt    \textcolor{inv_background inv_foreground}{^J} Justificar    \textcolor{inv_background inv_foreground}{^C} Pos cursor\textcolor{inv_background inv_foreground}{U} Desfazer     \textcolor{inv_background inv_foreground}{M-A} Marcar txt\textcolor{inv_background inv_foreground}{X} Sair    \textcolor{inv_background inv_foreground}{^R} Carregar\textcolor{inv_background inv_foreground}{^\} Substituir    \textcolor{inv_background inv_foreground}{^U} Colar txt     \textcolor{inv_background inv_foreground}{^T} Ortografia    \textcolor{inv_background inv_foreground}{^_} Ir p/ linha   \textcolor{inv_background inv_foreground}{M-E} Refazer      \textcolor{inv_background inv_foreground}{M-6} Copiar txt
[?12l[?25h[?25l[?12l[?25h[?1049l
[?1l>[?2004l]0;admlocal@devOps: ~/leantime_ansibleadmlocal@devOps:~/leantime_ansible$ nano .ansible-lintmolecule lint
\textcolor{ansi34}{INFO    } default scenario test matrix: dependency, lint
\textcolor{ansi34}{INFO    } Performing prerun\textcolor{ansi33}{...}
\textcolor{ansi34}{INFO    } Using .cache/roles/nunomourinho.leantime_ansible symlink to current repository in order to enable Ansible to find the role using its expected full name.
\textcolor{ansi34}{INFO    } Added \textcolor{ansi33}{ANSIBLE_ROLES_PATH}=~\textcolor{ansi35}{/.ansible/}\textcolor{ansi95}{roles}:\textcolor{ansi35}{/usr/share/ansible/}\textcolor{ansi95}{roles}:\textcolor{ansi35}{/etc/ansible/}\textcolor{ansi95}{roles}:.\textcolor{ansi35}{/.cache/}\textcolor{ansi95}{roles}
\textcolor{ansi34}{INFO    } \textcolor{ansi2 ansi36}{Running }\textcolor{ansi2 ansi32}{default}\textcolor{ansi2 ansi36}{ > }\textcolor{ansi2 ansi32}{dependency}
\textcolor{ansi31}{WARNING } Skipping, missing the requirements file.
\textcolor{ansi31}{WARNING } Skipping, missing the requirements file.
\textcolor{ansi34}{INFO    } \textcolor{ansi2 ansi36}{Running }\textcolor{ansi2 ansi32}{default}\textcolor{ansi2 ansi36}{ > }\textcolor{ansi2 ansi32}{lint}
COMMAND: set -e
yamllint .
ansible-lint

Loading custom .yamllint config file, this extends our internal yamllint config.
WARNING  Listing 1 violation(s) that are fatal
\textcolor{ansi91}{command-instead-of-shell}\textcolor{ansi2}{:} \textcolor{ansi31}{Use shell only when shell functionality is required}
\textcolor{ansi34}{molecule/default/verify.yml}:166 \textcolor{ansi2}{Task/Handler: Simulação: Verifica se é necessário criar a base de dados leantime_database}

You can skip specific rules or tags by adding them to your configuration file:
\textcolor{ansi2}{# .ansible-lint}
\textcolor{ansi94}{warn_list}:  \textcolor{ansi2}{# or 'skip_list' to silence them completely}
Finished with \textbf{\textcolor{ansi36}{0}} failure\textcolor{ansi1}{(}s\textcolor{ansi1}{)}, \textbf{\textcolor{ansi36}{1}} warning\textcolor{ansi1}{(}s\textcolor{ansi1}{)} on \textbf{\textcolor{ansi36}{38}} files.
]0;admlocal@devOps: ~/leantime_ansibleadmlocal@devOps:~/leantime_ansible$ molecule lintnano .ansible-lint
[?2004h[?1049h[?7h[?1h=[?1h=[?25l\textcolor{inv_background inv_foreground}{[ A ler... ]}\textcolor{inv_background inv_foreground}{[ 3 linhas lidas ]}\textcolor{inv_background inv_foreground}{  GNU nano 4.8                                                 .ansible-lint                                                              }
\textcolor{inv_background inv_foreground}{^G} Ajuda\textcolor{inv_background inv_foreground}{^O} Gravar\textcolor{inv_background inv_foreground}{^W} Procurar\textcolor{inv_background inv_foreground}{^K} Cortar txt    \textcolor{inv_background inv_foreground}{^J} Justificar    \textcolor{inv_background inv_foreground}{^C} Pos cursor    \textcolor{inv_background inv_foreground}{M-U} Desfazer     \textcolor{inv_background inv_foreground}{M-A} Marcar txt
\textcolor{inv_background inv_foreground}{^X} Sair\textcolor{inv_background inv_foreground}{^R} Carregar\textcolor{inv_background inv_foreground}{^\} Substituir    \textcolor{inv_background inv_foreground}{^U} Colar txt     \textcolor{inv_background inv_foreground}{^T} Ortografia    \textcolor{inv_background inv_foreground}{^_} Ir p/ linha   \textcolor{inv_background inv_foreground}{M-E} Refazer\textcolor{inv_background inv_foreground}{M-6} Copiar txt
[?12l[?25h[?25l\textcolor{inv_background inv_foreground}{Modificado}
[?12l[?25h[?25l[?12l[?25h[?25l[?12l[?25h[?25l[?12l[?25h[?25ls_list:  # or 'skip_list' to silence them completely
s[?12l[?25h[?25lk_list:  # or 'skip_list' to silence them completely
sk[?12l[?25h[?25li_list:  # or 'skip_list' to silence them completely
ski[?12l[?25h[?25lp_list:  # or 'skip_list' to silence them completely[?12l[?25h[?25l         \textcolor{inv_background inv_foreground}{M-D} Formato DOS\textcolor{inv_background inv_foreground}{M-A} Anexar\textcolor{inv_background inv_foreground}{B} Segurança\textcolor{inv_background inv_foreground}{C} Cancelar           \textcolor{inv_background inv_foreground}{M-M} Formato Mac\textcolor{inv_background inv_foreground}{M-P} Prepor\textcolor{inv_background inv_foreground}{^T} P/ ficheiros
\textcolor{inv_background inv_foreground}{Nome do ficheiro onde escrever: .ansible-lint                                                                                             }[?12l[?25h[?25l \textcolor{inv_background inv_foreground}{[ A escrever... ]}\textcolor{inv_background inv_foreground}{          }\textcolor{inv_background inv_foreground}{[ 3 linhas escritas ]}\textcolor{inv_background inv_foreground}{^O} Gravar\textcolor{inv_background inv_foreground}{^W} Procurar      \textcolor{inv_background inv_foreground}{^K} Cortar txt    \textcolor{inv_background inv_foreground}{^J} Justificar    \textcolor{inv_background inv_foreground}{^C} Pos cursor\textcolor{inv_background inv_foreground}{U} Desfazer     \textcolor{inv_background inv_foreground}{M-A} Marcar txt\textcolor{inv_background inv_foreground}{X} Sair    \textcolor{inv_background inv_foreground}{^R} Carregar\textcolor{inv_background inv_foreground}{^\} Substituir    \textcolor{inv_background inv_foreground}{^U} Colar txt     \textcolor{inv_background inv_foreground}{^T} Ortografia    \textcolor{inv_background inv_foreground}{^_} Ir p/ linha   \textcolor{inv_background inv_foreground}{M-E} Refazer      \textcolor{inv_background inv_foreground}{M-6} Copiar txt
[?12l[?25h[?25l_[?12l[?25h[?25ll[?12l[?25h[?25l
[?12l[?25h[?1049l
[?1l>[?2004l]0;admlocal@devOps: ~/leantime_ansibleadmlocal@devOps:~/leantime_ansible$ nano .ansible-lintmolecule lint
\textcolor{ansi34}{INFO    } default scenario test matrix: dependency, lint
\textcolor{ansi34}{INFO    } Performing prerun\textcolor{ansi33}{...}
\textcolor{ansi34}{INFO    } Using .cache/roles/nunomourinho.leantime_ansible symlink to current repository in order to enable Ansible to find the role using its expected full name.
\textcolor{ansi34}{INFO    } Added \textcolor{ansi33}{ANSIBLE_ROLES_PATH}=~\textcolor{ansi35}{/.ansible/}\textcolor{ansi95}{roles}:\textcolor{ansi35}{/usr/share/ansible/}\textcolor{ansi95}{roles}:\textcolor{ansi35}{/etc/ansible/}\textcolor{ansi95}{roles}:.\textcolor{ansi35}{/.cache/}\textcolor{ansi95}{roles}
\textcolor{ansi34}{INFO    } \textcolor{ansi2 ansi36}{Running }\textcolor{ansi2 ansi32}{default}\textcolor{ansi2 ansi36}{ > }\textcolor{ansi2 ansi32}{dependency}
\textcolor{ansi31}{WARNING } Skipping, missing the requirements file.
\textcolor{ansi31}{WARNING } Skipping, missing the requirements file.
\textcolor{ansi34}{INFO    } \textcolor{ansi2 ansi36}{Running }\textcolor{ansi2 ansi32}{default}\textcolor{ansi2 ansi36}{ > }\textcolor{ansi2 ansi32}{lint}
COMMAND: set -e
yamllint .
ansible-lint

Loading custom .yamllint config file, this extends our internal yamllint config.
]0;admlocal@devOps: ~/leantime_ansibleadmlocal@devOps:~/leantime_ansible$ molecule converge
\textcolor{ansi34}{INFO    } default scenario test matrix: dependency, create, prepare, converge
\textcolor{ansi34}{INFO    } Performing prerun\textcolor{ansi33}{...}
\textcolor{ansi34}{INFO    } Using .cache/roles/nunomourinho.leantime_ansible symlink to current repository in order to enable Ansible to find the role using its expected full name.
\textcolor{ansi34}{INFO    } Added \textcolor{ansi33}{ANSIBLE_ROLES_PATH}=~\textcolor{ansi35}{/.ansible/}\textcolor{ansi95}{roles}:\textcolor{ansi35}{/usr/share/ansible/}\textcolor{ansi95}{roles}:\textcolor{ansi35}{/etc/ansible/}\textcolor{ansi95}{roles}:.\textcolor{ansi35}{/.cache/}\textcolor{ansi95}{roles}
\textcolor{ansi34}{INFO    } \textcolor{ansi2 ansi36}{Running }\textcolor{ansi2 ansi32}{default}\textcolor{ansi2 ansi36}{ > }\textcolor{ansi2 ansi32}{dependency}
\textcolor{ansi31}{WARNING } Skipping, missing the requirements file.
\textcolor{ansi31}{WARNING } Skipping, missing the requirements file.
\textcolor{ansi34}{INFO    } \textcolor{ansi2 ansi36}{Running }\textcolor{ansi2 ansi32}{default}\textcolor{ansi2 ansi36}{ > }\textcolor{ansi2 ansi32}{create}
\textcolor{ansi31}{WARNING } Skipping, instances already created.
\textcolor{ansi34}{INFO    } \textcolor{ansi2 ansi36}{Running }\textcolor{ansi2 ansi32}{default}\textcolor{ansi2 ansi36}{ > }\textcolor{ansi2 ansi32}{prepare}
\textcolor{ansi31}{WARNING } Skipping, instances already prepared.
\textcolor{ansi34}{INFO    } \textcolor{ansi2 ansi36}{Running }\textcolor{ansi2 ansi32}{default}\textcolor{ansi2 ansi36}{ > }\textcolor{ansi2 ansi32}{converge}

PLAY [Converge] **************************************************************************************************************************

TASK [Gathering Facts] *******************************************************************************************************************
\textcolor{ansi32}{ok: [ubuntu-20.04]}

TASK [Include leantime_ansible] **********************************************************************************************************

TASK [leantime_ansible : Atualiza a cache (equivalente a apt update)] ********************************************************************
\textcolor{ansi33}{changed: [ubuntu-20.04]}

TASK [leantime_ansible : Atualiza o sistema operativo] ***********************************************************************************
\textcolor{ansi32}{ok: [ubuntu-20.04]}

TASK [leantime_ansible : Instalar as dependencias necessária ao programa leantime] *******************************************************
\textcolor{ansi32}{ok: [ubuntu-20.04]}

TASK [leantime_ansible : Instala o serviço apache2 no arranque do sistema] ***************************************************************
\textcolor{ansi32}{ok: [ubuntu-20.04]}

TASK [leantime_ansible : Rotina Auxiliar: Procura o caminho do ficheiro php.ini do servidor apache2] *************************************
\textcolor{ansi32}{ok: [ubuntu-20.04]}

TASK [leantime_ansible : Ativa várias opções no ficheiro php.ini, utilizando o módulo lineinfile.] ***************************************
\textcolor{ansi32}{ok: [ubuntu-20.04] => (item={'regexp': '^file_uploads', 'line': 'file_uploads = 1'})}
\textcolor{ansi32}{ok: [ubuntu-20.04] => (item={'regexp': '^upload_max_filesize', 'line': 'upload_max_filesize = 1G'})}
\textcolor{ansi32}{ok: [ubuntu-20.04] => (item={'regexp': '^max_file_uploads', 'line': 'max_file_uploads = 20'})}
\textcolor{ansi32}{ok: [ubuntu-20.04] => (item={'regexp': '^post_max_size', 'line': 'post_max_size = 2G'})}
\textcolor{ansi32}{ok: [ubuntu-20.04] => (item={'regexp': '^memory_limit', 'line': 'memory_limit = 2G'})}
\textcolor{ansi32}{ok: [ubuntu-20.04] => (item={'regexp': '^max_input_time', 'line': 'max_input_time = 3600'})}

TASK [leantime_ansible : Cria a directoria temporária leantime e a directoria de apache leantime] ****************************************
\textcolor{ansi32}{ok: [ubuntu-20.04] => (item={'path': '/tmp/leantime'})}
\textcolor{ansi32}{ok: [ubuntu-20.04] => (item={'path': '/var/www/leantime'})}

TASK [leantime_ansible : Cria a base de dados de mysql para o leantime] ******************************************************************
\textcolor{ansi33}{changed: [ubuntu-20.04]}

TASK [leantime_ansible : Cria o utilizador para a base de dados leantime] ****************************************************************
\textbf{\textcolor{ansi35}{[WARNING]: Module did not set no_log for update_password}}
\textcolor{ansi33}{changed: [ubuntu-20.04]}

PLAY RECAP *******************************************************************************************************************************
\textcolor{ansi33}{ubuntu-20.04}               : \textcolor{ansi32}{ok=10  } \textcolor{ansi33}{changed=3   } unreachable=0    failed=0    skipped=0    rescued=0    ignored=0


]0;admlocal@devOps: ~/leantime_ansibleadmlocal@devOps:~/leantime_ansible$ molecule idempotence
\textcolor{ansi34}{INFO    } default scenario test matrix: idempotence
\textcolor{ansi34}{INFO    } Performing prerun\textcolor{ansi33}{...}
\textcolor{ansi34}{INFO    } Using .cache/roles/nunomourinho.leantime_ansible symlink to current repository in order to enable Ansible to find the role using its expected full name.
\textcolor{ansi34}{INFO    } Added \textcolor{ansi33}{ANSIBLE_ROLES_PATH}=~\textcolor{ansi35}{/.ansible/}\textcolor{ansi95}{roles}:\textcolor{ansi35}{/usr/share/ansible/}\textcolor{ansi95}{roles}:\textcolor{ansi35}{/etc/ansible/}\textcolor{ansi95}{roles}:.\textcolor{ansi35}{/.cache/}\textcolor{ansi95}{roles}
\textcolor{ansi34}{INFO    } \textcolor{ansi2 ansi36}{Running }\textcolor{ansi2 ansi32}{default}\textcolor{ansi2 ansi36}{ > }\textcolor{ansi2 ansi32}{idempotence}

PLAY [Converge] **************************************************************************************************************************

TASK [Gathering Facts] *******************************************************************************************************************
\textcolor{ansi32}{ok: [ubuntu-20.04]}

TASK [Include leantime_ansible] **********************************************************************************************************

TASK [leantime_ansible : Atualiza a cache (equivalente a apt update)] ********************************************************************
\textcolor{ansi32}{ok: [ubuntu-20.04]}

TASK [leantime_ansible : Atualiza o sistema operativo] ***********************************************************************************
\textcolor{ansi32}{ok: [ubuntu-20.04]}

TASK [leantime_ansible : Instalar as dependencias necessária ao programa leantime] *******************************************************
\textcolor{ansi32}{ok: [ubuntu-20.04]}

TASK [leantime_ansible : Instala o serviço apache2 no arranque do sistema] ***************************************************************
\textcolor{ansi32}{ok: [ubuntu-20.04]}

TASK [leantime_ansible : Rotina Auxiliar: Procura o caminho do ficheiro php.ini do servidor apache2] *************************************
\textcolor{ansi32}{ok: [ubuntu-20.04]}

TASK [leantime_ansible : Ativa várias opções no ficheiro php.ini, utilizando o módulo lineinfile.] ***************************************
\textcolor{ansi32}{ok: [ubuntu-20.04] => (item={'regexp': '^file_uploads', 'line': 'file_uploads = 1'})}
\textcolor{ansi32}{ok: [ubuntu-20.04] => (item={'regexp': '^upload_max_filesize', 'line': 'upload_max_filesize = 1G'})}
\textcolor{ansi32}{ok: [ubuntu-20.04] => (item={'regexp': '^max_file_uploads', 'line': 'max_file_uploads = 20'})}
\textcolor{ansi32}{ok: [ubuntu-20.04] => (item={'regexp': '^post_max_size', 'line': 'post_max_size = 2G'})}
\textcolor{ansi32}{ok: [ubuntu-20.04] => (item={'regexp': '^memory_limit', 'line': 'memory_limit = 2G'})}
\textcolor{ansi32}{ok: [ubuntu-20.04] => (item={'regexp': '^max_input_time', 'line': 'max_input_time = 3600'})}

TASK [leantime_ansible : Cria a directoria temporária leantime e a directoria de apache leantime] ****************************************
\textcolor{ansi32}{ok: [ubuntu-20.04] => (item={'path': '/tmp/leantime'})}
\textcolor{ansi32}{ok: [ubuntu-20.04] => (item={'path': '/var/www/leantime'})}

TASK [leantime_ansible : Cria a base de dados de mysql para o leantime] ******************************************************************
\textcolor{ansi32}{ok: [ubuntu-20.04]}

TASK [leantime_ansible : Cria o utilizador para a base de dados leantime] ****************************************************************
\textbf{\textcolor{ansi35}{[WARNING]: Module did not set no_log for update_password}}
\textcolor{ansi33}{changed: [ubuntu-20.04]}

PLAY RECAP *******************************************************************************************************************************
\textcolor{ansi33}{ubuntu-20.04}               : \textcolor{ansi32}{ok=10  } \textcolor{ansi33}{changed=1   } unreachable=0    failed=0    skipped=0    rescued=0    ignored=0


\textbf{\textcolor{ansi31}{CRITICAL}} Idempotence test failed because of the following tasks:
*  => leantime_ansible : Cria o utilizador para a base de dados leantime
]0;admlocal@devOps: ~/leantime_ansibleadmlocal@devOps:~/leantime_ansible$ molecule verify
\textcolor{ansi34}{INFO    } default scenario test matrix: verify
\textcolor{ansi34}{INFO    } Performing prerun\textcolor{ansi33}{...}
\textcolor{ansi34}{INFO    } Using .cache/roles/nunomourinho.leantime_ansible symlink to current repository in order to enable Ansible to find the role using its expected full name.
\textcolor{ansi34}{INFO    } Added \textcolor{ansi33}{ANSIBLE_ROLES_PATH}=~\textcolor{ansi35}{/.ansible/}\textcolor{ansi95}{roles}:\textcolor{ansi35}{/usr/share/ansible/}\textcolor{ansi95}{roles}:\textcolor{ansi35}{/etc/ansible/}\textcolor{ansi95}{roles}:.\textcolor{ansi35}{/.cache/}\textcolor{ansi95}{roles}
\textcolor{ansi34}{INFO    } \textcolor{ansi2 ansi36}{Running }\textcolor{ansi2 ansi32}{default}\textcolor{ansi2 ansi36}{ > }\textcolor{ansi2 ansi32}{verify}
\textcolor{ansi34}{INFO    } Running Ansible Verifier

PLAY [Infraestrutura conduzida por testes] ***********************************************************************************************

TASK [Simulação: Atualizar a cache do sistema] *******************************************************************************************
\textcolor{ansi32}{ok: [ubuntu-20.04]}

TASK [teste: a cache encontra-se actualizada?] *******************************************************************************************
\textcolor{ansi32}{ok: [ubuntu-20.04] => {}
\textcolor{ansi32}{    "changed": false,}
\textcolor{ansi32}{    "msg": "SUCESSO: A cache está atualizada"}
\textcolor{ansi32}{}}

TASK [Atualizar o sistema operativo (equivalente a apt upgrade)] *************************************************************************
\textcolor{ansi32}{ok: [ubuntu-20.04]}

TASK [teste: o sistema operativo encontra-se atualizado?] ********************************************************************************
\textcolor{ansi32}{ok: [ubuntu-20.04] => {}
\textcolor{ansi32}{    "changed": false,}
\textcolor{ansi32}{    "msg": "SUCESSO: O sistema operativo está atualizado"}
\textcolor{ansi32}{}}

TASK [Teste: a cache encontra-se atualizada?] ********************************************************************************************
\textcolor{ansi32}{ok: [ubuntu-20.04] => {}
\textcolor{ansi32}{    "changed": false,}
\textcolor{ansi32}{    "msg": "SUCESSO: A cache está atualizada"}
\textcolor{ansi32}{}}

TASK [Simulação: testa se as aplicações dependencia do software leantime se encontram instaladas] ****************************************
\textcolor{ansi32}{ok: [ubuntu-20.04]}

TASK [Teste: as dependencias encontra-se instaladas?] ************************************************************************************
\textcolor{ansi32}{ok: [ubuntu-20.04] => {}
\textcolor{ansi32}{    "changed": false,}
\textcolor{ansi32}{    "msg": "SUCESSO: As dependencias estavam instaladas"}
\textcolor{ansi32}{}}

TASK [Simulação: Testar se o serviço apache2 se encontra instalado, iniciado e ativo no arranque] ****************************************
\textcolor{ansi32}{ok: [ubuntu-20.04]}

TASK [Teste: O serviço apache2 encontra-se ativo no arranque no sistema, e está iniciado?] ***********************************************
\textcolor{ansi32}{ok: [ubuntu-20.04] => {}
\textcolor{ansi32}{    "changed": false,}
\textcolor{ansi32}{    "msg": "SUCESSO: O serviço apache2 está correctamente instalado e inicia com o arranque do sistema"}
\textcolor{ansi32}{}}

TASK [Rotina Auxiliar> Procura o caminho do ficheiro php.ini do servidor apache2] ********************************************************
\textcolor{ansi32}{ok: [ubuntu-20.04]}

TASK [Simulação: Ativa a opção file_uploads no ficheiro php.ini, utilizando o módulo lineinfile.] ****************************************
\textcolor{ansi32}{ok: [ubuntu-20.04] => (item={'regexp': '^file_uploads', 'line': 'file_uploads = 1'})}
\textcolor{ansi32}{ok: [ubuntu-20.04] => (item={'regexp': '^upload_max_filesize', 'line': 'upload_max_filesize = 1G'})}
\textcolor{ansi32}{ok: [ubuntu-20.04] => (item={'regexp': '^max_file_uploads', 'line': 'max_file_uploads = 20'})}
\textcolor{ansi32}{ok: [ubuntu-20.04] => (item={'regexp': '^post_max_size', 'line': 'post_max_size = 2G'})}
\textcolor{ansi32}{ok: [ubuntu-20.04] => (item={'regexp': '^memory_limit', 'line': 'memory_limit = 2G'})}
\textcolor{ansi32}{ok: [ubuntu-20.04] => (item={'regexp': '^max_input_time', 'line': 'max_input_time = 3600'})}

TASK [Teste: As linhas do php.ini encontram-se alteradas ?] ******************************************************************************
\textcolor{ansi32}{ok: [ubuntu-20.04] => {}
\textcolor{ansi32}{    "changed": false,}
\textcolor{ansi32}{    "msg": "SUCESSO: O ficheiro php.ini foi alterado com sucesso"}
\textcolor{ansi32}{}}

TASK [Simulação: Obtem informação sobre a pastas /var/www/leantime] **********************************************************************
\textcolor{ansi32}{ok: [ubuntu-20.04]}

TASK [Teste: A pasta /var/www/leantime existe e tem as permissões certas?] ***************************************************************
\textcolor{ansi32}{ok: [ubuntu-20.04] => {}
\textcolor{ansi32}{    "changed": false,}
\textcolor{ansi32}{    "msg": "SUCESSO: Permissões correctas no site leantime"}
\textcolor{ansi32}{}}

TASK [Simulação: Verifica se é necessário criar a base de dados leantime_database] *******************************************************
\textcolor{ansi32}{ok: [ubuntu-20.04]}

TASK [Teste: A base de dados leantime_database existe ?] *********************************************************************************
\textcolor{ansi32}{ok: [ubuntu-20.04] => {}
\textcolor{ansi32}{    "changed": false,}
\textcolor{ansi32}{    "msg": "SUCESSO: A base de dados leantime_database já se encontra criada"}
\textcolor{ansi32}{}}

PLAY RECAP *******************************************************************************************************************************
\textcolor{ansi32}{ubuntu-20.04}               : \textcolor{ansi32}{ok=16  } changed=0    unreachable=0    failed=0    skipped=0    rescued=0    ignored=0

\textcolor{ansi34}{INFO    } Verifier completed successfully.
]0;admlocal@devOps: ~/leantime_ansibleadmlocal@devOps:~/leantime_ansible$ molecule verifyidempotence
\textcolor{ansi34}{INFO    } default scenario test matrix: idempotence
\textcolor{ansi34}{INFO    } Performing prerun\textcolor{ansi33}{...}
\textcolor{ansi34}{INFO    } Using .cache/roles/nunomourinho.leantime_ansible symlink to current repository in order to enable Ansible to find the role using its expected full name.
\textcolor{ansi34}{INFO    } Added \textcolor{ansi33}{ANSIBLE_ROLES_PATH}=~\textcolor{ansi35}{/.ansible/}\textcolor{ansi95}{roles}:\textcolor{ansi35}{/usr/share/ansible/}\textcolor{ansi95}{roles}:\textcolor{ansi35}{/etc/ansible/}\textcolor{ansi95}{roles}:.\textcolor{ansi35}{/.cache/}\textcolor{ansi95}{roles}
\textcolor{ansi34}{INFO    } \textcolor{ansi2 ansi36}{Running }\textcolor{ansi2 ansi32}{default}\textcolor{ansi2 ansi36}{ > }\textcolor{ansi2 ansi32}{idempotence}

PLAY [Converge] **************************************************************************************************************************

TASK [Gathering Facts] *******************************************************************************************************************
\textcolor{ansi32}{ok: [ubuntu-20.04]}

TASK [Include leantime_ansible] **********************************************************************************************************

TASK [leantime_ansible : Atualiza a cache (equivalente a apt update)] ********************************************************************
\textcolor{ansi32}{ok: [ubuntu-20.04]}

TASK [leantime_ansible : Atualiza o sistema operativo] ***********************************************************************************
\textcolor{ansi32}{ok: [ubuntu-20.04]}

TASK [leantime_ansible : Instalar as dependencias necessária ao programa leantime] *******************************************************
\textcolor{ansi32}{ok: [ubuntu-20.04]}

TASK [leantime_ansible : Instala o serviço apache2 no arranque do sistema] ***************************************************************
\textcolor{ansi32}{ok: [ubuntu-20.04]}

TASK [leantime_ansible : Rotina Auxiliar: Procura o caminho do ficheiro php.ini do servidor apache2] *************************************
\textcolor{ansi32}{ok: [ubuntu-20.04]}

TASK [leantime_ansible : Ativa várias opções no ficheiro php.ini, utilizando o módulo lineinfile.] ***************************************
\textcolor{ansi32}{ok: [ubuntu-20.04] => (item={'regexp': '^file_uploads', 'line': 'file_uploads = 1'})}
\textcolor{ansi32}{ok: [ubuntu-20.04] => (item={'regexp': '^upload_max_filesize', 'line': 'upload_max_filesize = 1G'})}
\textcolor{ansi32}{ok: [ubuntu-20.04] => (item={'regexp': '^max_file_uploads', 'line': 'max_file_uploads = 20'})}
\textcolor{ansi32}{ok: [ubuntu-20.04] => (item={'regexp': '^post_max_size', 'line': 'post_max_size = 2G'})}
\textcolor{ansi32}{ok: [ubuntu-20.04] => (item={'regexp': '^memory_limit', 'line': 'memory_limit = 2G'})}
\textcolor{ansi32}{ok: [ubuntu-20.04] => (item={'regexp': '^max_input_time', 'line': 'max_input_time = 3600'})}

TASK [leantime_ansible : Cria a directoria temporária leantime e a directoria de apache leantime] ****************************************
\textcolor{ansi32}{ok: [ubuntu-20.04] => (item={'path': '/tmp/leantime'})}
\textcolor{ansi32}{ok: [ubuntu-20.04] => (item={'path': '/var/www/leantime'})}

TASK [leantime_ansible : Cria a base de dados de mysql para o leantime] ******************************************************************
\textcolor{ansi32}{ok: [ubuntu-20.04]}

TASK [leantime_ansible : Cria o utilizador para a base de dados leantime] ****************************************************************
\textbf{\textcolor{ansi35}{[WARNING]: Module did not set no_log for update_password}}
\textcolor{ansi32}{ok: [ubuntu-20.04]}

PLAY RECAP *******************************************************************************************************************************
\textcolor{ansi32}{ubuntu-20.04}               : \textcolor{ansi32}{ok=10  } changed=0    unreachable=0    failed=0    skipped=0    rescued=0    ignored=0


\textcolor{ansi34}{INFO    } Idempotence completed successfully.
]0;admlocal@devOps: ~/leantime_ansibleadmlocal@devOps:~/leantime_ansible$ exit
exit
Há tarefas interrompidas.
]0;admlocal@devOps: ~/leantime_ansibleadmlocal@devOps:~/leantime_ansible$ exit
exit
[?1049l
[?1l>[?2004lRecebido SIGHUP ou SIGTERM

\end{Verbatim}
\end{document}
